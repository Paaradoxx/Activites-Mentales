\documentclass[15pt, mathserif]{beamer}

\usepackage[french]{babel}
\usepackage[T1]{fontenc}
\usepackage[utf8]{inputenc}
%\usepackage{esvect}
\usepackage{bm}
\usepackage{eurosym}
\usepackage{tikz}
\usepackage{pgf,tikz,pgfplots}
\pgfplotsset{compat=1.15}
\usepackage{mathrsfs}
\usetikzlibrary{arrows}
\usetikzlibrary{arrows.meta}

\usetikzlibrary{mindmap}
\usepackage{multicol}
\usepackage[tikz]{bclogo}
\usepackage{tkz-tab}
\usepackage{amsmath, tabu}
\usepackage{esvect} %\vv{AB} pour le vecteur AB

\DeclareMathOperator{\e}{e}

%% Tableau

\usepackage{makecell}
\setcellgapes{1pt}
\makegapedcells
\newcolumntype{R}[1]{>{\raggedleft\arraybackslash }b{#1}}
\newcolumntype{L}[1]{>{\raggedright\arraybackslash }b{#1}}
\newcolumntype{C}[1]{>{\centering\arraybackslash }b{#1}}


%pour avoir des parenthèses rondes dans le package fourier
\DeclareSymbolFont{cmoperators}   {OT1}{cmr} {m}{n}
\DeclareSymbolFont{cmlargesymbols}{OMX}{cmex}{m}{n}

\usefonttheme{professionalfonts} %permet d'enlever un bug avec fourier
\usepackage{fourier}
\DeclareMathDelimiter{(}{\mathopen} {cmoperators}{"28}{cmlargesymbols}{"00}
\DeclareMathDelimiter{)}{\mathclose}{cmoperators}{"29}{cmlargesymbols}{"01}

%Graphiques 

\usepackage{pgf,tikz,pgfplots}
\pgfplotsset{compat=1.15}
\usepackage{mathrsfs}
\usetikzlibrary{arrows}
\usetikzlibrary{mindmap}

%ensembles de nbres

\newcommand{\R}{\mathbb{R}}			%permet d'écrire le R "ensemble des réels"'
\newcommand{\N}{\mathbb{N}}			%permet d'écrire le N "ensemble des entiers naturels"
\newcommand{\Z}{\mathbb{Z}}			%permet d'écrire le Z "ensemble des entiers relatifs"
\newcommand{\Prem}{\mathbb{P}}	%permet d'écrire le P "ensemble des nombres premiers" (qui n'a pas marché avec le \P car il existe déjà)
\newcommand{\D}{\mathbb{D}}
\newcommand{\Df}{\mathcal{D}_f}
\newcommand{\Cf}{\mathcal{C}_f}

\newcommand{\Q}{\mathbb{Q}}


\newcommand{\st}[1]{$(#1_n)_{n \in \N}$}

\usetheme{Madrid}
\useoutertheme{miniframes} % Alternatively: miniframes, infolines, split
\useinnertheme{circles}
\definecolor{UBCblue}{rgb}{0.1, 0.25, 0.4} % UBC Blue (primary)
\definecolor{bordeaux}{RGB}{128,0,0}
\usecolortheme[named=UBCblue]{structure}

\usepackage{color} % J'aime bien définir mes couleurs
\definecolor{propcolor}{rgb}{0, 0.5, 1}
\definecolor{thcolor}{rgb}{0.6, 0.07, 0.07}
\colorlet{louis}{blue!70!green!60!white}
\colorlet{sakura}{pink!40!red}

\title{Activités Mentales}
\date{24 Août 2023}

\newcommand{\vco}[2]{\begin{pmatrix} #1 \\ #2 \end{pmatrix}} %Coordonnées de vecteur
\newenvironment{eq}{\begin{cases}\begin{tabu}{ccccc}}{\end{tabu}\end{cases}}
\newenvironment{eql}{\begin{cases}\begin{tabu}{cccccl}}{\end{tabu}\end{cases}}
\newenvironment{eqrl}{\begin{cases}\begin{tabu}{rl}}{\end{tabu}\end{cases}}

\newenvironment{Eq}{\begin{center}\begin{tabular}{rrcl}}{\end{tabular}\end{center}}
\newcommand{\ligneq}[2]{$\Longleftrightarrow$ & $#1$ & $=$ & $#2$ \\}
\newcommand{\Ligneq}[2]{ & $#1$ & $=$ & $#2$ \\}

\newenvironment{RPN}{\begin{center}\begin{tabular}{rrclcrcl}}{\end{tabular}\end{center}}
\newcommand{\Lignerpn}[4]{ & $#1$ & $=$ & $#2$ & ou & $#3$ & $=$ & $#4$ \\}
\newcommand{\lignerpn}[4]{$\Longleftrightarrow$ & $#1$ & $=$ & $#2$ & ou & $#3$ & $=$ & $#4$ \\}

\newenvironment{TRPN}{\begin{center}\begin{tabular}{rrclcrclcrcl}}{\end{tabular}\end{center}}
\newcommand{\Lignetrpn}[6]{ & $#1$ & $=$ & $#2$ & ou & $#3$ & $=$ & $#4$ & ou & $#5$ & $=$ & $#6$ \\}
\newcommand{\lignetrpn}[6]{$\Longleftrightarrow$ & $#1$ & $=$ & $#2$ & ou & $#3$ & $=$ & $#4$ & ou & $#5$ & $=$ & $#6$ \\}
\begin{document}

\begin{frame}
    \titlepage
\end{frame}

\begin{frame} 
	\frametitle{Question 1}
Soit $f$ la fonction définie sur $\mathbb{R}$ par $f(x) = 5x^2+7x+8$.

 Déterminer si $f$ admet une forme factorisée et préciser celle-ci dans le cas où elle existe. \bigskip

\end{frame}


\begin{frame} 
	\frametitle{Question 2}
Soit $f$ la fonction définie sur $\mathbb{R}$ par $f(x) = -4x^2+28x-24$.

 Déterminer si $f$ admet une forme factorisée et préciser celle-ci dans le cas où elle existe. \bigskip

\end{frame}


\begin{frame} 
	\frametitle{Question 3}
Soit $f$ la fonction définie sur $\mathbb{R}$ par $f(x) = -x^2+2x-1$.

 Déterminer si $f$ admet une forme factorisée et préciser celle-ci dans le cas où elle existe. \bigskip

\end{frame}


\begin{frame} 
	\frametitle{Question 4}
Soit $f$ la fonction définie sur $\mathbb{R}$ par $f(x) = 10x^2+7x+3$.

 Déterminer si $f$ admet une forme factorisée et préciser celle-ci dans le cas où elle existe. \bigskip

\end{frame}


\begin{frame} 
	\frametitle{Question 5}
Soit $f$ la fonction définie sur $\mathbb{R}$ par $f(x) = 7x^2+70x+168$.

 Déterminer si $f$ admet une forme factorisée et préciser celle-ci dans le cas où elle existe. \bigskip

\end{frame}


\begin{frame}
\vspace{-10mm}
	\frametitle{Correction 1}
$f$ est un polynôme de degré $2$ dont les coefficients sont $a =5, \; b =7$ et $c =8$.
 
 On a $\Delta = b^2-4ac =7^2-4 \times5\times8=49-160 = -111<0$.


 
 Comme $\Delta <0$, $f$ ne possède pas de racines réelles et n'admet donc pas de forme factorisée dans $\mathbb{R}$.\end{frame}


\begin{frame}
\vspace{-10mm}
	\frametitle{Correction 2}
$f$ est un polynôme de degré $2$ dont les coefficients sont $a =-4, \; b =28$ et $c =-24$.
 
 On a $\Delta = b^2-4ac =28^2-4 \times\left(-4\right)\times\left(-24\right)=784-384 = 400>0$.


 
 Comme $\Delta>0$, $f$ admet deux racines distinctes. 

\[ x_1 = \dfrac{-b-\sqrt{\Delta}}{2a} = \dfrac{-28-20}{2 \times \left(-4\right)} = \dfrac{-48}{-8} = 6 \] et \[ x_2 = \dfrac{-b+\sqrt{\Delta}}{2a} = \dfrac{-28+20}{2 \times \left(-4\right)} = \dfrac{-8}{-8}=1\] 

Finalement, la forme factorisée de $f$ est: $f(x) = a(x-x_1)(x-x_2) = -4(x-6)(x-1)$\end{frame}


\begin{frame}
\vspace{-10mm}
	\frametitle{Correction 3}
$f$ est un polynôme de degré $2$ dont les coefficients sont $a =-1, \; b =2$ et $c =-1$.
 
 On a $\Delta = b^2-4ac =2^2-4 \times\left(-1\right)\times\left(-1\right)=4-4 = 0$.


 
 Comme $\Delta =0$,  $f$ admet une unique racine $$x_0 = \dfrac{-b}{2a} = \dfrac{-2}{2 \times \left(-1\right)} = 1$$. 

 Finalement la forme factorisée de $f$ est: \[f(x) = a(x-x_0)^2 = -\left(x-1\right)^2\] 

 \bigskip\end{frame}


\begin{frame}
\vspace{-10mm}
	\frametitle{Correction 4}
$f$ est un polynôme de degré $2$ dont les coefficients sont $a =10, \; b =7$ et $c =3$.
 
 On a $\Delta = b^2-4ac =7^2-4 \times10\times3=49-120 = -71<0$.


 
 Comme $\Delta <0$, $f$ ne possède pas de racines réelles et n'admet donc pas de forme factorisée dans $\mathbb{R}$.\end{frame}


\begin{frame}
\vspace{-10mm}
	\frametitle{Correction 5}
$f$ est un polynôme de degré $2$ dont les coefficients sont $a =7, \; b =70$ et $c =168$.
 
 On a $\Delta = b^2-4ac =70^2-4 \times7\times168=4900-4704 = 196>0$.


 
 Comme $\Delta>0$, $f$ admet deux racines distinctes. 

\[ x_1 = \dfrac{-b-\sqrt{\Delta}}{2a} = \dfrac{-70-14}{2 \times 7} = \dfrac{-84}{14} = -6 \] et \[ x_2 = \dfrac{-b+\sqrt{\Delta}}{2a} = \dfrac{-70+14}{2 \times 7} = \dfrac{-56}{14}=-4\] 

Finalement, la forme factorisée de $f$ est: $f(x) = a(x-x_1)(x-x_2) = 7(x-\left(-6\right))(x-\left(-4\right))=7(x+6)(x+4)$\end{frame}




\end{document}