\documentclass[15pt, mathserif]{beamer}

\usepackage[french]{babel}
\usepackage[T1]{fontenc}
\usepackage[utf8]{inputenc}
%\usepackage{esvect}
\usepackage{bm}
\usepackage{eurosym}
\usepackage{tikz}
\usepackage{pgf,tikz,pgfplots}
\pgfplotsset{compat=1.15}
\usepackage{mathrsfs}
\usetikzlibrary{arrows}
\usetikzlibrary{arrows.meta}

\usetikzlibrary{mindmap}
\usepackage{multicol}
\usepackage[tikz]{bclogo}
\usepackage{tkz-tab}
\usepackage{amsmath, tabu}
\usepackage{esvect} %\vv{AB} pour le vecteur AB

\DeclareMathOperator{\e}{e}

%% Tableau

\usepackage{makecell}
\setcellgapes{1pt}
\makegapedcells
\newcolumntype{R}[1]{>{\raggedleft\arraybackslash }b{#1}}
\newcolumntype{L}[1]{>{\raggedright\arraybackslash }b{#1}}
\newcolumntype{C}[1]{>{\centering\arraybackslash }b{#1}}


%pour avoir des parenthèses rondes dans le package fourier
\DeclareSymbolFont{cmoperators}   {OT1}{cmr} {m}{n}
\DeclareSymbolFont{cmlargesymbols}{OMX}{cmex}{m}{n}

\usefonttheme{professionalfonts} %permet d'enlever un bug avec fourier
\usepackage{fourier}
\DeclareMathDelimiter{(}{\mathopen} {cmoperators}{"28}{cmlargesymbols}{"00}
\DeclareMathDelimiter{)}{\mathclose}{cmoperators}{"29}{cmlargesymbols}{"01}

%Graphiques 

\usepackage{pgf,tikz,pgfplots}
\pgfplotsset{compat=1.15}
\usepackage{mathrsfs}
\usetikzlibrary{arrows}
\usetikzlibrary{mindmap}

%ensembles de nbres

\newcommand{\R}{\mathbb{R}}			%permet d'écrire le R "ensemble des réels"'
\newcommand{\N}{\mathbb{N}}			%permet d'écrire le N "ensemble des entiers naturels"
\newcommand{\Z}{\mathbb{Z}}			%permet d'écrire le Z "ensemble des entiers relatifs"
\newcommand{\Prem}{\mathbb{P}}	%permet d'écrire le P "ensemble des nombres premiers" (qui n'a pas marché avec le \P car il existe déjà)
\newcommand{\D}{\mathbb{D}}
\newcommand{\Df}{\mathcal{D}_f}
\newcommand{\Cf}{\mathcal{C}_f}

\newcommand{\Q}{\mathbb{Q}}


\newcommand{\st}[1]{$(#1_n)_{n \in \N}$}

\usetheme{Madrid}
\useoutertheme{miniframes} % Alternatively: miniframes, infolines, split
\useinnertheme{circles}
\definecolor{UBCblue}{rgb}{0.1, 0.25, 0.4} % UBC Blue (primary)
\definecolor{bordeaux}{RGB}{128,0,0}
\usecolortheme[named=UBCblue]{structure}

\usepackage{color} % J'aime bien définir mes couleurs
\definecolor{propcolor}{rgb}{0, 0.5, 1}
\definecolor{thcolor}{rgb}{0.6, 0.07, 0.07}
\colorlet{louis}{blue!70!green!60!white}
\colorlet{sakura}{pink!40!red}

\title{Activités Mentales}
\date{24 Août 2023}

\newcommand{\vco}[2]{\begin{pmatrix} #1 \\ #2 \end{pmatrix}} %Coordonnées de vecteur
\newenvironment{eq}{\begin{cases}\begin{tabu}{ccccc}}{\end{tabu}\end{cases}}
\newenvironment{eql}{\begin{cases}\begin{tabu}{cccccl}}{\end{tabu}\end{cases}}
\newenvironment{eqrl}{\begin{cases}\begin{tabu}{rl}}{\end{tabu}\end{cases}}

\newenvironment{Eq}{\begin{center}\begin{tabular}{rrcl}}{\end{tabular}\end{center}}
\newcommand{\ligneq}[2]{$\Longleftrightarrow$ & $#1$ & $=$ & $#2$ \\}
\newcommand{\Ligneq}[2]{ & $#1$ & $=$ & $#2$ \\}

\newenvironment{RPN}{\begin{center}\begin{tabular}{rrclcrcl}}{\end{tabular}\end{center}}
\newcommand{\Lignerpn}[4]{ & $#1$ & $=$ & $#2$ & ou & $#3$ & $=$ & $#4$ \\}
\newcommand{\lignerpn}[4]{$\Longleftrightarrow$ & $#1$ & $=$ & $#2$ & ou & $#3$ & $=$ & $#4$ \\}

\newenvironment{TRPN}{\begin{center}\begin{tabular}{rrclcrclcrcl}}{\end{tabular}\end{center}}
\newcommand{\Lignetrpn}[6]{ & $#1$ & $=$ & $#2$ & ou & $#3$ & $=$ & $#4$ & ou & $#5$ & $=$ & $#6$ \\}
\newcommand{\lignetrpn}[6]{$\Longleftrightarrow$ & $#1$ & $=$ & $#2$ & ou & $#3$ & $=$ & $#4$ & ou & $#5$ & $=$ & $#6$ \\}
\begin{document}

\begin{frame}
    \titlepage
\end{frame}

\begin{frame} 
	\frametitle{Question 1}
On considère une fonction $f : x \mapsto -3x^2+5x+2$. \\ Le point $A$ d'abscisse 7 appartient à $\mathcal{C}_f$ qui est la courbe représentative de la fonction $f$. Calculer l'ordonnée de $A$.\end{frame}


\begin{frame} 
	\frametitle{Question 2}
On considère une fonction $f : x \mapsto -2x^2+6x-10$ Le point $A(-1;-18 )$ appartient-il à $\mathcal{C}_f$ la courbe représentative de la fonction $f$ ?\end{frame}


\begin{frame} 
	\frametitle{Question 3}
On considère une fonction $f : x \mapsto x^2+3x+8$. \\ Le point $A$ d'abscisse 1 appartient à $\mathcal{C}_f$ qui est la courbe représentative de la fonction $f$. Calculer l'ordonnée de $A$.\end{frame}


\begin{frame} 
	\frametitle{Question 4}
On considère une fonction $f : x \mapsto -3x^2-6x+3$. \\ Le point $A$ d'abscisse 4 appartient à $\mathcal{C}_f$ qui est la courbe représentative de la fonction $f$. Calculer l'ordonnée de $A$.\end{frame}


\begin{frame} 
	\frametitle{Question 5}
On considère une fonction $f : x \mapsto x^2+8x-4$ Le point $A(-6;-16 )$ appartient-il à $\mathcal{C}_f$ la courbe représentative de la fonction $f$ ?\end{frame}


\begin{frame}
\vspace{-10mm}
	\frametitle{Correction 1}
On considère une fonction $f : x \mapsto -3x^2+5x+2$ Le point $A$ d'abscisse 7 appartient à $\mathcal{C}_f$ qui est la courbe représentative de la fonction $f$. Calculer l'ordonnée de $A$. \\ Il faut calculer l'image de $7$ c'est à dire on calcule $f(7)$. On a : $$ f(7)=-3\times7^2+5\times7+2=-3\times49+35+2=-147+35+2=-110$$ Donc $A( 7;-110)$.\end{frame}


\begin{frame}
\vspace{-10mm}
	\frametitle{Correction 2}
On considère une fonction $f : x \mapsto -2x^2+6x-10$ Le point $A(-1;-18 )$ appartient-il à $\mathcal{C}_f$ la courbe représentative de la fonction $f$ ? \\ Il faut calculer l'image de $-1$ c'est à dire on calcule $f(-1)$. et on regarde si on obtient -18. On a : $$ f(-1)=-2\times\left(-1\right)^2+6\times\left(-1\right)-10=-2\times1-6-10=-2-6-10=-18$$ Donc $A( -1;-18) $ appartient bien à la courbe.\end{frame}


\begin{frame}
\vspace{-10mm}
	\frametitle{Correction 3}
On considère une fonction $f : x \mapsto x^2+3x+8$ Le point $A$ d'abscisse 1 appartient à $\mathcal{C}_f$ qui est la courbe représentative de la fonction $f$. Calculer l'ordonnée de $A$. \\ Il faut calculer l'image de $1$ c'est à dire on calcule $f(1)$. On a : $$ f(1)=1\times1^2+3\times1+8=1\times1+3+8=1+3+8=12$$ Donc $A( 1;12)$.\end{frame}


\begin{frame}
\vspace{-10mm}
	\frametitle{Correction 4}
On considère une fonction $f : x \mapsto -3x^2-6x+3$ Le point $A$ d'abscisse 4 appartient à $\mathcal{C}_f$ qui est la courbe représentative de la fonction $f$. Calculer l'ordonnée de $A$. \\ Il faut calculer l'image de $4$ c'est à dire on calcule $f(4)$. On a : $$ f(4)=-3\times4^2-6\times4+3=-3\times16-24+3=-48-24+3=-69$$ Donc $A( 4;-69)$.\end{frame}


\begin{frame}
\vspace{-10mm}
	\frametitle{Correction 5}
On considère une fonction $f : x \mapsto x^2+8x-4$ Le point $A(-6;-16 )$ appartient-il à $\mathcal{C}_f$ la courbe représentative de la fonction $f$ ? \\ Il faut calculer l'image de $-6$ c'est à dire on calcule $f(-6)$. et on regarde si on obtient -16. On a : $$ f(-6)=1\times\left(-6\right)^2+8\times\left(-6\right)-4=1\times36-48-4=36-48-4=-16$$ Donc $A( -6;-16) $ appartient bien à la courbe.\end{frame}




\end{document}