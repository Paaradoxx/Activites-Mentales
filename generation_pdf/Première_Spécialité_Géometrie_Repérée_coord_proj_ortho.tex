\documentclass[15pt, mathserif]{beamer}

\usepackage[french]{babel}
\usepackage[T1]{fontenc}
\usepackage[utf8]{inputenc}
%\usepackage{esvect}
\usepackage{bm}
\usepackage{eurosym}
\usepackage{tikz}
\usepackage{pgf,tikz,pgfplots}
\pgfplotsset{compat=1.15}
\usepackage{mathrsfs}
\usetikzlibrary{arrows}
\usetikzlibrary{arrows.meta}

\usetikzlibrary{mindmap}
\usepackage{multicol}
\usepackage[tikz]{bclogo}
\usepackage{tkz-tab}
\usepackage{amsmath, tabu}
\usepackage{esvect} %\vv{AB} pour le vecteur AB

\DeclareMathOperator{\e}{e}

%% Tableau

\usepackage{makecell}
\setcellgapes{1pt}
\makegapedcells
\newcolumntype{R}[1]{>{\raggedleft\arraybackslash }b{#1}}
\newcolumntype{L}[1]{>{\raggedright\arraybackslash }b{#1}}
\newcolumntype{C}[1]{>{\centering\arraybackslash }b{#1}}


%pour avoir des parenthèses rondes dans le package fourier
\DeclareSymbolFont{cmoperators}   {OT1}{cmr} {m}{n}
\DeclareSymbolFont{cmlargesymbols}{OMX}{cmex}{m}{n}

\usefonttheme{professionalfonts} %permet d'enlever un bug avec fourier
\usepackage{fourier}
\DeclareMathDelimiter{(}{\mathopen} {cmoperators}{"28}{cmlargesymbols}{"00}
\DeclareMathDelimiter{)}{\mathclose}{cmoperators}{"29}{cmlargesymbols}{"01}

%Graphiques 

\usepackage{pgf,tikz,pgfplots}
\pgfplotsset{compat=1.15}
\usepackage{mathrsfs}
\usetikzlibrary{arrows}
\usetikzlibrary{mindmap}

%ensembles de nbres

\newcommand{\R}{\mathbb{R}}			%permet d'écrire le R "ensemble des réels"'
\newcommand{\N}{\mathbb{N}}			%permet d'écrire le N "ensemble des entiers naturels"
\newcommand{\Z}{\mathbb{Z}}			%permet d'écrire le Z "ensemble des entiers relatifs"
\newcommand{\Prem}{\mathbb{P}}	%permet d'écrire le P "ensemble des nombres premiers" (qui n'a pas marché avec le \P car il existe déjà)
\newcommand{\D}{\mathbb{D}}
\newcommand{\Df}{\mathcal{D}_f}
\newcommand{\Cf}{\mathcal{C}_f}

\newcommand{\Q}{\mathbb{Q}}


\newcommand{\st}[1]{$(#1_n)_{n \in \N}$}

\usetheme{Madrid}
\useoutertheme{miniframes} % Alternatively: miniframes, infolines, split
\useinnertheme{circles}
\definecolor{UBCblue}{rgb}{0.1, 0.25, 0.4} % UBC Blue (primary)
\definecolor{bordeaux}{RGB}{128,0,0}
\usecolortheme[named=UBCblue]{structure}

\usepackage{color} % J'aime bien définir mes couleurs
\definecolor{propcolor}{rgb}{0, 0.5, 1}
\definecolor{thcolor}{rgb}{0.6, 0.07, 0.07}
\colorlet{louis}{blue!70!green!60!white}
\colorlet{sakura}{pink!40!red}

\title{Activités Mentales}
\date{24 Août 2023}

\newcommand{\vco}[2]{\begin{pmatrix} #1 \\ #2 \end{pmatrix}} %Coordonnées de vecteur
\newenvironment{eq}{\begin{cases}\begin{tabu}{ccccc}}{\end{tabu}\end{cases}}
\newenvironment{eql}{\begin{cases}\begin{tabu}{cccccl}}{\end{tabu}\end{cases}}
\newenvironment{eqrl}{\begin{cases}\begin{tabu}{rl}}{\end{tabu}\end{cases}}

\newenvironment{Eq}{\begin{center}\begin{tabular}{rrcl}}{\end{tabular}\end{center}}
\newcommand{\ligneq}[2]{$\Longleftrightarrow$ & $#1$ & $=$ & $#2$ \\}
\newcommand{\Ligneq}[2]{ & $#1$ & $=$ & $#2$ \\}

\newenvironment{RPN}{\begin{center}\begin{tabular}{rrclcrcl}}{\end{tabular}\end{center}}
\newcommand{\Lignerpn}[4]{ & $#1$ & $=$ & $#2$ & ou & $#3$ & $=$ & $#4$ \\}
\newcommand{\lignerpn}[4]{$\Longleftrightarrow$ & $#1$ & $=$ & $#2$ & ou & $#3$ & $=$ & $#4$ \\}

\newenvironment{TRPN}{\begin{center}\begin{tabular}{rrclcrclcrcl}}{\end{tabular}\end{center}}
\newcommand{\Lignetrpn}[6]{ & $#1$ & $=$ & $#2$ & ou & $#3$ & $=$ & $#4$ & ou & $#5$ & $=$ & $#6$ \\}
\newcommand{\lignetrpn}[6]{$\Longleftrightarrow$ & $#1$ & $=$ & $#2$ & ou & $#3$ & $=$ & $#4$ & ou & $#5$ & $=$ & $#6$ \\}
\begin{document}

\begin{frame}
    \titlepage
\end{frame}

\begin{frame} 
	\frametitle{Question 1}
Soit $d$ la droite d'équation cartésienne $-8x+5y-103=0$ et le point $A(-3;-2)$. 
 Déterminer les coordonnées du projeté orthogonal $H$ de $A$ sur $d$.\end{frame}


\begin{frame} 
	\frametitle{Question 2}
Soit $d$ la droite d'équation cartésienne $-3x+8y-110=0$ et le point $A(2;-22)$. 
 Déterminer les coordonnées du projeté orthogonal $H$ de $A$ sur $d$.\end{frame}


\begin{frame} 
	\frametitle{Question 3}
Soit $d$ la droite d'équation cartésienne $-9x+3y-117=0$ et le point $A(-8;5)$. 
 Déterminer les coordonnées du projeté orthogonal $H$ de $A$ sur $d$.\end{frame}


\begin{frame} 
	\frametitle{Question 4}
Soit $d$ la droite d'équation cartésienne $2x-2y-10=0$ et le point $A(5;-26)$. 
 Déterminer les coordonnées du projeté orthogonal $H$ de $A$ sur $d$.\end{frame}


\begin{frame} 
	\frametitle{Question 5}
Soit $d$ la droite d'équation cartésienne $8x+4y+88=0$ et le point $A(-5;-2)$. 
 Déterminer les coordonnées du projeté orthogonal $H$ de $A$ sur $d$.\end{frame}


\begin{frame}
\vspace{-10mm}
	\frametitle{Correction 1}
\vspace*{0.5cm} 
 Tout d'abord, $A \notin d$ car $-8\times \left(-3\right)+5\times \left(-2\right)-103=-89 \neq0$. 
 $H$ est l'intersection des deux droites $d$ et $(AH)$. 
 Déterminons une équation cartésienne de la droite $(AH)$ :   $d$ admet comme vecteur directeur $\vec{u} \begin{pmatrix} -5 \\ -8\end{pmatrix}$. Donc $\vec{u}$ est un vecteur normal à $(AH)$ et une équation cartésienne de $(AH)$ est : $$ -5x -8y+c=0 \quad \text{où $c$ est un réel à déterminer.}$$ $(AH)$ passe par $A$ si et seulement si :
 \begin{Eq} 
 	 \Ligneq{-5\times \left(-3\right)-8\times \left(-2\right)+c}{0} 
 	 \ligneq{31+c}{0} 
 	 \ligneq{c}{-31} 
 \end{Eq} $(AH)$ admet pour équation cartésienne $-5x -8y-31=0$. 
 
 \end{frame} 
 
 \begin{frame} 
 $H$ est l'intersection des deux droites $d$ et $(AH)$ donc ses coordonnées $(x;y)$ vérifient le système :
\begin{align*}
	(S)&\Leftrightarrow\begin{cases}-8x \quad + \quad 5y \quad -103 \quad &= \quad 0\\-5x \quad - \quad 8y \quad -31 \quad &= \quad 0\end{cases}\\&\Leftrightarrow\begin{eq}-8x&+&5y&=&103\\-5x&-&8y&=&31\end{eq}\\&\Leftrightarrow\begin{eql}40x&-&25y&=&-515& (L_1) \leftarrow -5\times (L_1) \\40x&+&64y&=&-248& (L_2) \leftarrow -8\times (L_2)\end{eql} \\
	&\Leftrightarrow\begin{eql}40x&-&25y&=&-515& (L_1) \\40x-40x&+&64y+25y&=&-248+515& (L_2) \leftarrow (L_2) - (L_1)\end{eql} \\
	&\Leftrightarrow\begin{eql}40x&-&25y&=&-515& (L_1) \\&&89y&=&267& (L_2)\end{eql} 
\end{align*} 

 \end{frame} 
 
 \begin{frame} 
 \vspace*{-1cm} 
\begin{align*}
	(S)&\Leftrightarrow\begin{eql}40x&-&25y&=&-515& (L_1) \\&&89y&=&267& (L_2)\end{eql} \\ &\Leftrightarrow \begin{eql} 40x&-&25y&=&-515 & \\& &y&=&\dfrac{267}{89} &= 3\end{eql}\\&\Leftrightarrow \begin{eqrl}40x-25\times3&=-515\\y&=3\end{eqrl}\\
	&\Leftrightarrow \begin{eqrl}40x&=-515+75\\y&=3\end{eqrl}\\
	&\Leftrightarrow \begin{eqrl}x&=\dfrac{-440}{40} = -11\\y&=3\end{eqrl}
\end{align*} D'où les solutions de $(S)$ sont $\left\{(-11~;~3)\right\}$. Donc $H$ a pour coordonnées $(-11~;~3)$.\end{frame}


\begin{frame}
\vspace{-10mm}
	\frametitle{Correction 2}
\vspace*{0.5cm} 
 Tout d'abord, $A \notin d$ car $-3\times 2+8\times \left(-22\right)-110=-292 \neq0$. 
 $H$ est l'intersection des deux droites $d$ et $(AH)$. 
 Déterminons une équation cartésienne de la droite $(AH)$ :   $d$ admet comme vecteur directeur $\vec{u} \begin{pmatrix} -8 \\ -3\end{pmatrix}$. Donc $\vec{u}$ est un vecteur normal à $(AH)$ et une équation cartésienne de $(AH)$ est : $$ -8x -3y+c=0 \quad \text{où $c$ est un réel à déterminer.}$$ $(AH)$ passe par $A$ si et seulement si :
 \begin{Eq} 
 	 \Ligneq{-8\times 2-3\times \left(-22\right)+c}{0} 
 	 \ligneq{50+c}{0} 
 	 \ligneq{c}{-50} 
 \end{Eq} $(AH)$ admet pour équation cartésienne $-8x -3y-50=0$. 
 
 \end{frame} 
 
 \begin{frame} 
 $H$ est l'intersection des deux droites $d$ et $(AH)$ donc ses coordonnées $(x;y)$ vérifient le système :
\begin{align*}
	(S)&\Leftrightarrow\begin{cases}-3x \quad + \quad 8y \quad -110 \quad &= \quad 0\\-8x \quad - \quad 3y \quad -50 \quad &= \quad 0\end{cases}\\&\Leftrightarrow\begin{eq}-3x&+&8y&=&110\\-8x&-&3y&=&50\end{eq}\\&\Leftrightarrow\begin{eql}24x&-&64y&=&-880& (L_1) \leftarrow -8\times (L_1) \\24x&+&9y&=&-150& (L_2) \leftarrow -3\times (L_2)\end{eql} \\
	&\Leftrightarrow\begin{eql}24x&-&64y&=&-880& (L_1) \\24x-24x&+&9y+64y&=&-150+880& (L_2) \leftarrow (L_2) - (L_1)\end{eql} \\
	&\Leftrightarrow\begin{eql}24x&-&64y&=&-880& (L_1) \\&&73y&=&730& (L_2)\end{eql} 
\end{align*} 

 \end{frame} 
 
 \begin{frame} 
 \vspace*{-1cm} 
\begin{align*}
	(S)&\Leftrightarrow\begin{eql}24x&-&64y&=&-880& (L_1) \\&&73y&=&730& (L_2)\end{eql} \\ &\Leftrightarrow \begin{eql} 24x&-&64y&=&-880 & \\& &y&=&\dfrac{730}{73} &= 10\end{eql}\\&\Leftrightarrow \begin{eqrl}24x-64\times10&=-880\\y&=10\end{eqrl}\\
	&\Leftrightarrow \begin{eqrl}24x&=-880+640\\y&=10\end{eqrl}\\
	&\Leftrightarrow \begin{eqrl}x&=\dfrac{-240}{24} = -10\\y&=10\end{eqrl}
\end{align*} D'où les solutions de $(S)$ sont $\left\{(-10~;~10)\right\}$. Donc $H$ a pour coordonnées $(-10~;~10)$.\end{frame}


\begin{frame}
\vspace{-10mm}
	\frametitle{Correction 3}
\vspace*{0.5cm} 
 Tout d'abord, $A \notin d$ car $-9\times \left(-8\right)+3\times 5-117=-30 \neq0$. 
 $H$ est l'intersection des deux droites $d$ et $(AH)$. 
 Déterminons une équation cartésienne de la droite $(AH)$ :   $d$ admet comme vecteur directeur $\vec{u} \begin{pmatrix} -3 \\ -9\end{pmatrix}$. Donc $\vec{u}$ est un vecteur normal à $(AH)$ et une équation cartésienne de $(AH)$ est : $$ -3x -9y+c=0 \quad \text{où $c$ est un réel à déterminer.}$$ $(AH)$ passe par $A$ si et seulement si :
 \begin{Eq} 
 	 \Ligneq{-3\times \left(-8\right)-9\times 5+c}{0} 
 	 \ligneq{-21+c}{0} 
 	 \ligneq{c}{21} 
 \end{Eq} $(AH)$ admet pour équation cartésienne $-3x -9y+21=0$. 
 
 \end{frame} 
 
 \begin{frame} 
 $H$ est l'intersection des deux droites $d$ et $(AH)$ donc ses coordonnées $(x;y)$ vérifient le système :
\begin{align*}
	(S)&\Leftrightarrow\begin{cases}-9x \quad + \quad 3y \quad -117 \quad &= \quad 0\\-3x \quad - \quad 9y \quad +21 \quad &= \quad 0\end{cases}\\&\Leftrightarrow\begin{eq}-9x&+&3y&=&117\\-3x&-&9y&=&-21\end{eq}\\&\Leftrightarrow\begin{eql}27x&-&9y&=&-351& (L_1) \leftarrow -3\times (L_1) \\27x&+&81y&=&189& (L_2) \leftarrow -9\times (L_2)\end{eql} \\
	&\Leftrightarrow\begin{eql}27x&-&9y&=&-351& (L_1) \\27x-27x&+&81y+9y&=&189+351& (L_2) \leftarrow (L_2) - (L_1)\end{eql} \\
	&\Leftrightarrow\begin{eql}27x&-&9y&=&-351& (L_1) \\&&90y&=&540& (L_2)\end{eql} 
\end{align*} 

 \end{frame} 
 
 \begin{frame} 
 \vspace*{-1cm} 
\begin{align*}
	(S)&\Leftrightarrow\begin{eql}27x&-&9y&=&-351& (L_1) \\&&90y&=&540& (L_2)\end{eql} \\ &\Leftrightarrow \begin{eql} 27x&-&9y&=&-351 & \\& &y&=&\dfrac{540}{90} &= 6\end{eql}\\&\Leftrightarrow \begin{eqrl}27x-9\times6&=-351\\y&=6\end{eqrl}\\
	&\Leftrightarrow \begin{eqrl}27x&=-351+54\\y&=6\end{eqrl}\\
	&\Leftrightarrow \begin{eqrl}x&=\dfrac{-297}{27} = -11\\y&=6\end{eqrl}
\end{align*} D'où les solutions de $(S)$ sont $\left\{(-11~;~6)\right\}$. Donc $H$ a pour coordonnées $(-11~;~6)$.\end{frame}


\begin{frame}
\vspace{-10mm}
	\frametitle{Correction 4}
\vspace*{0.5cm} 
 Tout d'abord, $A \notin d$ car $2\times 5-2\times \left(-26\right)-10=52 \neq0$. 
 $H$ est l'intersection des deux droites $d$ et $(AH)$. 
 Déterminons une équation cartésienne de la droite $(AH)$ :   $d$ admet comme vecteur directeur $\vec{u} \begin{pmatrix} 2 \\ 2\end{pmatrix}$. Donc $\vec{u}$ est un vecteur normal à $(AH)$ et une équation cartésienne de $(AH)$ est : $$ 2x +2y+c=0 \quad \text{où $c$ est un réel à déterminer.}$$ $(AH)$ passe par $A$ si et seulement si :
 \begin{Eq} 
 	 \Ligneq{2\times 5+2\times \left(-26\right)+c}{0} 
 	 \ligneq{-42+c}{0} 
 	 \ligneq{c}{42} 
 \end{Eq} $(AH)$ admet pour équation cartésienne $2x +2y+42=0$. 
 
 \end{frame} 
 
 \begin{frame} 
 $H$ est l'intersection des deux droites $d$ et $(AH)$ donc ses coordonnées $(x;y)$ vérifient le système :
\begin{align*}
	(S)&\Leftrightarrow\begin{cases}2x \quad - \quad 2y \quad -10 \quad &= \quad 0\\2x \quad + \quad 2y \quad +42 \quad &= \quad 0\end{cases}\\&\Leftrightarrow\begin{eq}2x&-&2y&=&10\\2x&+&2y&=&-42\end{eq}\\&\Leftrightarrow\begin{eql}4x&-&4y&=&20& (L_1) \leftarrow 2\times (L_1) \\4x&+&4y&=&-84& (L_2) \leftarrow 2\times (L_2)\end{eql} \\
	&\Leftrightarrow\begin{eql}4x&-&4y&=&20& (L_1) \\4x-4x&+&4y+4y&=&-84-20& (L_2) \leftarrow (L_2) - (L_1)\end{eql} \\
	&\Leftrightarrow\begin{eql}4x&-&4y&=&20& (L_1) \\&&8y&=&-104& (L_2)\end{eql} 
\end{align*} 

 \end{frame} 
 
 \begin{frame} 
 \vspace*{-1cm} 
\begin{align*}
	(S)&\Leftrightarrow\begin{eql}4x&-&4y&=&20& (L_1) \\&&8y&=&-104& (L_2)\end{eql} \\ &\Leftrightarrow \begin{eql} 4x&-&4y&=&20 & \\& &y&=&\dfrac{-104}{8} &= -13\end{eql}\\&\Leftrightarrow \begin{eqrl}4x-4\times\left(-13\right)&=20\\y&=-13\end{eqrl}\\
	&\Leftrightarrow \begin{eqrl}4x&=20-52\\y&=-13\end{eqrl}\\
	&\Leftrightarrow \begin{eqrl}x&=\dfrac{-32}{4} = -8\\y&=-13\end{eqrl}
\end{align*} D'où les solutions de $(S)$ sont $\left\{(-8~;~-13)\right\}$. Donc $H$ a pour coordonnées $(-8~;~-13)$.\end{frame}


\begin{frame}
\vspace{-10mm}
	\frametitle{Correction 5}
\vspace*{0.5cm} 
 Tout d'abord, $A \notin d$ car $8\times \left(-5\right)+4\times \left(-2\right)+88=40 \neq0$. 
 $H$ est l'intersection des deux droites $d$ et $(AH)$. 
 Déterminons une équation cartésienne de la droite $(AH)$ :   $d$ admet comme vecteur directeur $\vec{u} \begin{pmatrix} -4 \\ 8\end{pmatrix}$. Donc $\vec{u}$ est un vecteur normal à $(AH)$ et une équation cartésienne de $(AH)$ est : $$ -4x +8y+c=0 \quad \text{où $c$ est un réel à déterminer.}$$ $(AH)$ passe par $A$ si et seulement si :
 \begin{Eq} 
 	 \Ligneq{-4\times \left(-5\right)+8\times \left(-2\right)+c}{0} 
 	 \ligneq{4+c}{0} 
 	 \ligneq{c}{-4} 
 \end{Eq} $(AH)$ admet pour équation cartésienne $-4x +8y-4=0$. 
 
 \end{frame} 
 
 \begin{frame} 
 $H$ est l'intersection des deux droites $d$ et $(AH)$ donc ses coordonnées $(x;y)$ vérifient le système :
\begin{align*}
	(S)&\Leftrightarrow\begin{cases}8x \quad + \quad 4y \quad +88 \quad &= \quad 0\\-4x \quad + \quad 8y \quad -4 \quad &= \quad 0\end{cases}\\&\Leftrightarrow\begin{eq}8x&+&4y&=&-88\\-4x&+&8y&=&4\end{eq}\\&\Leftrightarrow\begin{eql}-32x&-&16y&=&352& (L_1) \leftarrow -4\times (L_1) \\-32x&+&64y&=&32& (L_2) \leftarrow 8\times (L_2)\end{eql} \\
	&\Leftrightarrow\begin{eql}-32x&-&16y&=&352& (L_1) \\-32x+32x&+&64y+16y&=&32-352& (L_2) \leftarrow (L_2) - (L_1)\end{eql} \\
	&\Leftrightarrow\begin{eql}-32x&-&16y&=&352& (L_1) \\&&80y&=&-320& (L_2)\end{eql} 
\end{align*} 

 \end{frame} 
 
 \begin{frame} 
 \vspace*{-1cm} 
\begin{align*}
	(S)&\Leftrightarrow\begin{eql}-32x&-&16y&=&352& (L_1) \\&&80y&=&-320& (L_2)\end{eql} \\ &\Leftrightarrow \begin{eql} -32x&-&16y&=&352 & \\& &y&=&\dfrac{-320}{80} &= -4\end{eql}\\&\Leftrightarrow \begin{eqrl}-32x-16\times\left(-4\right)&=352\\y&=-4\end{eqrl}\\
	&\Leftrightarrow \begin{eqrl}-32x&=352-64\\y&=-4\end{eqrl}\\
	&\Leftrightarrow \begin{eqrl}x&=\dfrac{288}{-32} = -9\\y&=-4\end{eqrl}
\end{align*} D'où les solutions de $(S)$ sont $\left\{(-9~;~-4)\right\}$. Donc $H$ a pour coordonnées $(-9~;~-4)$.\end{frame}




\end{document}