\documentclass[15pt, mathserif]{beamer}

\usepackage[french]{babel}
\usepackage[T1]{fontenc}
\usepackage[utf8]{inputenc}
%\usepackage{esvect}
\usepackage{bm}
\usepackage{eurosym}
\usepackage{tikz}
\usepackage{pgf,tikz,pgfplots}
\pgfplotsset{compat=1.15}
\usepackage{mathrsfs}
\usetikzlibrary{arrows}
\usetikzlibrary{arrows.meta}

\usetikzlibrary{mindmap}
\usepackage{multicol}
\usepackage[tikz]{bclogo}
\usepackage{tkz-tab}
\usepackage{amsmath, tabu}
\usepackage{esvect} %\vv{AB} pour le vecteur AB

\DeclareMathOperator{\e}{e}

%% Tableau

\usepackage{makecell}
\setcellgapes{1pt}
\makegapedcells
\newcolumntype{R}[1]{>{\raggedleft\arraybackslash }b{#1}}
\newcolumntype{L}[1]{>{\raggedright\arraybackslash }b{#1}}
\newcolumntype{C}[1]{>{\centering\arraybackslash }b{#1}}


%pour avoir des parenthèses rondes dans le package fourier
\DeclareSymbolFont{cmoperators}   {OT1}{cmr} {m}{n}
\DeclareSymbolFont{cmlargesymbols}{OMX}{cmex}{m}{n}

\usefonttheme{professionalfonts} %permet d'enlever un bug avec fourier
\usepackage{fourier}
\DeclareMathDelimiter{(}{\mathopen} {cmoperators}{"28}{cmlargesymbols}{"00}
\DeclareMathDelimiter{)}{\mathclose}{cmoperators}{"29}{cmlargesymbols}{"01}

%Graphiques 

\usepackage{pgf,tikz,pgfplots}
\pgfplotsset{compat=1.15}
\usepackage{mathrsfs}
\usetikzlibrary{arrows}
\usetikzlibrary{mindmap}

%ensembles de nbres

\newcommand{\R}{\mathbb{R}}			%permet d'écrire le R "ensemble des réels"'
\newcommand{\N}{\mathbb{N}}			%permet d'écrire le N "ensemble des entiers naturels"
\newcommand{\Z}{\mathbb{Z}}			%permet d'écrire le Z "ensemble des entiers relatifs"
\newcommand{\Prem}{\mathbb{P}}	%permet d'écrire le P "ensemble des nombres premiers" (qui n'a pas marché avec le \P car il existe déjà)
\newcommand{\D}{\mathbb{D}}
\newcommand{\Df}{\mathcal{D}_f}
\newcommand{\Cf}{\mathcal{C}_f}

\newcommand{\Q}{\mathbb{Q}}


\newcommand{\st}[1]{$(#1_n)_{n \in \N}$}

\usetheme{Madrid}
\useoutertheme{miniframes} % Alternatively: miniframes, infolines, split
\useinnertheme{circles}
\definecolor{UBCblue}{rgb}{0.1, 0.25, 0.4} % UBC Blue (primary)
\definecolor{bordeaux}{RGB}{128,0,0}
\usecolortheme[named=UBCblue]{structure}

\usepackage{color} % J'aime bien définir mes couleurs
\definecolor{propcolor}{rgb}{0, 0.5, 1}
\definecolor{thcolor}{rgb}{0.6, 0.07, 0.07}
\colorlet{louis}{blue!70!green!60!white}
\colorlet{sakura}{pink!40!red}

\title{Activités Mentales}
\date{24 Août 2023}

\newcommand{\vco}[2]{\begin{pmatrix} #1 \\ #2 \end{pmatrix}} %Coordonnées de vecteur
\newenvironment{eq}{\begin{cases}\begin{tabu}{ccccc}}{\end{tabu}\end{cases}}
\newenvironment{eql}{\begin{cases}\begin{tabu}{cccccl}}{\end{tabu}\end{cases}}
\newenvironment{eqrl}{\begin{cases}\begin{tabu}{rl}}{\end{tabu}\end{cases}}

\newenvironment{Eq}{\begin{center}\begin{tabular}{rrcl}}{\end{tabular}\end{center}}
\newcommand{\ligneq}[2]{$\Longleftrightarrow$ & $#1$ & $=$ & $#2$ \\}
\newcommand{\Ligneq}[2]{ & $#1$ & $=$ & $#2$ \\}

\newenvironment{RPN}{\begin{center}\begin{tabular}{rrclcrcl}}{\end{tabular}\end{center}}
\newcommand{\Lignerpn}[4]{ & $#1$ & $=$ & $#2$ & ou & $#3$ & $=$ & $#4$ \\}
\newcommand{\lignerpn}[4]{$\Longleftrightarrow$ & $#1$ & $=$ & $#2$ & ou & $#3$ & $=$ & $#4$ \\}

\newenvironment{TRPN}{\begin{center}\begin{tabular}{rrclcrclcrcl}}{\end{tabular}\end{center}}
\newcommand{\Lignetrpn}[6]{ & $#1$ & $=$ & $#2$ & ou & $#3$ & $=$ & $#4$ & ou & $#5$ & $=$ & $#6$ \\}
\newcommand{\lignetrpn}[6]{$\Longleftrightarrow$ & $#1$ & $=$ & $#2$ & ou & $#3$ & $=$ & $#4$ & ou & $#5$ & $=$ & $#6$ \\}
\begin{document}

\begin{frame}
    \titlepage
\end{frame}

\begin{frame} 
	\frametitle{Question 1}
\includegraphics[scale=0.01]{calculatrice}  On considère le tableau suivant 
 
 \begin{center} 
 \begin{tabular}{|p{2cm}|p{0.5cm}|p{0.5cm}|p{0.5cm}|p{0.5cm}|p{0.5cm}|p{0.5cm}|p{0.5cm}|p{0.5cm}|p{0.5cm}|p{0.5cm}|} 
 \hline 
  \centering Valeur & \centering 1& \centering 2& \centering 6& \centering 7& \centering 8& \centering 10& \centering 12& \centering 14& \centering 16& \centering 20\tabularnewline  
 \hline 
 \centering Effectif & \centering 8& \centering 2& \centering 2& \centering 3& \centering 5& \centering 10& \centering 3& \centering 7& \centering 1& \centering 7\tabularnewline  
 \hline 
 \end{tabular} 
 \end{center}  Calculer la médiane. \end{frame}


\begin{frame} 
	\frametitle{Question 2}
\includegraphics[scale=0.01]{calculatrice}  On considère le tableau suivant 
 
 \begin{center} 
 \begin{tabular}{|p{2cm}|p{0.5cm}|p{0.5cm}|p{0.5cm}|p{0.5cm}|p{0.5cm}|p{0.5cm}|p{0.5cm}|} 
 \hline 
  \centering Valeur & \centering 1& \centering 2& \centering 3& \centering 6& \centering 8& \centering 11& \centering 20\tabularnewline  
 \hline 
 \centering Effectif & \centering 4& \centering 9& \centering 5& \centering 8& \centering 2& \centering 7& \centering 3\tabularnewline  
 \hline 
 \end{tabular} 
 \end{center}  Calculer la médiane. \end{frame}


\begin{frame} 
	\frametitle{Question 3}
\includegraphics[scale=0.01]{calculatrice}  On considère le tableau suivant 
 
 \begin{center} 
 \begin{tabular}{|p{2cm}|p{0.5cm}|p{0.5cm}|p{0.5cm}|p{0.5cm}|p{0.5cm}|p{0.5cm}|p{0.5cm}|p{0.5cm}|p{0.5cm}|p{0.5cm}|} 
 \hline 
  \centering Valeur & \centering 2& \centering 10& \centering 11& \centering 12& \centering 13& \centering 14& \centering 16& \centering 17& \centering 19& \centering 20\tabularnewline  
 \hline 
 \centering Effectif & \centering 4& \centering 7& \centering 8& \centering 9& \centering 8& \centering 7& \centering 5& \centering 10& \centering 7& \centering 7\tabularnewline  
 \hline 
 \end{tabular} 
 \end{center}  Calculer la médiane. \end{frame}


\begin{frame} 
	\frametitle{Question 4}
\includegraphics[scale=0.01]{calculatrice}  On considère le tableau suivant 
 
 \begin{center} 
 \begin{tabular}{|p{2cm}|p{0.5cm}|p{0.5cm}|p{0.5cm}|p{0.5cm}|p{0.5cm}|p{0.5cm}|p{0.5cm}|p{0.5cm}|p{0.5cm}|p{0.5cm}|} 
 \hline 
  \centering Valeur & \centering 4& \centering 5& \centering 7& \centering 8& \centering 11& \centering 15& \centering 16& \centering 17& \centering 18& \centering 20\tabularnewline  
 \hline 
 \centering Effectif & \centering 6& \centering 8& \centering 1& \centering 3& \centering 5& \centering 10& \centering 6& \centering 6& \centering 1& \centering 1\tabularnewline  
 \hline 
 \end{tabular} 
 \end{center}  Calculer la médiane. \end{frame}


\begin{frame} 
	\frametitle{Question 5}
\includegraphics[scale=0.01]{calculatrice}  On considère le tableau suivant 
 
 \begin{center} 
 \begin{tabular}{|p{2cm}|p{0.5cm}|p{0.5cm}|p{0.5cm}|p{0.5cm}|p{0.5cm}|p{0.5cm}|} 
 \hline 
  \centering Valeur & \centering 3& \centering 4& \centering 9& \centering 12& \centering 14& \centering 15\tabularnewline  
 \hline 
 \centering Effectif & \centering 8& \centering 4& \centering 9& \centering 9& \centering 6& \centering 1\tabularnewline  
 \hline 
 \end{tabular} 
 \end{center}  Calculer la médiane. \end{frame}


\begin{frame}
\vspace{-10mm}
	\frametitle{Correction 1}
\begin{center} 
 \begin{tabular}{|p{2cm}|p{0.5cm}|p{0.5cm}|p{0.5cm}|p{0.5cm}|p{0.5cm}|p{0.5cm}|p{0.5cm}|p{0.5cm}|p{0.5cm}|p{0.5cm}|} 
 \hline 
  \centering Valeur & \centering 1& \centering 2& \centering 6& \centering 7& \centering 8& \centering 10& \centering 12& \centering 14& \centering 16& \centering 20\tabularnewline  
 \hline 
 \centering Effectif & \centering 8& \centering 2& \centering 2& \centering 3& \centering 5& \centering 10& \centering 3& \centering 7& \centering 1& \centering 7\tabularnewline  
 \hline 
 \centering ECC  & 8& \centering 10& \centering 12& \centering 15& \centering 20& \centering 30& \centering 33& \centering 40& \centering 41& \centering 48\tabularnewline  
 \hline 
 \end{tabular} 
 \end{center} Il y a 48 valeurs. Il s'agit d'un nombre pair, la médiane est la moyenne entre la 24e valeur et la 25e valeur.
 
 La médiane de cette série statistique est donc 10. 
 
 Cela signifie qu'il y a autant de personnes en dessous de 10 qu'au dessus de cette valeur.\end{frame}


\begin{frame}
\vspace{-10mm}
	\frametitle{Correction 2}
\begin{center} 
 \begin{tabular}{|p{2cm}|p{0.5cm}|p{0.5cm}|p{0.5cm}|p{0.5cm}|p{0.5cm}|p{0.5cm}|p{0.5cm}|} 
 \hline 
  \centering Valeur & \centering 1& \centering 2& \centering 3& \centering 6& \centering 8& \centering 11& \centering 20\tabularnewline  
 \hline 
 \centering Effectif & \centering 4& \centering 9& \centering 5& \centering 8& \centering 2& \centering 7& \centering 3\tabularnewline  
 \hline 
 \centering ECC  & 4& \centering 13& \centering 18& \centering 26& \centering 28& \centering 35& \centering 38\tabularnewline  
 \hline 
 \end{tabular} 
 \end{center} Il y a 38 valeurs. Il s'agit d'un nombre pair, la médiane est la moyenne entre la 19e valeur et la 20e valeur.
 
 La médiane de cette série statistique est donc 6. 
 
 Cela signifie qu'il y a autant de personnes en dessous de 6 qu'au dessus de cette valeur.\end{frame}


\begin{frame}
\vspace{-10mm}
	\frametitle{Correction 3}
\begin{center} 
 \begin{tabular}{|p{2cm}|p{0.5cm}|p{0.5cm}|p{0.5cm}|p{0.5cm}|p{0.5cm}|p{0.5cm}|p{0.5cm}|p{0.5cm}|p{0.5cm}|p{0.5cm}|} 
 \hline 
  \centering Valeur & \centering 2& \centering 10& \centering 11& \centering 12& \centering 13& \centering 14& \centering 16& \centering 17& \centering 19& \centering 20\tabularnewline  
 \hline 
 \centering Effectif & \centering 4& \centering 7& \centering 8& \centering 9& \centering 8& \centering 7& \centering 5& \centering 10& \centering 7& \centering 7\tabularnewline  
 \hline 
 \centering ECC  & 4& \centering 11& \centering 19& \centering 28& \centering 36& \centering 43& \centering 48& \centering 58& \centering 65& \centering 72\tabularnewline  
 \hline 
 \end{tabular} 
 \end{center} Il y a 72 valeurs. Il s'agit d'un nombre pair, la médiane est la moyenne entre la 36e valeur et la 37e valeur.
 
 La médiane de cette série statistique est donc 13.5. 
 
 Cela signifie qu'il y a autant de personnes en dessous de 13.5 qu'au dessus de cette valeur.\end{frame}


\begin{frame}
\vspace{-10mm}
	\frametitle{Correction 4}
\begin{center} 
 \begin{tabular}{|p{2cm}|p{0.5cm}|p{0.5cm}|p{0.5cm}|p{0.5cm}|p{0.5cm}|p{0.5cm}|p{0.5cm}|p{0.5cm}|p{0.5cm}|p{0.5cm}|} 
 \hline 
  \centering Valeur & \centering 4& \centering 5& \centering 7& \centering 8& \centering 11& \centering 15& \centering 16& \centering 17& \centering 18& \centering 20\tabularnewline  
 \hline 
 \centering Effectif & \centering 6& \centering 8& \centering 1& \centering 3& \centering 5& \centering 10& \centering 6& \centering 6& \centering 1& \centering 1\tabularnewline  
 \hline 
 \centering ECC  & 6& \centering 14& \centering 15& \centering 18& \centering 23& \centering 33& \centering 39& \centering 45& \centering 46& \centering 47\tabularnewline  
 \hline 
 \end{tabular} 
 \end{center} Il y a 47 valeurs. Il s'agit d'un nombre impair, la médiane se trouve donc à la 24e valeur (il s'agit de la valeur centrale).
 
 La médiane de cette série statistique est de 15. 
 
 Cela signifie qu'il y a autant de personnes en dessous de 15 qu'au dessus de cette valeur.\end{frame}


\begin{frame}
\vspace{-10mm}
	\frametitle{Correction 5}
\begin{center} 
 \begin{tabular}{|p{2cm}|p{0.5cm}|p{0.5cm}|p{0.5cm}|p{0.5cm}|p{0.5cm}|p{0.5cm}|} 
 \hline 
  \centering Valeur & \centering 3& \centering 4& \centering 9& \centering 12& \centering 14& \centering 15\tabularnewline  
 \hline 
 \centering Effectif & \centering 8& \centering 4& \centering 9& \centering 9& \centering 6& \centering 1\tabularnewline  
 \hline 
 \centering ECC  & 8& \centering 12& \centering 21& \centering 30& \centering 36& \centering 37\tabularnewline  
 \hline 
 \end{tabular} 
 \end{center} Il y a 37 valeurs. Il s'agit d'un nombre impair, la médiane se trouve donc à la 19e valeur (il s'agit de la valeur centrale).
 
 La médiane de cette série statistique est de 9. 
 
 Cela signifie qu'il y a autant de personnes en dessous de 9 qu'au dessus de cette valeur.\end{frame}




\end{document}