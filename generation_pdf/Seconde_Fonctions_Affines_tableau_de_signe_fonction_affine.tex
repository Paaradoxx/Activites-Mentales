\documentclass[15pt, mathserif]{beamer}

\usepackage[french]{babel}
\usepackage[T1]{fontenc}
\usepackage[utf8]{inputenc}
%\usepackage{esvect}
\usepackage{bm}
\usepackage{eurosym}
\usepackage{tikz}
\usepackage{pgf,tikz,pgfplots}
\pgfplotsset{compat=1.15}
\usepackage{mathrsfs}
\usetikzlibrary{arrows}
\usetikzlibrary{arrows.meta}

\usetikzlibrary{mindmap}
\usepackage{multicol}
\usepackage[tikz]{bclogo}
\usepackage{tkz-tab}
\usepackage{amsmath, tabu}
\usepackage{esvect} %\vv{AB} pour le vecteur AB

\DeclareMathOperator{\e}{e}

%% Tableau

\usepackage{makecell}
\setcellgapes{1pt}
\makegapedcells
\newcolumntype{R}[1]{>{\raggedleft\arraybackslash }b{#1}}
\newcolumntype{L}[1]{>{\raggedright\arraybackslash }b{#1}}
\newcolumntype{C}[1]{>{\centering\arraybackslash }b{#1}}


%pour avoir des parenthèses rondes dans le package fourier
\DeclareSymbolFont{cmoperators}   {OT1}{cmr} {m}{n}
\DeclareSymbolFont{cmlargesymbols}{OMX}{cmex}{m}{n}

\usefonttheme{professionalfonts} %permet d'enlever un bug avec fourier
\usepackage{fourier}
\DeclareMathDelimiter{(}{\mathopen} {cmoperators}{"28}{cmlargesymbols}{"00}
\DeclareMathDelimiter{)}{\mathclose}{cmoperators}{"29}{cmlargesymbols}{"01}

%Graphiques 

\usepackage{pgf,tikz,pgfplots}
\pgfplotsset{compat=1.15}
\usepackage{mathrsfs}
\usetikzlibrary{arrows}
\usetikzlibrary{mindmap}

%ensembles de nbres

\newcommand{\R}{\mathbb{R}}			%permet d'écrire le R "ensemble des réels"'
\newcommand{\N}{\mathbb{N}}			%permet d'écrire le N "ensemble des entiers naturels"
\newcommand{\Z}{\mathbb{Z}}			%permet d'écrire le Z "ensemble des entiers relatifs"
\newcommand{\Prem}{\mathbb{P}}	%permet d'écrire le P "ensemble des nombres premiers" (qui n'a pas marché avec le \P car il existe déjà)
\newcommand{\D}{\mathbb{D}}
\newcommand{\Df}{\mathcal{D}_f}
\newcommand{\Cf}{\mathcal{C}_f}

\newcommand{\Q}{\mathbb{Q}}


\newcommand{\st}[1]{$(#1_n)_{n \in \N}$}

\usetheme{Madrid}
\useoutertheme{miniframes} % Alternatively: miniframes, infolines, split
\useinnertheme{circles}
\definecolor{UBCblue}{rgb}{0.1, 0.25, 0.4} % UBC Blue (primary)
\definecolor{bordeaux}{RGB}{128,0,0}
\usecolortheme[named=UBCblue]{structure}

\usepackage{color} % J'aime bien définir mes couleurs
\definecolor{propcolor}{rgb}{0, 0.5, 1}
\definecolor{thcolor}{rgb}{0.6, 0.07, 0.07}
\colorlet{louis}{blue!70!green!60!white}
\colorlet{sakura}{pink!40!red}

\title{Activités Mentales}
\date{24 Août 2023}

\newcommand{\vco}[2]{\begin{pmatrix} #1 \\ #2 \end{pmatrix}} %Coordonnées de vecteur
\newenvironment{eq}{\begin{cases}\begin{tabu}{ccccc}}{\end{tabu}\end{cases}}
\newenvironment{eql}{\begin{cases}\begin{tabu}{cccccl}}{\end{tabu}\end{cases}}
\newenvironment{eqrl}{\begin{cases}\begin{tabu}{rl}}{\end{tabu}\end{cases}}

\newenvironment{Eq}{\begin{center}\begin{tabular}{rrcl}}{\end{tabular}\end{center}}
\newcommand{\ligneq}[2]{$\Longleftrightarrow$ & $#1$ & $=$ & $#2$ \\}
\newcommand{\Ligneq}[2]{ & $#1$ & $=$ & $#2$ \\}

\newenvironment{RPN}{\begin{center}\begin{tabular}{rrclcrcl}}{\end{tabular}\end{center}}
\newcommand{\Lignerpn}[4]{ & $#1$ & $=$ & $#2$ & ou & $#3$ & $=$ & $#4$ \\}
\newcommand{\lignerpn}[4]{$\Longleftrightarrow$ & $#1$ & $=$ & $#2$ & ou & $#3$ & $=$ & $#4$ \\}

\newenvironment{TRPN}{\begin{center}\begin{tabular}{rrclcrclcrcl}}{\end{tabular}\end{center}}
\newcommand{\Lignetrpn}[6]{ & $#1$ & $=$ & $#2$ & ou & $#3$ & $=$ & $#4$ & ou & $#5$ & $=$ & $#6$ \\}
\newcommand{\lignetrpn}[6]{$\Longleftrightarrow$ & $#1$ & $=$ & $#2$ & ou & $#3$ & $=$ & $#4$ & ou & $#5$ & $=$ & $#6$ \\}
\begin{document}

\begin{frame}
    \titlepage
\end{frame}

\begin{frame} 
	\frametitle{Question 1}
Dresser le tableau de signe de la fonction suivante : $$ f:x\mapsto-14x-2$$\end{frame}


\begin{frame} 
	\frametitle{Question 2}
Dresser le tableau de signe de la fonction suivante : $$ f:x\mapsto10x+17$$\end{frame}


\begin{frame} 
	\frametitle{Question 3}
Dresser le tableau de signe de la fonction suivante : $$ f:x\mapsto-2x+10$$\end{frame}


\begin{frame} 
	\frametitle{Question 4}
Dresser le tableau de signe de la fonction suivante : $$ f:x\mapsto17x-13$$\end{frame}


\begin{frame} 
	\frametitle{Question 5}
Dresser le tableau de signe de la fonction suivante : $$ f:x\mapsto-10x+1$$\end{frame}


\begin{frame}
\vspace{-10mm}
	\frametitle{Correction 1}
\vspace*{1cm} 
 \footnotesize{Dresser le tableau de signe de la fonction suivante : $ f:x\mapsto-14x-2$} 
 \begin{enumerate} 
 \item $f$ est décroissante car son coefficient directeur ($m=-14$) est négatif.
 \item On cherche ensuite à résoudre  $(E) : -14x-2=0 $	 
 \begin{align*} (E)& \Leftrightarrow -14x-2=0\\
		 	 & \Leftrightarrow -14x=2\\
			 & \Leftrightarrow x= \dfrac{2}{-14}=\dfrac{-1}{7}\\
	 \end{align*} 
 \item Ainsi : \\ 
 \begin{tikzpicture}[scale=1]
 \tkzTabInit{$x$/1, $f$/1}{$-\infty$, $\dfrac{-1}{7}$, $+\infty$} 
 \tkzTabLine{, +, z, -,} 
 \end{tikzpicture}
 \end{enumerate} 
 \end{frame}


\begin{frame}
\vspace{-10mm}
	\frametitle{Correction 2}
\vspace*{1cm} 
 \footnotesize{Dresser le tableau de signe de la fonction suivante : $ f:x\mapsto10x+17$} 
 \begin{enumerate} 
 \item $f$ est croissante car son coefficient directeur ($m=10)$ est positif.
 \item On cherche ensuite à résoudre  $(E) : 10x+17=0 $	 
 \begin{align*} (E)& \Leftrightarrow 10x+17=0\\
		 	 & \Leftrightarrow 10x=-17\\
			 & \Leftrightarrow x= \dfrac{-17}{10}=\dfrac{-17}{10}\\
	 \end{align*} 
 \item Ainsi : \\ 
 \begin{tikzpicture}[scale=1]
 \tkzTabInit{$x$/1, $f$/1}{$-\infty$, $\dfrac{-17}{10}$, $+\infty$} 
 \tkzTabLine{, -, z, +,} 
 \end{tikzpicture}
 \end{enumerate} 
 \end{frame}


\begin{frame}
\vspace{-10mm}
	\frametitle{Correction 3}
\vspace*{1cm} 
 \footnotesize{Dresser le tableau de signe de la fonction suivante : $ f:x\mapsto-2x+10$} 
 \begin{enumerate} 
 \item $f$ est décroissante car son coefficient directeur ($m=-2$) est négatif.
 \item On cherche ensuite à résoudre  $(E) : -2x+10=0 $	 
 \begin{align*} (E)& \Leftrightarrow -2x+10=0\\
		 	 & \Leftrightarrow -2x=-10\\
			 & \Leftrightarrow x= \dfrac{-10}{-2}=5\\
	 \end{align*} 
 \item Ainsi : \\ 
 \begin{tikzpicture}[scale=1]
 \tkzTabInit{$x$/1, $f$/1}{$-\infty$, $5$, $+\infty$} 
 \tkzTabLine{, +, z, -,} 
 \end{tikzpicture}
 \end{enumerate} 
 \end{frame}


\begin{frame}
\vspace{-10mm}
	\frametitle{Correction 4}
\vspace*{1cm} 
 \footnotesize{Dresser le tableau de signe de la fonction suivante : $ f:x\mapsto17x-13$} 
 \begin{enumerate} 
 \item $f$ est croissante car son coefficient directeur ($m=17)$ est positif.
 \item On cherche ensuite à résoudre  $(E) : 17x-13=0 $	 
 \begin{align*} (E)& \Leftrightarrow 17x-13=0\\
		 	 & \Leftrightarrow 17x=13\\
			 & \Leftrightarrow x= \dfrac{13}{17}=\dfrac{13}{17}\\
	 \end{align*} 
 \item Ainsi : \\ 
 \begin{tikzpicture}[scale=1]
 \tkzTabInit{$x$/1, $f$/1}{$-\infty$, $\dfrac{13}{17}$, $+\infty$} 
 \tkzTabLine{, -, z, +,} 
 \end{tikzpicture}
 \end{enumerate} 
 \end{frame}


\begin{frame}
\vspace{-10mm}
	\frametitle{Correction 5}
\vspace*{1cm} 
 \footnotesize{Dresser le tableau de signe de la fonction suivante : $ f:x\mapsto-10x+1$} 
 \begin{enumerate} 
 \item $f$ est décroissante car son coefficient directeur ($m=-10$) est négatif.
 \item On cherche ensuite à résoudre  $(E) : -10x+1=0 $	 
 \begin{align*} (E)& \Leftrightarrow -10x+1=0\\
		 	 & \Leftrightarrow -10x=-1\\
			 & \Leftrightarrow x= \dfrac{-1}{-10}=\dfrac{1}{10}\\
	 \end{align*} 
 \item Ainsi : \\ 
 \begin{tikzpicture}[scale=1]
 \tkzTabInit{$x$/1, $f$/1}{$-\infty$, $\dfrac{1}{10}$, $+\infty$} 
 \tkzTabLine{, +, z, -,} 
 \end{tikzpicture}
 \end{enumerate} 
 \end{frame}




\end{document}