\documentclass[15pt, mathserif]{beamer}

\usepackage[french]{babel}
\usepackage[T1]{fontenc}
\usepackage[utf8]{inputenc}
%\usepackage{esvect}
\usepackage{bm}
\usepackage{eurosym}
\usepackage{tikz}
\usepackage{pgf,tikz,pgfplots}
\pgfplotsset{compat=1.15}
\usepackage{mathrsfs}
\usetikzlibrary{arrows}
\usetikzlibrary{arrows.meta}

\usetikzlibrary{mindmap}
\usepackage{multicol}
\usepackage[tikz]{bclogo}
\usepackage{tkz-tab}
\usepackage{amsmath, tabu}
\usepackage{esvect} %\vv{AB} pour le vecteur AB

\DeclareMathOperator{\e}{e}

%% Tableau

\usepackage{makecell}
\setcellgapes{1pt}
\makegapedcells
\newcolumntype{R}[1]{>{\raggedleft\arraybackslash }b{#1}}
\newcolumntype{L}[1]{>{\raggedright\arraybackslash }b{#1}}
\newcolumntype{C}[1]{>{\centering\arraybackslash }b{#1}}


%pour avoir des parenthèses rondes dans le package fourier
\DeclareSymbolFont{cmoperators}   {OT1}{cmr} {m}{n}
\DeclareSymbolFont{cmlargesymbols}{OMX}{cmex}{m}{n}

\usefonttheme{professionalfonts} %permet d'enlever un bug avec fourier
\usepackage{fourier}
\DeclareMathDelimiter{(}{\mathopen} {cmoperators}{"28}{cmlargesymbols}{"00}
\DeclareMathDelimiter{)}{\mathclose}{cmoperators}{"29}{cmlargesymbols}{"01}

%Graphiques 

\usepackage{pgf,tikz,pgfplots}
\pgfplotsset{compat=1.15}
\usepackage{mathrsfs}
\usetikzlibrary{arrows}
\usetikzlibrary{mindmap}

%ensembles de nbres

\newcommand{\R}{\mathbb{R}}			%permet d'écrire le R "ensemble des réels"'
\newcommand{\N}{\mathbb{N}}			%permet d'écrire le N "ensemble des entiers naturels"
\newcommand{\Z}{\mathbb{Z}}			%permet d'écrire le Z "ensemble des entiers relatifs"
\newcommand{\Prem}{\mathbb{P}}	%permet d'écrire le P "ensemble des nombres premiers" (qui n'a pas marché avec le \P car il existe déjà)
\newcommand{\D}{\mathbb{D}}
\newcommand{\Df}{\mathcal{D}_f}
\newcommand{\Cf}{\mathcal{C}_f}

\newcommand{\Q}{\mathbb{Q}}


\newcommand{\st}[1]{$(#1_n)_{n \in \N}$}

\usetheme{Madrid}
\useoutertheme{miniframes} % Alternatively: miniframes, infolines, split
\useinnertheme{circles}
\definecolor{UBCblue}{rgb}{0.1, 0.25, 0.4} % UBC Blue (primary)
\definecolor{bordeaux}{RGB}{128,0,0}
\usecolortheme[named=UBCblue]{structure}

\usepackage{color} % J'aime bien définir mes couleurs
\definecolor{propcolor}{rgb}{0, 0.5, 1}
\definecolor{thcolor}{rgb}{0.6, 0.07, 0.07}
\colorlet{louis}{blue!70!green!60!white}
\colorlet{sakura}{pink!40!red}

\title{Activités Mentales}
\date{24 Août 2023}

\newcommand{\vco}[2]{\begin{pmatrix} #1 \\ #2 \end{pmatrix}} %Coordonnées de vecteur
\newenvironment{eq}{\begin{cases}\begin{tabu}{ccccc}}{\end{tabu}\end{cases}}
\newenvironment{eql}{\begin{cases}\begin{tabu}{cccccl}}{\end{tabu}\end{cases}}
\newenvironment{eqrl}{\begin{cases}\begin{tabu}{rl}}{\end{tabu}\end{cases}}

\newenvironment{Eq}{\begin{center}\begin{tabular}{rrcl}}{\end{tabular}\end{center}}
\newcommand{\ligneq}[2]{$\Longleftrightarrow$ & $#1$ & $=$ & $#2$ \\}
\newcommand{\Ligneq}[2]{ & $#1$ & $=$ & $#2$ \\}

\newenvironment{RPN}{\begin{center}\begin{tabular}{rrclcrcl}}{\end{tabular}\end{center}}
\newcommand{\Lignerpn}[4]{ & $#1$ & $=$ & $#2$ & ou & $#3$ & $=$ & $#4$ \\}
\newcommand{\lignerpn}[4]{$\Longleftrightarrow$ & $#1$ & $=$ & $#2$ & ou & $#3$ & $=$ & $#4$ \\}

\newenvironment{TRPN}{\begin{center}\begin{tabular}{rrclcrclcrcl}}{\end{tabular}\end{center}}
\newcommand{\Lignetrpn}[6]{ & $#1$ & $=$ & $#2$ & ou & $#3$ & $=$ & $#4$ & ou & $#5$ & $=$ & $#6$ \\}
\newcommand{\lignetrpn}[6]{$\Longleftrightarrow$ & $#1$ & $=$ & $#2$ & ou & $#3$ & $=$ & $#4$ & ou & $#5$ & $=$ & $#6$ \\}
\begin{document}

\begin{frame}
    \titlepage
\end{frame}

\begin{frame} 
	\frametitle{Question 1}
On considère le point $M\left(-4~;~5\right)$ et le vecteur $\vv{u}\vco{-1}{9}$.

\bigskip

Déterminer une équation cartésienne de la droite passant par $M$ et de vecteur directeur $\vv{u}$\end{frame}


\begin{frame} 
	\frametitle{Question 2}
On considère le point $M\left(-8~;~3\right)$ et le vecteur $\vv{u}\vco{9}{-6}$.

\bigskip

Déterminer une équation cartésienne de la droite passant par $M$ et de vecteur directeur $\vv{u}$\end{frame}


\begin{frame} 
	\frametitle{Question 3}
On considère le point $M\left(-3~;~-1\right)$ et le vecteur $\vv{u}\vco{3}{-9}$.

\bigskip

Déterminer une équation cartésienne de la droite passant par $M$ et de vecteur directeur $\vv{u}$\end{frame}


\begin{frame} 
	\frametitle{Question 4}
On considère le point $M\left(0~;~-4\right)$ et le vecteur $\vv{u}\vco{4}{8}$.

\bigskip

Déterminer une équation cartésienne de la droite passant par $M$ et de vecteur directeur $\vv{u}$\end{frame}


\begin{frame} 
	\frametitle{Question 5}
On considère le point $M\left(-8~;~-3\right)$ et le vecteur $\vv{u}\vco{7}{8}$.

\bigskip

Déterminer une équation cartésienne de la droite passant par $M$ et de vecteur directeur $\vv{u}$\end{frame}


\begin{frame}
\vspace{-10mm}
	\frametitle{Correction 1}
\vspace*{2em}
$d$ est de vecteur directeur $\vv{u}\vco{-1}{9}$ et passant par $M\left(-4~;~5\right)$.

Une équation cartésienne de la droite est de la forme $ax+by+c=0$.

Comme  $\vv{u}\vco{-1}{9}$ est un vecteur directeur de $d$, il est de la forme 

\smallskip

\hfil $\vco{-1}{9}=\vco{-b}{a} \Leftrightarrow \begin{cases} -1& = -b \\ 9&=a \end{cases} \Leftrightarrow \begin{cases} b &= 1\\ a &=9\end{cases}.$\smallskip

 L'équation est alors de la forme $9x+y + c = 0$. Or \[M(-4~;~5) \in d \Leftrightarrow 9\times \left(-4\right)+5+c=0 \Leftrightarrow -31+c = 0 \Leftrightarrow c = 31.\] Finalement une équation cartésienne de la droite passant par $M\left(-4~;~5\right)$ et de vecteur directeur $\vv{u}\vco{-1}{9}$ est $d:~9x+y+31=0.$\end{frame}


\begin{frame}
\vspace{-10mm}
	\frametitle{Correction 2}
\vspace*{2em}
$d$ est de vecteur directeur $\vv{u}\vco{9}{-6}$ et passant par $M\left(-8~;~3\right)$.

Une équation cartésienne de la droite est de la forme $ax+by+c=0$.

Comme  $\vv{u}\vco{9}{-6}$ est un vecteur directeur de $d$, il est de la forme 

\smallskip

\hfil $\vco{9}{-6}=\vco{-b}{a} \Leftrightarrow \begin{cases} 9& = -b \\ -6&=a \end{cases} \Leftrightarrow \begin{cases} b &= -9\\ a &=-6\end{cases}.$\smallskip

 L'équation est alors de la forme $-6x-9y + c = 0$. Or \[M(-8~;~3) \in d \Leftrightarrow -6\times \left(-8\right)-9\times 3+c=0 \Leftrightarrow 21+c = 0 \Leftrightarrow c = -21.\] Finalement une équation cartésienne de la droite passant par $M\left(-8~;~3\right)$ et de vecteur directeur $\vv{u}\vco{9}{-6}$ est $d:~-6x-9y-21=0.$\end{frame}


\begin{frame}
\vspace{-10mm}
	\frametitle{Correction 3}
\vspace*{2em}
$d$ est de vecteur directeur $\vv{u}\vco{3}{-9}$ et passant par $M\left(-3~;~-1\right)$.

Une équation cartésienne de la droite est de la forme $ax+by+c=0$.

Comme  $\vv{u}\vco{3}{-9}$ est un vecteur directeur de $d$, il est de la forme 

\smallskip

\hfil $\vco{3}{-9}=\vco{-b}{a} \Leftrightarrow \begin{cases} 3& = -b \\ -9&=a \end{cases} \Leftrightarrow \begin{cases} b &= -3\\ a &=-9\end{cases}.$\smallskip

 L'équation est alors de la forme $-9x-3y + c = 0$. Or \[M(-3~;~-1) \in d \Leftrightarrow -9\times \left(-3\right)-3\times \left(-1\right)+c=0 \Leftrightarrow 30+c = 0 \Leftrightarrow c = -30.\] Finalement une équation cartésienne de la droite passant par $M\left(-3~;~-1\right)$ et de vecteur directeur $\vv{u}\vco{3}{-9}$ est $d:~-9x-3y-30=0.$\end{frame}


\begin{frame}
\vspace{-10mm}
	\frametitle{Correction 4}
\vspace*{2em}
$d$ est de vecteur directeur $\vv{u}\vco{4}{8}$ et passant par $M\left(0~;~-4\right)$.

Une équation cartésienne de la droite est de la forme $ax+by+c=0$.

Comme  $\vv{u}\vco{4}{8}$ est un vecteur directeur de $d$, il est de la forme 

\smallskip

\hfil $\vco{4}{8}=\vco{-b}{a} \Leftrightarrow \begin{cases} 4& = -b \\ 8&=a \end{cases} \Leftrightarrow \begin{cases} b &= -4\\ a &=8\end{cases}.$\smallskip

 L'équation est alors de la forme $8x-4y + c = 0$. Or \[M(0~;~-4) \in d \Leftrightarrow 8\times 0-4\times \left(-4\right)+c=0 \Leftrightarrow 16+c = 0 \Leftrightarrow c = -16.\] Finalement une équation cartésienne de la droite passant par $M\left(0~;~-4\right)$ et de vecteur directeur $\vv{u}\vco{4}{8}$ est $d:~8x-4y-16=0.$\end{frame}


\begin{frame}
\vspace{-10mm}
	\frametitle{Correction 5}
\vspace*{2em}
$d$ est de vecteur directeur $\vv{u}\vco{7}{8}$ et passant par $M\left(-8~;~-3\right)$.

Une équation cartésienne de la droite est de la forme $ax+by+c=0$.

Comme  $\vv{u}\vco{7}{8}$ est un vecteur directeur de $d$, il est de la forme 

\smallskip

\hfil $\vco{7}{8}=\vco{-b}{a} \Leftrightarrow \begin{cases} 7& = -b \\ 8&=a \end{cases} \Leftrightarrow \begin{cases} b &= -7\\ a &=8\end{cases}.$\smallskip

 L'équation est alors de la forme $8x-7y + c = 0$. Or \[M(-8~;~-3) \in d \Leftrightarrow 8\times \left(-8\right)-7\times \left(-3\right)+c=0 \Leftrightarrow -43+c = 0 \Leftrightarrow c = 43.\] Finalement une équation cartésienne de la droite passant par $M\left(-8~;~-3\right)$ et de vecteur directeur $\vv{u}\vco{7}{8}$ est $d:~8x-7y+43=0.$\end{frame}




\end{document}