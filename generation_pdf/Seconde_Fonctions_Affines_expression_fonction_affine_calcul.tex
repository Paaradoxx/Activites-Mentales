\documentclass[15pt, mathserif]{beamer}

\usepackage[french]{babel}
\usepackage[T1]{fontenc}
\usepackage[utf8]{inputenc}
%\usepackage{esvect}
\usepackage{bm}
\usepackage{eurosym}
\usepackage{tikz}
\usepackage{pgf,tikz,pgfplots}
\pgfplotsset{compat=1.15}
\usepackage{mathrsfs}
\usetikzlibrary{arrows}
\usetikzlibrary{arrows.meta}

\usetikzlibrary{mindmap}
\usepackage{multicol}
\usepackage[tikz]{bclogo}
\usepackage{tkz-tab}
\usepackage{amsmath, tabu}
\usepackage{esvect} %\vv{AB} pour le vecteur AB

\DeclareMathOperator{\e}{e}

%% Tableau

\usepackage{makecell}
\setcellgapes{1pt}
\makegapedcells
\newcolumntype{R}[1]{>{\raggedleft\arraybackslash }b{#1}}
\newcolumntype{L}[1]{>{\raggedright\arraybackslash }b{#1}}
\newcolumntype{C}[1]{>{\centering\arraybackslash }b{#1}}


%pour avoir des parenthèses rondes dans le package fourier
\DeclareSymbolFont{cmoperators}   {OT1}{cmr} {m}{n}
\DeclareSymbolFont{cmlargesymbols}{OMX}{cmex}{m}{n}

\usefonttheme{professionalfonts} %permet d'enlever un bug avec fourier
\usepackage{fourier}
\DeclareMathDelimiter{(}{\mathopen} {cmoperators}{"28}{cmlargesymbols}{"00}
\DeclareMathDelimiter{)}{\mathclose}{cmoperators}{"29}{cmlargesymbols}{"01}

%Graphiques 

\usepackage{pgf,tikz,pgfplots}
\pgfplotsset{compat=1.15}
\usepackage{mathrsfs}
\usetikzlibrary{arrows}
\usetikzlibrary{mindmap}

%ensembles de nbres

\newcommand{\R}{\mathbb{R}}			%permet d'écrire le R "ensemble des réels"'
\newcommand{\N}{\mathbb{N}}			%permet d'écrire le N "ensemble des entiers naturels"
\newcommand{\Z}{\mathbb{Z}}			%permet d'écrire le Z "ensemble des entiers relatifs"
\newcommand{\Prem}{\mathbb{P}}	%permet d'écrire le P "ensemble des nombres premiers" (qui n'a pas marché avec le \P car il existe déjà)
\newcommand{\D}{\mathbb{D}}
\newcommand{\Df}{\mathcal{D}_f}
\newcommand{\Cf}{\mathcal{C}_f}

\newcommand{\Q}{\mathbb{Q}}


\newcommand{\st}[1]{$(#1_n)_{n \in \N}$}

\usetheme{Madrid}
\useoutertheme{miniframes} % Alternatively: miniframes, infolines, split
\useinnertheme{circles}
\definecolor{UBCblue}{rgb}{0.1, 0.25, 0.4} % UBC Blue (primary)
\definecolor{bordeaux}{RGB}{128,0,0}
\usecolortheme[named=UBCblue]{structure}

\usepackage{color} % J'aime bien définir mes couleurs
\definecolor{propcolor}{rgb}{0, 0.5, 1}
\definecolor{thcolor}{rgb}{0.6, 0.07, 0.07}
\colorlet{louis}{blue!70!green!60!white}
\colorlet{sakura}{pink!40!red}

\title{Activités Mentales}
\date{24 Août 2023}

\newcommand{\vco}[2]{\begin{pmatrix} #1 \\ #2 \end{pmatrix}} %Coordonnées de vecteur
\newenvironment{eq}{\begin{cases}\begin{tabu}{ccccc}}{\end{tabu}\end{cases}}
\newenvironment{eql}{\begin{cases}\begin{tabu}{cccccl}}{\end{tabu}\end{cases}}
\newenvironment{eqrl}{\begin{cases}\begin{tabu}{rl}}{\end{tabu}\end{cases}}

\newenvironment{Eq}{\begin{center}\begin{tabular}{rrcl}}{\end{tabular}\end{center}}
\newcommand{\ligneq}[2]{$\Longleftrightarrow$ & $#1$ & $=$ & $#2$ \\}
\newcommand{\Ligneq}[2]{ & $#1$ & $=$ & $#2$ \\}

\newenvironment{RPN}{\begin{center}\begin{tabular}{rrclcrcl}}{\end{tabular}\end{center}}
\newcommand{\Lignerpn}[4]{ & $#1$ & $=$ & $#2$ & ou & $#3$ & $=$ & $#4$ \\}
\newcommand{\lignerpn}[4]{$\Longleftrightarrow$ & $#1$ & $=$ & $#2$ & ou & $#3$ & $=$ & $#4$ \\}

\newenvironment{TRPN}{\begin{center}\begin{tabular}{rrclcrclcrcl}}{\end{tabular}\end{center}}
\newcommand{\Lignetrpn}[6]{ & $#1$ & $=$ & $#2$ & ou & $#3$ & $=$ & $#4$ & ou & $#5$ & $=$ & $#6$ \\}
\newcommand{\lignetrpn}[6]{$\Longleftrightarrow$ & $#1$ & $=$ & $#2$ & ou & $#3$ & $=$ & $#4$ & ou & $#5$ & $=$ & $#6$ \\}
\begin{document}

\begin{frame}
    \titlepage
\end{frame}

\begin{frame} 
	\frametitle{Question 1}
Quelle est l'expression de la fonction affine passant par les points de coordonnées (8;62) et (10;80) ?\end{frame}


\begin{frame} 
	\frametitle{Question 2}
Quelle est l'expression de la fonction affine passant par les points de coordonnées (8;-49) et (10;-61) ?\end{frame}


\begin{frame} 
	\frametitle{Question 3}
Quelle est l'expression de la fonction affine passant par les points de coordonnées (0;0) et (-5;-30) ?\end{frame}


\begin{frame} 
	\frametitle{Question 4}
Quelle est l'expression de la fonction affine passant par les points de coordonnées (-10;-14) et (-8;-10) ?\end{frame}


\begin{frame} 
	\frametitle{Question 5}
Quelle est l'expression de la fonction affine passant par les points de coordonnées (-9;58) et (-1;2) ?\end{frame}


\begin{frame}
\vspace{-10mm}
	\frametitle{Correction 1}
\vspace*{1cm} 
 \footnotesize{Quelle est l'expression de la fonction affine passant par les points de coordonnées (8;62) et (10;80) ? Il existe deux techniques :} 
 \begin{multicols}{2} 
 \begin{enumerate} 
 \item On résout un système : $$ \begin{array}{rcl} 
 & \textcolor{white}{\Leftrightarrow} & 
 \left 
 \{\begin{array}{rcl}8\times m + p&=&62 \\ 
 10\times m+p&=&80\end{array} \right. \\ 
 &\Leftrightarrow & \left 
 \{\begin{array}{rcl} p&=&62-8m \\ 
 10m+p&=&80\end{array} \right. \\ 
 &\Leftrightarrow & \left 
 \{\begin{array}{rcl} p&=&62-8m \\ 
 10m+(62-8m) &=&80\end{array} \right. \\ &\Leftrightarrow& \left \{\begin{array}{rcl}p&=&62-8m \\ 
 62+2m&=&80\end{array} \right. \\ &\Leftrightarrow& \left \{\begin{array}{rcl}p&=&62-8m \\ 
 2m&=&18\end{array} \right. \\  &\Leftrightarrow& \left \{\begin{array}{rcl} p&=&-10 \\  m&=&9\end{array}\right. \end{array}$$ 
 Ainsi on a $f:x\mapsto 9x-10$ 
 \columnbreak 
 \item 
 \footnotesize{On applique la formule du cours pour calculer $m$ :$$ \dfrac{f(x_1)-f(x_2)}{x_1-x_2}=\dfrac{62-80}{8-10}= \dfrac{-18}{-2}=9$$} \footnotesize{ Ainsi on a $f(x)= 9x +p $. 
  \\ On cherche maintenant la valeur de $p$. On sait que $f(8)=62$. On doit donc résoudre $(E): 9\times8+p=62$}	 
 \begin{align*} (E)& \Leftrightarrow 72+p=62\\
		 	 & \Leftrightarrow p=62-72\\
			 & \Leftrightarrow p=-10
	 \end{align*} 
 Ainsi on a $f:x\mapsto 9x-10$ 
 \end{enumerate} 
 \end{multicols} 
 \end{frame}


\begin{frame}
\vspace{-10mm}
	\frametitle{Correction 2}
\vspace*{1cm} 
 \footnotesize{Quelle est l'expression de la fonction affine passant par les points de coordonnées (8;-49) et (10;-61) ? Il existe deux techniques :} 
 \begin{multicols}{2} 
 \begin{enumerate} 
 \item On résout un système : $$ \begin{array}{rcl} 
 & \textcolor{white}{\Leftrightarrow} & 
 \left 
 \{\begin{array}{rcl}8\times m + p&=&-49 \\ 
 10\times m+p&=&-61\end{array} \right. \\ 
 &\Leftrightarrow & \left 
 \{\begin{array}{rcl} p&=&-49-8m \\ 
 10m+p&=&-61\end{array} \right. \\ 
 &\Leftrightarrow & \left 
 \{\begin{array}{rcl} p&=&-49-8m \\ 
 10m+(-49-8m) &=&-61\end{array} \right. \\ &\Leftrightarrow& \left \{\begin{array}{rcl}p&=&-49-8m \\ 
 -49+2m&=&-61\end{array} \right. \\ &\Leftrightarrow& \left \{\begin{array}{rcl}p&=&-49-8m \\ 
 2m&=&-12\end{array} \right. \\  &\Leftrightarrow& \left \{\begin{array}{rcl} p&=&-1 \\  m&=&-6\end{array}\right. \end{array}$$ 
 Ainsi on a $f:x\mapsto -6x-1$ 
 \columnbreak 
 \item 
 \footnotesize{On applique la formule du cours pour calculer $m$ :$$ \dfrac{f(x_1)-f(x_2)}{x_1-x_2}=\dfrac{-49-\left(-61\right)}{8-10}= \dfrac{12}{-2}=-6$$} \footnotesize{ Ainsi on a $f(x)= -6x +p $. 
  \\ On cherche maintenant la valeur de $p$. On sait que $f(8)=-49$. On doit donc résoudre $(E): -6\times8+p=-49$}	 
 \begin{align*} (E)& \Leftrightarrow -48+p=-49\\
		 	 & \Leftrightarrow p=-49+48\\
			 & \Leftrightarrow p=-1
	 \end{align*} 
 Ainsi on a $f:x\mapsto -6x-1$ 
 \end{enumerate} 
 \end{multicols} 
 \end{frame}


\begin{frame}
\vspace{-10mm}
	\frametitle{Correction 3}
\vspace*{1cm} 
 \footnotesize{Quelle est l'expression de la fonction affine passant par les points de coordonnées (0;0) et (-5;-30) ? Il existe deux techniques :} 
 \begin{multicols}{2} 
 \begin{enumerate} 
 \item On résout un système : $$ \begin{array}{rcl} 
 & \textcolor{white}{\Leftrightarrow} & 
 \left 
 \{\begin{array}{rcl}0\times m + p&=&0 \\ 
 -5\times m+p&=&-30\end{array} \right. \\ 
 &\Leftrightarrow & \left 
 \{\begin{array}{rcl} p&=&00 \\ 
 -5m+p&=&-30\end{array} \right. \\ 
 &\Leftrightarrow & \left 
 \{\begin{array}{rcl} p&=&00 \\ 
 -5m+(00 \\ 
  &=&-30\end{array} \right. \\ &\Leftrightarrow& \left \{\begin{array}{rcl}p&=&00 \\ 
 0-5m&=&-30\end{array} \right. \\ &\Leftrightarrow& \left \{\begin{array}{rcl}p&=&00 \\ 
 -5m&=&-30\end{array} \right. \\  &\Leftrightarrow& \left \{\begin{array}{rcl} p&=&0 \\  m&=&6\end{array}\right. \end{array}$$ 
 Ainsi on a $f:x\mapsto 6x$ 
 \columnbreak 
 \item 
 \footnotesize{On applique la formule du cours pour calculer $m$ :$$ \dfrac{f(x_1)-f(x_2)}{x_1-x_2}=\dfrac{0-\left(-30\right)}{0-\left(-5\right)}= \dfrac{30}{5}=6$$} \footnotesize{ Ainsi on a $f(x)= 6x +p $. 
  \\ On cherche maintenant la valeur de $p$. On sait que $f(0)=0$. On doit donc résoudre $(E): 6\times0+p=0$}	 
 \begin{align*} (E)& \Leftrightarrow 0+p=0\\
		 	 & \Leftrightarrow p=0\\
			 & \Leftrightarrow p=0
	 \end{align*} 
 Ainsi on a $f:x\mapsto 6x$ 
 \end{enumerate} 
 \end{multicols} 
 \end{frame}


\begin{frame}
\vspace{-10mm}
	\frametitle{Correction 4}
\vspace*{1cm} 
 \footnotesize{Quelle est l'expression de la fonction affine passant par les points de coordonnées (-10;-14) et (-8;-10) ? Il existe deux techniques :} 
 \begin{multicols}{2} 
 \begin{enumerate} 
 \item On résout un système : $$ \begin{array}{rcl} 
 & \textcolor{white}{\Leftrightarrow} & 
 \left 
 \{\begin{array}{rcl}-10\times m + p&=&-14 \\ 
 -8\times m+p&=&-10\end{array} \right. \\ 
 &\Leftrightarrow & \left 
 \{\begin{array}{rcl} p&=&-14+10m \\ 
 -8m+p&=&-10\end{array} \right. \\ 
 &\Leftrightarrow & \left 
 \{\begin{array}{rcl} p&=&-14+10m \\ 
 -8m+(-14+10m) &=&-10\end{array} \right. \\ &\Leftrightarrow& \left \{\begin{array}{rcl}p&=&-14+10m \\ 
 -14+2m&=&-10\end{array} \right. \\ &\Leftrightarrow& \left \{\begin{array}{rcl}p&=&-14+10m \\ 
 2m&=&4\end{array} \right. \\  &\Leftrightarrow& \left \{\begin{array}{rcl} p&=&6 \\  m&=&2\end{array}\right. \end{array}$$ 
 Ainsi on a $f:x\mapsto 2x+6$ 
 \columnbreak 
 \item 
 \footnotesize{On applique la formule du cours pour calculer $m$ :$$ \dfrac{f(x_1)-f(x_2)}{x_1-x_2}=\dfrac{-14-\left(-10\right)}{-10-\left(-8\right)}= \dfrac{-4}{-2}=2$$} \footnotesize{ Ainsi on a $f(x)= 2x +p $. 
  \\ On cherche maintenant la valeur de $p$. On sait que $f(-10)=-14$. On doit donc résoudre $(E): 2\times\left(-10\right)+p=-14$}	 
 \begin{align*} (E)& \Leftrightarrow -20+p=-14\\
		 	 & \Leftrightarrow p=-14+20\\
			 & \Leftrightarrow p=6
	 \end{align*} 
 Ainsi on a $f:x\mapsto 2x+6$ 
 \end{enumerate} 
 \end{multicols} 
 \end{frame}


\begin{frame}
\vspace{-10mm}
	\frametitle{Correction 5}
\vspace*{1cm} 
 \footnotesize{Quelle est l'expression de la fonction affine passant par les points de coordonnées (-9;58) et (-1;2) ? Il existe deux techniques :} 
 \begin{multicols}{2} 
 \begin{enumerate} 
 \item On résout un système : $$ \begin{array}{rcl} 
 & \textcolor{white}{\Leftrightarrow} & 
 \left 
 \{\begin{array}{rcl}-9\times m + p&=&58 \\ 
 -1\times m+p&=&2\end{array} \right. \\ 
 &\Leftrightarrow & \left 
 \{\begin{array}{rcl} p&=&58+9m \\ 
 -m+p&=&2\end{array} \right. \\ 
 &\Leftrightarrow & \left 
 \{\begin{array}{rcl} p&=&58+9m \\ 
 -m+(58+9m) &=&2\end{array} \right. \\ &\Leftrightarrow& \left \{\begin{array}{rcl}p&=&58+9m \\ 
 58+8m&=&2\end{array} \right. \\ &\Leftrightarrow& \left \{\begin{array}{rcl}p&=&58+9m \\ 
 8m&=&-56\end{array} \right. \\  &\Leftrightarrow& \left \{\begin{array}{rcl} p&=&-5 \\  m&=&-7\end{array}\right. \end{array}$$ 
 Ainsi on a $f:x\mapsto -7x-5$ 
 \columnbreak 
 \item 
 \footnotesize{On applique la formule du cours pour calculer $m$ :$$ \dfrac{f(x_1)-f(x_2)}{x_1-x_2}=\dfrac{58-2}{-9-\left(-1\right)}= \dfrac{56}{-8}=-7$$} \footnotesize{ Ainsi on a $f(x)= -7x +p $. 
  \\ On cherche maintenant la valeur de $p$. On sait que $f(-9)=58$. On doit donc résoudre $(E): -7\times\left(-9\right)+p=58$}	 
 \begin{align*} (E)& \Leftrightarrow 63+p=58\\
		 	 & \Leftrightarrow p=58-63\\
			 & \Leftrightarrow p=-5
	 \end{align*} 
 Ainsi on a $f:x\mapsto -7x-5$ 
 \end{enumerate} 
 \end{multicols} 
 \end{frame}




\end{document}