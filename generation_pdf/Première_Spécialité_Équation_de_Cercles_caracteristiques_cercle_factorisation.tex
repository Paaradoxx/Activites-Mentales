\documentclass[15pt, mathserif]{beamer}

\usepackage[french]{babel}
\usepackage[T1]{fontenc}
\usepackage[utf8]{inputenc}
%\usepackage{esvect}
\usepackage{bm}
\usepackage{eurosym}
\usepackage{tikz}
\usepackage{pgf,tikz,pgfplots}
\pgfplotsset{compat=1.15}
\usepackage{mathrsfs}
\usetikzlibrary{arrows}
\usetikzlibrary{arrows.meta}

\usetikzlibrary{mindmap}
\usepackage{multicol}
\usepackage[tikz]{bclogo}
\usepackage{tkz-tab}
\usepackage{amsmath, tabu}
\usepackage{esvect} %\vv{AB} pour le vecteur AB

\DeclareMathOperator{\e}{e}

%% Tableau

\usepackage{makecell}
\setcellgapes{1pt}
\makegapedcells
\newcolumntype{R}[1]{>{\raggedleft\arraybackslash }b{#1}}
\newcolumntype{L}[1]{>{\raggedright\arraybackslash }b{#1}}
\newcolumntype{C}[1]{>{\centering\arraybackslash }b{#1}}


%pour avoir des parenthèses rondes dans le package fourier
\DeclareSymbolFont{cmoperators}   {OT1}{cmr} {m}{n}
\DeclareSymbolFont{cmlargesymbols}{OMX}{cmex}{m}{n}

\usefonttheme{professionalfonts} %permet d'enlever un bug avec fourier
\usepackage{fourier}
\DeclareMathDelimiter{(}{\mathopen} {cmoperators}{"28}{cmlargesymbols}{"00}
\DeclareMathDelimiter{)}{\mathclose}{cmoperators}{"29}{cmlargesymbols}{"01}

%Graphiques 

\usepackage{pgf,tikz,pgfplots}
\pgfplotsset{compat=1.15}
\usepackage{mathrsfs}
\usetikzlibrary{arrows}
\usetikzlibrary{mindmap}

%ensembles de nbres

\newcommand{\R}{\mathbb{R}}			%permet d'écrire le R "ensemble des réels"'
\newcommand{\N}{\mathbb{N}}			%permet d'écrire le N "ensemble des entiers naturels"
\newcommand{\Z}{\mathbb{Z}}			%permet d'écrire le Z "ensemble des entiers relatifs"
\newcommand{\Prem}{\mathbb{P}}	%permet d'écrire le P "ensemble des nombres premiers" (qui n'a pas marché avec le \P car il existe déjà)
\newcommand{\D}{\mathbb{D}}
\newcommand{\Df}{\mathcal{D}_f}
\newcommand{\Cf}{\mathcal{C}_f}

\newcommand{\Q}{\mathbb{Q}}


\newcommand{\st}[1]{$(#1_n)_{n \in \N}$}

\usetheme{Madrid}
\useoutertheme{miniframes} % Alternatively: miniframes, infolines, split
\useinnertheme{circles}
\definecolor{UBCblue}{rgb}{0.1, 0.25, 0.4} % UBC Blue (primary)
\definecolor{bordeaux}{RGB}{128,0,0}
\usecolortheme[named=UBCblue]{structure}

\usepackage{color} % J'aime bien définir mes couleurs
\definecolor{propcolor}{rgb}{0, 0.5, 1}
\definecolor{thcolor}{rgb}{0.6, 0.07, 0.07}
\colorlet{louis}{blue!70!green!60!white}
\colorlet{sakura}{pink!40!red}

\title{Activités Mentales}
\date{24 Août 2023}

\newcommand{\vco}[2]{\begin{pmatrix} #1 \\ #2 \end{pmatrix}} %Coordonnées de vecteur
\newenvironment{eq}{\begin{cases}\begin{tabu}{ccccc}}{\end{tabu}\end{cases}}
\newenvironment{eql}{\begin{cases}\begin{tabu}{cccccl}}{\end{tabu}\end{cases}}
\newenvironment{eqrl}{\begin{cases}\begin{tabu}{rl}}{\end{tabu}\end{cases}}

\newenvironment{Eq}{\begin{center}\begin{tabular}{rrcl}}{\end{tabular}\end{center}}
\newcommand{\ligneq}[2]{$\Longleftrightarrow$ & $#1$ & $=$ & $#2$ \\}
\newcommand{\Ligneq}[2]{ & $#1$ & $=$ & $#2$ \\}

\newenvironment{RPN}{\begin{center}\begin{tabular}{rrclcrcl}}{\end{tabular}\end{center}}
\newcommand{\Lignerpn}[4]{ & $#1$ & $=$ & $#2$ & ou & $#3$ & $=$ & $#4$ \\}
\newcommand{\lignerpn}[4]{$\Longleftrightarrow$ & $#1$ & $=$ & $#2$ & ou & $#3$ & $=$ & $#4$ \\}

\newenvironment{TRPN}{\begin{center}\begin{tabular}{rrclcrclcrcl}}{\end{tabular}\end{center}}
\newcommand{\Lignetrpn}[6]{ & $#1$ & $=$ & $#2$ & ou & $#3$ & $=$ & $#4$ & ou & $#5$ & $=$ & $#6$ \\}
\newcommand{\lignetrpn}[6]{$\Longleftrightarrow$ & $#1$ & $=$ & $#2$ & ou & $#3$ & $=$ & $#4$ & ou & $#5$ & $=$ & $#6$ \\}
\begin{document}

\begin{frame}
    \titlepage
\end{frame}

\begin{frame} 
	\frametitle{Question 1}
Déterminer les caractéristiques du cercle d'équation: \[\mathcal{C} ~: x^2-2x + y^2-15y-2= 0.\]\end{frame}


\begin{frame} 
	\frametitle{Question 2}
Déterminer les caractéristiques du cercle d'équation: \[\mathcal{C} ~: x^2-8x + y^2+11y-8= 0.\]\end{frame}


\begin{frame} 
	\frametitle{Question 3}
Déterminer les caractéristiques du cercle d'équation: \[\mathcal{C} ~: x^2-9x + y^2+8y-9= 0.\]\end{frame}


\begin{frame} 
	\frametitle{Question 4}
Déterminer les caractéristiques du cercle d'équation: \[\mathcal{C} ~: x^2+12x + y^2+14y+12= 0.\]\end{frame}


\begin{frame} 
	\frametitle{Question 5}
Déterminer les caractéristiques du cercle d'équation: \[\mathcal{C} ~: x^2+6x + y^2+13y+6= 0.\]\end{frame}


\begin{frame}
\vspace{-10mm}
	\frametitle{Correction 1}
\begin{align*}\mathcal{C} ~: & x^2-2x + y^2-15y-2= 0 \\
	 \Leftrightarrow & x^2-2\times x \times 1+1^2-1^2+y^2-2\times y \times \dfrac{15}{2}+\left(\dfrac{15}{2}\right)^2- \left(\dfrac{15}{2}\right)^2-2= 0 \\
	 \Leftrightarrow & \left( x-1\right)^2-1+\left( y-\dfrac{15}{2}\right)^2-\dfrac{225}{4}-2= 0 \\
	 \Leftrightarrow & \left( x-1\right)^2+\left( y-\dfrac{15}{2}\right)^2 = \dfrac{237}{4}
 \end{align*} 

 \bigskip 

 Ainsi, $\mathcal{C}$ est le cercle de centre $\Omega \left( 1;\dfrac{15}{2}\right)$ et de rayon $r = \sqrt{\dfrac{237}{4}}= \dfrac{\sqrt{237}}{2}$.\end{frame}


\begin{frame}
\vspace{-10mm}
	\frametitle{Correction 2}
\begin{align*}\mathcal{C} ~: & x^2-8x + y^2+11y-8= 0 \\
	 \Leftrightarrow & x^2-2\times x \times 4+4^2-4^2+y^2+2\times y \times \dfrac{11}{2}+\left(\dfrac{11}{2}\right)^2- \left(\dfrac{11}{2}\right)^2-8= 0 \\
	 \Leftrightarrow & \left( x-4\right)^2-16+\left( y+\dfrac{11}{2}\right)^2-\dfrac{121}{4}-8= 0 \\
	 \Leftrightarrow & \left( x-4\right)^2+\left( y+\dfrac{11}{2}\right)^2 = \dfrac{217}{4}
 \end{align*} 

 \bigskip 

 Ainsi, $\mathcal{C}$ est le cercle de centre $\Omega \left( 4;\dfrac{-11}{2}\right)$ et de rayon $r = \sqrt{\dfrac{217}{4}}= \dfrac{\sqrt{217}}{2}$.\end{frame}


\begin{frame}
\vspace{-10mm}
	\frametitle{Correction 3}
\begin{align*}\mathcal{C} ~: & x^2-9x + y^2+8y-9= 0 \\
	 \Leftrightarrow & x^2-2\times x \times \dfrac{9}{2}+\left(\dfrac{9}{2}\right)^2- \left(\dfrac{9}{2}\right)^2+2\times y \times 4+4^2-4^2-9= 0 \\
	 \Leftrightarrow & \left( x-\dfrac{9}{2}\right)^2-\dfrac{81}{4}+\left( y+4\right)^2-16-9= 0 \\
	 \Leftrightarrow & \left( x-\dfrac{9}{2}\right)^2+\left( y+4\right)^2 = \dfrac{181}{4}
 \end{align*} 

 \bigskip 

 Ainsi, $\mathcal{C}$ est le cercle de centre $\Omega \left( \dfrac{9}{2};-4\right)$ et de rayon $r = \sqrt{\dfrac{181}{4}}= \dfrac{\sqrt{181}}{2}$.\end{frame}


\begin{frame}
\vspace{-10mm}
	\frametitle{Correction 4}
\begin{align*}\mathcal{C} ~: & x^2+12x + y^2+14y+12= 0 \\
	 \Leftrightarrow & x^2+2\times x \times 6+6^2-6^2+y^2+2\times y \times 7+7^2-7^2+12= 0 \\
	 \Leftrightarrow & \left( x+6\right)^2-36+\left( y+7\right)^2-49+12= 0 \\
	 \Leftrightarrow & \left( x+6\right)^2+\left( y+7\right)^2 = 73
 \end{align*} 

 \bigskip 

 Ainsi, $\mathcal{C}$ est le cercle de centre $\Omega \left( -6;-7\right)$ et de rayon $r = \sqrt{73}$.\end{frame}


\begin{frame}
\vspace{-10mm}
	\frametitle{Correction 5}
\begin{align*}\mathcal{C} ~: & x^2+6x + y^2+13y+6= 0 \\
	 \Leftrightarrow & x^2+2\times x \times 3+3^2-3^2+y^2+2\times y \times \dfrac{13}{2}+\left(\dfrac{13}{2}\right)^2- \left(\dfrac{13}{2}\right)^2+6= 0 \\
	 \Leftrightarrow & \left( x+3\right)^2-9+\left( y+\dfrac{13}{2}\right)^2-\dfrac{169}{4}+6= 0 \\
	 \Leftrightarrow & \left( x+3\right)^2+\left( y+\dfrac{13}{2}\right)^2 = \dfrac{181}{4}
 \end{align*} 

 \bigskip 

 Ainsi, $\mathcal{C}$ est le cercle de centre $\Omega \left( -3;\dfrac{-13}{2}\right)$ et de rayon $r = \sqrt{\dfrac{181}{4}}= \dfrac{\sqrt{181}}{2}$.\end{frame}




\end{document}