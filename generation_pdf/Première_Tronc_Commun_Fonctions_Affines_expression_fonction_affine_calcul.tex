\documentclass[15pt, mathserif]{beamer}

\usepackage[french]{babel}
\usepackage[T1]{fontenc}
\usepackage[utf8]{inputenc}
%\usepackage{esvect}
\usepackage{bm}
\usepackage{eurosym}
\usepackage{tikz}
\usepackage{pgf,tikz,pgfplots}
\pgfplotsset{compat=1.15}
\usepackage{mathrsfs}
\usetikzlibrary{arrows}
\usetikzlibrary{arrows.meta}

\usetikzlibrary{mindmap}
\usepackage{multicol}
\usepackage[tikz]{bclogo}
\usepackage{tkz-tab}
\usepackage{amsmath, tabu}
\usepackage{esvect} %\vv{AB} pour le vecteur AB

\DeclareMathOperator{\e}{e}

%% Tableau

\usepackage{makecell}
\setcellgapes{1pt}
\makegapedcells
\newcolumntype{R}[1]{>{\raggedleft\arraybackslash }b{#1}}
\newcolumntype{L}[1]{>{\raggedright\arraybackslash }b{#1}}
\newcolumntype{C}[1]{>{\centering\arraybackslash }b{#1}}


%pour avoir des parenthèses rondes dans le package fourier
\DeclareSymbolFont{cmoperators}   {OT1}{cmr} {m}{n}
\DeclareSymbolFont{cmlargesymbols}{OMX}{cmex}{m}{n}

\usefonttheme{professionalfonts} %permet d'enlever un bug avec fourier
\usepackage{fourier}
\DeclareMathDelimiter{(}{\mathopen} {cmoperators}{"28}{cmlargesymbols}{"00}
\DeclareMathDelimiter{)}{\mathclose}{cmoperators}{"29}{cmlargesymbols}{"01}

%Graphiques 

\usepackage{pgf,tikz,pgfplots}
\pgfplotsset{compat=1.15}
\usepackage{mathrsfs}
\usetikzlibrary{arrows}
\usetikzlibrary{mindmap}

%ensembles de nbres

\newcommand{\R}{\mathbb{R}}			%permet d'écrire le R "ensemble des réels"'
\newcommand{\N}{\mathbb{N}}			%permet d'écrire le N "ensemble des entiers naturels"
\newcommand{\Z}{\mathbb{Z}}			%permet d'écrire le Z "ensemble des entiers relatifs"
\newcommand{\Prem}{\mathbb{P}}	%permet d'écrire le P "ensemble des nombres premiers" (qui n'a pas marché avec le \P car il existe déjà)
\newcommand{\D}{\mathbb{D}}
\newcommand{\Df}{\mathcal{D}_f}
\newcommand{\Cf}{\mathcal{C}_f}

\newcommand{\Q}{\mathbb{Q}}


\newcommand{\st}[1]{$(#1_n)_{n \in \N}$}

\usetheme{Madrid}
\useoutertheme{miniframes} % Alternatively: miniframes, infolines, split
\useinnertheme{circles}
\definecolor{UBCblue}{rgb}{0.1, 0.25, 0.4} % UBC Blue (primary)
\definecolor{bordeaux}{RGB}{128,0,0}
\usecolortheme[named=UBCblue]{structure}

\usepackage{color} % J'aime bien définir mes couleurs
\definecolor{propcolor}{rgb}{0, 0.5, 1}
\definecolor{thcolor}{rgb}{0.6, 0.07, 0.07}
\colorlet{louis}{blue!70!green!60!white}
\colorlet{sakura}{pink!40!red}

\title{Activités Mentales}
\date{24 Août 2023}

\newcommand{\vco}[2]{\begin{pmatrix} #1 \\ #2 \end{pmatrix}} %Coordonnées de vecteur
\newenvironment{eq}{\begin{cases}\begin{tabu}{ccccc}}{\end{tabu}\end{cases}}
\newenvironment{eql}{\begin{cases}\begin{tabu}{cccccl}}{\end{tabu}\end{cases}}
\newenvironment{eqrl}{\begin{cases}\begin{tabu}{rl}}{\end{tabu}\end{cases}}

\newenvironment{Eq}{\begin{center}\begin{tabular}{rrcl}}{\end{tabular}\end{center}}
\newcommand{\ligneq}[2]{$\Longleftrightarrow$ & $#1$ & $=$ & $#2$ \\}
\newcommand{\Ligneq}[2]{ & $#1$ & $=$ & $#2$ \\}

\newenvironment{RPN}{\begin{center}\begin{tabular}{rrclcrcl}}{\end{tabular}\end{center}}
\newcommand{\Lignerpn}[4]{ & $#1$ & $=$ & $#2$ & ou & $#3$ & $=$ & $#4$ \\}
\newcommand{\lignerpn}[4]{$\Longleftrightarrow$ & $#1$ & $=$ & $#2$ & ou & $#3$ & $=$ & $#4$ \\}

\newenvironment{TRPN}{\begin{center}\begin{tabular}{rrclcrclcrcl}}{\end{tabular}\end{center}}
\newcommand{\Lignetrpn}[6]{ & $#1$ & $=$ & $#2$ & ou & $#3$ & $=$ & $#4$ & ou & $#5$ & $=$ & $#6$ \\}
\newcommand{\lignetrpn}[6]{$\Longleftrightarrow$ & $#1$ & $=$ & $#2$ & ou & $#3$ & $=$ & $#4$ & ou & $#5$ & $=$ & $#6$ \\}
\begin{document}

\begin{frame}
    \titlepage
\end{frame}

\begin{frame} 
	\frametitle{Question 1}
Quelle est l'expression de la fonction affine passant par les points de coordonnées (-7;-63) et (-10;-90) ?\end{frame}


\begin{frame} 
	\frametitle{Question 2}
Quelle est l'expression de la fonction affine passant par les points de coordonnées (1;-1) et (3;3) ?\end{frame}


\begin{frame} 
	\frametitle{Question 3}
Quelle est l'expression de la fonction affine passant par les points de coordonnées (-3;10) et (-5;16) ?\end{frame}


\begin{frame} 
	\frametitle{Question 4}
Quelle est l'expression de la fonction affine passant par les points de coordonnées (-10;-92) et (-7;-65) ?\end{frame}


\begin{frame} 
	\frametitle{Question 5}
Quelle est l'expression de la fonction affine passant par les points de coordonnées (3;29) et (-3;-19) ?\end{frame}


\begin{frame}
\vspace{-10mm}
	\frametitle{Correction 1}
\vspace*{1cm} 
 \footnotesize{Quelle est l'expression de la fonction affine passant par les points de coordonnées (-7;-63) et (-10;-90) ? Il existe deux techniques :} 
 \begin{multicols}{2} 
 \begin{enumerate} 
 \item On résout un système : $$ \begin{array}{rcl} 
 & \textcolor{white}{\Leftrightarrow} & 
 \left 
 \{\begin{array}{rcl}-7\times m + p&=&-63 \\ 
 -10\times m+p&=&-90\end{array} \right. \\ 
 &\Leftrightarrow & \left 
 \{\begin{array}{rcl} p&=&-63+7m \\ 
 -10m+p&=&-90\end{array} \right. \\ 
 &\Leftrightarrow & \left 
 \{\begin{array}{rcl} p&=&-63+7m \\ 
 -10m+(-63+7m) &=&-90\end{array} \right. \\ &\Leftrightarrow& \left \{\begin{array}{rcl}p&=&-63+7m \\ 
 -63-3m&=&-90\end{array} \right. \\ &\Leftrightarrow& \left \{\begin{array}{rcl}p&=&-63+7m \\ 
 -3m&=&-27\end{array} \right. \\  &\Leftrightarrow& \left \{\begin{array}{rcl} p&=&0 \\  m&=&9\end{array}\right. \end{array}$$ 
 Ainsi on a $f:x\mapsto 9x$ 
 \columnbreak 
 \item 
 \footnotesize{On applique la formule du cours pour calculer $m$ :$$ \dfrac{f(x_1)-f(x_2)}{x_1-x_2}=\dfrac{-63-\left(-90\right)}{-7-\left(-10\right)}= \dfrac{27}{3}=9$$} \footnotesize{ Ainsi on a $f(x)= 9x +p $. 
  \\ On cherche maintenant la valeur de $p$. On sait que $f(-7)=-63$. On doit donc résoudre $(E): 9\times\left(-7\right)+p=-63$}	 
 \begin{align*} (E)& \Leftrightarrow -63+p=-63\\
		 	 & \Leftrightarrow p=-63+63\\
			 & \Leftrightarrow p=0
	 \end{align*} 
 Ainsi on a $f:x\mapsto 9x$ 
 \end{enumerate} 
 \end{multicols} 
 \end{frame}


\begin{frame}
\vspace{-10mm}
	\frametitle{Correction 2}
\vspace*{1cm} 
 \footnotesize{Quelle est l'expression de la fonction affine passant par les points de coordonnées (1;-1) et (3;3) ? Il existe deux techniques :} 
 \begin{multicols}{2} 
 \begin{enumerate} 
 \item On résout un système : $$ \begin{array}{rcl} 
 & \textcolor{white}{\Leftrightarrow} & 
 \left 
 \{\begin{array}{rcl}1\times m + p&=&-1 \\ 
 3\times m+p&=&3\end{array} \right. \\ 
 &\Leftrightarrow & \left 
 \{\begin{array}{rcl} p&=&-1-1m \\ 
 3m+p&=&3\end{array} \right. \\ 
 &\Leftrightarrow & \left 
 \{\begin{array}{rcl} p&=&-1-m \\ 
 3m+(-1-m) &=&3\end{array} \right. \\ &\Leftrightarrow& \left \{\begin{array}{rcl}p&=&-1-m \\ 
 -1+2m&=&3\end{array} \right. \\ &\Leftrightarrow& \left \{\begin{array}{rcl}p&=&-1-m \\ 
 2m&=&4\end{array} \right. \\  &\Leftrightarrow& \left \{\begin{array}{rcl} p&=&-3 \\  m&=&2\end{array}\right. \end{array}$$ 
 Ainsi on a $f:x\mapsto 2x-3$ 
 \columnbreak 
 \item 
 \footnotesize{On applique la formule du cours pour calculer $m$ :$$ \dfrac{f(x_1)-f(x_2)}{x_1-x_2}=\dfrac{-1-3}{1-3}= \dfrac{-4}{-2}=2$$} \footnotesize{ Ainsi on a $f(x)= 2x +p $. 
  \\ On cherche maintenant la valeur de $p$. On sait que $f(1)=-1$. On doit donc résoudre $(E): 2\times1+p=-1$}	 
 \begin{align*} (E)& \Leftrightarrow 2+p=-1\\
		 	 & \Leftrightarrow p=-1-2\\
			 & \Leftrightarrow p=-3
	 \end{align*} 
 Ainsi on a $f:x\mapsto 2x-3$ 
 \end{enumerate} 
 \end{multicols} 
 \end{frame}


\begin{frame}
\vspace{-10mm}
	\frametitle{Correction 3}
\vspace*{1cm} 
 \footnotesize{Quelle est l'expression de la fonction affine passant par les points de coordonnées (-3;10) et (-5;16) ? Il existe deux techniques :} 
 \begin{multicols}{2} 
 \begin{enumerate} 
 \item On résout un système : $$ \begin{array}{rcl} 
 & \textcolor{white}{\Leftrightarrow} & 
 \left 
 \{\begin{array}{rcl}-3\times m + p&=&10 \\ 
 -5\times m+p&=&16\end{array} \right. \\ 
 &\Leftrightarrow & \left 
 \{\begin{array}{rcl} p&=&10+3m \\ 
 -5m+p&=&16\end{array} \right. \\ 
 &\Leftrightarrow & \left 
 \{\begin{array}{rcl} p&=&10+3m \\ 
 -5m+(10+3m) &=&16\end{array} \right. \\ &\Leftrightarrow& \left \{\begin{array}{rcl}p&=&10+3m \\ 
 10-2m&=&16\end{array} \right. \\ &\Leftrightarrow& \left \{\begin{array}{rcl}p&=&10+3m \\ 
 -2m&=&6\end{array} \right. \\  &\Leftrightarrow& \left \{\begin{array}{rcl} p&=&1 \\  m&=&-3\end{array}\right. \end{array}$$ 
 Ainsi on a $f:x\mapsto -3x+1$ 
 \columnbreak 
 \item 
 \footnotesize{On applique la formule du cours pour calculer $m$ :$$ \dfrac{f(x_1)-f(x_2)}{x_1-x_2}=\dfrac{10-16}{-3-\left(-5\right)}= \dfrac{-6}{2}=-3$$} \footnotesize{ Ainsi on a $f(x)= -3x +p $. 
  \\ On cherche maintenant la valeur de $p$. On sait que $f(-3)=10$. On doit donc résoudre $(E): -3\times\left(-3\right)+p=10$}	 
 \begin{align*} (E)& \Leftrightarrow 9+p=10\\
		 	 & \Leftrightarrow p=10-9\\
			 & \Leftrightarrow p=1
	 \end{align*} 
 Ainsi on a $f:x\mapsto -3x+1$ 
 \end{enumerate} 
 \end{multicols} 
 \end{frame}


\begin{frame}
\vspace{-10mm}
	\frametitle{Correction 4}
\vspace*{1cm} 
 \footnotesize{Quelle est l'expression de la fonction affine passant par les points de coordonnées (-10;-92) et (-7;-65) ? Il existe deux techniques :} 
 \begin{multicols}{2} 
 \begin{enumerate} 
 \item On résout un système : $$ \begin{array}{rcl} 
 & \textcolor{white}{\Leftrightarrow} & 
 \left 
 \{\begin{array}{rcl}-10\times m + p&=&-92 \\ 
 -7\times m+p&=&-65\end{array} \right. \\ 
 &\Leftrightarrow & \left 
 \{\begin{array}{rcl} p&=&-92+10m \\ 
 -7m+p&=&-65\end{array} \right. \\ 
 &\Leftrightarrow & \left 
 \{\begin{array}{rcl} p&=&-92+10m \\ 
 -7m+(-92+10m) &=&-65\end{array} \right. \\ &\Leftrightarrow& \left \{\begin{array}{rcl}p&=&-92+10m \\ 
 -92+3m&=&-65\end{array} \right. \\ &\Leftrightarrow& \left \{\begin{array}{rcl}p&=&-92+10m \\ 
 3m&=&27\end{array} \right. \\  &\Leftrightarrow& \left \{\begin{array}{rcl} p&=&-2 \\  m&=&9\end{array}\right. \end{array}$$ 
 Ainsi on a $f:x\mapsto 9x-2$ 
 \columnbreak 
 \item 
 \footnotesize{On applique la formule du cours pour calculer $m$ :$$ \dfrac{f(x_1)-f(x_2)}{x_1-x_2}=\dfrac{-92-\left(-65\right)}{-10-\left(-7\right)}= \dfrac{-27}{-3}=9$$} \footnotesize{ Ainsi on a $f(x)= 9x +p $. 
  \\ On cherche maintenant la valeur de $p$. On sait que $f(-10)=-92$. On doit donc résoudre $(E): 9\times\left(-10\right)+p=-92$}	 
 \begin{align*} (E)& \Leftrightarrow -90+p=-92\\
		 	 & \Leftrightarrow p=-92+90\\
			 & \Leftrightarrow p=-2
	 \end{align*} 
 Ainsi on a $f:x\mapsto 9x-2$ 
 \end{enumerate} 
 \end{multicols} 
 \end{frame}


\begin{frame}
\vspace{-10mm}
	\frametitle{Correction 5}
\vspace*{1cm} 
 \footnotesize{Quelle est l'expression de la fonction affine passant par les points de coordonnées (3;29) et (-3;-19) ? Il existe deux techniques :} 
 \begin{multicols}{2} 
 \begin{enumerate} 
 \item On résout un système : $$ \begin{array}{rcl} 
 & \textcolor{white}{\Leftrightarrow} & 
 \left 
 \{\begin{array}{rcl}3\times m + p&=&29 \\ 
 -3\times m+p&=&-19\end{array} \right. \\ 
 &\Leftrightarrow & \left 
 \{\begin{array}{rcl} p&=&29-3m \\ 
 -3m+p&=&-19\end{array} \right. \\ 
 &\Leftrightarrow & \left 
 \{\begin{array}{rcl} p&=&29-3m \\ 
 -3m+(29-3m) &=&-19\end{array} \right. \\ &\Leftrightarrow& \left \{\begin{array}{rcl}p&=&29-3m \\ 
 29-6m&=&-19\end{array} \right. \\ &\Leftrightarrow& \left \{\begin{array}{rcl}p&=&29-3m \\ 
 -6m&=&-48\end{array} \right. \\  &\Leftrightarrow& \left \{\begin{array}{rcl} p&=&5 \\  m&=&8\end{array}\right. \end{array}$$ 
 Ainsi on a $f:x\mapsto 8x+5$ 
 \columnbreak 
 \item 
 \footnotesize{On applique la formule du cours pour calculer $m$ :$$ \dfrac{f(x_1)-f(x_2)}{x_1-x_2}=\dfrac{29-\left(-19\right)}{3-\left(-3\right)}= \dfrac{48}{6}=8$$} \footnotesize{ Ainsi on a $f(x)= 8x +p $. 
  \\ On cherche maintenant la valeur de $p$. On sait que $f(3)=29$. On doit donc résoudre $(E): 8\times3+p=29$}	 
 \begin{align*} (E)& \Leftrightarrow 24+p=29\\
		 	 & \Leftrightarrow p=29-24\\
			 & \Leftrightarrow p=5
	 \end{align*} 
 Ainsi on a $f:x\mapsto 8x+5$ 
 \end{enumerate} 
 \end{multicols} 
 \end{frame}




\end{document}