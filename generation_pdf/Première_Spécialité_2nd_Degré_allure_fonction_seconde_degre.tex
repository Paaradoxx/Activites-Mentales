\documentclass[15pt, mathserif]{beamer}

\usepackage[french]{babel}
\usepackage[T1]{fontenc}
\usepackage[utf8]{inputenc}
%\usepackage{esvect}
\usepackage{bm}
\usepackage{eurosym}
\usepackage{tikz}
\usepackage{pgf,tikz,pgfplots}
\pgfplotsset{compat=1.15}
\usepackage{mathrsfs}
\usetikzlibrary{arrows}
\usetikzlibrary{arrows.meta}

\usetikzlibrary{mindmap}
\usepackage{multicol}
\usepackage[tikz]{bclogo}
\usepackage{tkz-tab}
\usepackage{amsmath, tabu}
\usepackage{esvect} %\vv{AB} pour le vecteur AB

\DeclareMathOperator{\e}{e}

%% Tableau

\usepackage{makecell}
\setcellgapes{1pt}
\makegapedcells
\newcolumntype{R}[1]{>{\raggedleft\arraybackslash }b{#1}}
\newcolumntype{L}[1]{>{\raggedright\arraybackslash }b{#1}}
\newcolumntype{C}[1]{>{\centering\arraybackslash }b{#1}}


%pour avoir des parenthèses rondes dans le package fourier
\DeclareSymbolFont{cmoperators}   {OT1}{cmr} {m}{n}
\DeclareSymbolFont{cmlargesymbols}{OMX}{cmex}{m}{n}

\usefonttheme{professionalfonts} %permet d'enlever un bug avec fourier
\usepackage{fourier}
\DeclareMathDelimiter{(}{\mathopen} {cmoperators}{"28}{cmlargesymbols}{"00}
\DeclareMathDelimiter{)}{\mathclose}{cmoperators}{"29}{cmlargesymbols}{"01}

%Graphiques 

\usepackage{pgf,tikz,pgfplots}
\pgfplotsset{compat=1.15}
\usepackage{mathrsfs}
\usetikzlibrary{arrows}
\usetikzlibrary{mindmap}

%ensembles de nbres

\newcommand{\R}{\mathbb{R}}			%permet d'écrire le R "ensemble des réels"'
\newcommand{\N}{\mathbb{N}}			%permet d'écrire le N "ensemble des entiers naturels"
\newcommand{\Z}{\mathbb{Z}}			%permet d'écrire le Z "ensemble des entiers relatifs"
\newcommand{\Prem}{\mathbb{P}}	%permet d'écrire le P "ensemble des nombres premiers" (qui n'a pas marché avec le \P car il existe déjà)
\newcommand{\D}{\mathbb{D}}
\newcommand{\Df}{\mathcal{D}_f}
\newcommand{\Cf}{\mathcal{C}_f}

\newcommand{\Q}{\mathbb{Q}}


\newcommand{\st}[1]{$(#1_n)_{n \in \N}$}

\usetheme{Madrid}
\useoutertheme{miniframes} % Alternatively: miniframes, infolines, split
\useinnertheme{circles}
\definecolor{UBCblue}{rgb}{0.1, 0.25, 0.4} % UBC Blue (primary)
\definecolor{bordeaux}{RGB}{128,0,0}
\usecolortheme[named=UBCblue]{structure}

\usepackage{color} % J'aime bien définir mes couleurs
\definecolor{propcolor}{rgb}{0, 0.5, 1}
\definecolor{thcolor}{rgb}{0.6, 0.07, 0.07}
\colorlet{louis}{blue!70!green!60!white}
\colorlet{sakura}{pink!40!red}

\title{Activités Mentales}
\date{24 Août 2023}

\newcommand{\vco}[2]{\begin{pmatrix} #1 \\ #2 \end{pmatrix}} %Coordonnées de vecteur
\newenvironment{eq}{\begin{cases}\begin{tabu}{ccccc}}{\end{tabu}\end{cases}}
\newenvironment{eql}{\begin{cases}\begin{tabu}{cccccl}}{\end{tabu}\end{cases}}
\newenvironment{eqrl}{\begin{cases}\begin{tabu}{rl}}{\end{tabu}\end{cases}}

\newenvironment{Eq}{\begin{center}\begin{tabular}{rrcl}}{\end{tabular}\end{center}}
\newcommand{\ligneq}[2]{$\Longleftrightarrow$ & $#1$ & $=$ & $#2$ \\}
\newcommand{\Ligneq}[2]{ & $#1$ & $=$ & $#2$ \\}

\newenvironment{RPN}{\begin{center}\begin{tabular}{rrclcrcl}}{\end{tabular}\end{center}}
\newcommand{\Lignerpn}[4]{ & $#1$ & $=$ & $#2$ & ou & $#3$ & $=$ & $#4$ \\}
\newcommand{\lignerpn}[4]{$\Longleftrightarrow$ & $#1$ & $=$ & $#2$ & ou & $#3$ & $=$ & $#4$ \\}

\newenvironment{TRPN}{\begin{center}\begin{tabular}{rrclcrclcrcl}}{\end{tabular}\end{center}}
\newcommand{\Lignetrpn}[6]{ & $#1$ & $=$ & $#2$ & ou & $#3$ & $=$ & $#4$ & ou & $#5$ & $=$ & $#6$ \\}
\newcommand{\lignetrpn}[6]{$\Longleftrightarrow$ & $#1$ & $=$ & $#2$ & ou & $#3$ & $=$ & $#4$ & ou & $#5$ & $=$ & $#6$ \\}
\begin{document}

\begin{frame}
    \titlepage
\end{frame}

\begin{frame} 
	\frametitle{Question 1}
 Soit $f$ la fonction définie pour tout $x \in \R$ par $f(x)=x^2+3$. Tracer l'allure de la courbe représentative de la fonction $f$\end{frame}


\begin{frame} 
	\frametitle{Question 2}
 Soit $f$ la fonction définie pour tout $x \in \R$ par $f(x)=2.5(x+5)(x+1)$. 
 
  Tracer l'allure de la courbe représentative de la fonction $f$\end{frame}


\begin{frame} 
	\frametitle{Question 3}
 Soit $f$ la fonction définie pour tout $x \in \R$ par $f(x)=1.5x^2+3$. Tracer l'allure de la courbe représentative de la fonction $f$\end{frame}


\begin{frame} 
	\frametitle{Question 4}
 Soit $f$ la fonction définie pour tout $x \in \R$ par $f(x)=-0.5x^2+2$. Tracer l'allure de la courbe représentative de la fonction $f$\end{frame}


\begin{frame} 
	\frametitle{Question 5}
 Soit $f$ la fonction définie pour tout $x \in \R$ par $f(x)=(x+1)(x-5)$. 
 
  Tracer l'allure de la courbe représentative de la fonction $f$\end{frame}


\begin{frame}
\vspace{-10mm}
	\frametitle{Correction 1}
\vspace*{1cm} 
 Soit $f$ la fonction définie pour tout $x \in \R$ par $f(x)=x^2 +3$. Tracer l'allure de la courbe représentative de la fonction $f$. 
 
 \begin{multicols}{2} 
 On a $a=1>0$ donc la courbe a 'la forme d'un sourire'. Puis la fonction est de la forme $ax^2+b$ donc admet l'axe des ordonnées comme axe de symétrie, l'abscisse de son sommet est 0 et il nous reste à calculer l'image de 0. $$f(0)= 1\times 0^2+3=3$$ 
 
  \columnbreak  
 
 \begin{tikzpicture} 
 \draw[black, thick, ->] (-3.0, 1) -- (3.0, 1); 
 \draw[black, thick, ->] (0,0) -- (0, 5); 
 \begin{scope} 
 \clip  (-3, 0) rectangle  (3, 5) ; 
 \draw[thick, domain = -3:3, samples = 1000] plot (\x, {(\x)*(\x)+3}); 
 \end{scope} 
  \draw[black, thick] (0.1, 3) -- (-0.1, 3); 
 \draw[left] (0, 3) node[scale = 1] {$3$};
 \end{tikzpicture} 
 \end{multicols} 
 \end{frame}


\begin{frame}
\vspace{-10mm}
	\frametitle{Correction 2}
 \vspace*{1cm} Soit $f$ la fonction définie pour tout $x \in \R$ par $f(x)=2.5(x+5)(x+1)$. 
 
  Tracer l'allure de la courbe représentative de la fonction $f$ 
 
 \begin{multicols}{2} 
 On a $a=2.5>0$ donc la courbe a 'la forme d'un sourire'. Puis la fonction est de la forme $a(x-x_1)(x-x_2)$ donc on connait les deux racines qui sont -5 et -1. Donc la courbe coupe l'axe des abscisses en -5 et -1. 
 
  \columnbreak  
 
 \begin{tikzpicture} 
 \draw[black, thick, ->] (-3.0, 2) -- (3.0, 2); 
 \draw[black, thick, ->] (2,0) -- (2, 5); 
 \begin{scope} 
 \clip  (-3, 0) rectangle  (3, 5); 
 \draw[thick, domain = -3:3, samples = 1000] plot (\x, {(\x+1.5)*(\x+0.5)+1}); 
 \end{scope} 
 \draw[below] (-2.5, 2) node[scale = 1] {$-5$}; 
 \draw[below] (-0.3, 2) node[scale = 1] {$-1$}; 
 \end{tikzpicture} 
 \end{multicols} 
 \end{frame}


\begin{frame}
\vspace{-10mm}
	\frametitle{Correction 3}
\vspace*{1cm} 
 Soit $f$ la fonction définie pour tout $x \in \R$ par $f(x)=1.5x^2 +3$. Tracer l'allure de la courbe représentative de la fonction $f$. 
 
 \begin{multicols}{2} 
 On a $a=1.5>0$ donc la courbe a 'la forme d'un sourire'. Puis la fonction est de la forme $ax^2+b$ donc admet l'axe des ordonnées comme axe de symétrie, l'abscisse de son sommet est 0 et il nous reste à calculer l'image de 0. $$f(0)= 1.5\times 0^2+3=3$$ 
 
  \columnbreak  
 
 \begin{tikzpicture} 
 \draw[black, thick, ->] (-3.0, 1) -- (3.0, 1); 
 \draw[black, thick, ->] (0,0) -- (0, 5); 
 \begin{scope} 
 \clip  (-3, 0) rectangle  (3, 5) ; 
 \draw[thick, domain = -3:3, samples = 1000] plot (\x, {(\x)*(\x)+3}); 
 \end{scope} 
  \draw[black, thick] (0.1, 3) -- (-0.1, 3); 
 \draw[left] (0, 3) node[scale = 1] {$3$};
 \end{tikzpicture} 
 \end{multicols} 
 \end{frame}


\begin{frame}
\vspace{-10mm}
	\frametitle{Correction 4}
\vspace*{1cm} 
 Soit $f$ la fonction définie pour tout $x \in \R$ par $f(x)=-0.5x^2 +2$. Tracer l'allure de la courbe représentative de la fonction $f$. 
 
 \begin{multicols}{2} 
 On a $a=-0.5<0$ donc la courbe a 'la forme inverse d'un sourire'. Puis la fonction est de la forme $ax^2+b$ donc admet l'axe des ordonnées comme axe de symétrie, l'abscisse de son sommet est 0 et il nous reste à calculer l'image de 0. $$f(0)= -0.5\times 0^2+2=2$$ 
 
  \columnbreak  
 
 \begin{tikzpicture} 
 \draw[black, thick, ->] (-3.0, 1) -- (3.0, 1); 
 \draw[black, thick, ->] (0,0) -- (0, 5); 
 \begin{scope} 
 \clip  (-3, 0) rectangle  (3, 5) ; 
 \draw[thick, domain = -3:3, samples = 1000] plot (\x, {-(\x)*(\x)+3}); 
 \end{scope} 
 \draw[black, thick] (0.1, 3) -- (-0.1, 3); 
 \draw[left] (0, 3)node[scale = 1] {$2$};
 \end{tikzpicture} 
 \end{multicols} 
 \end{frame}


\begin{frame}
\vspace{-10mm}
	\frametitle{Correction 5}
 \vspace*{1cm} Soit $f$ la fonction définie pour tout $x \in \R$ par $f(x)=(x+1)(x-5)$. 
 
  Tracer l'allure de la courbe représentative de la fonction $f$ 
 
 \begin{multicols}{2} 
 On a $a=1>0$ donc la courbe a 'la forme d'un sourire'. Puis la fonction est de la forme $a(x-x_1)(x-x_2)$ donc on connait les deux racines qui sont -1 et 5. Donc la courbe coupe l'axe des abscisses en -1 et 5. 
 
  \columnbreak  
 
 \begin{tikzpicture} 
 \draw[black, thick, ->] (-3.0, 2) -- (3.0, 2); 
 \draw[black, thick, ->] (-1,0) -- (-1, 5); 
 \begin{scope} 
 \clip  (-3, 0) rectangle  (3, 5) ; 
 \draw[thick, domain = -3:3, samples = 1000] plot (\x, {(\x+2)*(\x-1)+2.5}); 
 \end{scope} 
  \draw[below] (-2, 2) node[scale = 1] {$-1$}; 
 \draw[below] (1, 2) node[scale = 1] {$5$}; 
 \end{tikzpicture} 
 \end{multicols} 
 \end{frame}




\end{document}