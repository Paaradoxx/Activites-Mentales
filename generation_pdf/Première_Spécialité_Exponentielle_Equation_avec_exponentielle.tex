\documentclass[15pt, mathserif]{beamer}

\usepackage[french]{babel}
\usepackage[T1]{fontenc}
\usepackage[utf8]{inputenc}
%\usepackage{esvect}
\usepackage{bm}
\usepackage{eurosym}
\usepackage{tikz}
\usepackage{pgf,tikz,pgfplots}
\pgfplotsset{compat=1.15}
\usepackage{mathrsfs}
\usetikzlibrary{arrows}
\usetikzlibrary{arrows.meta}

\usetikzlibrary{mindmap}
\usepackage{multicol}
\usepackage[tikz]{bclogo}
\usepackage{tkz-tab}
\usepackage{amsmath, tabu}
\usepackage{esvect} %\vv{AB} pour le vecteur AB

\DeclareMathOperator{\e}{e}

%% Tableau

\usepackage{makecell}
\setcellgapes{1pt}
\makegapedcells
\newcolumntype{R}[1]{>{\raggedleft\arraybackslash }b{#1}}
\newcolumntype{L}[1]{>{\raggedright\arraybackslash }b{#1}}
\newcolumntype{C}[1]{>{\centering\arraybackslash }b{#1}}


%pour avoir des parenthèses rondes dans le package fourier
\DeclareSymbolFont{cmoperators}   {OT1}{cmr} {m}{n}
\DeclareSymbolFont{cmlargesymbols}{OMX}{cmex}{m}{n}

\usefonttheme{professionalfonts} %permet d'enlever un bug avec fourier
\usepackage{fourier}
\DeclareMathDelimiter{(}{\mathopen} {cmoperators}{"28}{cmlargesymbols}{"00}
\DeclareMathDelimiter{)}{\mathclose}{cmoperators}{"29}{cmlargesymbols}{"01}

%Graphiques 

\usepackage{pgf,tikz,pgfplots}
\pgfplotsset{compat=1.15}
\usepackage{mathrsfs}
\usetikzlibrary{arrows}
\usetikzlibrary{mindmap}

%ensembles de nbres

\newcommand{\R}{\mathbb{R}}			%permet d'écrire le R "ensemble des réels"'
\newcommand{\N}{\mathbb{N}}			%permet d'écrire le N "ensemble des entiers naturels"
\newcommand{\Z}{\mathbb{Z}}			%permet d'écrire le Z "ensemble des entiers relatifs"
\newcommand{\Prem}{\mathbb{P}}	%permet d'écrire le P "ensemble des nombres premiers" (qui n'a pas marché avec le \P car il existe déjà)
\newcommand{\D}{\mathbb{D}}
\newcommand{\Df}{\mathcal{D}_f}
\newcommand{\Cf}{\mathcal{C}_f}

\newcommand{\Q}{\mathbb{Q}}


\newcommand{\st}[1]{$(#1_n)_{n \in \N}$}

\usetheme{Madrid}
\useoutertheme{miniframes} % Alternatively: miniframes, infolines, split
\useinnertheme{circles}
\definecolor{UBCblue}{rgb}{0.1, 0.25, 0.4} % UBC Blue (primary)
\definecolor{bordeaux}{RGB}{128,0,0}
\usecolortheme[named=UBCblue]{structure}

\usepackage{color} % J'aime bien définir mes couleurs
\definecolor{propcolor}{rgb}{0, 0.5, 1}
\definecolor{thcolor}{rgb}{0.6, 0.07, 0.07}
\colorlet{louis}{blue!70!green!60!white}
\colorlet{sakura}{pink!40!red}

\title{Activités Mentales}
\date{24 Août 2023}

\newcommand{\vco}[2]{\begin{pmatrix} #1 \\ #2 \end{pmatrix}} %Coordonnées de vecteur
\newenvironment{eq}{\begin{cases}\begin{tabu}{ccccc}}{\end{tabu}\end{cases}}
\newenvironment{eql}{\begin{cases}\begin{tabu}{cccccl}}{\end{tabu}\end{cases}}
\newenvironment{eqrl}{\begin{cases}\begin{tabu}{rl}}{\end{tabu}\end{cases}}

\newenvironment{Eq}{\begin{center}\begin{tabular}{rrcl}}{\end{tabular}\end{center}}
\newcommand{\ligneq}[2]{$\Longleftrightarrow$ & $#1$ & $=$ & $#2$ \\}
\newcommand{\Ligneq}[2]{ & $#1$ & $=$ & $#2$ \\}

\newenvironment{RPN}{\begin{center}\begin{tabular}{rrclcrcl}}{\end{tabular}\end{center}}
\newcommand{\Lignerpn}[4]{ & $#1$ & $=$ & $#2$ & ou & $#3$ & $=$ & $#4$ \\}
\newcommand{\lignerpn}[4]{$\Longleftrightarrow$ & $#1$ & $=$ & $#2$ & ou & $#3$ & $=$ & $#4$ \\}

\newenvironment{TRPN}{\begin{center}\begin{tabular}{rrclcrclcrcl}}{\end{tabular}\end{center}}
\newcommand{\Lignetrpn}[6]{ & $#1$ & $=$ & $#2$ & ou & $#3$ & $=$ & $#4$ & ou & $#5$ & $=$ & $#6$ \\}
\newcommand{\lignetrpn}[6]{$\Longleftrightarrow$ & $#1$ & $=$ & $#2$ & ou & $#3$ & $=$ & $#4$ & ou & $#5$ & $=$ & $#6$ \\}
\begin{document}

\begin{frame}
    \titlepage
\end{frame}

\begin{frame} 
	\frametitle{Question 1}
	Résoudre dans $\mathbb{R}$ l'équation suivante: \[(E):~ \e^{-9x+4}-\e^{8x-4}=0\]\end{frame}


\begin{frame} 
	\frametitle{Question 2}
Résoudre dans $\mathbb{R}$ l'équation suivante: \[(E):~ \e^{ -x^2+11x}-\e^{x-4}=0 \] 

\end{frame}


\begin{frame} 
	\frametitle{Question 3}
	Résoudre dans $\mathbb{R}$ l'équation suivante: \[(E):~ \e^{6x+1}-\e^{4x-6}=0\]\end{frame}


\begin{frame} 
	\frametitle{Question 4}
Résoudre dans $\mathbb{R}$ l'équation suivante: \[(E):~ \e^{ -2x^2+5x}-\e^{2x+6}=0 \] 

\end{frame}


\begin{frame} 
	\frametitle{Question 5}
Résoudre dans $\mathbb{R}$ l'équation suivante: \[(E):~ \e^{ -6x^2+2x}-\e^{x-10}=0 \] 

\end{frame}


\begin{frame}
\vspace{-10mm}
	\frametitle{Correction 1}
	\begin{align*} (E)& \Leftrightarrow \e^{-9x+4}-\e^{8x-4}=0 \\
		&\Leftrightarrow \e^{-9x+4}=\e^{8x-4} \\
		&\Leftrightarrow -9x+4=8x-4\\
		&\Leftrightarrow -9x+4-8x=8x-4-8x\\
		&\Leftrightarrow -17x+4=-4\\
		&\Leftrightarrow -17x+4-4=-4-4\\
		&\Leftrightarrow -17x=-8\\
		&\Leftrightarrow \dfrac{-17x}{-17}=\dfrac{-8}{-17} \\
		&\Leftrightarrow x= \dfrac{8}{17}
	\end{align*}
	Finalement l'ensemble des solutions de $(E)$ est $S = \left\{\dfrac{8}{17}\right\}$.
\end{frame}


\begin{frame}
\vspace{-10mm}
	\frametitle{Correction 2}
	\begin{align*} (E)& \Leftrightarrow \e^{-x^2+11x}-\e^{x-4}=0 \\
		&\Leftrightarrow \e^{-x^2+11x}=\e^{x-4} \\
		&\Leftrightarrow -x^2+11x=x-4\\
		&\Leftrightarrow -x^2+10x+4=0 
 \end{align*} 
 
 On note $f$ la fonction définie sur $\R$ par $f(x)=-x^2+10x+4$ 
 
 $f$ est un polynôme de degré $2$ dont les coefficients sont $a =-1, \; b =10$ et $c =4$.
 
 On a $\Delta = b^2-4ac =10^2-4 \times\left(-1\right)\times4=100+16 = 116>0$.

 
 \end{frame} 
 
 \begin{frame} 
 

 
 Comme $\Delta > 0$, $f$ admet deux racines distinctes. 

\[x_1 = \dfrac{-b-\sqrt{\Delta}}{2a} = \dfrac{-10-2\sqrt{29}}{2 \times \left(-1\right)} = \dfrac{-10-2\sqrt{29}}{-2}=5+\sqrt{29}\] et \[ x_2 = \dfrac{-b+\sqrt{\Delta}}{2a} = \dfrac{-10+2\sqrt{29}}{2 \times \left(-1\right)} = \dfrac{-10+2\sqrt{29}}{-2}  =5-\sqrt{29}\] L'ensemble des solutions de l'équation $(E)$ est $S=\left\{ 5+\sqrt{29};5-\sqrt{29}\right\}$\end{frame}


\begin{frame}
\vspace{-10mm}
	\frametitle{Correction 3}
	\begin{align*} (E)& \Leftrightarrow \e^{6x+1}-\e^{4x-6}=0 \\
		&\Leftrightarrow \e^{6x+1}=\e^{4x-6} \\
		&\Leftrightarrow 6x+1=4x-6\\
		&\Leftrightarrow 6x+1-4x=4x-6-4x\\
		&\Leftrightarrow 2x+1=-6\\
		&\Leftrightarrow 2x+1-1=-6-1\\
		&\Leftrightarrow 2x=-7\\
		&\Leftrightarrow \dfrac{2x}{2}=\dfrac{-7}{2} \\
		&\Leftrightarrow x= \dfrac{-7}{2}
	\end{align*}
	Finalement l'ensemble des solutions de $(E)$ est $S = \left\{\dfrac{-7}{2}\right\}$.
\end{frame}


\begin{frame}
\vspace{-10mm}
	\frametitle{Correction 4}
	\begin{align*} (E)& \Leftrightarrow \e^{-2x^2+5x}-\e^{2x+6}=0 \\
		&\Leftrightarrow \e^{-2x^2+5x}=\e^{2x+6} \\
		&\Leftrightarrow -2x^2+5x=2x+6\\
		&\Leftrightarrow -2x^2+3x-6=0 
 \end{align*} 
 
 On note $f$ la fonction définie sur $\R$ par $f(x)=-2x^2+3x-6$ 
 
 $f$ est un polynôme de degré $2$ dont les coefficients sont $a =-2, \; b =3$ et $c =-6$.
 
 On a $\Delta = b^2-4ac =3^2-4 \times\left(-2\right)\times\left(-6\right)=9-48 = -39<0$.


 
 Comme $\Delta <0$, $f$ ne possède pas de racines réelles et l'équation n'admet pas de solution : $S=\emptyset$.\end{frame}


\begin{frame}
\vspace{-10mm}
	\frametitle{Correction 5}
	\begin{align*} (E)& \Leftrightarrow \e^{-6x^2+2x}-\e^{x-10}=0 \\
		&\Leftrightarrow \e^{-6x^2+2x}=\e^{x-10} \\
		&\Leftrightarrow -6x^2+2x=x-10\\
		&\Leftrightarrow -6x^2+x+10=0 
 \end{align*} 
 
 On note $f$ la fonction définie sur $\R$ par $f(x)=-6x^2+x+10$ 
 
 $f$ est un polynôme de degré $2$ dont les coefficients sont $a =-6, \; b =1$ et $c =10$.
 
 On a $\Delta = b^2-4ac =1^2-4 \times\left(-6\right)\times10=1+240 = 241>0$.

 
 \end{frame} 
 
 \begin{frame} 
 

 
 Comme $\Delta > 0$, $f$ admet deux racines distinctes. 

\[x_1 = \dfrac{-b-\sqrt{\Delta}}{2a} = \dfrac{-1-\sqrt{241}}{2 \times \left(-6\right)} = \dfrac{-1-\sqrt{241}}{-12}= \dfrac{1+\sqrt{241}}{12}\] et \[ x_2 = \dfrac{-b+\sqrt{\Delta}}{2a} = \dfrac{-1+\sqrt{241}}{2 \times \left(-6\right)} = \dfrac{-1+\sqrt{241}}{-12}  = \dfrac{1-\sqrt{241}}{12}\] L'ensemble des solutions de l'équation $(E)$ est $S=\left\{ \dfrac{1+\sqrt{241}}{12};\dfrac{1-\sqrt{241}}{12}\right\}$\end{frame}




\end{document}