\documentclass[15pt, mathserif]{beamer}

\usepackage[french]{babel}
\usepackage[T1]{fontenc}
\usepackage[utf8]{inputenc}
%\usepackage{esvect}
\usepackage{bm}
\usepackage{eurosym}
\usepackage{tikz}
\usepackage{pgf,tikz,pgfplots}
\pgfplotsset{compat=1.15}
\usepackage{mathrsfs}
\usetikzlibrary{arrows}
\usetikzlibrary{arrows.meta}

\usetikzlibrary{mindmap}
\usepackage{multicol}
\usepackage[tikz]{bclogo}
\usepackage{tkz-tab}
\usepackage{amsmath, tabu}
\usepackage{esvect} %\vv{AB} pour le vecteur AB

\DeclareMathOperator{\e}{e}

%% Tableau

\usepackage{makecell}
\setcellgapes{1pt}
\makegapedcells
\newcolumntype{R}[1]{>{\raggedleft\arraybackslash }b{#1}}
\newcolumntype{L}[1]{>{\raggedright\arraybackslash }b{#1}}
\newcolumntype{C}[1]{>{\centering\arraybackslash }b{#1}}


%pour avoir des parenthèses rondes dans le package fourier
\DeclareSymbolFont{cmoperators}   {OT1}{cmr} {m}{n}
\DeclareSymbolFont{cmlargesymbols}{OMX}{cmex}{m}{n}

\usefonttheme{professionalfonts} %permet d'enlever un bug avec fourier
\usepackage{fourier}
\DeclareMathDelimiter{(}{\mathopen} {cmoperators}{"28}{cmlargesymbols}{"00}
\DeclareMathDelimiter{)}{\mathclose}{cmoperators}{"29}{cmlargesymbols}{"01}

%Graphiques 

\usepackage{pgf,tikz,pgfplots}
\pgfplotsset{compat=1.15}
\usepackage{mathrsfs}
\usetikzlibrary{arrows}
\usetikzlibrary{mindmap}

%ensembles de nbres

\newcommand{\R}{\mathbb{R}}			%permet d'écrire le R "ensemble des réels"'
\newcommand{\N}{\mathbb{N}}			%permet d'écrire le N "ensemble des entiers naturels"
\newcommand{\Z}{\mathbb{Z}}			%permet d'écrire le Z "ensemble des entiers relatifs"
\newcommand{\Prem}{\mathbb{P}}	%permet d'écrire le P "ensemble des nombres premiers" (qui n'a pas marché avec le \P car il existe déjà)
\newcommand{\D}{\mathbb{D}}
\newcommand{\Df}{\mathcal{D}_f}
\newcommand{\Cf}{\mathcal{C}_f}

\newcommand{\Q}{\mathbb{Q}}


\newcommand{\st}[1]{$(#1_n)_{n \in \N}$}

\usetheme{Madrid}
\useoutertheme{miniframes} % Alternatively: miniframes, infolines, split
\useinnertheme{circles}
\definecolor{UBCblue}{rgb}{0.1, 0.25, 0.4} % UBC Blue (primary)
\definecolor{bordeaux}{RGB}{128,0,0}
\usecolortheme[named=UBCblue]{structure}

\usepackage{color} % J'aime bien définir mes couleurs
\definecolor{propcolor}{rgb}{0, 0.5, 1}
\definecolor{thcolor}{rgb}{0.6, 0.07, 0.07}
\colorlet{louis}{blue!70!green!60!white}
\colorlet{sakura}{pink!40!red}

\title{Activités Mentales}
\date{24 Août 2023}

\newcommand{\vco}[2]{\begin{pmatrix} #1 \\ #2 \end{pmatrix}} %Coordonnées de vecteur
\newenvironment{eq}{\begin{cases}\begin{tabu}{ccccc}}{\end{tabu}\end{cases}}
\newenvironment{eql}{\begin{cases}\begin{tabu}{cccccl}}{\end{tabu}\end{cases}}
\newenvironment{eqrl}{\begin{cases}\begin{tabu}{rl}}{\end{tabu}\end{cases}}

\newenvironment{Eq}{\begin{center}\begin{tabular}{rrcl}}{\end{tabular}\end{center}}
\newcommand{\ligneq}[2]{$\Longleftrightarrow$ & $#1$ & $=$ & $#2$ \\}
\newcommand{\Ligneq}[2]{ & $#1$ & $=$ & $#2$ \\}

\newenvironment{RPN}{\begin{center}\begin{tabular}{rrclcrcl}}{\end{tabular}\end{center}}
\newcommand{\Lignerpn}[4]{ & $#1$ & $=$ & $#2$ & ou & $#3$ & $=$ & $#4$ \\}
\newcommand{\lignerpn}[4]{$\Longleftrightarrow$ & $#1$ & $=$ & $#2$ & ou & $#3$ & $=$ & $#4$ \\}

\newenvironment{TRPN}{\begin{center}\begin{tabular}{rrclcrclcrcl}}{\end{tabular}\end{center}}
\newcommand{\Lignetrpn}[6]{ & $#1$ & $=$ & $#2$ & ou & $#3$ & $=$ & $#4$ & ou & $#5$ & $=$ & $#6$ \\}
\newcommand{\lignetrpn}[6]{$\Longleftrightarrow$ & $#1$ & $=$ & $#2$ & ou & $#3$ & $=$ & $#4$ & ou & $#5$ & $=$ & $#6$ \\}
\begin{document}

\begin{frame}
    \titlepage
\end{frame}

\begin{frame} 
	\frametitle{Question 1}
On considère la fonction $f$ définie sur $\R$ par $f(x)=12x^3 -6x^2-3x+13$. \begin{enumerate} 
 	 \item Donner l'expression de la dérivée de la fonction $f$ que l'on notera $f'$. 
 	 \item Montrer que pour tout $x \in \R$, on a $3(-6x-1)(-2x+1)=36x^2-12x-3$. 
 	 \item Construire le tableau de signe de la fonction définie sur $\R$ par 
 \hfil$3(-6x-1)(-2x+1)$ 
 	 \item En déduire les variations de la fonction $f$. 
 
 \end{enumerate} 
 
 \end{frame}


\begin{frame} 
	\frametitle{Question 2}
On considère la fonction $f$ définie sur $\R$ par $f(x)=20x^3 -38x^2-40x+19$. \begin{enumerate} 
 	 \item Donner l'expression de la dérivée de la fonction $f$ que l'on notera $f'$. 
 	 \item Montrer que pour tout $x \in \R$, on a $-4(5x+2)(-3x+5)=60x^2-76x-40$. 
 	 \item Construire le tableau de signe de la fonction définie sur $\R$ par 
 \hfil$-4(5x+2)(-3x+5)$ 
 	 \item En déduire les variations de la fonction $f$. 
 
 \end{enumerate} 
 
 \end{frame}


\begin{frame} 
	\frametitle{Question 3}
On considère la fonction $f$ définie sur $\R$ par $f(x)=4x^3 +6x^2-72x+14$. \begin{enumerate} 
 	 \item Donner l'expression de la dérivée de la fonction $f$ que l'on notera $f'$. 
 	 \item Montrer que pour tout $x \in \R$, on a $-6(x+3)(-2x+4)=12x^2+12x-72$. 
 	 \item Construire le tableau de signe de la fonction définie sur $\R$ par 
 \hfil$-6(x+3)(-2x+4)$ 
 	 \item En déduire les variations de la fonction $f$. 
 
 \end{enumerate} 
 
 \end{frame}


\begin{frame} 
	\frametitle{Question 4}
On considère la fonction $f$ définie sur $\R$ par $f(x)=-8x^3 +50x^2+100x-19$. \begin{enumerate} 
 	 \item Donner l'expression de la dérivée de la fonction $f$ que l'on notera $f'$. 
 	 \item Montrer que pour tout $x \in \R$, on a $4(6x+5)(-x+5)=-24x^2+100x+100$. 
 	 \item Construire le tableau de signe de la fonction définie sur $\R$ par 
 \hfil$4(6x+5)(-x+5)$ 
 	 \item En déduire les variations de la fonction $f$. 
 
 \end{enumerate} 
 
 \end{frame}


\begin{frame} 
	\frametitle{Question 5}
On considère la fonction $f$ définie sur $\R$ par $f(x)=12x^3 +24x^2-48x+23$. \begin{enumerate} 
 	 \item Donner l'expression de la dérivée de la fonction $f$ que l'on notera $f'$. 
 	 \item Montrer que pour tout $x \in \R$, on a $3(-2x-4)(-6x+4)=36x^2+48x-48$. 
 	 \item Construire le tableau de signe de la fonction définie sur $\R$ par 
 \hfil$3(-2x-4)(-6x+4)$ 
 	 \item En déduire les variations de la fonction $f$. 
 
 \end{enumerate} 
 
 \end{frame}


\begin{frame}
\vspace{-10mm}
	\frametitle{Correction 1}
\begin{enumerate} 
 	 \item Soit $x \in \R$, on a $$f(x)=12\textcolor{blue}{x^3}-6\textcolor{blue}{x^2}-3\textcolor{blue}{x+13}$$
 
 On a alors pour tout $x \in  \R$, $$f'(x)= 12\times \textcolor{blue}{3x^2} -6\times \textcolor{blue}{2x}-3\times \textcolor{blue}{1}+\textcolor{blue}{0}=36x^2-12x-3$$
 	 \item Soit $x \in \R$, \begin{align*} 
 3(-6x-1)(-2x+1) & = 3\left( 12x^2 -6x +2x -1\right) \\ 
 &=  3\left( 12x^2 -4x -1\right) \\ 
 &= 36x^2 -12x -3
 \end{align*} \end{enumerate} 
 
 \end{frame} 
 
 \begin{frame} 
 
 \begin{enumerate} 
 \setcounter{enumi}{2} 
 
 	 \item On pose $A(x)= -6x-1$ et $B(x) = -2x+1$.
 \bigskip 
 \begin{itemize}
	\item $A$ est une fonction affine avec $m =-6<0$. $f$ est donc décroissante sur $\mathbb{R}$. Elle est donc d'abord positive puis négative. .

	 De plus $A(x) = 0 \Leftrightarrow x = \dfrac{-1}{6}$. 
 \bigskip 
	\item $B$ est une fonction affine avec $m =-2<0$. $B$ est donc décroissante sur $\mathbb{R}$. Elle est donc d'abord positive puis négative. sur $\mathbb{R}$.

	 De plus $B(x) = 0 \Leftrightarrow x = \dfrac{1}{2}$.
\end{itemize}
 On compare les deux racines obtenues : $ \dfrac{-1}{6} < \dfrac{1}{2}$ 
 \end{enumerate} 
 
 \end{frame}


\begin{frame}On rappelle que $A(x) = -6x-1$ et $B(x) = -2x+1$ et $f'(x) = 3(-6x-1)(-2x+1)$. Son tableau de signe est alors 

\medskip \hfil
\begin{tikzpicture}[scale = 0.75]
	\tkzTabInit[lgt = 1.5]{$x$/1.25, $3$/ 1, $A(x)$ / 1, $B(x)$ / 1, $f'(x)$/1}{$-\infty$, $\dfrac{-1}{6}$, $\dfrac{1}{2}$, $+\infty$}
	\tkzTabLine{ , +, t, +, t, +, }
	\tkzTabLine{ , +, z, -, t, -, }
	\tkzTabLine{ , +, t, +, z, -, }
	\tkzTabLine{ , +, z, -, z, +, }
	\end{tikzpicture}

 \begin{enumerate} 
 \setcounter{enumi}{3} 
 	 \item On en déduit les variations de la fonction $f$ : 

  \medskip \hfil
\begin{tikzpicture}[scale = 0.75]
	\tkzTabInit[lgt = 1.5]{$x$/1.25, $f$/1}{$-\infty$, $\dfrac{-1}{6}$, $\dfrac{1}{2}$, $+\infty$}
	\tkzTabVar{-/ , +/ ,-/,+/}
	
 \end{tikzpicture}

 \end{enumerate} 
 
\end{frame}


\begin{frame}
\vspace{-10mm}
	\frametitle{Correction 2}
\begin{enumerate} 
 	 \item Soit $x \in \R$, on a $$f(x)=20\textcolor{blue}{x^3}-38\textcolor{blue}{x^2}-40\textcolor{blue}{x+19}$$
 
 On a alors pour tout $x \in  \R$, $$f'(x)= 20\times \textcolor{blue}{3x^2} -38\times \textcolor{blue}{2x}-40\times \textcolor{blue}{1}+\textcolor{blue}{0}=60x^2-76x-40$$
 	 \item Soit $x \in \R$, \begin{align*} 
 -4(5x+2)(-3x+5) & = -4\left( -15x^2 +25x -6x +10\right) \\ 
 &=  -4\left( -15x^2 +19x +10\right) \\ 
 &= 60x^2 -76x -40
 \end{align*} \end{enumerate} 
 
 \end{frame} 
 
 \begin{frame} 
 
 \begin{enumerate} 
 \setcounter{enumi}{2} 
 
 	 \item On pose $A(x)= 5x+2$ et $B(x) = -3x+5$.
 \bigskip 
 \begin{itemize}
	\item $A$ est une fonction affine avec $m =5>0$. $f$ est donc croissante sur $\mathbb{R}$. Elle est donc d'abord négative puis positive. .

	 De plus $A(x) = 0 \Leftrightarrow x = \dfrac{-2}{5}$. 
 \bigskip 
	\item $B$ est une fonction affine avec $m =-3<0$. $B$ est donc décroissante sur $\mathbb{R}$. Elle est donc d'abord positive puis négative. sur $\mathbb{R}$.

	 De plus $B(x) = 0 \Leftrightarrow x = \dfrac{5}{3}$.
\end{itemize}
 On compare les deux racines obtenues : $ \dfrac{-2}{5} < \dfrac{5}{3}$ 
 \end{enumerate} 
 
 \end{frame}


\begin{frame}On rappelle que $A(x) = 5x+2$ et $B(x) = -3x+5$ et $f'(x) = -4(5x+2)(-3x+5)$. Son tableau de signe est alors 

\medskip \hfil
\begin{tikzpicture}[scale = 0.75]
	\tkzTabInit[lgt = 1.5]{$x$/1.25, $-4$/ 1, $A(x)$ / 1, $B(x)$ / 1, $f'(x)$/1}{$-\infty$, $\dfrac{-2}{5}$, $\dfrac{5}{3}$, $+\infty$}
	\tkzTabLine{ , -, t, -, t, -, }
	\tkzTabLine{ , -, z, +, t, +, }
	\tkzTabLine{ , +, t, +, z, -, }
	\tkzTabLine{ , +, z, -, z, +, }
	\end{tikzpicture}

 \begin{enumerate} 
 \setcounter{enumi}{3} 
 	 \item On en déduit les variations de la fonction $f$ : 

  \medskip \hfil
\begin{tikzpicture}[scale = 0.75]
	\tkzTabInit[lgt = 1.5]{$x$/1.25, $f$/1}{$-\infty$, $\dfrac{-2}{5}$, $\dfrac{5}{3}$, $+\infty$}
	\tkzTabVar{-/ , +/ ,-/,+/}
	
 \end{tikzpicture}

 \end{enumerate} 
 
\end{frame}


\begin{frame}
\vspace{-10mm}
	\frametitle{Correction 3}
\begin{enumerate} 
 	 \item Soit $x \in \R$, on a $$f(x)=4\textcolor{blue}{x^3}+6\textcolor{blue}{x^2}-72\textcolor{blue}{x+14}$$
 
 On a alors pour tout $x \in  \R$, $$f'(x)= 4\times \textcolor{blue}{3x^2} +6\times \textcolor{blue}{2x}-72\times \textcolor{blue}{1}+\textcolor{blue}{0}=12x^2+12x-72$$
 	 \item Soit $x \in \R$, \begin{align*} 
 -6(x+3)(-2x+4) & = -6\left( -2x^2 +4x -6x +12\right) \\ 
 &=  -6\left( -2x^2 -2x +12\right) \\ 
 &= 12x^2 +12x -72
 \end{align*} \end{enumerate} 
 
 \end{frame} 
 
 \begin{frame} 
 
 \begin{enumerate} 
 \setcounter{enumi}{2} 
 
 	 \item On pose $A(x)= x+3$ et $B(x) = -2x+4$.
 \bigskip 
 \begin{itemize}
	\item $A$ est une fonction affine avec $m =1>0$. $f$ est donc croissante sur $\mathbb{R}$. Elle est donc d'abord négative puis positive. .

	 De plus $A(x) = 0 \Leftrightarrow x = -3$. 
 \bigskip 
	\item $B$ est une fonction affine avec $m =-2<0$. $B$ est donc décroissante sur $\mathbb{R}$. Elle est donc d'abord positive puis négative. sur $\mathbb{R}$.

	 De plus $B(x) = 0 \Leftrightarrow x = 2$.
\end{itemize}
 On compare les deux racines obtenues : $ -3 < 2$ 
 \end{enumerate} 
 
 \end{frame}


\begin{frame}On rappelle que $A(x) = x+3$ et $B(x) = -2x+4$ et $f'(x) = -6(x+3)(-2x+4)$. Son tableau de signe est alors 

\medskip \hfil
\begin{tikzpicture}[scale = 0.75]
	\tkzTabInit[lgt = 1.5]{$x$/1.25, $-6$/ 1, $A(x)$ / 1, $B(x)$ / 1, $f'(x)$/1}{$-\infty$, $-3$, $2$, $+\infty$}
	\tkzTabLine{ , -, t, -, t, -, }
	\tkzTabLine{ , -, z, +, t, +, }
	\tkzTabLine{ , +, t, +, z, -, }
	\tkzTabLine{ , +, z, -, z, +, }
	\end{tikzpicture}

 \begin{enumerate} 
 \setcounter{enumi}{3} 
 	 \item On en déduit les variations de la fonction $f$ : 

  \medskip \hfil
\begin{tikzpicture}[scale = 0.75]
	\tkzTabInit[lgt = 1.5]{$x$/1.25, $f$/1}{$-\infty$, $-3$, $2$, $+\infty$}
	\tkzTabVar{-/ , +/ ,-/,+/}
	
 \end{tikzpicture}

 \end{enumerate} 
 
\end{frame}


\begin{frame}
\vspace{-10mm}
	\frametitle{Correction 4}
\begin{enumerate} 
 	 \item Soit $x \in \R$, on a $$f(x)=-8\textcolor{blue}{x^3}+50\textcolor{blue}{x^2}+100\textcolor{blue}{x-19}$$
 
 On a alors pour tout $x \in  \R$, $$f'(x)= -8\times \textcolor{blue}{3x^2} +50\times \textcolor{blue}{2x}+100\times \textcolor{blue}{1}+\textcolor{blue}{0}=-24x^2+100x+100$$
 	 \item Soit $x \in \R$, \begin{align*} 
 4(6x+5)(-x+5) & = 4\left( -6x^2 +30x -5x +25\right) \\ 
 &=  4\left( -6x^2 +25x +25\right) \\ 
 &= -24x^2 +100x +100
 \end{align*} \end{enumerate} 
 
 \end{frame} 
 
 \begin{frame} 
 
 \begin{enumerate} 
 \setcounter{enumi}{2} 
 
 	 \item On pose $A(x)= 6x+5$ et $B(x) = -x+5$.
 \bigskip 
 \begin{itemize}
	\item $A$ est une fonction affine avec $m =6>0$. $f$ est donc croissante sur $\mathbb{R}$. Elle est donc d'abord négative puis positive. .

	 De plus $A(x) = 0 \Leftrightarrow x = \dfrac{-5}{6}$. 
 \bigskip 
	\item $B$ est une fonction affine avec $m =-1<0$. $B$ est donc décroissante sur $\mathbb{R}$. Elle est donc d'abord positive puis négative. sur $\mathbb{R}$.

	 De plus $B(x) = 0 \Leftrightarrow x = 5$.
\end{itemize}
 On compare les deux racines obtenues : $ \dfrac{-5}{6} < 5$ 
 \end{enumerate} 
 
 \end{frame}


\begin{frame}On rappelle que $A(x) = 6x+5$ et $B(x) = -x+5$ et $f'(x) = 4(6x+5)(-x+5)$. Son tableau de signe est alors 

\medskip \hfil
\begin{tikzpicture}[scale = 0.75]
	\tkzTabInit[lgt = 1.5]{$x$/1.25, $4$/ 1, $A(x)$ / 1, $B(x)$ / 1, $f'(x)$/1}{$-\infty$, $\dfrac{-5}{6}$, $5$, $+\infty$}
	\tkzTabLine{ , +, t, +, t, +, }
	\tkzTabLine{ , -, z, +, t, +, }
	\tkzTabLine{ , +, t, +, z, -, }
	\tkzTabLine{ , -, z, +, z, -, }
	\end{tikzpicture}

 \begin{enumerate} 
 \setcounter{enumi}{3} 
 	 \item On en déduit les variations de la fonction $f$ : 

  \medskip \hfil
\begin{tikzpicture}[scale = 0.75]
	\tkzTabInit[lgt = 1.5]{$x$/1.25, $f$/1}{$-\infty$, $\dfrac{-5}{6}$, $5$, $+\infty$}
	\tkzTabVar{+/ , -/ ,+/,-/}
	
 \end{tikzpicture}

 \end{enumerate} 
 
\end{frame}


\begin{frame}
\vspace{-10mm}
	\frametitle{Correction 5}
\begin{enumerate} 
 	 \item Soit $x \in \R$, on a $$f(x)=12\textcolor{blue}{x^3}+24\textcolor{blue}{x^2}-48\textcolor{blue}{x+23}$$
 
 On a alors pour tout $x \in  \R$, $$f'(x)= 12\times \textcolor{blue}{3x^2} +24\times \textcolor{blue}{2x}-48\times \textcolor{blue}{1}+\textcolor{blue}{0}=36x^2+48x-48$$
 	 \item Soit $x \in \R$, \begin{align*} 
 3(-2x-4)(-6x+4) & = 3\left( 12x^2 -8x +24x -16\right) \\ 
 &=  3\left( 12x^2 +16x -16\right) \\ 
 &= 36x^2 +48x -48
 \end{align*} \end{enumerate} 
 
 \end{frame} 
 
 \begin{frame} 
 
 \begin{enumerate} 
 \setcounter{enumi}{2} 
 
 	 \item On pose $A(x)= -2x-4$ et $B(x) = -6x+4$.
 \bigskip 
 \begin{itemize}
	\item $A$ est une fonction affine avec $m =-2<0$. $f$ est donc décroissante sur $\mathbb{R}$. Elle est donc d'abord positive puis négative. .

	 De plus $A(x) = 0 \Leftrightarrow x = -2$. 
 \bigskip 
	\item $B$ est une fonction affine avec $m =-6<0$. $B$ est donc décroissante sur $\mathbb{R}$. Elle est donc d'abord positive puis négative. sur $\mathbb{R}$.

	 De plus $B(x) = 0 \Leftrightarrow x = \dfrac{2}{3}$.
\end{itemize}
 On compare les deux racines obtenues : $ -2 < \dfrac{2}{3}$ 
 \end{enumerate} 
 
 \end{frame}


\begin{frame}On rappelle que $A(x) = -2x-4$ et $B(x) = -6x+4$ et $f'(x) = 3(-2x-4)(-6x+4)$. Son tableau de signe est alors 

\medskip \hfil
\begin{tikzpicture}[scale = 0.75]
	\tkzTabInit[lgt = 1.5]{$x$/1.25, $3$/ 1, $A(x)$ / 1, $B(x)$ / 1, $f'(x)$/1}{$-\infty$, $-2$, $\dfrac{2}{3}$, $+\infty$}
	\tkzTabLine{ , +, t, +, t, +, }
	\tkzTabLine{ , +, z, -, t, -, }
	\tkzTabLine{ , +, t, +, z, -, }
	\tkzTabLine{ , +, z, -, z, +, }
	\end{tikzpicture}

 \begin{enumerate} 
 \setcounter{enumi}{3} 
 	 \item On en déduit les variations de la fonction $f$ : 

  \medskip \hfil
\begin{tikzpicture}[scale = 0.75]
	\tkzTabInit[lgt = 1.5]{$x$/1.25, $f$/1}{$-\infty$, $-2$, $\dfrac{2}{3}$, $+\infty$}
	\tkzTabVar{-/ , +/ ,-/,+/}
	
 \end{tikzpicture}

 \end{enumerate} 
 
\end{frame}




\end{document}