\documentclass[15pt, mathserif]{beamer}

\usepackage[french]{babel}
\usepackage[T1]{fontenc}
\usepackage[utf8]{inputenc}
%\usepackage{esvect}
\usepackage{bm}
\usepackage{eurosym}
\usepackage{tikz}
\usepackage{pgf,tikz,pgfplots}
\pgfplotsset{compat=1.15}
\usepackage{mathrsfs}
\usetikzlibrary{arrows}
\usetikzlibrary{arrows.meta}

\usetikzlibrary{mindmap}
\usepackage{multicol}
\usepackage[tikz]{bclogo}
\usepackage{tkz-tab}
\usepackage{amsmath, tabu}
\usepackage{esvect} %\vv{AB} pour le vecteur AB

\DeclareMathOperator{\e}{e}

%% Tableau

\usepackage{makecell}
\setcellgapes{1pt}
\makegapedcells
\newcolumntype{R}[1]{>{\raggedleft\arraybackslash }b{#1}}
\newcolumntype{L}[1]{>{\raggedright\arraybackslash }b{#1}}
\newcolumntype{C}[1]{>{\centering\arraybackslash }b{#1}}


%pour avoir des parenthèses rondes dans le package fourier
\DeclareSymbolFont{cmoperators}   {OT1}{cmr} {m}{n}
\DeclareSymbolFont{cmlargesymbols}{OMX}{cmex}{m}{n}

\usefonttheme{professionalfonts} %permet d'enlever un bug avec fourier
\usepackage{fourier}
\DeclareMathDelimiter{(}{\mathopen} {cmoperators}{"28}{cmlargesymbols}{"00}
\DeclareMathDelimiter{)}{\mathclose}{cmoperators}{"29}{cmlargesymbols}{"01}

%Graphiques 

\usepackage{pgf,tikz,pgfplots}
\pgfplotsset{compat=1.15}
\usepackage{mathrsfs}
\usetikzlibrary{arrows}
\usetikzlibrary{mindmap}

%ensembles de nbres

\newcommand{\R}{\mathbb{R}}			%permet d'écrire le R "ensemble des réels"'
\newcommand{\N}{\mathbb{N}}			%permet d'écrire le N "ensemble des entiers naturels"
\newcommand{\Z}{\mathbb{Z}}			%permet d'écrire le Z "ensemble des entiers relatifs"
\newcommand{\Prem}{\mathbb{P}}	%permet d'écrire le P "ensemble des nombres premiers" (qui n'a pas marché avec le \P car il existe déjà)
\newcommand{\D}{\mathbb{D}}
\newcommand{\Df}{\mathcal{D}_f}
\newcommand{\Cf}{\mathcal{C}_f}

\newcommand{\Q}{\mathbb{Q}}


\newcommand{\st}[1]{$(#1_n)_{n \in \N}$}

\usetheme{Madrid}
\useoutertheme{miniframes} % Alternatively: miniframes, infolines, split
\useinnertheme{circles}
\definecolor{UBCblue}{rgb}{0.1, 0.25, 0.4} % UBC Blue (primary)
\definecolor{bordeaux}{RGB}{128,0,0}
\usecolortheme[named=UBCblue]{structure}

\usepackage{color} % J'aime bien définir mes couleurs
\definecolor{propcolor}{rgb}{0, 0.5, 1}
\definecolor{thcolor}{rgb}{0.6, 0.07, 0.07}
\colorlet{louis}{blue!70!green!60!white}
\colorlet{sakura}{pink!40!red}

\title{Activités Mentales}
\date{24 Août 2023}

\newcommand{\vco}[2]{\begin{pmatrix} #1 \\ #2 \end{pmatrix}} %Coordonnées de vecteur
\newenvironment{eq}{\begin{cases}\begin{tabu}{ccccc}}{\end{tabu}\end{cases}}
\newenvironment{eql}{\begin{cases}\begin{tabu}{cccccl}}{\end{tabu}\end{cases}}
\newenvironment{eqrl}{\begin{cases}\begin{tabu}{rl}}{\end{tabu}\end{cases}}

\newenvironment{Eq}{\begin{center}\begin{tabular}{rrcl}}{\end{tabular}\end{center}}
\newcommand{\ligneq}[2]{$\Longleftrightarrow$ & $#1$ & $=$ & $#2$ \\}
\newcommand{\Ligneq}[2]{ & $#1$ & $=$ & $#2$ \\}

\newenvironment{RPN}{\begin{center}\begin{tabular}{rrclcrcl}}{\end{tabular}\end{center}}
\newcommand{\Lignerpn}[4]{ & $#1$ & $=$ & $#2$ & ou & $#3$ & $=$ & $#4$ \\}
\newcommand{\lignerpn}[4]{$\Longleftrightarrow$ & $#1$ & $=$ & $#2$ & ou & $#3$ & $=$ & $#4$ \\}

\newenvironment{TRPN}{\begin{center}\begin{tabular}{rrclcrclcrcl}}{\end{tabular}\end{center}}
\newcommand{\Lignetrpn}[6]{ & $#1$ & $=$ & $#2$ & ou & $#3$ & $=$ & $#4$ & ou & $#5$ & $=$ & $#6$ \\}
\newcommand{\lignetrpn}[6]{$\Longleftrightarrow$ & $#1$ & $=$ & $#2$ & ou & $#3$ & $=$ & $#4$ & ou & $#5$ & $=$ & $#6$ \\}
\begin{document}

\begin{frame}
    \titlepage
\end{frame}

\begin{frame} 
	\frametitle{Question 1}
On considère la fonction $f$ définie sur $\mathbb{R}$ d'expression $f(x) = -8x^{2}-30x+44$.

\begin{enumerate}
\item
	Montrer que $(-2x-10)(4x-5)=-8x^{2}-30x+50$.
	\item En déduire le ou les antécédents de $-6$ par $f$.
\end{enumerate}


\end{frame}


\begin{frame} 
	\frametitle{Question 2}
On considère la fonction $f$ définie sur $\mathbb{R}$ d'expression $f(x) = -27x^{2}+57x+46$.

\begin{enumerate}
\item
	Montrer que $(-3x+8)(9x+5)=-27x^{2}+57x+40$.
	\item En déduire le ou les antécédents de $6$ par $f$.
\end{enumerate}


\end{frame}


\begin{frame} 
	\frametitle{Question 3}
On considère la fonction $f$ définie sur $\mathbb{R}$ d'expression $f(x) = 4x^{2}+8x-6$.

\begin{enumerate}
\item
	Montrer que $(-4x-4)(-x-1)=4x^{2}+8x+4$.
	\item En déduire le ou les antécédents de $-10$ par $f$.
\end{enumerate}


\end{frame}


\begin{frame} 
	\frametitle{Question 4}
On considère la fonction $f$ définie sur $\mathbb{R}$ d'expression $f(x) = -8x^{2}+20x-5$.

\begin{enumerate}
\item
	Montrer que $(-4x+2)(2x-4)=-8x^{2}+20x-8$.
	\item En déduire le ou les antécédents de $3$ par $f$.
\end{enumerate}


\end{frame}


\begin{frame} 
	\frametitle{Question 5}
On considère la fonction $f$ définie sur $\mathbb{R}$ d'expression $f(x) = -4x^{2}+30x-39$.

\begin{enumerate}
\item
	Montrer que $(-4x+6)(x-6)=-4x^{2}+30x-36$.
	\item En déduire le ou les antécédents de $-3$ par $f$.
\end{enumerate}


\end{frame}


\begin{frame}
\vspace{-10mm}
	\frametitle{Correction 1}
1)

\begin{align*}(-2x-10)(4x-5)&=-8x^{2}+10x-40x+50\\
	&=-8x^{2}-30x+50\end{align*}
\end{frame}

\begin{frame}
\vspace*{2em}2) 

 	\begin{tabular}{ccc} $f(x) = -6$ & $\Leftrightarrow$ & $-8x^{2}-30x+44=-6$  \\
		& $\Leftrightarrow$ & $-8x^{2}-30x+50=0$  \\
		& $\Leftrightarrow$ &  $(-2x-10)(4x-5)=0$  \quad \text{D'après 1)}\\
		& $\Leftrightarrow$ &  \hbox to 1.75cm {\hfill $-2x-10= 0$\hfill} \quad  ou \quad  \hbox to 1.75cm {\hfill $4x-5=0$\hfill} \\
		 & $\Leftrightarrow$ & \hbox to 1.75cm {\hfill $-2x = 10$\hfill} \quad  ou \quad \hbox to 1.75cm {\hfill $4x = 5$\hfill} \\[1.5ex]
		 & $\Leftrightarrow$ & \hbox to 1.75cm {\hfill $x= \dfrac{10}{-2}$\hfill} \quad  ou \quad \hbox to 1.75cm {\hfill $x= \dfrac{5}{4}$\hfill} \\[2.5ex]
		 & $\Leftrightarrow$ & \hbox to 1.75cm {\hfill $x = -5$\hfill} \quad  ou \quad \hbox to 1.75cm {\hfill $x = \dfrac{5}{4}$\hfill}
	\end{tabular}

\bigskip

L'ensemble des solutions de $(E)$ est $S=\left\{-5,~\dfrac{5}{4}\right\}$.

Les antécédents de $-6$ par $f$ sont donc $-5$ et $\dfrac{5}{4}$.\end{frame}


\begin{frame}
\vspace{-10mm}
	\frametitle{Correction 2}
1)

\begin{align*}(-3x+8)(9x+5)&=-27x^{2}-15x+72x+40\\
	&=-27x^{2}+57x+40\end{align*}
\end{frame}

\begin{frame}
\vspace*{2em}2) 

 	\begin{tabular}{ccc} $f(x) = 6$ & $\Leftrightarrow$ & $-27x^{2}+57x+46=6$  \\
		& $\Leftrightarrow$ & $-27x^{2}+57x+40=0$  \\
		& $\Leftrightarrow$ &  $(-3x+8)(9x+5)=0$  \quad \text{D'après 1)}\\
		& $\Leftrightarrow$ &  \hbox to 1.75cm {\hfill $-3x+8= 0$\hfill} \quad  ou \quad  \hbox to 1.75cm {\hfill $9x+5=0$\hfill} \\
		 & $\Leftrightarrow$ & \hbox to 1.75cm {\hfill $-3x = -8$\hfill} \quad  ou \quad \hbox to 1.75cm {\hfill $9x = -5$\hfill} \\[1.5ex]
		 & $\Leftrightarrow$ & \hbox to 1.75cm {\hfill $x= \dfrac{-8}{-3}$\hfill} \quad  ou \quad \hbox to 1.75cm {\hfill $x= \dfrac{-5}{9}$\hfill} \\[2.5ex]
		 & $\Leftrightarrow$ & \hbox to 1.75cm {\hfill $x = \dfrac{8}{3}$\hfill} \quad  ou \quad \hbox to 1.75cm {\hfill $x = \dfrac{-5}{9}$\hfill}
	\end{tabular}

\bigskip

L'ensemble des solutions de $(E)$ est $S=\left\{\dfrac{8}{3},~\dfrac{-5}{9}\right\}$.

Les antécédents de $6$ par $f$ sont donc $\dfrac{8}{3}$ et $\dfrac{-5}{9}$.\end{frame}


\begin{frame}
\vspace{-10mm}
	\frametitle{Correction 3}
1)

\begin{align*}(-4x-4)(-x-1)&=4x^{2}+4x+4x+4\\
	&=4x^{2}+8x+4\end{align*}
\end{frame}

\begin{frame}
\vspace*{2em}2) 

 	\begin{tabular}{ccc} $f(x) = -10$ & $\Leftrightarrow$ & $4x^{2}+8x-6=-10$  \\
		& $\Leftrightarrow$ & $4x^{2}+8x+4=0$  \\
		& $\Leftrightarrow$ &  $(-4x-4)(-x-1)=0$  \quad \text{D'après 1)}\\
		& $\Leftrightarrow$ &  \hbox to 1.75cm {\hfill $-4x-4= 0$\hfill} \quad  ou \quad  \hbox to 1.75cm {\hfill $-x-1=0$\hfill} \\
		 & $\Leftrightarrow$ & \hbox to 1.75cm {\hfill $-4x = 4$\hfill} \quad  ou \quad \hbox to 1.75cm {\hfill $-x = 1$\hfill} \\[1.5ex]
		 & $\Leftrightarrow$ & \hbox to 1.75cm {\hfill $x= \dfrac{4}{-4}$\hfill} \quad  ou \quad \hbox to 1.75cm {\hfill $x= \dfrac{1}{-1}$\hfill} \\[2.5ex]
		 & $\Leftrightarrow$ & \hbox to 1.75cm {\hfill $x = -1$\hfill} \quad  ou \quad \hbox to 1.75cm {\hfill $x = -1$\hfill}
	\end{tabular}

\bigskip

L'ensemble des solutions de $(E)$ est $S=\left\{-1\right\}$.

L'antécédent de $-10$ par $f$ est donc $-1$.\end{frame}


\begin{frame}
\vspace{-10mm}
	\frametitle{Correction 4}
1)

\begin{align*}(-4x+2)(2x-4)&=-8x^{2}+16x+4x-8\\
	&=-8x^{2}+20x-8\end{align*}
\end{frame}

\begin{frame}
\vspace*{2em}2) 

 	\begin{tabular}{ccc} $f(x) = 3$ & $\Leftrightarrow$ & $-8x^{2}+20x-5=3$  \\
		& $\Leftrightarrow$ & $-8x^{2}+20x-8=0$  \\
		& $\Leftrightarrow$ &  $(-4x+2)(2x-4)=0$  \quad \text{D'après 1)}\\
		& $\Leftrightarrow$ &  \hbox to 1.75cm {\hfill $-4x+2= 0$\hfill} \quad  ou \quad  \hbox to 1.75cm {\hfill $2x-4=0$\hfill} \\
		 & $\Leftrightarrow$ & \hbox to 1.75cm {\hfill $-4x = -2$\hfill} \quad  ou \quad \hbox to 1.75cm {\hfill $2x = 4$\hfill} \\[1.5ex]
		 & $\Leftrightarrow$ & \hbox to 1.75cm {\hfill $x= \dfrac{-2}{-4}$\hfill} \quad  ou \quad \hbox to 1.75cm {\hfill $x= \dfrac{4}{2}$\hfill} \\[2.5ex]
		 & $\Leftrightarrow$ & \hbox to 1.75cm {\hfill $x = \dfrac{1}{2}$\hfill} \quad  ou \quad \hbox to 1.75cm {\hfill $x = 2$\hfill}
	\end{tabular}

\bigskip

L'ensemble des solutions de $(E)$ est $S=\left\{\dfrac{1}{2},~2\right\}$.

Les antécédents de $3$ par $f$ sont donc $\dfrac{1}{2}$ et $2$.\end{frame}


\begin{frame}
\vspace{-10mm}
	\frametitle{Correction 5}
1)

\begin{align*}(-4x+6)(x-6)&=-4x^{2}+24x+6x-36\\
	&=-4x^{2}+30x-36\end{align*}
\end{frame}

\begin{frame}
\vspace*{2em}2) 

 	\begin{tabular}{ccc} $f(x) = -3$ & $\Leftrightarrow$ & $-4x^{2}+30x-39=-3$  \\
		& $\Leftrightarrow$ & $-4x^{2}+30x-36=0$  \\
		& $\Leftrightarrow$ &  $(-4x+6)(x-6)=0$  \quad \text{D'après 1)}\\
		& $\Leftrightarrow$ &  \hbox to 1.75cm {\hfill $-4x+6= 0$\hfill} \quad  ou \quad  \hbox to 1.75cm {\hfill $x-6=0$\hfill} \\
		 & $\Leftrightarrow$ & \hbox to 1.75cm {\hfill $-4x = -6$\hfill} \quad  ou \quad \hbox to 1.75cm {\hfill $x = 6$\hfill} \\[1.5ex]
		 & $\Leftrightarrow$ & \hbox to 1.75cm {\hfill $x= \dfrac{-6}{-4}$\hfill} \quad  ou \quad \hbox to 1.75cm {\hfill $x= \dfrac{6}{1}$\hfill} \\[2.5ex]
		 & $\Leftrightarrow$ & \hbox to 1.75cm {\hfill $x = \dfrac{3}{2}$\hfill} \quad  ou \quad \hbox to 1.75cm {\hfill $x = 6$\hfill}
	\end{tabular}

\bigskip

L'ensemble des solutions de $(E)$ est $S=\left\{\dfrac{3}{2},~6\right\}$.

Les antécédents de $-3$ par $f$ sont donc $\dfrac{3}{2}$ et $6$.\end{frame}




\end{document}