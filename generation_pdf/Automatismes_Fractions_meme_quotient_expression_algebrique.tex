\documentclass[15pt, mathserif]{beamer}

\usepackage[french]{babel}
\usepackage[T1]{fontenc}
\usepackage[utf8]{inputenc}
%\usepackage{esvect}
\usepackage{bm}
\usepackage{eurosym}
\usepackage{tikz}
\usepackage{pgf,tikz,pgfplots}
\pgfplotsset{compat=1.15}
\usepackage{mathrsfs}
\usetikzlibrary{arrows}
\usetikzlibrary{arrows.meta}

\usetikzlibrary{mindmap}
\usepackage{multicol}
\usepackage[tikz]{bclogo}
\usepackage{tkz-tab}
\usepackage{amsmath, tabu}
\usepackage{esvect} %\vv{AB} pour le vecteur AB

\DeclareMathOperator{\e}{e}

%% Tableau

\usepackage{makecell}
\setcellgapes{1pt}
\makegapedcells
\newcolumntype{R}[1]{>{\raggedleft\arraybackslash }b{#1}}
\newcolumntype{L}[1]{>{\raggedright\arraybackslash }b{#1}}
\newcolumntype{C}[1]{>{\centering\arraybackslash }b{#1}}


%pour avoir des parenthèses rondes dans le package fourier
\DeclareSymbolFont{cmoperators}   {OT1}{cmr} {m}{n}
\DeclareSymbolFont{cmlargesymbols}{OMX}{cmex}{m}{n}

\usefonttheme{professionalfonts} %permet d'enlever un bug avec fourier
\usepackage{fourier}
\DeclareMathDelimiter{(}{\mathopen} {cmoperators}{"28}{cmlargesymbols}{"00}
\DeclareMathDelimiter{)}{\mathclose}{cmoperators}{"29}{cmlargesymbols}{"01}

%Graphiques 

\usepackage{pgf,tikz,pgfplots}
\pgfplotsset{compat=1.15}
\usepackage{mathrsfs}
\usetikzlibrary{arrows}
\usetikzlibrary{mindmap}

%ensembles de nbres

\newcommand{\R}{\mathbb{R}}			%permet d'écrire le R "ensemble des réels"'
\newcommand{\N}{\mathbb{N}}			%permet d'écrire le N "ensemble des entiers naturels"
\newcommand{\Z}{\mathbb{Z}}			%permet d'écrire le Z "ensemble des entiers relatifs"
\newcommand{\Prem}{\mathbb{P}}	%permet d'écrire le P "ensemble des nombres premiers" (qui n'a pas marché avec le \P car il existe déjà)
\newcommand{\D}{\mathbb{D}}
\newcommand{\Df}{\mathcal{D}_f}
\newcommand{\Cf}{\mathcal{C}_f}

\newcommand{\Q}{\mathbb{Q}}


\newcommand{\st}[1]{$(#1_n)_{n \in \N}$}

\usetheme{Madrid}
\useoutertheme{miniframes} % Alternatively: miniframes, infolines, split
\useinnertheme{circles}
\definecolor{UBCblue}{rgb}{0.1, 0.25, 0.4} % UBC Blue (primary)
\definecolor{bordeaux}{RGB}{128,0,0}
\usecolortheme[named=UBCblue]{structure}

\usepackage{color} % J'aime bien définir mes couleurs
\definecolor{propcolor}{rgb}{0, 0.5, 1}
\definecolor{thcolor}{rgb}{0.6, 0.07, 0.07}
\colorlet{louis}{blue!70!green!60!white}
\colorlet{sakura}{pink!40!red}

\title{Activités Mentales}
\date{24 Août 2023}

\newcommand{\vco}[2]{\begin{pmatrix} #1 \\ #2 \end{pmatrix}} %Coordonnées de vecteur
\newenvironment{eq}{\begin{cases}\begin{tabu}{ccccc}}{\end{tabu}\end{cases}}
\newenvironment{eql}{\begin{cases}\begin{tabu}{cccccl}}{\end{tabu}\end{cases}}
\newenvironment{eqrl}{\begin{cases}\begin{tabu}{rl}}{\end{tabu}\end{cases}}

\newenvironment{Eq}{\begin{center}\begin{tabular}{rrcl}}{\end{tabular}\end{center}}
\newcommand{\ligneq}[2]{$\Longleftrightarrow$ & $#1$ & $=$ & $#2$ \\}
\newcommand{\Ligneq}[2]{ & $#1$ & $=$ & $#2$ \\}

\newenvironment{RPN}{\begin{center}\begin{tabular}{rrclcrcl}}{\end{tabular}\end{center}}
\newcommand{\Lignerpn}[4]{ & $#1$ & $=$ & $#2$ & ou & $#3$ & $=$ & $#4$ \\}
\newcommand{\lignerpn}[4]{$\Longleftrightarrow$ & $#1$ & $=$ & $#2$ & ou & $#3$ & $=$ & $#4$ \\}

\newenvironment{TRPN}{\begin{center}\begin{tabular}{rrclcrclcrcl}}{\end{tabular}\end{center}}
\newcommand{\Lignetrpn}[6]{ & $#1$ & $=$ & $#2$ & ou & $#3$ & $=$ & $#4$ & ou & $#5$ & $=$ & $#6$ \\}
\newcommand{\lignetrpn}[6]{$\Longleftrightarrow$ & $#1$ & $=$ & $#2$ & ou & $#3$ & $=$ & $#4$ & ou & $#5$ & $=$ & $#6$ \\}
\begin{document}

\begin{frame}
    \titlepage
\end{frame}

\begin{frame} 
	\frametitle{Question 1}
Mettre au même dénominateur l'expression suivante: \[\dfrac{6x+2}{x-6} + \dfrac{6x+5}{x+6}\]\end{frame}


\begin{frame} 
	\frametitle{Question 2}
Mettre au même dénominateur l'expression suivante: \[\dfrac{5x-5}{x-2} + \dfrac{-2x-2}{x+3}\]\end{frame}


\begin{frame} 
	\frametitle{Question 3}
Mettre au même dénominateur l'expression suivante: \[\dfrac{2x-2}{x+6} + \dfrac{3x-1}{x+2}\]\end{frame}


\begin{frame} 
	\frametitle{Question 4}
Mettre au même dénominateur l'expression suivante: \[\dfrac{3x+1}{x+3} + \dfrac{7x-2}{x+6}\]\end{frame}


\begin{frame} 
	\frametitle{Question 5}
Mettre au même dénominateur l'expression suivante: \[\dfrac{x+2}{x-5} + \dfrac{2x+3}{x+2}\]\end{frame}


\begin{frame}
\vspace{-10mm}
	\frametitle{Correction 1}


\bigskip 

Je multiplie la première fraction par $x+6$ qui est le dénominateur de la deuxième fraction et la deuxième fraction par $x-6$ qui est le dénominateur de la première fraction. On a alors 

 \begin{align*} \dfrac{6x+2}{x-6} + \dfrac{6x+5}{x+6} &= \dfrac{(6x+2)(x+6)}{(x-6)(x+6)} + \dfrac{(6x+5)(x-6)}{(x+6)(x-6)}\\ &= \dfrac{6x^2+36x+2x+12}{(x-6)(x+6)}+ \dfrac{6x^2-36x+5x-30}{(x+6)(x-6)} \\
	&=\dfrac{6x^2+38x+12+6x^2-31x-30}{(x-6)(x+6)}\\
	&=\dfrac{12x^2+7x-18}{(x-6)(x+6)}
\end{align*}\end{frame}


\begin{frame}
\vspace{-10mm}
	\frametitle{Correction 2}


\bigskip 

Je multiplie la première fraction par $x+3$ qui est le dénominateur de la deuxième fraction et la deuxième fraction par $x-2$ qui est le dénominateur de la première fraction. On a alors 

 \begin{align*} \dfrac{5x-5}{x-2} + \dfrac{-2x-2}{x+3} &= \dfrac{(5x-5)(x+3)}{(x-2)(x+3)} + \dfrac{(-2x-2)(x-2)}{(x+3)(x-2)}\\ &= \dfrac{5x^2+15x-5x-15}{(x-2)(x+3)}+ \dfrac{-2x^2+4x-2x+4}{(x+3)(x-2)} \\
	&=\dfrac{5x^2+10x-15-2x^2+2x+4}{(x-2)(x+3)}\\
	&=\dfrac{3x^2+12x-11}{(x-2)(x+3)}
\end{align*}\end{frame}


\begin{frame}
\vspace{-10mm}
	\frametitle{Correction 3}


\bigskip 

Je multiplie la première fraction par $x+2$ qui est le dénominateur de la deuxième fraction et la deuxième fraction par $x+6$ qui est le dénominateur de la première fraction. On a alors 

 \begin{align*} \dfrac{2x-2}{x+6} + \dfrac{3x-1}{x+2} &= \dfrac{(2x-2)(x+2)}{(x+6)(x+2)} + \dfrac{(3x-1)(x+6)}{(x+2)(x+6)}\\ &= \dfrac{2x^2+4x-2x-4}{(x+6)(x+2)}+ \dfrac{3x^2+18x-x-6}{(x+2)(x+6)} \\
	&=\dfrac{2x^2+2x-4+3x^2+17x-6}{(x+6)(x+2)}\\
	&=\dfrac{5x^2+19x-10}{(x+6)(x+2)}
\end{align*}\end{frame}


\begin{frame}
\vspace{-10mm}
	\frametitle{Correction 4}


\bigskip 

Je multiplie la première fraction par $x+6$ qui est le dénominateur de la deuxième fraction et la deuxième fraction par $x+3$ qui est le dénominateur de la première fraction. On a alors 

 \begin{align*} \dfrac{3x+1}{x+3} + \dfrac{7x-2}{x+6} &= \dfrac{(3x+1)(x+6)}{(x+3)(x+6)} + \dfrac{(7x-2)(x+3)}{(x+6)(x+3)}\\ &= \dfrac{3x^2+18x+x+6}{(x+3)(x+6)}+ \dfrac{7x^2+21x-2x-6}{(x+6)(x+3)} \\
	&=\dfrac{3x^2+19x+6+7x^2+19x-6}{(x+3)(x+6)}\\
	&=\dfrac{10x^2+38x}{(x+3)(x+6)}
\end{align*}\end{frame}


\begin{frame}
\vspace{-10mm}
	\frametitle{Correction 5}


\bigskip 

Je multiplie la première fraction par $x+2$ qui est le dénominateur de la deuxième fraction et la deuxième fraction par $x-5$ qui est le dénominateur de la première fraction. On a alors 

 \begin{align*} \dfrac{x+2}{x-5} + \dfrac{2x+3}{x+2} &= \dfrac{(x+2)(x+2)}{(x-5)(x+2)} + \dfrac{(2x+3)(x-5)}{(x+2)(x-5)}\\ &= \dfrac{x^2+2x+2x+4}{(x-5)(x+2)}+ \dfrac{2x^2-10x+3x-15}{(x+2)(x-5)} \\
	&=\dfrac{x^2+4x+4+2x^2-7x-15}{(x-5)(x+2)}\\
	&=\dfrac{3x^2-3x-11}{(x-5)(x+2)}
\end{align*}\end{frame}




\end{document}