\documentclass[15pt, mathserif]{beamer}

\usepackage[french]{babel}
\usepackage[T1]{fontenc}
\usepackage[utf8]{inputenc}
%\usepackage{esvect}
\usepackage{bm}
\usepackage{eurosym}
\usepackage{tikz}
\usepackage{pgf,tikz,pgfplots}
\pgfplotsset{compat=1.15}
\usepackage{mathrsfs}
\usetikzlibrary{arrows}
\usetikzlibrary{arrows.meta}

\usetikzlibrary{mindmap}
\usepackage{multicol}
\usepackage[tikz]{bclogo}
\usepackage{tkz-tab}
\usepackage{amsmath, tabu}
\usepackage{esvect} %\vv{AB} pour le vecteur AB

\DeclareMathOperator{\e}{e}

%% Tableau

\usepackage{makecell}
\setcellgapes{1pt}
\makegapedcells
\newcolumntype{R}[1]{>{\raggedleft\arraybackslash }b{#1}}
\newcolumntype{L}[1]{>{\raggedright\arraybackslash }b{#1}}
\newcolumntype{C}[1]{>{\centering\arraybackslash }b{#1}}


%pour avoir des parenthèses rondes dans le package fourier
\DeclareSymbolFont{cmoperators}   {OT1}{cmr} {m}{n}
\DeclareSymbolFont{cmlargesymbols}{OMX}{cmex}{m}{n}

\usefonttheme{professionalfonts} %permet d'enlever un bug avec fourier
\usepackage{fourier}
\DeclareMathDelimiter{(}{\mathopen} {cmoperators}{"28}{cmlargesymbols}{"00}
\DeclareMathDelimiter{)}{\mathclose}{cmoperators}{"29}{cmlargesymbols}{"01}

%Graphiques 

\usepackage{pgf,tikz,pgfplots}
\pgfplotsset{compat=1.15}
\usepackage{mathrsfs}
\usetikzlibrary{arrows}
\usetikzlibrary{mindmap}

%ensembles de nbres

\newcommand{\R}{\mathbb{R}}			%permet d'écrire le R "ensemble des réels"'
\newcommand{\N}{\mathbb{N}}			%permet d'écrire le N "ensemble des entiers naturels"
\newcommand{\Z}{\mathbb{Z}}			%permet d'écrire le Z "ensemble des entiers relatifs"
\newcommand{\Prem}{\mathbb{P}}	%permet d'écrire le P "ensemble des nombres premiers" (qui n'a pas marché avec le \P car il existe déjà)
\newcommand{\D}{\mathbb{D}}
\newcommand{\Df}{\mathcal{D}_f}
\newcommand{\Cf}{\mathcal{C}_f}

\newcommand{\Q}{\mathbb{Q}}


\newcommand{\st}[1]{$(#1_n)_{n \in \N}$}

\usetheme{Madrid}
\useoutertheme{miniframes} % Alternatively: miniframes, infolines, split
\useinnertheme{circles}
\definecolor{UBCblue}{rgb}{0.1, 0.25, 0.4} % UBC Blue (primary)
\definecolor{bordeaux}{RGB}{128,0,0}
\usecolortheme[named=UBCblue]{structure}

\usepackage{color} % J'aime bien définir mes couleurs
\definecolor{propcolor}{rgb}{0, 0.5, 1}
\definecolor{thcolor}{rgb}{0.6, 0.07, 0.07}
\colorlet{louis}{blue!70!green!60!white}
\colorlet{sakura}{pink!40!red}

\title{Activités Mentales}
\date{24 Août 2023}

\newcommand{\vco}[2]{\begin{pmatrix} #1 \\ #2 \end{pmatrix}} %Coordonnées de vecteur
\newenvironment{eq}{\begin{cases}\begin{tabu}{ccccc}}{\end{tabu}\end{cases}}
\newenvironment{eql}{\begin{cases}\begin{tabu}{cccccl}}{\end{tabu}\end{cases}}
\newenvironment{eqrl}{\begin{cases}\begin{tabu}{rl}}{\end{tabu}\end{cases}}

\newenvironment{Eq}{\begin{center}\begin{tabular}{rrcl}}{\end{tabular}\end{center}}
\newcommand{\ligneq}[2]{$\Longleftrightarrow$ & $#1$ & $=$ & $#2$ \\}
\newcommand{\Ligneq}[2]{ & $#1$ & $=$ & $#2$ \\}

\newenvironment{RPN}{\begin{center}\begin{tabular}{rrclcrcl}}{\end{tabular}\end{center}}
\newcommand{\Lignerpn}[4]{ & $#1$ & $=$ & $#2$ & ou & $#3$ & $=$ & $#4$ \\}
\newcommand{\lignerpn}[4]{$\Longleftrightarrow$ & $#1$ & $=$ & $#2$ & ou & $#3$ & $=$ & $#4$ \\}

\newenvironment{TRPN}{\begin{center}\begin{tabular}{rrclcrclcrcl}}{\end{tabular}\end{center}}
\newcommand{\Lignetrpn}[6]{ & $#1$ & $=$ & $#2$ & ou & $#3$ & $=$ & $#4$ & ou & $#5$ & $=$ & $#6$ \\}
\newcommand{\lignetrpn}[6]{$\Longleftrightarrow$ & $#1$ & $=$ & $#2$ & ou & $#3$ & $=$ & $#4$ & ou & $#5$ & $=$ & $#6$ \\}
\begin{document}

\begin{frame}
    \titlepage
\end{frame}

\begin{frame} 
	\frametitle{Question 1}
On considère la suite $(u_n)_{n\in\mathbb{N}}$ définie par la relation de récurrence suivante:\[\begin{cases} u_{n+1} = 2u_n+30\\ u_0 = -17\end{cases}.\]

Démontrer par récurrence que la suite $(u_n)_{n\in\mathbb{N}}$ est  croissante .\end{frame}


\begin{frame} 
	\frametitle{Question 2}
On considère la suite $(u_n)_{n\in\mathbb{N}}$ définie par la relation de récurrence suivante:\[\begin{cases} u_{n+1} = 6u_n+15\\ u_0 = 13\end{cases}.\]

Démontrer par récurrence que la suite $(u_n)_{n\in\mathbb{N}}$ est  croissante .\end{frame}


\begin{frame} 
	\frametitle{Question 3}
On considère la suite $(u_n)_{n\in\mathbb{N}}$ définie par la relation de récurrence suivante:\[\begin{cases} u_{n+1} = 5u_n+25\\ u_0 = -17\end{cases}.\]

Démontrer par récurrence que la suite $(u_n)_{n\in\mathbb{N}}$ est  croissante .\end{frame}


\begin{frame} 
	\frametitle{Question 4}
On considère la suite $(u_n)_{n\in\mathbb{N}}$ définie par la relation de récurrence suivante:\[\begin{cases} u_{n+1} = 10u_n+37\\ u_0 = -3\end{cases}.\]

Démontrer par récurrence que la suite $(u_n)_{n\in\mathbb{N}}$ est  croissante .\end{frame}


\begin{frame} 
	\frametitle{Question 5}
On considère la suite $(u_n)_{n\in\mathbb{N}}$ définie par la relation de récurrence suivante:\[\begin{cases} u_{n+1} = 6u_n+39\\ u_0 = -15\end{cases}.\]

Démontrer par récurrence que la suite $(u_n)_{n\in\mathbb{N}}$ est  croissante .\end{frame}


\begin{frame}
\vspace{-10mm}
	\frametitle{Correction 1}
\vspace{6.5mm}

On rappelle que $\begin{cases} u_{n+1} = 2u_n+30\\ u_0 = -17\end{cases}$.

\medskip

 On pose pour tout entier $n \in \mathbb{N}$ l'hypothèse de récurrence $H_n:~"u_n  \leq u_{n+1}"$.

\medskip

Initialisation: On a $u_0 =-17$ et $u_1 = 2u_0+30= 2\times\left(-17\right)+30=-4$.

 Comme $-17 \leq -4$, on a bien $u_0  \leq u_{0+1}$ et $H_0$ est vraie.

\medskip

Hérédité: Supposons $H_k$ vraie pour \textbf\underline{{$k$ fixé}} et montrons que $H_{k+1}$ est vraie. C'est-à-dire, montrons que si $u_k  \leq u_{k+1}$ alors $u_{k+1}  \leq u_{k+2}$.Or \begin{align*} u_k  \leq u_{k+1} &\Rightarrow 2u_k \leq 2u_{k+1}\quad \text{car } 2>0\\
	 &\Rightarrow2u_k+30 \leq 2u_{k+1}+30\\
	 &\Rightarrow u_{k+1}  \leq u_{k+2}
\end{align*}

Donc $H_{k} \Rightarrow H_{k+1}$. On a finalement démontré par récurrence que pour tout $n \in\mathbb{N},~ u_n \leq u_{n+1}$ et la suite est donc croissante.\end{frame}


\begin{frame}
\vspace{-10mm}
	\frametitle{Correction 2}
\vspace{6.5mm}

On rappelle que $\begin{cases} u_{n+1} = 6u_n+15\\ u_0 = 13\end{cases}$.

\medskip

 On pose pour tout entier $n \in \mathbb{N}$ l'hypothèse de récurrence $H_n:~"u_n  \leq u_{n+1}"$.

\medskip

Initialisation: On a $u_0 =13$ et $u_1 = 6u_0+15= 6\times13+15=93$.

 Comme $13 \leq 93$, on a bien $u_0  \leq u_{0+1}$ et $H_0$ est vraie.

\medskip

Hérédité: Supposons $H_k$ vraie pour \textbf\underline{{$k$ fixé}} et montrons que $H_{k+1}$ est vraie. C'est-à-dire, montrons que si $u_k  \leq u_{k+1}$ alors $u_{k+1}  \leq u_{k+2}$.Or \begin{align*} u_k  \leq u_{k+1} &\Rightarrow 6u_k \leq 6u_{k+1}\quad \text{car } 6>0\\
	 &\Rightarrow6u_k+15 \leq 6u_{k+1}+15\\
	 &\Rightarrow u_{k+1}  \leq u_{k+2}
\end{align*}

Donc $H_{k} \Rightarrow H_{k+1}$. On a finalement démontré par récurrence que pour tout $n \in\mathbb{N},~ u_n \leq u_{n+1}$ et la suite est donc croissante.\end{frame}


\begin{frame}
\vspace{-10mm}
	\frametitle{Correction 3}
\vspace{6.5mm}

On rappelle que $\begin{cases} u_{n+1} = 5u_n+25\\ u_0 = -17\end{cases}$.

\medskip

 On pose pour tout entier $n \in \mathbb{N}$ l'hypothèse de récurrence $H_n:~"u_n  \leq u_{n+1}"$.

\medskip

Initialisation: On a $u_0 =-17$ et $u_1 = 5u_0+25= 5\times\left(-17\right)+25=-60$.

 Comme $-17 \leq -60$, on a bien $u_0  \leq u_{0+1}$ et $H_0$ est vraie.

\medskip

Hérédité: Supposons $H_k$ vraie pour \textbf\underline{{$k$ fixé}} et montrons que $H_{k+1}$ est vraie. C'est-à-dire, montrons que si $u_k  \leq u_{k+1}$ alors $u_{k+1}  \leq u_{k+2}$.Or \begin{align*} u_k  \leq u_{k+1} &\Rightarrow 5u_k \leq 5u_{k+1}\quad \text{car } 5>0\\
	 &\Rightarrow5u_k+25 \leq 5u_{k+1}+25\\
	 &\Rightarrow u_{k+1}  \leq u_{k+2}
\end{align*}

Donc $H_{k} \Rightarrow H_{k+1}$. On a finalement démontré par récurrence que pour tout $n \in\mathbb{N},~ u_n \leq u_{n+1}$ et la suite est donc croissante.\end{frame}


\begin{frame}
\vspace{-10mm}
	\frametitle{Correction 4}
\vspace{6.5mm}

On rappelle que $\begin{cases} u_{n+1} = 10u_n+37\\ u_0 = -3\end{cases}$.

\medskip

 On pose pour tout entier $n \in \mathbb{N}$ l'hypothèse de récurrence $H_n:~"u_n  \leq u_{n+1}"$.

\medskip

Initialisation: On a $u_0 =-3$ et $u_1 = 10u_0+37= 10\times\left(-3\right)+37=7$.

 Comme $-3 \leq 7$, on a bien $u_0  \leq u_{0+1}$ et $H_0$ est vraie.

\medskip

Hérédité: Supposons $H_k$ vraie pour \textbf\underline{{$k$ fixé}} et montrons que $H_{k+1}$ est vraie. C'est-à-dire, montrons que si $u_k  \leq u_{k+1}$ alors $u_{k+1}  \leq u_{k+2}$.Or \begin{align*} u_k  \leq u_{k+1} &\Rightarrow 10u_k \leq 10u_{k+1}\quad \text{car } 10>0\\
	 &\Rightarrow10u_k+37 \leq 10u_{k+1}+37\\
	 &\Rightarrow u_{k+1}  \leq u_{k+2}
\end{align*}

Donc $H_{k} \Rightarrow H_{k+1}$. On a finalement démontré par récurrence que pour tout $n \in\mathbb{N},~ u_n \leq u_{n+1}$ et la suite est donc croissante.\end{frame}


\begin{frame}
\vspace{-10mm}
	\frametitle{Correction 5}
\vspace{6.5mm}

On rappelle que $\begin{cases} u_{n+1} = 6u_n+39\\ u_0 = -15\end{cases}$.

\medskip

 On pose pour tout entier $n \in \mathbb{N}$ l'hypothèse de récurrence $H_n:~"u_n  \leq u_{n+1}"$.

\medskip

Initialisation: On a $u_0 =-15$ et $u_1 = 6u_0+39= 6\times\left(-15\right)+39=-51$.

 Comme $-15 \leq -51$, on a bien $u_0  \leq u_{0+1}$ et $H_0$ est vraie.

\medskip

Hérédité: Supposons $H_k$ vraie pour \textbf\underline{{$k$ fixé}} et montrons que $H_{k+1}$ est vraie. C'est-à-dire, montrons que si $u_k  \leq u_{k+1}$ alors $u_{k+1}  \leq u_{k+2}$.Or \begin{align*} u_k  \leq u_{k+1} &\Rightarrow 6u_k \leq 6u_{k+1}\quad \text{car } 6>0\\
	 &\Rightarrow6u_k+39 \leq 6u_{k+1}+39\\
	 &\Rightarrow u_{k+1}  \leq u_{k+2}
\end{align*}

Donc $H_{k} \Rightarrow H_{k+1}$. On a finalement démontré par récurrence que pour tout $n \in\mathbb{N},~ u_n \leq u_{n+1}$ et la suite est donc croissante.\end{frame}




\end{document}