\documentclass[15pt, mathserif]{beamer}

\usepackage[french]{babel}
\usepackage[T1]{fontenc}
\usepackage[utf8]{inputenc}
%\usepackage{esvect}
\usepackage{bm}
\usepackage{eurosym}
\usepackage{tikz}
\usepackage{pgf,tikz,pgfplots}
\pgfplotsset{compat=1.15}
\usepackage{mathrsfs}
\usetikzlibrary{arrows}
\usetikzlibrary{arrows.meta}

\usetikzlibrary{mindmap}
\usepackage{multicol}
\usepackage[tikz]{bclogo}
\usepackage{tkz-tab}
\usepackage{amsmath, tabu}
\usepackage{esvect} %\vv{AB} pour le vecteur AB

\DeclareMathOperator{\e}{e}

%% Tableau

\usepackage{makecell}
\setcellgapes{1pt}
\makegapedcells
\newcolumntype{R}[1]{>{\raggedleft\arraybackslash }b{#1}}
\newcolumntype{L}[1]{>{\raggedright\arraybackslash }b{#1}}
\newcolumntype{C}[1]{>{\centering\arraybackslash }b{#1}}


%pour avoir des parenthèses rondes dans le package fourier
\DeclareSymbolFont{cmoperators}   {OT1}{cmr} {m}{n}
\DeclareSymbolFont{cmlargesymbols}{OMX}{cmex}{m}{n}

\usefonttheme{professionalfonts} %permet d'enlever un bug avec fourier
\usepackage{fourier}
\DeclareMathDelimiter{(}{\mathopen} {cmoperators}{"28}{cmlargesymbols}{"00}
\DeclareMathDelimiter{)}{\mathclose}{cmoperators}{"29}{cmlargesymbols}{"01}

%Graphiques 

\usepackage{pgf,tikz,pgfplots}
\pgfplotsset{compat=1.15}
\usepackage{mathrsfs}
\usetikzlibrary{arrows}
\usetikzlibrary{mindmap}

%ensembles de nbres

\newcommand{\R}{\mathbb{R}}			%permet d'écrire le R "ensemble des réels"'
\newcommand{\N}{\mathbb{N}}			%permet d'écrire le N "ensemble des entiers naturels"
\newcommand{\Z}{\mathbb{Z}}			%permet d'écrire le Z "ensemble des entiers relatifs"
\newcommand{\Prem}{\mathbb{P}}	%permet d'écrire le P "ensemble des nombres premiers" (qui n'a pas marché avec le \P car il existe déjà)
\newcommand{\D}{\mathbb{D}}
\newcommand{\Df}{\mathcal{D}_f}
\newcommand{\Cf}{\mathcal{C}_f}

\newcommand{\Q}{\mathbb{Q}}


\newcommand{\st}[1]{$(#1_n)_{n \in \N}$}

\usetheme{Madrid}
\useoutertheme{miniframes} % Alternatively: miniframes, infolines, split
\useinnertheme{circles}
\definecolor{UBCblue}{rgb}{0.1, 0.25, 0.4} % UBC Blue (primary)
\definecolor{bordeaux}{RGB}{128,0,0}
\usecolortheme[named=UBCblue]{structure}

\usepackage{color} % J'aime bien définir mes couleurs
\definecolor{propcolor}{rgb}{0, 0.5, 1}
\definecolor{thcolor}{rgb}{0.6, 0.07, 0.07}
\colorlet{louis}{blue!70!green!60!white}
\colorlet{sakura}{pink!40!red}

\title{Activités Mentales}
\date{24 Août 2023}

\newcommand{\vco}[2]{\begin{pmatrix} #1 \\ #2 \end{pmatrix}} %Coordonnées de vecteur
\newenvironment{eq}{\begin{cases}\begin{tabu}{ccccc}}{\end{tabu}\end{cases}}
\newenvironment{eql}{\begin{cases}\begin{tabu}{cccccl}}{\end{tabu}\end{cases}}
\newenvironment{eqrl}{\begin{cases}\begin{tabu}{rl}}{\end{tabu}\end{cases}}

\newenvironment{Eq}{\begin{center}\begin{tabular}{rrcl}}{\end{tabular}\end{center}}
\newcommand{\ligneq}[2]{$\Longleftrightarrow$ & $#1$ & $=$ & $#2$ \\}
\newcommand{\Ligneq}[2]{ & $#1$ & $=$ & $#2$ \\}

\newenvironment{RPN}{\begin{center}\begin{tabular}{rrclcrcl}}{\end{tabular}\end{center}}
\newcommand{\Lignerpn}[4]{ & $#1$ & $=$ & $#2$ & ou & $#3$ & $=$ & $#4$ \\}
\newcommand{\lignerpn}[4]{$\Longleftrightarrow$ & $#1$ & $=$ & $#2$ & ou & $#3$ & $=$ & $#4$ \\}

\newenvironment{TRPN}{\begin{center}\begin{tabular}{rrclcrclcrcl}}{\end{tabular}\end{center}}
\newcommand{\Lignetrpn}[6]{ & $#1$ & $=$ & $#2$ & ou & $#3$ & $=$ & $#4$ & ou & $#5$ & $=$ & $#6$ \\}
\newcommand{\lignetrpn}[6]{$\Longleftrightarrow$ & $#1$ & $=$ & $#2$ & ou & $#3$ & $=$ & $#4$ & ou & $#5$ & $=$ & $#6$ \\}
\begin{document}

\begin{frame}
    \titlepage
\end{frame}

\begin{frame} 
	\frametitle{Question 1}
On considère le tableau ci-dessous récapitulant le nombres de personnes appartenant au groupe A, au groupe B, aux deux groupes ou à aucun des deux.\begin{center} 
 \begin{tabular}{|c|c|c|c|} 
 \cline{2-4} 
 \multicolumn{1}{c|}{} & A & $\overline{A}$ & Total \\\hline 
 B   &18  &2& 20 \\\hline 
 $\overline{B}$   &162 & 18 & 180 \\\hline 
 Total   &180&20 &200 \\\hline  
 \end{tabular} 
 \end{center} On choisit une personne au hasard. On considère les évènements suivants : 
 \begin{itemize} 
 \item $A$ l'évènement 'la personne tirée appartient au groupe A'. 
 \item $B$ l'évènement 'la personne tirée appartient au groupe B'. 
 \end{itemize} 
 \begin{enumerate} 
 \item Que signifie  $\mathbb{P}(\overline{B} \cap A)$. La calculer. 
 \item Que signifie $\mathbb{P}_{A} (B \cap A)$ ? La calculer. 
  \end{enumerate} 
 \end{frame}


\begin{frame} 
	\frametitle{Question 2}
On considère le tableau ci-dessous récapitulant le nombres de personnes appartenant au groupe A, au groupe B, aux deux groupes ou à aucun des deux.\begin{center} 
 \begin{tabular}{|c|c|c|c|} 
 \cline{2-4} 
 \multicolumn{1}{c|}{} & A & $\overline{A}$ & Total \\\hline 
 B   &108  &252& 360 \\\hline 
 $\overline{B}$   &162 & 378 & 540 \\\hline 
 Total   &270&630 &900 \\\hline  
 \end{tabular} 
 \end{center} On choisit une personne au hasard. On considère les évènements suivants : 
 \begin{itemize} 
 \item $A$ l'évènement 'la personne tirée appartient au groupe A'. 
 \item $B$ l'évènement 'la personne tirée appartient au groupe B'. 
 \end{itemize} 
 \begin{enumerate} 
 \item Que signifie $\mathbb{P}(\overline{A} \cap B)$. La calculer. 
 \item Que signifie $\mathbb{P}_{B} (B \cap A)$ ? La calculer. 
  \end{enumerate} 
 \end{frame}


\begin{frame} 
	\frametitle{Question 3}
On considère le tableau ci-dessous récapitulant le nombres de personnes appartenant au groupe A, au groupe B, aux deux groupes ou à aucun des deux.\begin{center} 
 \begin{tabular}{|c|c|c|c|} 
 \cline{2-4} 
 \multicolumn{1}{c|}{} & A & $\overline{A}$ & Total \\\hline 
 B   &90  &90& 180 \\\hline 
 $\overline{B}$   &360 & 360 & 720 \\\hline 
 Total   &450&450 &900 \\\hline  
 \end{tabular} 
 \end{center} On choisit une personne au hasard. On considère les évènements suivants : 
 \begin{itemize} 
 \item $A$ l'évènement 'la personne tirée appartient au groupe A'. 
 \item $B$ l'évènement 'la personne tirée appartient au groupe B'. 
 \end{itemize} 
 \begin{enumerate} 
 \item Que signifie $\mathbb{P}(\overline{A} \cap \overline{B})$ ? La calculer. 
 \item Que signifie $\mathbb{P}_{\overline{A}} (\overline{A} \cap B)$. La calculer. 
  \end{enumerate} 
 \end{frame}


\begin{frame} 
	\frametitle{Question 4}
On considère le tableau ci-dessous récapitulant le nombres de personnes appartenant au groupe A, au groupe B, aux deux groupes ou à aucun des deux.\begin{center} 
 \begin{tabular}{|c|c|c|c|} 
 \cline{2-4} 
 \multicolumn{1}{c|}{} & A & $\overline{A}$ & Total \\\hline 
 B   &33  &297& 330 \\\hline 
 $\overline{B}$   &77 & 693 & 770 \\\hline 
 Total   &110&990 &1100 \\\hline  
 \end{tabular} 
 \end{center} On choisit une personne au hasard. On considère les évènements suivants : 
 \begin{itemize} 
 \item $A$ l'évènement 'la personne tirée appartient au groupe A'. 
 \item $B$ l'évènement 'la personne tirée appartient au groupe B'. 
 \end{itemize} 
 \begin{enumerate} 
 \item Que signifie $\mathbb{P}(\overline{A})$ ? La calculer. 
 \item Que signifie $\mathbb{P}_{\overline{B}} (\overline{B} \cap A)$. La calculer. 
  \end{enumerate} 
 \end{frame}


\begin{frame} 
	\frametitle{Question 5}
On considère le tableau ci-dessous récapitulant le nombres de personnes appartenant au groupe A, au groupe B, aux deux groupes ou à aucun des deux.\begin{center} 
 \begin{tabular}{|c|c|c|c|} 
 \cline{2-4} 
 \multicolumn{1}{c|}{} & A & $\overline{A}$ & Total \\\hline 
 B   &576  &864& 1440 \\\hline 
 $\overline{B}$   &144 & 216 & 360 \\\hline 
 Total   &720&1080 &1800 \\\hline  
 \end{tabular} 
 \end{center} On choisit une personne au hasard. On considère les évènements suivants : 
 \begin{itemize} 
 \item $A$ l'évènement 'la personne tirée appartient au groupe A'. 
 \item $B$ l'évènement 'la personne tirée appartient au groupe B'. 
 \end{itemize} 
 \begin{enumerate} 
 \item Que signifie $\mathbb{P}(\overline{A} \cap B)$. La calculer. 
 \item Que signifie $\mathbb{P}_{B} (B \cap A)$ ? La calculer. 
  \end{enumerate} 
 \end{frame}


\begin{frame}
\vspace{-10mm}
	\frametitle{Correction 1}

 $\mathbb{P}(\overline{B} \cap A)$ signifie que l'on cherche la probabilité d'avoir tiré une personne n'appartenant pas à B mais appartenant à A. $$\mathbb{P}(\overline{B} \cap A)=\dfrac{162}{200 }= \dfrac{81}{100}$$
 \\ $\mathbb{P}_{A} (B \cap A)$ signifie que l'on cherche la probabilité d'avoir tiré une personne appartenant à B et A parmi les personnes appartenant à A. $$\mathbb{P}_{A} (B \cap A)=\dfrac{18}{180 }= \dfrac{1}{10}$$ 
 
\end{frame}


\begin{frame}
\vspace{-10mm}
	\frametitle{Correction 2}

 $\mathbb{P}(\overline{A} \cap B)$ signifie que l'on cherche la probabilité d'avoir tiré une personne  n'appartenant pas à A mais appartenant à B. $$\mathbb{P}(\overline{A} \cap B)=\dfrac{252}{900 }= \dfrac{7}{25}$$
 \\ $\mathbb{P}_{B} (B \cap A)$ signifie que l'on cherche la probabilité d'avoir tiré une personne appartenant à B et A parmi les personnes appartenant à B. $$\mathbb{P}_{B} (B \cap A) =\dfrac{108}{360 }= \dfrac{3}{10}$$
\end{frame}


\begin{frame}
\vspace{-10mm}
	\frametitle{Correction 3}

 $\mathbb{P}(\overline{A} \cap \overline{B})$ signifie que l'on cherche la probabilité d'avoir tiré une personne n'appartenant à aucun des deux groupes. On a $$\mathbb{P}(\overline{A} \cap \overline{B})=\dfrac{360}{900 }= \dfrac{2}{5}$$
 \\ $\mathbb{P}_{\overline{A}} (\overline{A} \cap B)$ signifie que l'on cherche la probabilité d'avoir tiré une personne n'appartenant pas à A mais appartenant à B parmi les personnes n'appartenant pas à A. $$\mathbb{P}_{\overline{A}} (\overline{A} \cap B)=\dfrac{90}{450 }= \dfrac{1}{5}$$
\end{frame}


\begin{frame}
\vspace{-10mm}
	\frametitle{Correction 4}

 $\mathbb{P}(\overline{A})$ signifie que l'on cherche la probabilité d'avoir tiré une personne n'appartenant pas à A.  $$\mathbb{P}(\overline{A})=\dfrac{990}{1100 }= \dfrac{9}{10}$$
 \\ $\mathbb{P}_{\overline{B}} (\overline{B} \cap A)$ signifie que l'on cherche la probabilité d'avoir tiré une personne n'appartenant pas à B mais appartenant à A parmi les personnes n'appartenant pas à B. $$\mathbb{P}_{\overline{B}} (\overline{B} \cap A)=\dfrac{77}{770 }= \dfrac{1}{10}$$
\end{frame}


\begin{frame}
\vspace{-10mm}
	\frametitle{Correction 5}

 $\mathbb{P}(\overline{A} \cap B)$ signifie que l'on cherche la probabilité d'avoir tiré une personne  n'appartenant pas à A mais appartenant à B. $$\mathbb{P}(\overline{A} \cap B)=\dfrac{864}{1800 }= \dfrac{12}{25}$$
 \\ $\mathbb{P}_{B} (B \cap A)$ signifie que l'on cherche la probabilité d'avoir tiré une personne appartenant à B et A parmi les personnes appartenant à B. $$\mathbb{P}_{B} (B \cap A) =\dfrac{576}{1440 }= \dfrac{2}{5}$$
\end{frame}




\end{document}