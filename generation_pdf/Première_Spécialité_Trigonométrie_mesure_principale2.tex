\documentclass[15pt, mathserif]{beamer}

\usepackage[french]{babel}
\usepackage[T1]{fontenc}
\usepackage[utf8]{inputenc}
%\usepackage{esvect}
\usepackage{bm}
\usepackage{eurosym}
\usepackage{tikz}
\usepackage{pgf,tikz,pgfplots}
\pgfplotsset{compat=1.15}
\usepackage{mathrsfs}
\usetikzlibrary{arrows}
\usetikzlibrary{arrows.meta}

\usetikzlibrary{mindmap}
\usepackage{multicol}
\usepackage[tikz]{bclogo}
\usepackage{tkz-tab}
\usepackage{amsmath, tabu}
\usepackage{esvect} %\vv{AB} pour le vecteur AB

\DeclareMathOperator{\e}{e}

%% Tableau

\usepackage{makecell}
\setcellgapes{1pt}
\makegapedcells
\newcolumntype{R}[1]{>{\raggedleft\arraybackslash }b{#1}}
\newcolumntype{L}[1]{>{\raggedright\arraybackslash }b{#1}}
\newcolumntype{C}[1]{>{\centering\arraybackslash }b{#1}}


%pour avoir des parenthèses rondes dans le package fourier
\DeclareSymbolFont{cmoperators}   {OT1}{cmr} {m}{n}
\DeclareSymbolFont{cmlargesymbols}{OMX}{cmex}{m}{n}

\usefonttheme{professionalfonts} %permet d'enlever un bug avec fourier
\usepackage{fourier}
\DeclareMathDelimiter{(}{\mathopen} {cmoperators}{"28}{cmlargesymbols}{"00}
\DeclareMathDelimiter{)}{\mathclose}{cmoperators}{"29}{cmlargesymbols}{"01}

%Graphiques 

\usepackage{pgf,tikz,pgfplots}
\pgfplotsset{compat=1.15}
\usepackage{mathrsfs}
\usetikzlibrary{arrows}
\usetikzlibrary{mindmap}

%ensembles de nbres

\newcommand{\R}{\mathbb{R}}			%permet d'écrire le R "ensemble des réels"'
\newcommand{\N}{\mathbb{N}}			%permet d'écrire le N "ensemble des entiers naturels"
\newcommand{\Z}{\mathbb{Z}}			%permet d'écrire le Z "ensemble des entiers relatifs"
\newcommand{\Prem}{\mathbb{P}}	%permet d'écrire le P "ensemble des nombres premiers" (qui n'a pas marché avec le \P car il existe déjà)
\newcommand{\D}{\mathbb{D}}
\newcommand{\Df}{\mathcal{D}_f}
\newcommand{\Cf}{\mathcal{C}_f}

\newcommand{\Q}{\mathbb{Q}}


\newcommand{\st}[1]{$(#1_n)_{n \in \N}$}

\usetheme{Madrid}
\useoutertheme{miniframes} % Alternatively: miniframes, infolines, split
\useinnertheme{circles}
\definecolor{UBCblue}{rgb}{0.1, 0.25, 0.4} % UBC Blue (primary)
\definecolor{bordeaux}{RGB}{128,0,0}
\usecolortheme[named=UBCblue]{structure}

\usepackage{color} % J'aime bien définir mes couleurs
\definecolor{propcolor}{rgb}{0, 0.5, 1}
\definecolor{thcolor}{rgb}{0.6, 0.07, 0.07}
\colorlet{louis}{blue!70!green!60!white}
\colorlet{sakura}{pink!40!red}

\title{Activités Mentales}
\date{24 Août 2023}

\newcommand{\vco}[2]{\begin{pmatrix} #1 \\ #2 \end{pmatrix}} %Coordonnées de vecteur
\newenvironment{eq}{\begin{cases}\begin{tabu}{ccccc}}{\end{tabu}\end{cases}}
\newenvironment{eql}{\begin{cases}\begin{tabu}{cccccl}}{\end{tabu}\end{cases}}
\newenvironment{eqrl}{\begin{cases}\begin{tabu}{rl}}{\end{tabu}\end{cases}}

\newenvironment{Eq}{\begin{center}\begin{tabular}{rrcl}}{\end{tabular}\end{center}}
\newcommand{\ligneq}[2]{$\Longleftrightarrow$ & $#1$ & $=$ & $#2$ \\}
\newcommand{\Ligneq}[2]{ & $#1$ & $=$ & $#2$ \\}

\newenvironment{RPN}{\begin{center}\begin{tabular}{rrclcrcl}}{\end{tabular}\end{center}}
\newcommand{\Lignerpn}[4]{ & $#1$ & $=$ & $#2$ & ou & $#3$ & $=$ & $#4$ \\}
\newcommand{\lignerpn}[4]{$\Longleftrightarrow$ & $#1$ & $=$ & $#2$ & ou & $#3$ & $=$ & $#4$ \\}

\newenvironment{TRPN}{\begin{center}\begin{tabular}{rrclcrclcrcl}}{\end{tabular}\end{center}}
\newcommand{\Lignetrpn}[6]{ & $#1$ & $=$ & $#2$ & ou & $#3$ & $=$ & $#4$ & ou & $#5$ & $=$ & $#6$ \\}
\newcommand{\lignetrpn}[6]{$\Longleftrightarrow$ & $#1$ & $=$ & $#2$ & ou & $#3$ & $=$ & $#4$ & ou & $#5$ & $=$ & $#6$ \\}
\begin{document}

\begin{frame}
    \titlepage
\end{frame}

\begin{frame} 
	\frametitle{Question 1}
Déterminer la mesure principale de l'angle $\dfrac{116\pi}{5}$ puis le placer sur le cercle trigonométrique.\end{frame}


\begin{frame} 
	\frametitle{Question 2}
Déterminer la mesure principale de l'angle $\dfrac{48\pi}{7}$ puis le placer sur le cercle trigonométrique.\end{frame}


\begin{frame} 
	\frametitle{Question 3}
Déterminer la mesure principale de l'angle $\dfrac{-117\pi}{11}$ puis le placer sur le cercle trigonométrique.\end{frame}


\begin{frame} 
	\frametitle{Question 4}
Déterminer la mesure principale de l'angle $\dfrac{-150\pi}{7}$ puis le placer sur le cercle trigonométrique.\end{frame}


\begin{frame} 
	\frametitle{Question 5}
Déterminer la mesure principale de l'angle $\dfrac{11\pi}{3}$ puis le placer sur le cercle trigonométrique.\end{frame}


\begin{frame}
\vspace{-10mm}
	\frametitle{Correction 1}
\begin{minipage}{0.45 \linewidth} 
 \vspace*{1cm} 
 Méthode 'M. Herr'
	\begin{align*}
		\dfrac{116\pi}{5} &= \dfrac{116}{5}\times \dfrac{2\pi}{2} \\
		&=\dfrac{116}{10} \times 2 \pi\\
		&=\dfrac{(10\times 11+6) \times 2 \pi}{10}\\
		&=\dfrac{10\times 11 \times 2 \pi}{10}+\dfrac{6\times 2\pi}{10}\\
		&=11\times 2\pi+\dfrac{6\pi}{5}
	\end{align*}
\end{minipage}
\hfil
\begin{minipage}{0.5 \linewidth}
	\begin{tikzpicture}[scale = 0.65]
		\draw[thick] (0,0) circle (2);
		\draw[-{Straight Barb[length = 0.5mm]}] (-2.25,0) -- (2.25, 0);
		\draw[-{Straight Barb[length = 0.5mm]}] (0,-2.25) -- (0, 2.25);
		\begin{scope}[rotate = 72.0]
	\draw[dotted] (0,0) -- (2,0);
	\draw[thick] (1.9, 0) -- (2.1,0);
	\end{scope}

\begin{scope}[rotate = 108.0]
	\draw[dotted] (0,0) -- (2,0);
	\draw[thick] (1.9, 0) -- (2.1,0);
	\end{scope}

\begin{scope}[rotate = 144.0]
	\draw[dotted] (0,0) -- (2,0);
	\draw[thick] (1.9, 0) -- (2.1,0);
	\end{scope}

\begin{scope}[rotate = 252.0]
	\draw[dotted] (0,0) -- (2,0);
	\draw[thick] (1.9, 0) -- (2.1,0);
	\end{scope}

\begin{scope}[rotate = 288.0]
	\draw[dotted] (0,0) -- (2,0);
	\draw[thick] (1.9, 0) -- (2.1,0);
	\end{scope}

\begin{scope}[rotate = 324.0]
	\draw[dotted] (0,0) -- (2,0);
	\draw[thick] (1.9, 0) -- (2.1,0);
	\end{scope}

\begin{scope}[rotate = -144.0]
	\draw[ bordeaux, thick] (0,0) -- (2,0);
	\draw[bordeaux, thick] (1.9, 0) -- (2.1,0);
	\draw[bordeaux] (2.3, 0) node [below] {\small{$\dfrac{-4\pi}{5}$} };
\end{scope}

\begin{scope}[rotate = 36.0]
	\draw[thick, dotted, louis] (0,0) -- (2,0);
	\draw[thick, louis] (1.9, 0) -- (2.1,0) node[above right] {\small{$ \dfrac{\pi}{5}$} };
\end{scope}

\end{tikzpicture}
\end{minipage}

Or $\dfrac{6\pi}{5}>\pi$, on fait un tour de moins en retirant $2\pi$: $\dfrac{6\pi}{5}-2\pi = \dfrac{-4\pi}{5}$.

Comme $-\pi <-\dfrac{4\pi}{5}\leq \pi$, la mesure principale de $\dfrac{116\pi}{5}$ est $\dfrac{-4\pi}{5}$.
 \end{frame} 
   \begin{frame} 
 \begin{minipage}{0.45 \linewidth}
 Méthode 'Mme Chartier' : 
 
  \bigskip 
 
 $\dfrac{116\pi}{5\times 2\pi} = \dfrac{116}{10}\simeq 12$ 
 
 \medskip 
 
 Or $116=10\times 12-4$ 
 
 Ainsi, \begin{align*} 
 \dfrac{116\pi}{5} &=(10\times 12-4)\dfrac{\pi}{5} \\ 
 	 	 &=10\times 12\times \dfrac{\pi}{5}-4\times \dfrac{\pi}{5} \\ 
 	 	 &= 12\times 2\pi -\dfrac{4\pi}{5}
 \end{align*} 
 \end{minipage}\hfil \begin{minipage}{0.5 \linewidth}
	\begin{tikzpicture}[scale = 0.65]
		\draw[thick] (0,0) circle (2);
		\draw[-{Straight Barb[length = 0.5mm]}] (-2.25,0) -- (2.25, 0);
		\draw[-{Straight Barb[length = 0.5mm]}] (0,-2.25) -- (0, 2.25);
		\begin{scope}[rotate = 72.0]
	\draw[dotted] (0,0) -- (2,0);
	\draw[thick] (1.9, 0) -- (2.1,0);
	\end{scope}

\begin{scope}[rotate = 108.0]
	\draw[dotted] (0,0) -- (2,0);
	\draw[thick] (1.9, 0) -- (2.1,0);
	\end{scope}

\begin{scope}[rotate = 144.0]
	\draw[dotted] (0,0) -- (2,0);
	\draw[thick] (1.9, 0) -- (2.1,0);
	\end{scope}

\begin{scope}[rotate = 252.0]
	\draw[dotted] (0,0) -- (2,0);
	\draw[thick] (1.9, 0) -- (2.1,0);
	\end{scope}

\begin{scope}[rotate = 288.0]
	\draw[dotted] (0,0) -- (2,0);
	\draw[thick] (1.9, 0) -- (2.1,0);
	\end{scope}

\begin{scope}[rotate = 324.0]
	\draw[dotted] (0,0) -- (2,0);
	\draw[thick] (1.9, 0) -- (2.1,0);
	\end{scope}

\begin{scope}[rotate = -144.0]
	\draw[ bordeaux, thick] (0,0) -- (2,0);
	\draw[bordeaux, thick] (1.9, 0) -- (2.1,0);
	\draw[bordeaux] (2.3, 0) node [below] {\small{$\dfrac{-4\pi}{5}$} };
\end{scope}

\begin{scope}[rotate = 36.0]
	\draw[thick, dotted, louis] (0,0) -- (2,0);
	\draw[thick, louis] (1.9, 0) -- (2.1,0) node[above right] {\small{$ \dfrac{\pi}{5}$} };
\end{scope}

\end{tikzpicture}
\end{minipage}

 \bigskip 
 
 Comme $-\pi < -\dfrac{4\pi}{5}\leq \pi$, la mesure principale de $\dfrac{116\pi}{5}$ est $-\dfrac{4\pi}{5}$ 
 
 \end{frame}


\begin{frame}
\vspace{-10mm}
	\frametitle{Correction 2}
\begin{minipage}{0.45 \linewidth} 
 \vspace*{1cm} 
 Méthode 'M. Herr'
	\begin{align*}
		\dfrac{48\pi}{7} &= \dfrac{48}{7}\times \dfrac{2\pi}{2} \\
		&=\dfrac{48}{14} \times 2 \pi\\
		&=\dfrac{(14\times 3+6) \times 2 \pi}{14}\\
		&=\dfrac{14\times 3 \times 2 \pi}{14}+\dfrac{6\times 2\pi}{14}\\
		&=3\times 2\pi+\dfrac{6\pi}{7}
	\end{align*}
\end{minipage}
\hfil
\begin{minipage}{0.5 \linewidth}
	\begin{tikzpicture}[scale = 0.65]
		\draw[thick] (0,0) circle (2);
		\draw[-{Straight Barb[length = 0.5mm]}] (-2.25,0) -- (2.25, 0);
		\draw[-{Straight Barb[length = 0.5mm]}] (0,-2.25) -- (0, 2.25);
		\begin{scope}[rotate = 51.42]
	\draw[dotted] (0,0) -- (2,0);
	\draw[thick] (1.9, 0) -- (2.1,0);
	\end{scope}

\begin{scope}[rotate = 77.13]
	\draw[dotted] (0,0) -- (2,0);
	\draw[thick] (1.9, 0) -- (2.1,0);
	\end{scope}

\begin{scope}[rotate = 102.84]
	\draw[dotted] (0,0) -- (2,0);
	\draw[thick] (1.9, 0) -- (2.1,0);
	\end{scope}

\begin{scope}[rotate = 128.55]
	\draw[dotted] (0,0) -- (2,0);
	\draw[thick] (1.9, 0) -- (2.1,0);
	\end{scope}

\begin{scope}[rotate = 179.97]
	\draw[dotted] (0,0) -- (2,0);
	\draw[thick] (1.9, 0) -- (2.1,0);
	\end{scope}

\begin{scope}[rotate = 205.68]
	\draw[dotted] (0,0) -- (2,0);
	\draw[thick] (1.9, 0) -- (2.1,0);
	\end{scope}

\begin{scope}[rotate = 231.39000000000001]
	\draw[dotted] (0,0) -- (2,0);
	\draw[thick] (1.9, 0) -- (2.1,0);
	\end{scope}

\begin{scope}[rotate = 257.1]
	\draw[dotted] (0,0) -- (2,0);
	\draw[thick] (1.9, 0) -- (2.1,0);
	\end{scope}

\begin{scope}[rotate = 282.81]
	\draw[dotted] (0,0) -- (2,0);
	\draw[thick] (1.9, 0) -- (2.1,0);
	\end{scope}

\begin{scope}[rotate = 308.52]
	\draw[dotted] (0,0) -- (2,0);
	\draw[thick] (1.9, 0) -- (2.1,0);
	\end{scope}

\begin{scope}[rotate = 334.23]
	\draw[dotted] (0,0) -- (2,0);
	\draw[thick] (1.9, 0) -- (2.1,0);
	\end{scope}

\begin{scope}[rotate = 154.26]
	\draw[ bordeaux, thick] (0,0) -- (2,0);
	\draw[bordeaux, thick] (1.9, 0) -- (2.1,0);
	\draw[bordeaux] (2.3, 0) node [above] {\small{$\dfrac{6\pi}{7}$} };
\end{scope}

\begin{scope}[rotate = 25.71]
	\draw[thick, dotted, louis] (0,0) -- (2,0);
	\draw[thick, louis] (1.9, 0) -- (2.1,0) node[above right] {\small{$ \dfrac{\pi}{7}$} };
\end{scope}

\end{tikzpicture}
\end{minipage}

Comme $-\pi < \dfrac{6\pi}{7}\leq \pi$, la mesure principale de $\dfrac{48\pi}{7}$ est $\dfrac{-6\pi}{7}$.
 \end{frame} 
   \begin{frame} 
 \begin{minipage}{0.45 \linewidth}
 Méthode 'Mme Chartier' : 
 
  \bigskip 
 
 $\dfrac{48\pi}{7\times 2\pi} = \dfrac{48}{14}\simeq 3$ 
 
 \medskip 
 
 Or $48=14\times 3+6$ 
 
 Ainsi, \begin{align*} 
 \dfrac{48\pi}{7} &=(14\times 3+6)\dfrac{\pi}{7} \\ 
 &=14\times 3\times \dfrac{\pi}{7}+6\times \dfrac{\pi}{7} \\ 
 &= 3\times 2\pi +\dfrac{6\pi}{7}
 \end{align*} 
 \end{minipage}\hfil \begin{minipage}{0.5 \linewidth}
	\begin{tikzpicture}[scale = 0.65]
		\draw[thick] (0,0) circle (2);
		\draw[-{Straight Barb[length = 0.5mm]}] (-2.25,0) -- (2.25, 0);
		\draw[-{Straight Barb[length = 0.5mm]}] (0,-2.25) -- (0, 2.25);
		\begin{scope}[rotate = 51.42]
	\draw[dotted] (0,0) -- (2,0);
	\draw[thick] (1.9, 0) -- (2.1,0);
	\end{scope}

\begin{scope}[rotate = 77.13]
	\draw[dotted] (0,0) -- (2,0);
	\draw[thick] (1.9, 0) -- (2.1,0);
	\end{scope}

\begin{scope}[rotate = 102.84]
	\draw[dotted] (0,0) -- (2,0);
	\draw[thick] (1.9, 0) -- (2.1,0);
	\end{scope}

\begin{scope}[rotate = 128.55]
	\draw[dotted] (0,0) -- (2,0);
	\draw[thick] (1.9, 0) -- (2.1,0);
	\end{scope}

\begin{scope}[rotate = 179.97]
	\draw[dotted] (0,0) -- (2,0);
	\draw[thick] (1.9, 0) -- (2.1,0);
	\end{scope}

\begin{scope}[rotate = 205.68]
	\draw[dotted] (0,0) -- (2,0);
	\draw[thick] (1.9, 0) -- (2.1,0);
	\end{scope}

\begin{scope}[rotate = 231.39000000000001]
	\draw[dotted] (0,0) -- (2,0);
	\draw[thick] (1.9, 0) -- (2.1,0);
	\end{scope}

\begin{scope}[rotate = 257.1]
	\draw[dotted] (0,0) -- (2,0);
	\draw[thick] (1.9, 0) -- (2.1,0);
	\end{scope}

\begin{scope}[rotate = 282.81]
	\draw[dotted] (0,0) -- (2,0);
	\draw[thick] (1.9, 0) -- (2.1,0);
	\end{scope}

\begin{scope}[rotate = 308.52]
	\draw[dotted] (0,0) -- (2,0);
	\draw[thick] (1.9, 0) -- (2.1,0);
	\end{scope}

\begin{scope}[rotate = 334.23]
	\draw[dotted] (0,0) -- (2,0);
	\draw[thick] (1.9, 0) -- (2.1,0);
	\end{scope}

\begin{scope}[rotate = 154.26]
	\draw[ bordeaux, thick] (0,0) -- (2,0);
	\draw[bordeaux, thick] (1.9, 0) -- (2.1,0);
	\draw[bordeaux] (2.3, 0) node [above] {\small{$\dfrac{6\pi}{7}$} };
\end{scope}

\begin{scope}[rotate = 25.71]
	\draw[thick, dotted, louis] (0,0) -- (2,0);
	\draw[thick, louis] (1.9, 0) -- (2.1,0) node[above right] {\small{$ \dfrac{\pi}{7}$} };
\end{scope}

\end{tikzpicture}
\end{minipage}

 \bigskip 
 
 Comme $-\pi < \dfrac{6\pi}{7}\leq \pi$, la mesure principale de $\dfrac{48\pi}{7}$ est $\dfrac{6\pi}{7}$ 
 
 \end{frame}


\begin{frame}
\vspace{-10mm}
	\frametitle{Correction 3}
\begin{minipage}{0.45 \linewidth} 
 \vspace*{1cm} 
 Méthode 'M. Herr'
	\begin{align*}
		\dfrac{-117\pi}{11} &= \dfrac{-117}{11}\times \dfrac{2\pi}{2} \\
		&=\dfrac{-117}{22} \times 2 \pi\\
		&=\dfrac{-(22\times 5+7) \times 2 \pi}{22}\\
		&=-\dfrac{22\times 5 \times 2 \pi}{22}-\dfrac{7\times 2\pi}{22}\\
		&=-5\times 2\pi-\dfrac{7\pi}{11}
	\end{align*}
\end{minipage}
\hfil
\begin{minipage}{0.5 \linewidth}
	\begin{tikzpicture}[scale = 0.65]
		\draw[thick] (0,0) circle (2);
		\draw[-{Straight Barb[length = 0.5mm]}] (-2.25,0) -- (2.25, 0);
		\draw[-{Straight Barb[length = 0.5mm]}] (0,-2.25) -- (0, 2.25);
		\begin{scope}[rotate = 32.72]
	\draw[dotted] (0,0) -- (2,0);
	\draw[thick] (1.9, 0) -- (2.1,0);
	\end{scope}

\begin{scope}[rotate = 49.08]
	\draw[dotted] (0,0) -- (2,0);
	\draw[thick] (1.9, 0) -- (2.1,0);
	\end{scope}

\begin{scope}[rotate = 65.44]
	\draw[dotted] (0,0) -- (2,0);
	\draw[thick] (1.9, 0) -- (2.1,0);
	\end{scope}

\begin{scope}[rotate = 81.8]
	\draw[dotted] (0,0) -- (2,0);
	\draw[thick] (1.9, 0) -- (2.1,0);
	\end{scope}

\begin{scope}[rotate = 98.16]
	\draw[dotted] (0,0) -- (2,0);
	\draw[thick] (1.9, 0) -- (2.1,0);
	\end{scope}

\begin{scope}[rotate = 114.52]
	\draw[dotted] (0,0) -- (2,0);
	\draw[thick] (1.9, 0) -- (2.1,0);
	\end{scope}

\begin{scope}[rotate = 130.88]
	\draw[dotted] (0,0) -- (2,0);
	\draw[thick] (1.9, 0) -- (2.1,0);
	\end{scope}

\begin{scope}[rotate = 147.24]
	\draw[dotted] (0,0) -- (2,0);
	\draw[thick] (1.9, 0) -- (2.1,0);
	\end{scope}

\begin{scope}[rotate = 163.6]
	\draw[dotted] (0,0) -- (2,0);
	\draw[thick] (1.9, 0) -- (2.1,0);
	\end{scope}

\begin{scope}[rotate = 179.95999999999998]
	\draw[dotted] (0,0) -- (2,0);
	\draw[thick] (1.9, 0) -- (2.1,0);
	\end{scope}

\begin{scope}[rotate = 196.32]
	\draw[dotted] (0,0) -- (2,0);
	\draw[thick] (1.9, 0) -- (2.1,0);
	\end{scope}

\begin{scope}[rotate = 212.68]
	\draw[dotted] (0,0) -- (2,0);
	\draw[thick] (1.9, 0) -- (2.1,0);
	\end{scope}

\begin{scope}[rotate = 229.04]
	\draw[dotted] (0,0) -- (2,0);
	\draw[thick] (1.9, 0) -- (2.1,0);
	\end{scope}

\begin{scope}[rotate = 261.76]
	\draw[dotted] (0,0) -- (2,0);
	\draw[thick] (1.9, 0) -- (2.1,0);
	\end{scope}

\begin{scope}[rotate = 278.12]
	\draw[dotted] (0,0) -- (2,0);
	\draw[thick] (1.9, 0) -- (2.1,0);
	\end{scope}

\begin{scope}[rotate = 294.48]
	\draw[dotted] (0,0) -- (2,0);
	\draw[thick] (1.9, 0) -- (2.1,0);
	\end{scope}

\begin{scope}[rotate = 310.84]
	\draw[dotted] (0,0) -- (2,0);
	\draw[thick] (1.9, 0) -- (2.1,0);
	\end{scope}

\begin{scope}[rotate = 327.2]
	\draw[dotted] (0,0) -- (2,0);
	\draw[thick] (1.9, 0) -- (2.1,0);
	\end{scope}

\begin{scope}[rotate = 343.56]
	\draw[dotted] (0,0) -- (2,0);
	\draw[thick] (1.9, 0) -- (2.1,0);
	\end{scope}

\begin{scope}[rotate = -114.52]
	\draw[ bordeaux, thick] (0,0) -- (2,0);
	\draw[bordeaux, thick] (1.9, 0) -- (2.1,0);
	\draw[bordeaux] (2.3, 0) node [below] {\small{$\dfrac{-7\pi}{11}$} };
\end{scope}

\begin{scope}[rotate = 16.36]
	\draw[thick, dotted, louis] (0,0) -- (2,0);
	\draw[thick, louis] (1.9, 0) -- (2.1,0) node[above right] {\small{$ \dfrac{\pi}{11}$} };
\end{scope}

\end{tikzpicture}
\end{minipage}

Comme $-\pi < -\dfrac{7\pi}{11}\leq \pi$, la mesure principale de $\dfrac{-117\pi}{11}$ est $\dfrac{-7\pi}{11}$.
 \end{frame} 
   \begin{frame} 
 \begin{minipage}{0.45 \linewidth}
 Méthode 'Mme Chartier' : 
 
  \bigskip 
 
 $\dfrac{-117\pi}{11\times 2\pi} = \dfrac{-117}{22}\simeq -5$ 
 
 \medskip 
 
 Or $-117=22\times \left(-5\right)-7$ 
 
 Ainsi, \begin{align*} 
 \dfrac{-117\pi}{11} &=(22\times \left(-5\right)-7)\dfrac{\pi}{11} \\ 
 	 	 &=22\times \left(-5\right)\times \dfrac{\pi}{11}-7\times \dfrac{\pi}{11} \\ 
 	 	 &= -5\times 2\pi -\dfrac{7\pi}{11}
 \end{align*} 
 \end{minipage}\hfil \begin{minipage}{0.5 \linewidth}
	\begin{tikzpicture}[scale = 0.65]
		\draw[thick] (0,0) circle (2);
		\draw[-{Straight Barb[length = 0.5mm]}] (-2.25,0) -- (2.25, 0);
		\draw[-{Straight Barb[length = 0.5mm]}] (0,-2.25) -- (0, 2.25);
		\begin{scope}[rotate = 32.72]
	\draw[dotted] (0,0) -- (2,0);
	\draw[thick] (1.9, 0) -- (2.1,0);
	\end{scope}

\begin{scope}[rotate = 49.08]
	\draw[dotted] (0,0) -- (2,0);
	\draw[thick] (1.9, 0) -- (2.1,0);
	\end{scope}

\begin{scope}[rotate = 65.44]
	\draw[dotted] (0,0) -- (2,0);
	\draw[thick] (1.9, 0) -- (2.1,0);
	\end{scope}

\begin{scope}[rotate = 81.8]
	\draw[dotted] (0,0) -- (2,0);
	\draw[thick] (1.9, 0) -- (2.1,0);
	\end{scope}

\begin{scope}[rotate = 98.16]
	\draw[dotted] (0,0) -- (2,0);
	\draw[thick] (1.9, 0) -- (2.1,0);
	\end{scope}

\begin{scope}[rotate = 114.52]
	\draw[dotted] (0,0) -- (2,0);
	\draw[thick] (1.9, 0) -- (2.1,0);
	\end{scope}

\begin{scope}[rotate = 130.88]
	\draw[dotted] (0,0) -- (2,0);
	\draw[thick] (1.9, 0) -- (2.1,0);
	\end{scope}

\begin{scope}[rotate = 147.24]
	\draw[dotted] (0,0) -- (2,0);
	\draw[thick] (1.9, 0) -- (2.1,0);
	\end{scope}

\begin{scope}[rotate = 163.6]
	\draw[dotted] (0,0) -- (2,0);
	\draw[thick] (1.9, 0) -- (2.1,0);
	\end{scope}

\begin{scope}[rotate = 179.95999999999998]
	\draw[dotted] (0,0) -- (2,0);
	\draw[thick] (1.9, 0) -- (2.1,0);
	\end{scope}

\begin{scope}[rotate = 196.32]
	\draw[dotted] (0,0) -- (2,0);
	\draw[thick] (1.9, 0) -- (2.1,0);
	\end{scope}

\begin{scope}[rotate = 212.68]
	\draw[dotted] (0,0) -- (2,0);
	\draw[thick] (1.9, 0) -- (2.1,0);
	\end{scope}

\begin{scope}[rotate = 229.04]
	\draw[dotted] (0,0) -- (2,0);
	\draw[thick] (1.9, 0) -- (2.1,0);
	\end{scope}

\begin{scope}[rotate = 261.76]
	\draw[dotted] (0,0) -- (2,0);
	\draw[thick] (1.9, 0) -- (2.1,0);
	\end{scope}

\begin{scope}[rotate = 278.12]
	\draw[dotted] (0,0) -- (2,0);
	\draw[thick] (1.9, 0) -- (2.1,0);
	\end{scope}

\begin{scope}[rotate = 294.48]
	\draw[dotted] (0,0) -- (2,0);
	\draw[thick] (1.9, 0) -- (2.1,0);
	\end{scope}

\begin{scope}[rotate = 310.84]
	\draw[dotted] (0,0) -- (2,0);
	\draw[thick] (1.9, 0) -- (2.1,0);
	\end{scope}

\begin{scope}[rotate = 327.2]
	\draw[dotted] (0,0) -- (2,0);
	\draw[thick] (1.9, 0) -- (2.1,0);
	\end{scope}

\begin{scope}[rotate = 343.56]
	\draw[dotted] (0,0) -- (2,0);
	\draw[thick] (1.9, 0) -- (2.1,0);
	\end{scope}

\begin{scope}[rotate = -114.52]
	\draw[ bordeaux, thick] (0,0) -- (2,0);
	\draw[bordeaux, thick] (1.9, 0) -- (2.1,0);
	\draw[bordeaux] (2.3, 0) node [below] {\small{$\dfrac{-7\pi}{11}$} };
\end{scope}

\begin{scope}[rotate = 16.36]
	\draw[thick, dotted, louis] (0,0) -- (2,0);
	\draw[thick, louis] (1.9, 0) -- (2.1,0) node[above right] {\small{$ \dfrac{\pi}{11}$} };
\end{scope}

\end{tikzpicture}
\end{minipage}

 \bigskip 
 
 Comme $-\pi < -\dfrac{7\pi}{11}\leq \pi$, la mesure principale de $\dfrac{-117\pi}{11}$ est $-\dfrac{7\pi}{11}$ 
 
 \end{frame}


\begin{frame}
\vspace{-10mm}
	\frametitle{Correction 4}
\begin{minipage}{0.45 \linewidth} 
 \vspace*{1cm} 
 Méthode 'M. Herr'
	\begin{align*}
		\dfrac{-150\pi}{7} &= \dfrac{-150}{7}\times \dfrac{2\pi}{2} \\
		&=\dfrac{-150}{14} \times 2 \pi\\
		&=\dfrac{-(14\times 10+10) \times 2 \pi}{14}\\
		&=-\dfrac{14\times 10 \times 2 \pi}{14}-\dfrac{10\times 2\pi}{14}\\
		&=-10\times 2\pi-\dfrac{10\pi}{7}
	\end{align*}
\end{minipage}
\hfil
\begin{minipage}{0.5 \linewidth}
	\begin{tikzpicture}[scale = 0.65]
		\draw[thick] (0,0) circle (2);
		\draw[-{Straight Barb[length = 0.5mm]}] (-2.25,0) -- (2.25, 0);
		\draw[-{Straight Barb[length = 0.5mm]}] (0,-2.25) -- (0, 2.25);
		\begin{scope}[rotate = 51.42]
	\draw[dotted] (0,0) -- (2,0);
	\draw[thick] (1.9, 0) -- (2.1,0);
	\end{scope}

\begin{scope}[rotate = 77.13]
	\draw[dotted] (0,0) -- (2,0);
	\draw[thick] (1.9, 0) -- (2.1,0);
	\end{scope}

\begin{scope}[rotate = 128.55]
	\draw[dotted] (0,0) -- (2,0);
	\draw[thick] (1.9, 0) -- (2.1,0);
	\end{scope}

\begin{scope}[rotate = 154.26]
	\draw[dotted] (0,0) -- (2,0);
	\draw[thick] (1.9, 0) -- (2.1,0);
	\end{scope}

\begin{scope}[rotate = 179.97]
	\draw[dotted] (0,0) -- (2,0);
	\draw[thick] (1.9, 0) -- (2.1,0);
	\end{scope}

\begin{scope}[rotate = 205.68]
	\draw[dotted] (0,0) -- (2,0);
	\draw[thick] (1.9, 0) -- (2.1,0);
	\end{scope}

\begin{scope}[rotate = 231.39000000000001]
	\draw[dotted] (0,0) -- (2,0);
	\draw[thick] (1.9, 0) -- (2.1,0);
	\end{scope}

\begin{scope}[rotate = 257.1]
	\draw[dotted] (0,0) -- (2,0);
	\draw[thick] (1.9, 0) -- (2.1,0);
	\end{scope}

\begin{scope}[rotate = 282.81]
	\draw[dotted] (0,0) -- (2,0);
	\draw[thick] (1.9, 0) -- (2.1,0);
	\end{scope}

\begin{scope}[rotate = 308.52]
	\draw[dotted] (0,0) -- (2,0);
	\draw[thick] (1.9, 0) -- (2.1,0);
	\end{scope}

\begin{scope}[rotate = 334.23]
	\draw[dotted] (0,0) -- (2,0);
	\draw[thick] (1.9, 0) -- (2.1,0);
	\end{scope}

\begin{scope}[rotate = 102.84]
	\draw[ bordeaux, thick] (0,0) -- (2,0);
	\draw[bordeaux, thick] (1.9, 0) -- (2.1,0);
	\draw[bordeaux] (2.3, 0) node [above] {\small{$\dfrac{4\pi}{7}$} };
\end{scope}

\begin{scope}[rotate = 25.71]
	\draw[thick, dotted, louis] (0,0) -- (2,0);
	\draw[thick, louis] (1.9, 0) -- (2.1,0) node[above right] {\small{$ \dfrac{\pi}{7}$} };
\end{scope}

\end{tikzpicture}
\end{minipage}

Or $-\dfrac{10\pi}{7}\leq-\pi$, on fait un tour de plus en rajoutant $2\pi$: $-\dfrac{10\pi}{7}+2\pi = \dfrac{4\pi}{7}$.

Comme $-\pi <\dfrac{4\pi}{7}\leq \pi$, la mesure principale de $\dfrac{-150\pi}{7}$ est $\dfrac{4\pi}{7}$.
 \end{frame} 
   \begin{frame} 
 \begin{minipage}{0.45 \linewidth}
 Méthode 'Mme Chartier' : 
 
  \bigskip 
 
 $\dfrac{-150\pi}{7\times 2\pi} = \dfrac{-150}{14}\simeq -11$ 
 
 \medskip 
 
 Or $-150=14\times \left(-11\right)+4$ 
 
 Ainsi, \begin{align*} 
 \dfrac{-150\pi}{7} &=(14\times \left(-11\right)+4)\dfrac{\pi}{7} \\ 
 &=14\times \left(-11\right)\times \dfrac{\pi}{7}+4\times \dfrac{\pi}{7} \\ 
 &= -11\times 2\pi +\dfrac{4\pi}{7}
 \end{align*} 
 \end{minipage}\hfil \begin{minipage}{0.5 \linewidth}
	\begin{tikzpicture}[scale = 0.65]
		\draw[thick] (0,0) circle (2);
		\draw[-{Straight Barb[length = 0.5mm]}] (-2.25,0) -- (2.25, 0);
		\draw[-{Straight Barb[length = 0.5mm]}] (0,-2.25) -- (0, 2.25);
		\begin{scope}[rotate = 51.42]
	\draw[dotted] (0,0) -- (2,0);
	\draw[thick] (1.9, 0) -- (2.1,0);
	\end{scope}

\begin{scope}[rotate = 77.13]
	\draw[dotted] (0,0) -- (2,0);
	\draw[thick] (1.9, 0) -- (2.1,0);
	\end{scope}

\begin{scope}[rotate = 128.55]
	\draw[dotted] (0,0) -- (2,0);
	\draw[thick] (1.9, 0) -- (2.1,0);
	\end{scope}

\begin{scope}[rotate = 154.26]
	\draw[dotted] (0,0) -- (2,0);
	\draw[thick] (1.9, 0) -- (2.1,0);
	\end{scope}

\begin{scope}[rotate = 179.97]
	\draw[dotted] (0,0) -- (2,0);
	\draw[thick] (1.9, 0) -- (2.1,0);
	\end{scope}

\begin{scope}[rotate = 205.68]
	\draw[dotted] (0,0) -- (2,0);
	\draw[thick] (1.9, 0) -- (2.1,0);
	\end{scope}

\begin{scope}[rotate = 231.39000000000001]
	\draw[dotted] (0,0) -- (2,0);
	\draw[thick] (1.9, 0) -- (2.1,0);
	\end{scope}

\begin{scope}[rotate = 257.1]
	\draw[dotted] (0,0) -- (2,0);
	\draw[thick] (1.9, 0) -- (2.1,0);
	\end{scope}

\begin{scope}[rotate = 282.81]
	\draw[dotted] (0,0) -- (2,0);
	\draw[thick] (1.9, 0) -- (2.1,0);
	\end{scope}

\begin{scope}[rotate = 308.52]
	\draw[dotted] (0,0) -- (2,0);
	\draw[thick] (1.9, 0) -- (2.1,0);
	\end{scope}

\begin{scope}[rotate = 334.23]
	\draw[dotted] (0,0) -- (2,0);
	\draw[thick] (1.9, 0) -- (2.1,0);
	\end{scope}

\begin{scope}[rotate = 102.84]
	\draw[ bordeaux, thick] (0,0) -- (2,0);
	\draw[bordeaux, thick] (1.9, 0) -- (2.1,0);
	\draw[bordeaux] (2.3, 0) node [above] {\small{$\dfrac{4\pi}{7}$} };
\end{scope}

\begin{scope}[rotate = 25.71]
	\draw[thick, dotted, louis] (0,0) -- (2,0);
	\draw[thick, louis] (1.9, 0) -- (2.1,0) node[above right] {\small{$ \dfrac{\pi}{7}$} };
\end{scope}

\end{tikzpicture}
\end{minipage}

 \bigskip 
 
 Comme $-\pi < \dfrac{4\pi}{7}\leq \pi$, la mesure principale de $\dfrac{-150\pi}{7}$ est $\dfrac{4\pi}{7}$ 
 
 \end{frame}


\begin{frame}
\vspace{-10mm}
	\frametitle{Correction 5}
\begin{minipage}{0.45 \linewidth} 
 \vspace*{1cm} 
 Méthode 'M. Herr'
	\begin{align*}
		\dfrac{11\pi}{3} &= \dfrac{11}{3}\times \dfrac{2\pi}{2} \\
		&=\dfrac{11}{6} \times 2 \pi\\
		&=\dfrac{(6\times 1+5) \times 2 \pi}{6}\\
		&=\dfrac{6\times 1 \times 2 \pi}{6}+\dfrac{5\times 2\pi}{6}\\
		&=1\times 2\pi+\dfrac{5\pi}{3}
	\end{align*}
\end{minipage}
\hfil
\begin{minipage}{0.5 \linewidth}
	\begin{tikzpicture}[scale = 0.65]
		\draw[thick] (0,0) circle (2);
		\draw[-{Straight Barb[length = 0.5mm]}] (-2.25,0) -- (2.25, 0);
		\draw[-{Straight Barb[length = 0.5mm]}] (0,-2.25) -- (0, 2.25);
		\begin{scope}[rotate = 120.0]
	\draw[dotted] (0,0) -- (2,0);
	\draw[thick] (1.9, 0) -- (2.1,0);
	\end{scope}

\begin{scope}[rotate = 240.0]
	\draw[dotted] (0,0) -- (2,0);
	\draw[thick] (1.9, 0) -- (2.1,0);
	\end{scope}

\begin{scope}[rotate = -60.0]
	\draw[ bordeaux, thick] (0,0) -- (2,0);
	\draw[bordeaux, thick] (1.9, 0) -- (2.1,0);
	\draw[bordeaux] (2.3, 0) node [below] {\small{$\dfrac{-\pi}{3}$} };
\end{scope}

\begin{scope}[rotate = 60.0]
	\draw[thick, dotted, louis] (0,0) -- (2,0);
	\draw[thick, louis] (1.9, 0) -- (2.1,0) node[above right] {\small{$ \dfrac{\pi}{3}$} };
\end{scope}

\end{tikzpicture}
\end{minipage}

Or $\dfrac{5\pi}{3}>\pi$, on fait un tour de moins en retirant $2\pi$: $\dfrac{5\pi}{3}-2\pi = \dfrac{-\pi}{3}$.

Comme $-\pi <-\dfrac{\pi}{3}\leq \pi$, la mesure principale de $\dfrac{11\pi}{3}$ est $\dfrac{-\pi}{3}$.
 \end{frame} 
   \begin{frame} 
 \begin{minipage}{0.45 \linewidth}
 Méthode 'Mme Chartier' : 
 
  \bigskip 
 
 $\dfrac{11\pi}{3\times 2\pi} = \dfrac{11}{6}\simeq 2$ 
 
 \medskip 
 
 Or $11=6\times 2-1$ 
 
 Ainsi, \begin{align*} 
 \dfrac{11\pi}{3} &=(6\times 2-1)\dfrac{\pi}{3} \\ 
 	 	 &=6\times 2\times \dfrac{\pi}{3}-1\times \dfrac{\pi}{3} \\ 
 	 	 &= 2\times 2\pi -\dfrac{\pi}{3}
 \end{align*} 
 \end{minipage}\hfil \begin{minipage}{0.5 \linewidth}
	\begin{tikzpicture}[scale = 0.65]
		\draw[thick] (0,0) circle (2);
		\draw[-{Straight Barb[length = 0.5mm]}] (-2.25,0) -- (2.25, 0);
		\draw[-{Straight Barb[length = 0.5mm]}] (0,-2.25) -- (0, 2.25);
		\begin{scope}[rotate = 120.0]
	\draw[dotted] (0,0) -- (2,0);
	\draw[thick] (1.9, 0) -- (2.1,0);
	\end{scope}

\begin{scope}[rotate = 240.0]
	\draw[dotted] (0,0) -- (2,0);
	\draw[thick] (1.9, 0) -- (2.1,0);
	\end{scope}

\begin{scope}[rotate = -60.0]
	\draw[ bordeaux, thick] (0,0) -- (2,0);
	\draw[bordeaux, thick] (1.9, 0) -- (2.1,0);
	\draw[bordeaux] (2.3, 0) node [below] {\small{$\dfrac{-\pi}{3}$} };
\end{scope}

\begin{scope}[rotate = 60.0]
	\draw[thick, dotted, louis] (0,0) -- (2,0);
	\draw[thick, louis] (1.9, 0) -- (2.1,0) node[above right] {\small{$ \dfrac{\pi}{3}$} };
\end{scope}

\end{tikzpicture}
\end{minipage}

 \bigskip 
 
 Comme $-\pi < -\dfrac{\pi}{3}\leq \pi$, la mesure principale de $\dfrac{11\pi}{3}$ est $-\dfrac{\pi}{3}$ 
 
 \end{frame}




\end{document}