\documentclass[15pt, mathserif]{beamer}

\usepackage[french]{babel}
\usepackage[T1]{fontenc}
\usepackage[utf8]{inputenc}
%\usepackage{esvect}
\usepackage{bm}
\usepackage{eurosym}
\usepackage{tikz}
\usepackage{pgf,tikz,pgfplots}
\pgfplotsset{compat=1.15}
\usepackage{mathrsfs}
\usetikzlibrary{arrows}
\usetikzlibrary{arrows.meta}

\usetikzlibrary{mindmap}
\usepackage{multicol}
\usepackage[tikz]{bclogo}
\usepackage{tkz-tab}
\usepackage{amsmath, tabu}
\usepackage{esvect} %\vv{AB} pour le vecteur AB

\DeclareMathOperator{\e}{e}

%% Tableau

\usepackage{makecell}
\setcellgapes{1pt}
\makegapedcells
\newcolumntype{R}[1]{>{\raggedleft\arraybackslash }b{#1}}
\newcolumntype{L}[1]{>{\raggedright\arraybackslash }b{#1}}
\newcolumntype{C}[1]{>{\centering\arraybackslash }b{#1}}


%pour avoir des parenthèses rondes dans le package fourier
\DeclareSymbolFont{cmoperators}   {OT1}{cmr} {m}{n}
\DeclareSymbolFont{cmlargesymbols}{OMX}{cmex}{m}{n}

\usefonttheme{professionalfonts} %permet d'enlever un bug avec fourier
\usepackage{fourier}
\DeclareMathDelimiter{(}{\mathopen} {cmoperators}{"28}{cmlargesymbols}{"00}
\DeclareMathDelimiter{)}{\mathclose}{cmoperators}{"29}{cmlargesymbols}{"01}

%Graphiques 

\usepackage{pgf,tikz,pgfplots}
\pgfplotsset{compat=1.15}
\usepackage{mathrsfs}
\usetikzlibrary{arrows}
\usetikzlibrary{mindmap}

%ensembles de nbres

\newcommand{\R}{\mathbb{R}}			%permet d'écrire le R "ensemble des réels"'
\newcommand{\N}{\mathbb{N}}			%permet d'écrire le N "ensemble des entiers naturels"
\newcommand{\Z}{\mathbb{Z}}			%permet d'écrire le Z "ensemble des entiers relatifs"
\newcommand{\Prem}{\mathbb{P}}	%permet d'écrire le P "ensemble des nombres premiers" (qui n'a pas marché avec le \P car il existe déjà)
\newcommand{\D}{\mathbb{D}}
\newcommand{\Df}{\mathcal{D}_f}
\newcommand{\Cf}{\mathcal{C}_f}

\newcommand{\Q}{\mathbb{Q}}


\newcommand{\st}[1]{$(#1_n)_{n \in \N}$}

\usetheme{Madrid}
\useoutertheme{miniframes} % Alternatively: miniframes, infolines, split
\useinnertheme{circles}
\definecolor{UBCblue}{rgb}{0.1, 0.25, 0.4} % UBC Blue (primary)
\definecolor{bordeaux}{RGB}{128,0,0}
\usecolortheme[named=UBCblue]{structure}

\usepackage{color} % J'aime bien définir mes couleurs
\definecolor{propcolor}{rgb}{0, 0.5, 1}
\definecolor{thcolor}{rgb}{0.6, 0.07, 0.07}
\colorlet{louis}{blue!70!green!60!white}
\colorlet{sakura}{pink!40!red}

\title{Activités Mentales}
\date{24 Août 2023}

\newcommand{\vco}[2]{\begin{pmatrix} #1 \\ #2 \end{pmatrix}} %Coordonnées de vecteur
\newenvironment{eq}{\begin{cases}\begin{tabu}{ccccc}}{\end{tabu}\end{cases}}
\newenvironment{eql}{\begin{cases}\begin{tabu}{cccccl}}{\end{tabu}\end{cases}}
\newenvironment{eqrl}{\begin{cases}\begin{tabu}{rl}}{\end{tabu}\end{cases}}

\newenvironment{Eq}{\begin{center}\begin{tabular}{rrcl}}{\end{tabular}\end{center}}
\newcommand{\ligneq}[2]{$\Longleftrightarrow$ & $#1$ & $=$ & $#2$ \\}
\newcommand{\Ligneq}[2]{ & $#1$ & $=$ & $#2$ \\}

\newenvironment{RPN}{\begin{center}\begin{tabular}{rrclcrcl}}{\end{tabular}\end{center}}
\newcommand{\Lignerpn}[4]{ & $#1$ & $=$ & $#2$ & ou & $#3$ & $=$ & $#4$ \\}
\newcommand{\lignerpn}[4]{$\Longleftrightarrow$ & $#1$ & $=$ & $#2$ & ou & $#3$ & $=$ & $#4$ \\}

\newenvironment{TRPN}{\begin{center}\begin{tabular}{rrclcrclcrcl}}{\end{tabular}\end{center}}
\newcommand{\Lignetrpn}[6]{ & $#1$ & $=$ & $#2$ & ou & $#3$ & $=$ & $#4$ & ou & $#5$ & $=$ & $#6$ \\}
\newcommand{\lignetrpn}[6]{$\Longleftrightarrow$ & $#1$ & $=$ & $#2$ & ou & $#3$ & $=$ & $#4$ & ou & $#5$ & $=$ & $#6$ \\}
\begin{document}

\begin{frame}
    \titlepage
\end{frame}

\begin{frame} 
	\frametitle{Question 1}
On dispose d'un jeu de 52 cartes. On pioche successivement 18 cartes avec remise. Les tirages sont indépendants.

\medskip

Quelle la probabilité d'avoir tiré 10 cartes avec un carreau dessiné dessus?\end{frame}


\begin{frame} 
	\frametitle{Question 2}
On s'intéresse à une entreprise de location de trotinettes. D'expérience, 9\% des trotinettes sont  endommagées.

 Un contrôleur décide de tester les produits de l'entreprise il choisit au hasard 20 trotinettes. La grande quantité de trotinettes fait qu'on peut assimiler cette expérience à un tirage avec remise.

\medskip

Quelle est la probabilité que le contrôleur ait en sa possession 19 trotinettes endommagées?\end{frame}


\begin{frame} 
	\frametitle{Question 3}
On dispose d'un jeu de 32 cartes. On pioche successivement 14 cartes avec remise. Les tirages sont indépendants.

\medskip

Quelle la probabilité d'avoir tiré 8 cartes avec un carreau dessiné dessus?\end{frame}


\begin{frame} 
	\frametitle{Question 4}
On dispose d'une urne contenant 40 boules de couleur. Dans cette urne il y a 19 boules Bleues et 21 boules Violettes.

On tire successivement et avec remise $12$ boules.

\medskip

Quelle est la probabilité d'obtenir exactement 9 boules de couleur Violette ?\end{frame}


\begin{frame} 
	\frametitle{Question 5}
On s'intéresse à une entreprise de location de trotinettes. D'expérience, 12\% des trotinettes sont  endommagées.

 Un contrôleur décide de tester les produits de l'entreprise il choisit au hasard 13 trotinettes. La grande quantité de trotinettes fait qu'on peut assimiler cette expérience à un tirage avec remise.

\medskip

Quelle est la probabilité que le contrôleur ait en sa possession 10 trotinettes endommagées?\end{frame}


\begin{frame}
\vspace{-10mm}
	\frametitle{Correction 1}
On répète $18$ fois de façon identique et indépendante l'épreuve de Bernoulli "On tire une carte au hasard" de succès S: "la carte est un carreau"  de probabilité $p =\dfrac{1}{4}$.

Soit $X$ la variable aléatoire qui compte le nombre de succès à l'issue des $18$ répétitions.

 Alors $X \leadsto \mathcal{B}\left(18,\dfrac{1}{4}\right)$.

Ainsi on cherche $\mathbb{P}(X = 10) = \begin{pmatrix}18\\10\end{pmatrix}\times\left(\dfrac{1}{4}\right)^{10}\times\left(\dfrac{3}{4}\right)^{18-10}$.\end{frame}


\begin{frame}
\vspace{-10mm}
	\frametitle{Correction 2}
On répète $20$ fois de façon identique et indépendante l'épreuve de Bernoulli "On contrôle une trotinette" de succès S: "La trotinette est endommagée" de probabilité $p =\dfrac{9}{100}$.

Soit $X$ la variable aléatoire qui compte le nombre de succès à l'issue des $20$ répétitions.

 Alors $X \leadsto \mathcal{B}\left(20,\dfrac{9}{100}\right)$.

Ainsi on cherche $\mathbb{P}(X = 19) = \begin{pmatrix}20\\19\end{pmatrix}\times\left(\dfrac{9}{100}\right)^{19}\times\left(\dfrac{91}{100}\right)^{20-19}$.\end{frame}


\begin{frame}
\vspace{-10mm}
	\frametitle{Correction 3}
On répète $14$ fois de façon identique et indépendante l'épreuve de Bernoulli "On tire une carte au hasard" de succès S: "la carte est un carreau"  de probabilité $p =\dfrac{1}{4}$.

Soit $X$ la variable aléatoire qui compte le nombre de succès à l'issue des $14$ répétitions.

 Alors $X \leadsto \mathcal{B}\left(14,\dfrac{1}{4}\right)$.

Ainsi on cherche $\mathbb{P}(X = 8) = \begin{pmatrix}14\\8\end{pmatrix}\times\left(\dfrac{1}{4}\right)^{8}\times\left(\dfrac{3}{4}\right)^{14-8}$.\end{frame}


\begin{frame}
\vspace{-10mm}
	\frametitle{Correction 4}
On répète $12$ fois de façon identique et indépendante l'épreuve de Bernoulli "On tire une boule de l'urne" de succès S: "la boule est Violette"  de probabilité $p =\dfrac{21}{40}$.

Soit $X$ la variable aléatoire qui compte le nombre de succès à l'issue des $12$ répétitions.

 Alors $X \leadsto \mathcal{B}\left(12,\dfrac{21}{40}\right)$.

Ainsi on cherche $\mathbb{P}(X = 9) = \begin{pmatrix}12\\9\end{pmatrix}\times\left(\dfrac{21}{40}\right)^{9}\times\left(\dfrac{19}{40}\right)^{12-9}$.\end{frame}


\begin{frame}
\vspace{-10mm}
	\frametitle{Correction 5}
On répète $13$ fois de façon identique et indépendante l'épreuve de Bernoulli "On contrôle une trotinette" de succès S: "La trotinette est endommagée" de probabilité $p =\dfrac{3}{25}$.

Soit $X$ la variable aléatoire qui compte le nombre de succès à l'issue des $13$ répétitions.

 Alors $X \leadsto \mathcal{B}\left(13,\dfrac{3}{25}\right)$.

Ainsi on cherche $\mathbb{P}(X = 10) = \begin{pmatrix}13\\10\end{pmatrix}\times\left(\dfrac{3}{25}\right)^{10}\times\left(\dfrac{22}{25}\right)^{13-10}$.\end{frame}




\end{document}