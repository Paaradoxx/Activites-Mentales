\documentclass[15pt, mathserif]{beamer}

\usepackage[french]{babel}
\usepackage[T1]{fontenc}
\usepackage[utf8]{inputenc}
%\usepackage{esvect}
\usepackage{bm}
\usepackage{eurosym}
\usepackage{tikz}
\usepackage{pgf,tikz,pgfplots}
\pgfplotsset{compat=1.15}
\usepackage{mathrsfs}
\usetikzlibrary{arrows}
\usetikzlibrary{arrows.meta}

\usetikzlibrary{mindmap}
\usepackage{multicol}
\usepackage[tikz]{bclogo}
\usepackage{tkz-tab}
\usepackage{amsmath, tabu}
\usepackage{esvect} %\vv{AB} pour le vecteur AB

\DeclareMathOperator{\e}{e}

%% Tableau

\usepackage{makecell}
\setcellgapes{1pt}
\makegapedcells
\newcolumntype{R}[1]{>{\raggedleft\arraybackslash }b{#1}}
\newcolumntype{L}[1]{>{\raggedright\arraybackslash }b{#1}}
\newcolumntype{C}[1]{>{\centering\arraybackslash }b{#1}}


%pour avoir des parenthèses rondes dans le package fourier
\DeclareSymbolFont{cmoperators}   {OT1}{cmr} {m}{n}
\DeclareSymbolFont{cmlargesymbols}{OMX}{cmex}{m}{n}

\usefonttheme{professionalfonts} %permet d'enlever un bug avec fourier
\usepackage{fourier}
\DeclareMathDelimiter{(}{\mathopen} {cmoperators}{"28}{cmlargesymbols}{"00}
\DeclareMathDelimiter{)}{\mathclose}{cmoperators}{"29}{cmlargesymbols}{"01}

%Graphiques 

\usepackage{pgf,tikz,pgfplots}
\pgfplotsset{compat=1.15}
\usepackage{mathrsfs}
\usetikzlibrary{arrows}
\usetikzlibrary{mindmap}

%ensembles de nbres

\newcommand{\R}{\mathbb{R}}			%permet d'écrire le R "ensemble des réels"'
\newcommand{\N}{\mathbb{N}}			%permet d'écrire le N "ensemble des entiers naturels"
\newcommand{\Z}{\mathbb{Z}}			%permet d'écrire le Z "ensemble des entiers relatifs"
\newcommand{\Prem}{\mathbb{P}}	%permet d'écrire le P "ensemble des nombres premiers" (qui n'a pas marché avec le \P car il existe déjà)
\newcommand{\D}{\mathbb{D}}
\newcommand{\Df}{\mathcal{D}_f}
\newcommand{\Cf}{\mathcal{C}_f}

\newcommand{\Q}{\mathbb{Q}}


\newcommand{\st}[1]{$(#1_n)_{n \in \N}$}

\usetheme{Madrid}
\useoutertheme{miniframes} % Alternatively: miniframes, infolines, split
\useinnertheme{circles}
\definecolor{UBCblue}{rgb}{0.1, 0.25, 0.4} % UBC Blue (primary)
\definecolor{bordeaux}{RGB}{128,0,0}
\usecolortheme[named=UBCblue]{structure}

\usepackage{color} % J'aime bien définir mes couleurs
\definecolor{propcolor}{rgb}{0, 0.5, 1}
\definecolor{thcolor}{rgb}{0.6, 0.07, 0.07}
\colorlet{louis}{blue!70!green!60!white}
\colorlet{sakura}{pink!40!red}

\title{Activités Mentales}
\date{24 Août 2023}

\newcommand{\vco}[2]{\begin{pmatrix} #1 \\ #2 \end{pmatrix}} %Coordonnées de vecteur
\newenvironment{eq}{\begin{cases}\begin{tabu}{ccccc}}{\end{tabu}\end{cases}}
\newenvironment{eql}{\begin{cases}\begin{tabu}{cccccl}}{\end{tabu}\end{cases}}
\newenvironment{eqrl}{\begin{cases}\begin{tabu}{rl}}{\end{tabu}\end{cases}}

\newenvironment{Eq}{\begin{center}\begin{tabular}{rrcl}}{\end{tabular}\end{center}}
\newcommand{\ligneq}[2]{$\Longleftrightarrow$ & $#1$ & $=$ & $#2$ \\}
\newcommand{\Ligneq}[2]{ & $#1$ & $=$ & $#2$ \\}

\newenvironment{RPN}{\begin{center}\begin{tabular}{rrclcrcl}}{\end{tabular}\end{center}}
\newcommand{\Lignerpn}[4]{ & $#1$ & $=$ & $#2$ & ou & $#3$ & $=$ & $#4$ \\}
\newcommand{\lignerpn}[4]{$\Longleftrightarrow$ & $#1$ & $=$ & $#2$ & ou & $#3$ & $=$ & $#4$ \\}

\newenvironment{TRPN}{\begin{center}\begin{tabular}{rrclcrclcrcl}}{\end{tabular}\end{center}}
\newcommand{\Lignetrpn}[6]{ & $#1$ & $=$ & $#2$ & ou & $#3$ & $=$ & $#4$ & ou & $#5$ & $=$ & $#6$ \\}
\newcommand{\lignetrpn}[6]{$\Longleftrightarrow$ & $#1$ & $=$ & $#2$ & ou & $#3$ & $=$ & $#4$ & ou & $#5$ & $=$ & $#6$ \\}
\begin{document}

\begin{frame}
    \titlepage
\end{frame}

\begin{frame} 
	\frametitle{Question 1}
Résoudre le système suivant \[(S):~\begin{eq}-6x&+&7y&=&68\\9x&+&9y&=&171\end{eq}\]\end{frame}


\begin{frame} 
	\frametitle{Question 2}
Résoudre le système suivant \[(S):~\begin{eq}-9x&+&6y&=&-57\\-5x&-&3y&=&-114\end{eq}\]\end{frame}


\begin{frame} 
	\frametitle{Question 3}
Résoudre le système suivant \[(S):~\begin{eq}-3x&+&6y&=&-57\\-4x&+&4y&=&-44\end{eq}\]\end{frame}


\begin{frame} 
	\frametitle{Question 4}
Résoudre le système suivant \[(S):~\begin{eq}2x&+&7y&=&38\\-7x&-&5y&=&-94\end{eq}\]\end{frame}


\begin{frame} 
	\frametitle{Question 5}
Résoudre le système suivant \[(S):~\begin{eq}4x&+&2y&=&-40\\9x&-&2y&=&-38\end{eq}\]\end{frame}


\begin{frame}
\vspace{-10mm}
	\frametitle{Correction 1}
On résout le système par combinaisons linéaires.

\vspace*{-2em}
\begin{align*}
	(S)&\Leftrightarrow\begin{eq}-6x&+&7y&=&68\\9x&+&9y&=&171\end{eq}\\&\Leftrightarrow\begin{eql}-54x&+&63y&=&612& (L_1) \leftarrow 9\times (L_1) \\-54x&-&54y&=&-1026& (L_2) \leftarrow -6\times (L_2)\end{eql} \\
	&\Leftrightarrow\begin{eql}-54x&+&63y&=&612& (L_1) \\-54x+54x&-&54y-63y&=&-1026-612& (L_2) \leftarrow (L_2) - (L_1)\end{eql} \\
	&\Leftrightarrow\begin{eql}-54x&+&63y&=&612& (L_1) \\&-&117y&=&-1638& (L_2)\end{eql} 
\end{align*}

\end{frame}

\begin{frame}

\vspace*{-2em}
\begin{align*}
	(S)&\Leftrightarrow\begin{eql}-54x&+&63y&=&612& (L_1) \\&-&117y&=&-1638& (L_2)\end{eql} \\ &\Leftrightarrow \begin{eql} -54x&+&63y&=&612 & \\& &y&=&\dfrac{-1638}{-117} &= 14\end{eql}\\&\Leftrightarrow \begin{eqrl}-54x+63\times14&=612\\y&=14\end{eqrl}\\
	&\Leftrightarrow \begin{eqrl}-54x&=612-882\\y&=14\end{eqrl}\\
	&\Leftrightarrow \begin{eqrl}x&=\dfrac{-270}{-54} = 5\\y&=14\end{eqrl}
\end{align*}D'où les solutions de $(S)$ sont $\left\{(5~;~14)\right\}$.\end{frame}


\begin{frame}
\vspace{-10mm}
	\frametitle{Correction 2}
On résout le système par combinaisons linéaires.

\vspace*{-2em}
\begin{align*}
	(S)&\Leftrightarrow\begin{eq}-9x&+&6y&=&-57\\-5x&-&3y&=&-114\end{eq}\\&\Leftrightarrow\begin{eql}45x&-&30y&=&285& (L_1) \leftarrow -5\times (L_1) \\45x&+&27y&=&1026& (L_2) \leftarrow -9\times (L_2)\end{eql} \\
	&\Leftrightarrow\begin{eql}45x&-&30y&=&285& (L_1) \\45x-45x&+&27y+30y&=&1026-285& (L_2) \leftarrow (L_2) - (L_1)\end{eql} \\
	&\Leftrightarrow\begin{eql}45x&-&30y&=&285& (L_1) \\&&57y&=&741& (L_2)\end{eql} 
\end{align*}

\end{frame}

\begin{frame}

\vspace*{-2em}
\begin{align*}
	(S)&\Leftrightarrow\begin{eql}45x&-&30y&=&285& (L_1) \\&&57y&=&741& (L_2)\end{eql} \\ &\Leftrightarrow \begin{eql} 45x&-&30y&=&285 & \\& &y&=&\dfrac{741}{57} &= 13\end{eql}\\&\Leftrightarrow \begin{eqrl}45x-30\times13&=285\\y&=13\end{eqrl}\\
	&\Leftrightarrow \begin{eqrl}45x&=285+390\\y&=13\end{eqrl}\\
	&\Leftrightarrow \begin{eqrl}x&=\dfrac{675}{45} = 15\\y&=13\end{eqrl}
\end{align*}D'où les solutions de $(S)$ sont $\left\{(15~;~13)\right\}$.\end{frame}


\begin{frame}
\vspace{-10mm}
	\frametitle{Correction 3}
On résout le système par combinaisons linéaires.

\vspace*{-2em}
\begin{align*}
	(S)&\Leftrightarrow\begin{eq}-3x&+&6y&=&-57\\-4x&+&4y&=&-44\end{eq}\\&\Leftrightarrow\begin{eql}12x&-&24y&=&228& (L_1) \leftarrow -4\times (L_1) \\12x&-&12y&=&132& (L_2) \leftarrow -3\times (L_2)\end{eql} \\
	&\Leftrightarrow\begin{eql}12x&-&24y&=&228& (L_1) \\12x-12x&-&12y+24y&=&132-228& (L_2) \leftarrow (L_2) - (L_1)\end{eql} \\
	&\Leftrightarrow\begin{eql}12x&-&24y&=&228& (L_1) \\&&12y&=&-96& (L_2)\end{eql} 
\end{align*}

\end{frame}

\begin{frame}

\vspace*{-2em}
\begin{align*}
	(S)&\Leftrightarrow\begin{eql}12x&-&24y&=&228& (L_1) \\&&12y&=&-96& (L_2)\end{eql} \\ &\Leftrightarrow \begin{eql} 12x&-&24y&=&228 & \\& &y&=&\dfrac{-96}{12} &= -8\end{eql}\\&\Leftrightarrow \begin{eqrl}12x-24\times\left(-8\right)&=228\\y&=-8\end{eqrl}\\
	&\Leftrightarrow \begin{eqrl}12x&=228-192\\y&=-8\end{eqrl}\\
	&\Leftrightarrow \begin{eqrl}x&=\dfrac{36}{12} = 3\\y&=-8\end{eqrl}
\end{align*}D'où les solutions de $(S)$ sont $\left\{(3~;~-8)\right\}$.\end{frame}


\begin{frame}
\vspace{-10mm}
	\frametitle{Correction 4}
On résout le système par combinaisons linéaires.

\vspace*{-2em}
\begin{align*}
	(S)&\Leftrightarrow\begin{eq}2x&+&7y&=&38\\-7x&-&5y&=&-94\end{eq}\\&\Leftrightarrow\begin{eql}-14x&-&49y&=&-266& (L_1) \leftarrow -7\times (L_1) \\-14x&-&10y&=&-188& (L_2) \leftarrow 2\times (L_2)\end{eql} \\
	&\Leftrightarrow\begin{eql}-14x&-&49y&=&-266& (L_1) \\-14x+14x&-&10y+49y&=&-188+266& (L_2) \leftarrow (L_2) - (L_1)\end{eql} \\
	&\Leftrightarrow\begin{eql}-14x&-&49y&=&-266& (L_1) \\&&39y&=&78& (L_2)\end{eql} 
\end{align*}

\end{frame}

\begin{frame}

\vspace*{-2em}
\begin{align*}
	(S)&\Leftrightarrow\begin{eql}-14x&-&49y&=&-266& (L_1) \\&&39y&=&78& (L_2)\end{eql} \\ &\Leftrightarrow \begin{eql} -14x&-&49y&=&-266 & \\& &y&=&\dfrac{78}{39} &= 2\end{eql}\\&\Leftrightarrow \begin{eqrl}-14x-49\times2&=-266\\y&=2\end{eqrl}\\
	&\Leftrightarrow \begin{eqrl}-14x&=-266+98\\y&=2\end{eqrl}\\
	&\Leftrightarrow \begin{eqrl}x&=\dfrac{-168}{-14} = 12\\y&=2\end{eqrl}
\end{align*}D'où les solutions de $(S)$ sont $\left\{(12~;~2)\right\}$.\end{frame}


\begin{frame}
\vspace{-10mm}
	\frametitle{Correction 5}
On résout le système par combinaisons linéaires.

\vspace*{-2em}
\begin{align*}
	(S)&\Leftrightarrow\begin{eq}4x&+&2y&=&-40\\9x&-&2y&=&-38\end{eq}\\&\Leftrightarrow\begin{eql}36x&+&18y&=&-360& (L_1) \leftarrow 9\times (L_1) \\36x&-&8y&=&-152& (L_2) \leftarrow 4\times (L_2)\end{eql} \\
	&\Leftrightarrow\begin{eql}36x&+&18y&=&-360& (L_1) \\36x-36x&-&8y-18y&=&-152+360& (L_2) \leftarrow (L_2) - (L_1)\end{eql} \\
	&\Leftrightarrow\begin{eql}36x&+&18y&=&-360& (L_1) \\&-&26y&=&208& (L_2)\end{eql} 
\end{align*}

\end{frame}

\begin{frame}

\vspace*{-2em}
\begin{align*}
	(S)&\Leftrightarrow\begin{eql}36x&+&18y&=&-360& (L_1) \\&-&26y&=&208& (L_2)\end{eql} \\ &\Leftrightarrow \begin{eql} 36x&+&18y&=&-360 & \\& &y&=&\dfrac{208}{-26} &= -8\end{eql}\\&\Leftrightarrow \begin{eqrl}36x+18\times\left(-8\right)&=-360\\y&=-8\end{eqrl}\\
	&\Leftrightarrow \begin{eqrl}36x&=-360+144\\y&=-8\end{eqrl}\\
	&\Leftrightarrow \begin{eqrl}x&=\dfrac{-216}{36} = -6\\y&=-8\end{eqrl}
\end{align*}D'où les solutions de $(S)$ sont $\left\{(-6~;~-8)\right\}$.\end{frame}




\end{document}