\documentclass[15pt, mathserif]{beamer}

\usepackage[french]{babel}
\usepackage[T1]{fontenc}
\usepackage[utf8]{inputenc}
%\usepackage{esvect}
\usepackage{bm}
\usepackage{eurosym}
\usepackage{tikz}
\usepackage{pgf,tikz,pgfplots}
\pgfplotsset{compat=1.15}
\usepackage{mathrsfs}
\usetikzlibrary{arrows}
\usetikzlibrary{arrows.meta}

\usetikzlibrary{mindmap}
\usepackage{multicol}
\usepackage[tikz]{bclogo}
\usepackage{tkz-tab}
\usepackage{amsmath, tabu}
\usepackage{esvect} %\vv{AB} pour le vecteur AB

\DeclareMathOperator{\e}{e}

%% Tableau

\usepackage{makecell}
\setcellgapes{1pt}
\makegapedcells
\newcolumntype{R}[1]{>{\raggedleft\arraybackslash }b{#1}}
\newcolumntype{L}[1]{>{\raggedright\arraybackslash }b{#1}}
\newcolumntype{C}[1]{>{\centering\arraybackslash }b{#1}}


%pour avoir des parenthèses rondes dans le package fourier
\DeclareSymbolFont{cmoperators}   {OT1}{cmr} {m}{n}
\DeclareSymbolFont{cmlargesymbols}{OMX}{cmex}{m}{n}

\usefonttheme{professionalfonts} %permet d'enlever un bug avec fourier
\usepackage{fourier}
\DeclareMathDelimiter{(}{\mathopen} {cmoperators}{"28}{cmlargesymbols}{"00}
\DeclareMathDelimiter{)}{\mathclose}{cmoperators}{"29}{cmlargesymbols}{"01}

%Graphiques 

\usepackage{pgf,tikz,pgfplots}
\pgfplotsset{compat=1.15}
\usepackage{mathrsfs}
\usetikzlibrary{arrows}
\usetikzlibrary{mindmap}

%ensembles de nbres

\newcommand{\R}{\mathbb{R}}			%permet d'écrire le R "ensemble des réels"'
\newcommand{\N}{\mathbb{N}}			%permet d'écrire le N "ensemble des entiers naturels"
\newcommand{\Z}{\mathbb{Z}}			%permet d'écrire le Z "ensemble des entiers relatifs"
\newcommand{\Prem}{\mathbb{P}}	%permet d'écrire le P "ensemble des nombres premiers" (qui n'a pas marché avec le \P car il existe déjà)
\newcommand{\D}{\mathbb{D}}
\newcommand{\Df}{\mathcal{D}_f}
\newcommand{\Cf}{\mathcal{C}_f}

\newcommand{\Q}{\mathbb{Q}}


\newcommand{\st}[1]{$(#1_n)_{n \in \N}$}

\usetheme{Madrid}
\useoutertheme{miniframes} % Alternatively: miniframes, infolines, split
\useinnertheme{circles}
\definecolor{UBCblue}{rgb}{0.1, 0.25, 0.4} % UBC Blue (primary)
\definecolor{bordeaux}{RGB}{128,0,0}
\usecolortheme[named=UBCblue]{structure}

\usepackage{color} % J'aime bien définir mes couleurs
\definecolor{propcolor}{rgb}{0, 0.5, 1}
\definecolor{thcolor}{rgb}{0.6, 0.07, 0.07}
\colorlet{louis}{blue!70!green!60!white}
\colorlet{sakura}{pink!40!red}

\title{Activités Mentales}
\date{24 Août 2023}

\newcommand{\vco}[2]{\begin{pmatrix} #1 \\ #2 \end{pmatrix}} %Coordonnées de vecteur
\newenvironment{eq}{\begin{cases}\begin{tabu}{ccccc}}{\end{tabu}\end{cases}}
\newenvironment{eql}{\begin{cases}\begin{tabu}{cccccl}}{\end{tabu}\end{cases}}
\newenvironment{eqrl}{\begin{cases}\begin{tabu}{rl}}{\end{tabu}\end{cases}}

\newenvironment{Eq}{\begin{center}\begin{tabular}{rrcl}}{\end{tabular}\end{center}}
\newcommand{\ligneq}[2]{$\Longleftrightarrow$ & $#1$ & $=$ & $#2$ \\}
\newcommand{\Ligneq}[2]{ & $#1$ & $=$ & $#2$ \\}

\newenvironment{RPN}{\begin{center}\begin{tabular}{rrclcrcl}}{\end{tabular}\end{center}}
\newcommand{\Lignerpn}[4]{ & $#1$ & $=$ & $#2$ & ou & $#3$ & $=$ & $#4$ \\}
\newcommand{\lignerpn}[4]{$\Longleftrightarrow$ & $#1$ & $=$ & $#2$ & ou & $#3$ & $=$ & $#4$ \\}

\newenvironment{TRPN}{\begin{center}\begin{tabular}{rrclcrclcrcl}}{\end{tabular}\end{center}}
\newcommand{\Lignetrpn}[6]{ & $#1$ & $=$ & $#2$ & ou & $#3$ & $=$ & $#4$ & ou & $#5$ & $=$ & $#6$ \\}
\newcommand{\lignetrpn}[6]{$\Longleftrightarrow$ & $#1$ & $=$ & $#2$ & ou & $#3$ & $=$ & $#4$ & ou & $#5$ & $=$ & $#6$ \\}
\begin{document}

\begin{frame}
    \titlepage
\end{frame}

\begin{frame} 
	\frametitle{Question 1}
Quelle est l'image de 0 par la fonction $f$ dont la représentation graphique est donnée ci-dessous ? \\ \begin{center} 
 \begin{tikzpicture}[line cap=round,line join=round,>=triangle 45,x=1cm,y=1cm] 
 \begin{axis}[ x=0.6cm,y=0.6cm,axis lines=middle, ymajorgrids=true, xmajorgrids=true,xmin=-4.64, xmax=4.920000000000016, ymin=-4.709999999999996, ymax=4.67, xtick={-4,-3,...,5}, ytick={-4,-3,...,4},] \clip(-4.64,-4.71) rectangle (4.92,4.67);
 \draw[line width=1pt,smooth,samples=100,domain=-4.64:5.920000000000016] plot(\x,{0.1*((\x)+3)*((\x))*(\x)+1}); 
 \draw (4.06,3.25) node[anchor=north west] {$\mathcal{C}_f$}; 
 \begin{scriptsize}
 \draw[color=black] (-3.76,-8.38) node {$f$}; 
 \end{scriptsize}
 \end{axis} 
 \end{tikzpicture} 
 \end{center}\end{frame}


\begin{frame} 
	\frametitle{Question 2}
Donner le ou les antécédents de 0 par la fonction $f$ dont la représentation graphique est donnée ci-dessous. \\ \begin{center} 
 \begin{tikzpicture}[line cap=round,line join=round,>=triangle 45,x=1cm,y=1cm] 
 \begin{axis}[ x=0.6cm,y=0.6cm,axis lines=middle, ymajorgrids=true, xmajorgrids=true,xmin=-4.64, xmax=4.920000000000016, ymin=-4.709999999999996, ymax=4.67, xtick={-4,-3,...,5}, ytick={-4,-3,...,4},] \clip(-4.64,-4.71) rectangle (4.92,4.67);
 \draw[line width=1pt,smooth,samples=100,domain=-4.64:5.920000000000016] plot(\x,{0.1*((\x)+3)*((\x)+2)*(\x)}); 
 \draw (4.06,3.25) node[anchor=north west] {$\mathcal{C}_f$}; 
 \begin{scriptsize}
 \draw[color=black] (-3.76,-8.38) node {$f$}; 
 \end{scriptsize}
 \end{axis} 
 \end{tikzpicture} 
 \end{center}\end{frame}


\begin{frame} 
	\frametitle{Question 3}
Donner le ou les antécédents de 2 par la fonction $f$ dont la représentation graphique est donnée ci-dessous. \\ \begin{center} 
 \begin{tikzpicture}[line cap=round,line join=round,>=triangle 45,x=1cm,y=1cm] 
 \begin{axis}[ x=0.6cm,y=0.6cm,axis lines=middle, ymajorgrids=true, xmajorgrids=true,xmin=-4.64, xmax=4.920000000000016, ymin=-4.709999999999996, ymax=4.67, xtick={-4,-3,...,5}, ytick={-4,-3,...,4},] \clip(-4.64,-4.71) rectangle (4.92,4.67);
 \draw[line width=1pt,smooth,samples=100,domain=-4.64:5.920000000000016] plot(\x,{0.1*((\x))*((\x))*(\x)+2}); 
 \draw (4.06,3.25) node[anchor=north west] {$\mathcal{C}_f$}; 
 \begin{scriptsize}
 \draw[color=black] (-3.76,-8.38) node {$f$}; 
 \end{scriptsize}
 \end{axis} 
 \end{tikzpicture} 
 \end{center}\end{frame}


\begin{frame} 
	\frametitle{Question 4}
Donner le ou les antécédents de -3 par la fonction $f$ dont la représentation graphique est donnée ci-dessous. \\ \begin{center} 
 \begin{tikzpicture}[line cap=round,line join=round,>=triangle 45,x=1cm,y=1cm] 
 \begin{axis}[ x=0.6cm,y=0.6cm,axis lines=middle, ymajorgrids=true, xmajorgrids=true,xmin=-4.64, xmax=4.920000000000016, ymin=-4.709999999999996, ymax=4.67, xtick={-4,-3,...,5}, ytick={-4,-3,...,4},] \clip(-4.64,-4.71) rectangle (4.92,4.67);
 \draw[line width=1pt,smooth,samples=100,domain=-4.64:5.920000000000016] plot(\x,{0.1*((\x)-2)*((\x)+2)*(\x)-3}); 
 \draw (4.06,3.25) node[anchor=north west] {$\mathcal{C}_f$}; 
 \begin{scriptsize}
 \draw[color=black] (-3.76,-8.38) node {$f$}; 
 \end{scriptsize}
 \end{axis} 
 \end{tikzpicture} 
 \end{center}\end{frame}


\begin{frame} 
	\frametitle{Question 5}
Donner le ou les antécédents de 3 par la fonction $f$ dont la représentation graphique est donnée ci-dessous. \\ \begin{center} 
 \begin{tikzpicture}[line cap=round,line join=round,>=triangle 45,x=1cm,y=1cm] 
 \begin{axis}[ x=0.6cm,y=0.6cm,axis lines=middle, ymajorgrids=true, xmajorgrids=true,xmin=-4.64, xmax=4.920000000000016, ymin=-4.709999999999996, ymax=4.67, xtick={-4,-3,...,5}, ytick={-4,-3,...,4},] \clip(-4.64,-4.71) rectangle (4.92,4.67);
 \draw[line width=1pt,smooth,samples=100,domain=-4.64:5.920000000000016] plot(\x,{0.1*((\x)-2)*((\x)+2)*(\x)+3}); 
 \draw (4.06,3.25) node[anchor=north west] {$\mathcal{C}_f$}; 
 \begin{scriptsize}
 \draw[color=black] (-3.76,-8.38) node {$f$}; 
 \end{scriptsize}
 \end{axis} 
 \end{tikzpicture} 
 \end{center}\end{frame}


\begin{frame}
\vspace{-10mm}
	\frametitle{Correction 1}
\vspace*{1cm} Quelle est l'image de 0 par la fonction $f$ dont la représentation graphique est donnée ci-dessous ? \\ \begin{center}  
 \definecolor{ccqqqq}{rgb}{0.8,0,0} \begin{tikzpicture}[line cap=round,line join=round,>=triangle 45,x=1cm,y=1cm] 
 \begin{axis}[ x=0.6cm,y=0.6cm,axis lines=middle, ymajorgrids=true, xmajorgrids=true,xmin=-4.64, xmax=4.920000000000016, ymin=-4.709999999999996, ymax=4.67, xtick={-4,-3,...,5}, ytick={-4,-3,...,4},] \clip(-4.64,-4.71) rectangle (4.92,4.67);
 \draw[line width=1pt,smooth,samples=100,domain=-4.64:5.920000000000016] plot(\x,{0.1*((\x)+3)*((\x))*(\x)+1}); 
 \draw (4.06,3.25) node[anchor=north west] {$\mathcal{C}_f$}; 
 \draw [line width=1pt,dash pattern=on 1pt off 1pt,color=ccqqqq] (0,0)-- (0,1); 
 \draw [line width=1pt,dash pattern=on 1pt off 1pt,color=ccqqqq] (0,1)-- (0,1); 
 \end{axis} 
 \end{tikzpicture} \end{center} Donc l'image de 0 est 1\end{frame}


\begin{frame}
\vspace{-10mm}
	\frametitle{Correction 2}
\vspace*{1cm} Donner le ou les antécédents de  0 par la fonction $f$ dont la représentation graphique est donnée ci-dessous. \\ \begin{center}  
 \definecolor{ccqqqq}{rgb}{0.8,0,0} \begin{tikzpicture}[line cap=round,line join=round,>=triangle 45,x=1cm,y=1cm] 
 \begin{axis}[ x=0.6cm,y=0.6cm,axis lines=middle, ymajorgrids=true, xmajorgrids=true,xmin=-4.64, xmax=4.920000000000016, ymin=-4.709999999999996, ymax=4.67, xtick={-4,-3,...,5}, ytick={-4,-3,...,4},] \clip(-4.64,-4.71) rectangle (4.92,4.67);
 \draw[line width=1pt,smooth,samples=100,domain=-4.64:5.920000000000016] plot(\x,{0.1*((\x)+3)*((\x)+2)*(\x)}); 
 \draw (4.06,3.25) node[anchor=north west] {$\mathcal{C}_f$}; 
 \draw[line width=1pt,color=ccqqqq,domain=-4.64:5.92] plot(\x,{(0-0*\x)/1}); 
 \draw[line width=1pt,dash pattern=on 1pt off 1pt,color=ccqqqq] (-3,0)-- (-3,0); 
 \draw [line width=1pt,dash pattern=on 1pt off 1pt,color=ccqqqq] (-2,0)-- (-2,0); 
 \draw [line width=1pt,dash pattern=on 1pt off 1pt,color=ccqqqq] (0,0)-- (0,0); ); 
 \end{axis} 
 \end{tikzpicture} \end{center} \vspace*{-0.5cm} Le ou les antécédents de 0 sont 0, -3 et -2.\end{frame}


\begin{frame}
\vspace{-10mm}
	\frametitle{Correction 3}
\vspace*{1cm} Donner le ou les antécédents de  2 par la fonction $f$ dont la représentation graphique est donnée ci-dessous. \\ \begin{center}  
 \definecolor{ccqqqq}{rgb}{0.8,0,0} \begin{tikzpicture}[line cap=round,line join=round,>=triangle 45,x=1cm,y=1cm] 
 \begin{axis}[ x=0.6cm,y=0.6cm,axis lines=middle, ymajorgrids=true, xmajorgrids=true,xmin=-4.64, xmax=4.920000000000016, ymin=-4.709999999999996, ymax=4.67, xtick={-4,-3,...,5}, ytick={-4,-3,...,4},] \clip(-4.64,-4.71) rectangle (4.92,4.67);
 \draw[line width=1pt,smooth,samples=100,domain=-4.64:5.920000000000016] plot(\x,{0.1*((\x))*((\x))*(\x)+2}); 
 \draw (4.06,3.25) node[anchor=north west] {$\mathcal{C}_f$}; 
 \draw[line width=1pt,color=ccqqqq,domain=-4.64:5.92] plot(\x,{(2-0*\x)/1}); 
 \draw[line width=1pt,dash pattern=on 1pt off 1pt,color=ccqqqq] (0,2)-- (0,0); 
 \draw [line width=1pt,dash pattern=on 1pt off 1pt,color=ccqqqq] (0,2)-- (0,0); 
 \draw [line width=1pt,dash pattern=on 1pt off 1pt,color=ccqqqq] (0,2)-- (0,0); ); 
 \end{axis} 
 \end{tikzpicture} \end{center} \vspace*{-0.5cm} Le ou les antécédents de 2 sont 0, 0 et 0.\end{frame}


\begin{frame}
\vspace{-10mm}
	\frametitle{Correction 4}
\vspace*{1cm} Donner le ou les antécédents de  -3 par la fonction $f$ dont la représentation graphique est donnée ci-dessous. \\ \begin{center}  
 \definecolor{ccqqqq}{rgb}{0.8,0,0} \begin{tikzpicture}[line cap=round,line join=round,>=triangle 45,x=1cm,y=1cm] 
 \begin{axis}[ x=0.6cm,y=0.6cm,axis lines=middle, ymajorgrids=true, xmajorgrids=true,xmin=-4.64, xmax=4.920000000000016, ymin=-4.709999999999996, ymax=4.67, xtick={-4,-3,...,5}, ytick={-4,-3,...,4},] \clip(-4.64,-4.71) rectangle (4.92,4.67);
 \draw[line width=1pt,smooth,samples=100,domain=-4.64:5.920000000000016] plot(\x,{0.1*((\x)-2)*((\x)+2)*(\x)-3}); 
 \draw (4.06,3.25) node[anchor=north west] {$\mathcal{C}_f$}; 
 \draw[line width=1pt,color=ccqqqq,domain=-4.64:5.92] plot(\x,{(-3-0*\x)/1}); 
 \draw[line width=1pt,dash pattern=on 1pt off 1pt,color=ccqqqq] (2,-3)-- (2,0); 
 \draw [line width=1pt,dash pattern=on 1pt off 1pt,color=ccqqqq] (-2,-3)-- (-2,0); 
 \draw [line width=1pt,dash pattern=on 1pt off 1pt,color=ccqqqq] (0,-3)-- (0,0); ); 
 \end{axis} 
 \end{tikzpicture} \end{center} \vspace*{-0.5cm} Le ou les antécédents de -3 sont 0, 2 et -2.\end{frame}


\begin{frame}
\vspace{-10mm}
	\frametitle{Correction 5}
\vspace*{1cm} Donner le ou les antécédents de  3 par la fonction $f$ dont la représentation graphique est donnée ci-dessous. \\ \begin{center}  
 \definecolor{ccqqqq}{rgb}{0.8,0,0} \begin{tikzpicture}[line cap=round,line join=round,>=triangle 45,x=1cm,y=1cm] 
 \begin{axis}[ x=0.6cm,y=0.6cm,axis lines=middle, ymajorgrids=true, xmajorgrids=true,xmin=-4.64, xmax=4.920000000000016, ymin=-4.709999999999996, ymax=4.67, xtick={-4,-3,...,5}, ytick={-4,-3,...,4},] \clip(-4.64,-4.71) rectangle (4.92,4.67);
 \draw[line width=1pt,smooth,samples=100,domain=-4.64:5.920000000000016] plot(\x,{0.1*((\x)-2)*((\x)+2)*(\x)+3}); 
 \draw (4.06,3.25) node[anchor=north west] {$\mathcal{C}_f$}; 
 \draw[line width=1pt,color=ccqqqq,domain=-4.64:5.92] plot(\x,{(3-0*\x)/1}); 
 \draw[line width=1pt,dash pattern=on 1pt off 1pt,color=ccqqqq] (2,3)-- (2,0); 
 \draw [line width=1pt,dash pattern=on 1pt off 1pt,color=ccqqqq] (-2,3)-- (-2,0); 
 \draw [line width=1pt,dash pattern=on 1pt off 1pt,color=ccqqqq] (0,3)-- (0,0); ); 
 \end{axis} 
 \end{tikzpicture} \end{center} \vspace*{-0.5cm} Le ou les antécédents de 3 sont 0, 2 et -2.\end{frame}




\end{document}