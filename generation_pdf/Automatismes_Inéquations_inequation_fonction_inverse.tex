\documentclass[15pt, mathserif]{beamer}

\usepackage[french]{babel}
\usepackage[T1]{fontenc}
\usepackage[utf8]{inputenc}
%\usepackage{esvect}
\usepackage{bm}
\usepackage{eurosym}
\usepackage{tikz}
\usepackage{pgf,tikz,pgfplots}
\pgfplotsset{compat=1.15}
\usepackage{mathrsfs}
\usetikzlibrary{arrows}
\usetikzlibrary{arrows.meta}

\usetikzlibrary{mindmap}
\usepackage{multicol}
\usepackage[tikz]{bclogo}
\usepackage{tkz-tab}
\usepackage{amsmath, tabu}
\usepackage{esvect} %\vv{AB} pour le vecteur AB

\DeclareMathOperator{\e}{e}

%% Tableau

\usepackage{makecell}
\setcellgapes{1pt}
\makegapedcells
\newcolumntype{R}[1]{>{\raggedleft\arraybackslash }b{#1}}
\newcolumntype{L}[1]{>{\raggedright\arraybackslash }b{#1}}
\newcolumntype{C}[1]{>{\centering\arraybackslash }b{#1}}


%pour avoir des parenthèses rondes dans le package fourier
\DeclareSymbolFont{cmoperators}   {OT1}{cmr} {m}{n}
\DeclareSymbolFont{cmlargesymbols}{OMX}{cmex}{m}{n}

\usefonttheme{professionalfonts} %permet d'enlever un bug avec fourier
\usepackage{fourier}
\DeclareMathDelimiter{(}{\mathopen} {cmoperators}{"28}{cmlargesymbols}{"00}
\DeclareMathDelimiter{)}{\mathclose}{cmoperators}{"29}{cmlargesymbols}{"01}

%Graphiques 

\usepackage{pgf,tikz,pgfplots}
\pgfplotsset{compat=1.15}
\usepackage{mathrsfs}
\usetikzlibrary{arrows}
\usetikzlibrary{mindmap}

%ensembles de nbres

\newcommand{\R}{\mathbb{R}}			%permet d'écrire le R "ensemble des réels"'
\newcommand{\N}{\mathbb{N}}			%permet d'écrire le N "ensemble des entiers naturels"
\newcommand{\Z}{\mathbb{Z}}			%permet d'écrire le Z "ensemble des entiers relatifs"
\newcommand{\Prem}{\mathbb{P}}	%permet d'écrire le P "ensemble des nombres premiers" (qui n'a pas marché avec le \P car il existe déjà)
\newcommand{\D}{\mathbb{D}}
\newcommand{\Df}{\mathcal{D}_f}
\newcommand{\Cf}{\mathcal{C}_f}

\newcommand{\Q}{\mathbb{Q}}


\newcommand{\st}[1]{$(#1_n)_{n \in \N}$}

\usetheme{Madrid}
\useoutertheme{miniframes} % Alternatively: miniframes, infolines, split
\useinnertheme{circles}
\definecolor{UBCblue}{rgb}{0.1, 0.25, 0.4} % UBC Blue (primary)
\definecolor{bordeaux}{RGB}{128,0,0}
\usecolortheme[named=UBCblue]{structure}

\usepackage{color} % J'aime bien définir mes couleurs
\definecolor{propcolor}{rgb}{0, 0.5, 1}
\definecolor{thcolor}{rgb}{0.6, 0.07, 0.07}
\colorlet{louis}{blue!70!green!60!white}
\colorlet{sakura}{pink!40!red}

\title{Activités Mentales}
\date{24 Août 2023}

\newcommand{\vco}[2]{\begin{pmatrix} #1 \\ #2 \end{pmatrix}} %Coordonnées de vecteur
\newenvironment{eq}{\begin{cases}\begin{tabu}{ccccc}}{\end{tabu}\end{cases}}
\newenvironment{eql}{\begin{cases}\begin{tabu}{cccccl}}{\end{tabu}\end{cases}}
\newenvironment{eqrl}{\begin{cases}\begin{tabu}{rl}}{\end{tabu}\end{cases}}

\newenvironment{Eq}{\begin{center}\begin{tabular}{rrcl}}{\end{tabular}\end{center}}
\newcommand{\ligneq}[2]{$\Longleftrightarrow$ & $#1$ & $=$ & $#2$ \\}
\newcommand{\Ligneq}[2]{ & $#1$ & $=$ & $#2$ \\}

\newenvironment{RPN}{\begin{center}\begin{tabular}{rrclcrcl}}{\end{tabular}\end{center}}
\newcommand{\Lignerpn}[4]{ & $#1$ & $=$ & $#2$ & ou & $#3$ & $=$ & $#4$ \\}
\newcommand{\lignerpn}[4]{$\Longleftrightarrow$ & $#1$ & $=$ & $#2$ & ou & $#3$ & $=$ & $#4$ \\}

\newenvironment{TRPN}{\begin{center}\begin{tabular}{rrclcrclcrcl}}{\end{tabular}\end{center}}
\newcommand{\Lignetrpn}[6]{ & $#1$ & $=$ & $#2$ & ou & $#3$ & $=$ & $#4$ & ou & $#5$ & $=$ & $#6$ \\}
\newcommand{\lignetrpn}[6]{$\Longleftrightarrow$ & $#1$ & $=$ & $#2$ & ou & $#3$ & $=$ & $#4$ & ou & $#5$ & $=$ & $#6$ \\}
\begin{document}

\begin{frame}
    \titlepage
\end{frame}

\begin{frame} 
	\frametitle{Question 1}
Résoudre dans $\R$ l'inéquation $\dfrac{10-2x}{x}>-1$\end{frame}


\begin{frame} 
	\frametitle{Question 2}
Résoudre dans $\R$ l'inéquation $\dfrac{4-4x}{x}-3>1$\end{frame}


\begin{frame} 
	\frametitle{Question 3}
Résoudre dans $\R$ l'inéquation $\dfrac{10}{x}-1>-2$\end{frame}


\begin{frame} 
	\frametitle{Question 4}
Résoudre dans $\R$ l'inéquation $\dfrac{-5-4x}{x}+3>-9$\end{frame}


\begin{frame} 
	\frametitle{Question 5}
Résoudre dans $\R$ l'inéquation $\dfrac{3+9x}{x}-6>4$\end{frame}


\begin{frame}
\vspace{-10mm}
	\frametitle{Correction 1}
 \begin{multicols}{2} 
 
 \begin{align*} & \dfrac{10-2x}{x}>-1\\ 
 \Leftrightarrow & \dfrac{10-2x}{x}>-1\\ 
 \Leftrightarrow & \dfrac{10}{x}-2>-1\\ 
 \Leftrightarrow & \dfrac{10}{x}>1\\ 
 \Leftrightarrow & \dfrac{1}{x}>\dfrac{1}{10} 
 \end{align*}
 
 \begin{tikzpicture}[x=0.5cm,y=0.5cm] 
 \draw[->] (-4,0) -- (4,0);  
 \draw[->] (0,-4.5) -- (0,4.5);
 \draw plot[domain=-4:-0.25] (\x,{1/(\x)});  
 \draw plot[domain=0.25:4] (\x,{1/(\x)}); 
 
 \draw (-4,2) -- (4,2); 
 \draw (0,2) node[below left] {\tiny{$\dfrac{1}{10}$}}; 
 \draw[blue,dotted] (0.5,0) -- (0.5,2); 
 \draw[blue] (0.5,0) node[below] {\tiny{10}};
 \end{tikzpicture} 
 \\ 
 L'ensemble des solutions de l'équation est $S=]0;\dfrac{10}{1}[$
 \end{multicols} 
 \end{frame}


\begin{frame}
\vspace{-10mm}
	\frametitle{Correction 2}
 \begin{multicols}{2} 
 
 \begin{align*} & \dfrac{4-4x}{x}-3>1\\ 
 \Leftrightarrow & \dfrac{4-4x}{x}>4\\ 
 \Leftrightarrow & \dfrac{4}{x}-4>4\\ 
 \Leftrightarrow & \dfrac{4}{x}>8\\ 
 \Leftrightarrow & \dfrac{1}{x}>2 
 \end{align*}
 
 \begin{tikzpicture}[x=0.5cm,y=0.5cm] 
 \draw[->] (-4,0) -- (4,0);  
 \draw[->] (0,-4.5) -- (0,4.5);
 \draw plot[domain=-4:-0.25] (\x,{1/(\x)});  
 \draw plot[domain=0.25:4] (\x,{1/(\x)}); 
 
 \draw (-4,2) -- (4,2); 
 \draw (0,2) node[below left] {\tiny{$2$}};
 \draw[blue,dotted] (0.5,0) -- (0.5,2); 
 \draw[blue] (0.5,0) node[below] {\tiny{$\dfrac{1}{2}$}};
 \end{tikzpicture} 
 \\ 
 L'ensemble des solutions de l'équation est $S=]0;\dfrac{1}{2}[$
 \end{multicols} 
 \end{frame}


\begin{frame}
\vspace{-10mm}
	\frametitle{Correction 3}
 \begin{multicols}{2} 
 
 \begin{align*} & \dfrac{10}{x}-1>-2\\ 
 \Leftrightarrow & \dfrac{10}{x}>-1\\ 
 \Leftrightarrow & \dfrac{10}{x}>-1\\ 
 \Leftrightarrow & \dfrac{10}{x}>-1\\ 
 \Leftrightarrow & \dfrac{1}{x}>\dfrac{-1}{10} 
 \end{align*}
 
 \begin{tikzpicture}[x=0.5cm,y=0.5cm] 
 \draw[->] (-4,0) -- (4,0);  
 \draw[->] (0,-4.5) -- (0,4.5);
 \draw plot[domain=-4:-0.25] (\x,{1/(\x)});  
 \draw plot[domain=0.25:4] (\x,{1/(\x)}); 
 
 \draw (-4,-2) -- (4,-2); \draw (0,-2) node[below right] {\tiny{$\dfrac{-1}{10}$}}; 
 \draw[blue,dotted] (-0.5,0) -- (-0.5,-2); 
 \draw[blue] (-0.5,0) node[above] {\tiny{-10}};
 \end{tikzpicture} 
 \\ 
 L'ensemble des solutions de l'équation est $S=]-\infty;\dfrac{-10}{1}[ \cup ]0;+\infty[$
 \end{multicols} 
 \end{frame}


\begin{frame}
\vspace{-10mm}
	\frametitle{Correction 4}
 \begin{multicols}{2} 
 
 \begin{align*} & \dfrac{-5-4x}{x}+3>-9\\ 
 \Leftrightarrow & \dfrac{-5-4x}{x}>-12\\ 
 \Leftrightarrow & \dfrac{-5}{x}-4>-12\\ 
 \Leftrightarrow & \dfrac{-5}{x}>-8\\ 
 \Leftrightarrow & \dfrac{1}{x}<\dfrac{8}{5} 
 \end{align*}
 
 \begin{tikzpicture}[x=0.5cm,y=0.5cm] 
 \draw[->] (-4,0) -- (4,0);  
 \draw[->] (0,-4.5) -- (0,4.5);
 \draw plot[domain=-4:-0.25] (\x,{1/(\x)});  
 \draw plot[domain=0.25:4] (\x,{1/(\x)}); 
 
 \draw (-4,2) -- (4,2); 
 \draw (0,2) node[below left] {\tiny{$\dfrac{8}{5}$}}; 
 \draw[blue,dotted] (0.5,0) -- (0.5,2); 
 \draw[blue] (0.5,0) node[below] {\tiny{$\dfrac{5}{8}$}};
 \end{tikzpicture} 
 \\ 
 L'ensemble des solutions de l'équation est $S=]-\infty;0[ \cup ]0;\dfrac{-5}{-8}[$
 \end{multicols} 
 \end{frame}


\begin{frame}
\vspace{-10mm}
	\frametitle{Correction 5}
 \begin{multicols}{2} 
 
 \begin{align*} & \dfrac{3+9x}{x}-6>4\\ 
 \Leftrightarrow & \dfrac{3+9x}{x}>10\\ 
 \Leftrightarrow & \dfrac{3}{x}+9>10\\ 
 \Leftrightarrow & \dfrac{3}{x}>1\\ 
 \Leftrightarrow & \dfrac{1}{x}>\dfrac{1}{3} 
 \end{align*}
 
 \begin{tikzpicture}[x=0.5cm,y=0.5cm] 
 \draw[->] (-4,0) -- (4,0);  
 \draw[->] (0,-4.5) -- (0,4.5);
 \draw plot[domain=-4:-0.25] (\x,{1/(\x)});  
 \draw plot[domain=0.25:4] (\x,{1/(\x)}); 
 
 \draw (-4,2) -- (4,2); 
 \draw (0,2) node[below left] {\tiny{$\dfrac{1}{3}$}}; 
 \draw[blue,dotted] (0.5,0) -- (0.5,2); 
 \draw[blue] (0.5,0) node[below] {\tiny{3}};
 \end{tikzpicture} 
 \\ 
 L'ensemble des solutions de l'équation est $S=]0;\dfrac{3}{1}[$
 \end{multicols} 
 \end{frame}




\end{document}