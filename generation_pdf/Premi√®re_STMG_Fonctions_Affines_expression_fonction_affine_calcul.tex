\documentclass[15pt, mathserif]{beamer}

\usepackage[french]{babel}
\usepackage[T1]{fontenc}
\usepackage[utf8]{inputenc}
%\usepackage{esvect}
\usepackage{bm}
\usepackage{eurosym}
\usepackage{tikz}
\usepackage{pgf,tikz,pgfplots}
\pgfplotsset{compat=1.15}
\usepackage{mathrsfs}
\usetikzlibrary{arrows}
\usetikzlibrary{arrows.meta}

\usetikzlibrary{mindmap}
\usepackage{multicol}
\usepackage[tikz]{bclogo}
\usepackage{tkz-tab}
\usepackage{amsmath, tabu}
\usepackage{esvect} %\vv{AB} pour le vecteur AB

\DeclareMathOperator{\e}{e}

%% Tableau

\usepackage{makecell}
\setcellgapes{1pt}
\makegapedcells
\newcolumntype{R}[1]{>{\raggedleft\arraybackslash }b{#1}}
\newcolumntype{L}[1]{>{\raggedright\arraybackslash }b{#1}}
\newcolumntype{C}[1]{>{\centering\arraybackslash }b{#1}}


%pour avoir des parenthèses rondes dans le package fourier
\DeclareSymbolFont{cmoperators}   {OT1}{cmr} {m}{n}
\DeclareSymbolFont{cmlargesymbols}{OMX}{cmex}{m}{n}

\usefonttheme{professionalfonts} %permet d'enlever un bug avec fourier
\usepackage{fourier}
\DeclareMathDelimiter{(}{\mathopen} {cmoperators}{"28}{cmlargesymbols}{"00}
\DeclareMathDelimiter{)}{\mathclose}{cmoperators}{"29}{cmlargesymbols}{"01}

%Graphiques 

\usepackage{pgf,tikz,pgfplots}
\pgfplotsset{compat=1.15}
\usepackage{mathrsfs}
\usetikzlibrary{arrows}
\usetikzlibrary{mindmap}

%ensembles de nbres

\newcommand{\R}{\mathbb{R}}			%permet d'écrire le R "ensemble des réels"'
\newcommand{\N}{\mathbb{N}}			%permet d'écrire le N "ensemble des entiers naturels"
\newcommand{\Z}{\mathbb{Z}}			%permet d'écrire le Z "ensemble des entiers relatifs"
\newcommand{\Prem}{\mathbb{P}}	%permet d'écrire le P "ensemble des nombres premiers" (qui n'a pas marché avec le \P car il existe déjà)
\newcommand{\D}{\mathbb{D}}
\newcommand{\Df}{\mathcal{D}_f}
\newcommand{\Cf}{\mathcal{C}_f}

\newcommand{\Q}{\mathbb{Q}}


\newcommand{\st}[1]{$(#1_n)_{n \in \N}$}

\usetheme{Madrid}
\useoutertheme{miniframes} % Alternatively: miniframes, infolines, split
\useinnertheme{circles}
\definecolor{UBCblue}{rgb}{0.1, 0.25, 0.4} % UBC Blue (primary)
\definecolor{bordeaux}{RGB}{128,0,0}
\usecolortheme[named=UBCblue]{structure}

\usepackage{color} % J'aime bien définir mes couleurs
\definecolor{propcolor}{rgb}{0, 0.5, 1}
\definecolor{thcolor}{rgb}{0.6, 0.07, 0.07}
\colorlet{louis}{blue!70!green!60!white}
\colorlet{sakura}{pink!40!red}

\title{Activités Mentales}
\date{24 Août 2023}

\newcommand{\vco}[2]{\begin{pmatrix} #1 \\ #2 \end{pmatrix}} %Coordonnées de vecteur
\newenvironment{eq}{\begin{cases}\begin{tabu}{ccccc}}{\end{tabu}\end{cases}}
\newenvironment{eql}{\begin{cases}\begin{tabu}{cccccl}}{\end{tabu}\end{cases}}
\newenvironment{eqrl}{\begin{cases}\begin{tabu}{rl}}{\end{tabu}\end{cases}}

\newenvironment{Eq}{\begin{center}\begin{tabular}{rrcl}}{\end{tabular}\end{center}}
\newcommand{\ligneq}[2]{$\Longleftrightarrow$ & $#1$ & $=$ & $#2$ \\}
\newcommand{\Ligneq}[2]{ & $#1$ & $=$ & $#2$ \\}

\newenvironment{RPN}{\begin{center}\begin{tabular}{rrclcrcl}}{\end{tabular}\end{center}}
\newcommand{\Lignerpn}[4]{ & $#1$ & $=$ & $#2$ & ou & $#3$ & $=$ & $#4$ \\}
\newcommand{\lignerpn}[4]{$\Longleftrightarrow$ & $#1$ & $=$ & $#2$ & ou & $#3$ & $=$ & $#4$ \\}

\newenvironment{TRPN}{\begin{center}\begin{tabular}{rrclcrclcrcl}}{\end{tabular}\end{center}}
\newcommand{\Lignetrpn}[6]{ & $#1$ & $=$ & $#2$ & ou & $#3$ & $=$ & $#4$ & ou & $#5$ & $=$ & $#6$ \\}
\newcommand{\lignetrpn}[6]{$\Longleftrightarrow$ & $#1$ & $=$ & $#2$ & ou & $#3$ & $=$ & $#4$ & ou & $#5$ & $=$ & $#6$ \\}
\begin{document}

\begin{frame}
    \titlepage
\end{frame}

\begin{frame} 
	\frametitle{Question 1}
Quelle est l'expression de la fonction affine passant par les points de coordonnées (-10;2) et (10;2) ?\end{frame}


\begin{frame} 
	\frametitle{Question 2}
Quelle est l'expression de la fonction affine passant par les points de coordonnées (-8;68) et (6;-44) ?\end{frame}


\begin{frame} 
	\frametitle{Question 3}
Quelle est l'expression de la fonction affine passant par les points de coordonnées (3;-37) et (9;-91) ?\end{frame}


\begin{frame} 
	\frametitle{Question 4}
Quelle est l'expression de la fonction affine passant par les points de coordonnées (-1;8) et (-4;32) ?\end{frame}


\begin{frame} 
	\frametitle{Question 5}
Quelle est l'expression de la fonction affine passant par les points de coordonnées (7;13) et (-10;-21) ?\end{frame}


\begin{frame}
\vspace{-10mm}
	\frametitle{Correction 1}
\vspace*{1cm} 
 \footnotesize{Quelle est l'expression de la fonction affine passant par les points de coordonnées (-10;2) et (10;2) ? Il existe deux techniques :} 
 \begin{multicols}{2} 
 \begin{enumerate} 
 \item On résout un système : $$ \begin{array}{rcl} 
 & \textcolor{white}{\Leftrightarrow} & 
 \left 
 \{\begin{array}{rcl}-10\times m + p&=&2 \\ 
 10\times m+p&=&2\end{array} \right. \\ 
 &\Leftrightarrow & \left 
 \{\begin{array}{rcl} p&=&2+10m \\ 
 10m+p&=&2\end{array} \right. \\ 
 &\Leftrightarrow & \left 
 \{\begin{array}{rcl} p&=&2+10m \\ 
 10m+(2+10m) &=&2\end{array} \right. \\ &\Leftrightarrow& \left \{\begin{array}{rcl}p&=&2+10m \\ 
 2+20m&=&2\end{array} \right. \\ &\Leftrightarrow& \left \{\begin{array}{rcl}p&=&2+10m \\ 
 20m&=&0\end{array} \right. \\  &\Leftrightarrow& \left \{\begin{array}{rcl} p&=&2 \\  m&=&0\end{array}\right. \end{array}$$ 
 Ainsi on a $f:x\mapsto 2$ 
 \columnbreak 
 \item 
 \footnotesize{On applique la formule du cours pour calculer $m$ :$$ \dfrac{f(x_1)-f(x_2)}{x_1-x_2}=\dfrac{2-2}{-10-10}= \dfrac{0}{-20}=0$$} \footnotesize{ Ainsi on a $f(x)= p$
  On cherche maintenant la valeur de $p$. On sait que $f(-10)=2$. On doit donc résoudre $(E): 0\times\left(-10\right)+p=2$}	 
 \begin{align*} (E)& \Leftrightarrow 0+p=2\\
		 	 & \Leftrightarrow p=2\\
			 & \Leftrightarrow p=2
	 \end{align*} 
 Ainsi on a $f:x\mapsto 2$ 
 \end{enumerate} 
 \end{multicols} 
 \end{frame}


\begin{frame}
\vspace{-10mm}
	\frametitle{Correction 2}
\vspace*{1cm} 
 \footnotesize{Quelle est l'expression de la fonction affine passant par les points de coordonnées (-8;68) et (6;-44) ? Il existe deux techniques :} 
 \begin{multicols}{2} 
 \begin{enumerate} 
 \item On résout un système : $$ \begin{array}{rcl} 
 & \textcolor{white}{\Leftrightarrow} & 
 \left 
 \{\begin{array}{rcl}-8\times m + p&=&68 \\ 
 6\times m+p&=&-44\end{array} \right. \\ 
 &\Leftrightarrow & \left 
 \{\begin{array}{rcl} p&=&68+8m \\ 
 6m+p&=&-44\end{array} \right. \\ 
 &\Leftrightarrow & \left 
 \{\begin{array}{rcl} p&=&68+8m \\ 
 6m+(68+8m) &=&-44\end{array} \right. \\ &\Leftrightarrow& \left \{\begin{array}{rcl}p&=&68+8m \\ 
 68+14m&=&-44\end{array} \right. \\ &\Leftrightarrow& \left \{\begin{array}{rcl}p&=&68+8m \\ 
 14m&=&-112\end{array} \right. \\  &\Leftrightarrow& \left \{\begin{array}{rcl} p&=&4 \\  m&=&-8\end{array}\right. \end{array}$$ 
 Ainsi on a $f:x\mapsto -8x+4$ 
 \columnbreak 
 \item 
 \footnotesize{On applique la formule du cours pour calculer $m$ :$$ \dfrac{f(x_1)-f(x_2)}{x_1-x_2}=\dfrac{68-\left(-44\right)}{-8-6}= \dfrac{112}{-14}=-8$$} \footnotesize{ Ainsi on a $f(x)= -8x +p $. 
  \\ On cherche maintenant la valeur de $p$. On sait que $f(-8)=68$. On doit donc résoudre $(E): -8\times\left(-8\right)+p=68$}	 
 \begin{align*} (E)& \Leftrightarrow 64+p=68\\
		 	 & \Leftrightarrow p=68-64\\
			 & \Leftrightarrow p=4
	 \end{align*} 
 Ainsi on a $f:x\mapsto -8x+4$ 
 \end{enumerate} 
 \end{multicols} 
 \end{frame}


\begin{frame}
\vspace{-10mm}
	\frametitle{Correction 3}
\vspace*{1cm} 
 \footnotesize{Quelle est l'expression de la fonction affine passant par les points de coordonnées (3;-37) et (9;-91) ? Il existe deux techniques :} 
 \begin{multicols}{2} 
 \begin{enumerate} 
 \item On résout un système : $$ \begin{array}{rcl} 
 & \textcolor{white}{\Leftrightarrow} & 
 \left 
 \{\begin{array}{rcl}3\times m + p&=&-37 \\ 
 9\times m+p&=&-91\end{array} \right. \\ 
 &\Leftrightarrow & \left 
 \{\begin{array}{rcl} p&=&-37-3m \\ 
 9m+p&=&-91\end{array} \right. \\ 
 &\Leftrightarrow & \left 
 \{\begin{array}{rcl} p&=&-37-3m \\ 
 9m+(-37-3m) &=&-91\end{array} \right. \\ &\Leftrightarrow& \left \{\begin{array}{rcl}p&=&-37-3m \\ 
 -37+6m&=&-91\end{array} \right. \\ &\Leftrightarrow& \left \{\begin{array}{rcl}p&=&-37-3m \\ 
 6m&=&-54\end{array} \right. \\  &\Leftrightarrow& \left \{\begin{array}{rcl} p&=&-10 \\  m&=&-9\end{array}\right. \end{array}$$ 
 Ainsi on a $f:x\mapsto -9x-10$ 
 \columnbreak 
 \item 
 \footnotesize{On applique la formule du cours pour calculer $m$ :$$ \dfrac{f(x_1)-f(x_2)}{x_1-x_2}=\dfrac{-37-\left(-91\right)}{3-9}= \dfrac{54}{-6}=-9$$} \footnotesize{ Ainsi on a $f(x)= -9x +p $. 
  \\ On cherche maintenant la valeur de $p$. On sait que $f(3)=-37$. On doit donc résoudre $(E): -9\times3+p=-37$}	 
 \begin{align*} (E)& \Leftrightarrow -27+p=-37\\
		 	 & \Leftrightarrow p=-37+27\\
			 & \Leftrightarrow p=-10
	 \end{align*} 
 Ainsi on a $f:x\mapsto -9x-10$ 
 \end{enumerate} 
 \end{multicols} 
 \end{frame}


\begin{frame}
\vspace{-10mm}
	\frametitle{Correction 4}
\vspace*{1cm} 
 \footnotesize{Quelle est l'expression de la fonction affine passant par les points de coordonnées (-1;8) et (-4;32) ? Il existe deux techniques :} 
 \begin{multicols}{2} 
 \begin{enumerate} 
 \item On résout un système : $$ \begin{array}{rcl} 
 & \textcolor{white}{\Leftrightarrow} & 
 \left 
 \{\begin{array}{rcl}-1\times m + p&=&8 \\ 
 -4\times m+p&=&32\end{array} \right. \\ 
 &\Leftrightarrow & \left 
 \{\begin{array}{rcl} p&=&8+1m \\ 
 -4m+p&=&32\end{array} \right. \\ 
 &\Leftrightarrow & \left 
 \{\begin{array}{rcl} p&=&8+m \\ 
 -4m+(8+m) &=&32\end{array} \right. \\ &\Leftrightarrow& \left \{\begin{array}{rcl}p&=&8+m \\ 
 8-3m&=&32\end{array} \right. \\ &\Leftrightarrow& \left \{\begin{array}{rcl}p&=&8+m \\ 
 -3m&=&24\end{array} \right. \\  &\Leftrightarrow& \left \{\begin{array}{rcl} p&=&0 \\  m&=&-8\end{array}\right. \end{array}$$ 
 Ainsi on a $f:x\mapsto -8x$ 
 \columnbreak 
 \item 
 \footnotesize{On applique la formule du cours pour calculer $m$ :$$ \dfrac{f(x_1)-f(x_2)}{x_1-x_2}=\dfrac{8-32}{-1-\left(-4\right)}= \dfrac{-24}{3}=-8$$} \footnotesize{ Ainsi on a $f(x)= -8x +p $. 
  \\ On cherche maintenant la valeur de $p$. On sait que $f(-1)=8$. On doit donc résoudre $(E): -8\times\left(-1\right)+p=8$}	 
 \begin{align*} (E)& \Leftrightarrow 8+p=8\\
		 	 & \Leftrightarrow p=8-8\\
			 & \Leftrightarrow p=0
	 \end{align*} 
 Ainsi on a $f:x\mapsto -8x$ 
 \end{enumerate} 
 \end{multicols} 
 \end{frame}


\begin{frame}
\vspace{-10mm}
	\frametitle{Correction 5}
\vspace*{1cm} 
 \footnotesize{Quelle est l'expression de la fonction affine passant par les points de coordonnées (7;13) et (-10;-21) ? Il existe deux techniques :} 
 \begin{multicols}{2} 
 \begin{enumerate} 
 \item On résout un système : $$ \begin{array}{rcl} 
 & \textcolor{white}{\Leftrightarrow} & 
 \left 
 \{\begin{array}{rcl}7\times m + p&=&13 \\ 
 -10\times m+p&=&-21\end{array} \right. \\ 
 &\Leftrightarrow & \left 
 \{\begin{array}{rcl} p&=&13-7m \\ 
 -10m+p&=&-21\end{array} \right. \\ 
 &\Leftrightarrow & \left 
 \{\begin{array}{rcl} p&=&13-7m \\ 
 -10m+(13-7m) &=&-21\end{array} \right. \\ &\Leftrightarrow& \left \{\begin{array}{rcl}p&=&13-7m \\ 
 13-17m&=&-21\end{array} \right. \\ &\Leftrightarrow& \left \{\begin{array}{rcl}p&=&13-7m \\ 
 -17m&=&-34\end{array} \right. \\  &\Leftrightarrow& \left \{\begin{array}{rcl} p&=&-1 \\  m&=&2\end{array}\right. \end{array}$$ 
 Ainsi on a $f:x\mapsto 2x-1$ 
 \columnbreak 
 \item 
 \footnotesize{On applique la formule du cours pour calculer $m$ :$$ \dfrac{f(x_1)-f(x_2)}{x_1-x_2}=\dfrac{13-\left(-21\right)}{7-\left(-10\right)}= \dfrac{34}{17}=2$$} \footnotesize{ Ainsi on a $f(x)= 2x +p $. 
  \\ On cherche maintenant la valeur de $p$. On sait que $f(7)=13$. On doit donc résoudre $(E): 2\times7+p=13$}	 
 \begin{align*} (E)& \Leftrightarrow 14+p=13\\
		 	 & \Leftrightarrow p=13-14\\
			 & \Leftrightarrow p=-1
	 \end{align*} 
 Ainsi on a $f:x\mapsto 2x-1$ 
 \end{enumerate} 
 \end{multicols} 
 \end{frame}




\end{document}