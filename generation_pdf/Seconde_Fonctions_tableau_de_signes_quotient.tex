\documentclass[15pt, mathserif]{beamer}

\usepackage[french]{babel}
\usepackage[T1]{fontenc}
\usepackage[utf8]{inputenc}
%\usepackage{esvect}
\usepackage{bm}
\usepackage{eurosym}
\usepackage{tikz}
\usepackage{pgf,tikz,pgfplots}
\pgfplotsset{compat=1.15}
\usepackage{mathrsfs}
\usetikzlibrary{arrows}
\usetikzlibrary{arrows.meta}

\usetikzlibrary{mindmap}
\usepackage{multicol}
\usepackage[tikz]{bclogo}
\usepackage{tkz-tab}
\usepackage{amsmath, tabu}
\usepackage{esvect} %\vv{AB} pour le vecteur AB

\DeclareMathOperator{\e}{e}

%% Tableau

\usepackage{makecell}
\setcellgapes{1pt}
\makegapedcells
\newcolumntype{R}[1]{>{\raggedleft\arraybackslash }b{#1}}
\newcolumntype{L}[1]{>{\raggedright\arraybackslash }b{#1}}
\newcolumntype{C}[1]{>{\centering\arraybackslash }b{#1}}


%pour avoir des parenthèses rondes dans le package fourier
\DeclareSymbolFont{cmoperators}   {OT1}{cmr} {m}{n}
\DeclareSymbolFont{cmlargesymbols}{OMX}{cmex}{m}{n}

\usefonttheme{professionalfonts} %permet d'enlever un bug avec fourier
\usepackage{fourier}
\DeclareMathDelimiter{(}{\mathopen} {cmoperators}{"28}{cmlargesymbols}{"00}
\DeclareMathDelimiter{)}{\mathclose}{cmoperators}{"29}{cmlargesymbols}{"01}

%Graphiques 

\usepackage{pgf,tikz,pgfplots}
\pgfplotsset{compat=1.15}
\usepackage{mathrsfs}
\usetikzlibrary{arrows}
\usetikzlibrary{mindmap}

%ensembles de nbres

\newcommand{\R}{\mathbb{R}}			%permet d'écrire le R "ensemble des réels"'
\newcommand{\N}{\mathbb{N}}			%permet d'écrire le N "ensemble des entiers naturels"
\newcommand{\Z}{\mathbb{Z}}			%permet d'écrire le Z "ensemble des entiers relatifs"
\newcommand{\Prem}{\mathbb{P}}	%permet d'écrire le P "ensemble des nombres premiers" (qui n'a pas marché avec le \P car il existe déjà)
\newcommand{\D}{\mathbb{D}}
\newcommand{\Df}{\mathcal{D}_f}
\newcommand{\Cf}{\mathcal{C}_f}

\newcommand{\Q}{\mathbb{Q}}


\newcommand{\st}[1]{$(#1_n)_{n \in \N}$}

\usetheme{Madrid}
\useoutertheme{miniframes} % Alternatively: miniframes, infolines, split
\useinnertheme{circles}
\definecolor{UBCblue}{rgb}{0.1, 0.25, 0.4} % UBC Blue (primary)
\definecolor{bordeaux}{RGB}{128,0,0}
\usecolortheme[named=UBCblue]{structure}

\usepackage{color} % J'aime bien définir mes couleurs
\definecolor{propcolor}{rgb}{0, 0.5, 1}
\definecolor{thcolor}{rgb}{0.6, 0.07, 0.07}
\colorlet{louis}{blue!70!green!60!white}
\colorlet{sakura}{pink!40!red}

\title{Activités Mentales}
\date{24 Août 2023}

\newcommand{\vco}[2]{\begin{pmatrix} #1 \\ #2 \end{pmatrix}} %Coordonnées de vecteur
\newenvironment{eq}{\begin{cases}\begin{tabu}{ccccc}}{\end{tabu}\end{cases}}
\newenvironment{eql}{\begin{cases}\begin{tabu}{cccccl}}{\end{tabu}\end{cases}}
\newenvironment{eqrl}{\begin{cases}\begin{tabu}{rl}}{\end{tabu}\end{cases}}

\newenvironment{Eq}{\begin{center}\begin{tabular}{rrcl}}{\end{tabular}\end{center}}
\newcommand{\ligneq}[2]{$\Longleftrightarrow$ & $#1$ & $=$ & $#2$ \\}
\newcommand{\Ligneq}[2]{ & $#1$ & $=$ & $#2$ \\}

\newenvironment{RPN}{\begin{center}\begin{tabular}{rrclcrcl}}{\end{tabular}\end{center}}
\newcommand{\Lignerpn}[4]{ & $#1$ & $=$ & $#2$ & ou & $#3$ & $=$ & $#4$ \\}
\newcommand{\lignerpn}[4]{$\Longleftrightarrow$ & $#1$ & $=$ & $#2$ & ou & $#3$ & $=$ & $#4$ \\}

\newenvironment{TRPN}{\begin{center}\begin{tabular}{rrclcrclcrcl}}{\end{tabular}\end{center}}
\newcommand{\Lignetrpn}[6]{ & $#1$ & $=$ & $#2$ & ou & $#3$ & $=$ & $#4$ & ou & $#5$ & $=$ & $#6$ \\}
\newcommand{\lignetrpn}[6]{$\Longleftrightarrow$ & $#1$ & $=$ & $#2$ & ou & $#3$ & $=$ & $#4$ & ou & $#5$ & $=$ & $#6$ \\}
\begin{document}

\begin{frame}
    \titlepage
\end{frame}

\begin{frame} 
	\frametitle{Question 1}
Résoudre l'inéquation \[ \dfrac{-15x-15}{13x+1}<0\]\end{frame}


\begin{frame} 
	\frametitle{Question 2}
Résoudre l'inéquation \[ \dfrac{5x-2}{-10x-6}\geq0\]\end{frame}


\begin{frame} 
	\frametitle{Question 3}
Résoudre l'inéquation \[ \dfrac{-4x-8}{-5x+3}>0\]\end{frame}


\begin{frame} 
	\frametitle{Question 4}
Résoudre l'inéquation \[ \dfrac{3x+7}{-4x-2}\geq0\]\end{frame}


\begin{frame} 
	\frametitle{Question 5}
Résoudre l'inéquation \[ \dfrac{-12x+4}{3x+11}\geq0\]\end{frame}


\begin{frame}
\vspace{-10mm}
	\frametitle{Correction 1}
\vspace*{1cm} 
  On pose $A(x) = \dfrac{-15x-15}{13x+1} = \dfrac{f(x)}{g(x)}$ avec $f(x) = -15x-15$ et $g(x) = 13x+1$.

 On cherche quand le quotient s'annule et les potentielles valeurs interdites. Pour cela, on résout $A(x)=0$ en utilisant la Règle du quotient nul : 
 
 $\begin{array}{crclcrcl} 
 
 	  & \dfrac{-15x-15}{13x+1} & = & 0 \\ 
 	  \Leftrightarrow & -15x-15 & =& 0 & \text{et} & 13x+1\neq & 0 \\ 
 	 \Leftrightarrow & -15x&=&15& \text{et} & 13x & \neq & -1 \\ 
 	 \Leftrightarrow & x&=&-1 & \text{et} & x &\neq&\dfrac{-1}{13}
 
 \end{array}$ 
 	 \begin{itemize} 
	\item  $f$ est une fonction affine avec $m =-15<0$. $f$ est donc décroissante sur $\mathbb{R}$.
	\item $g$ est une fonction affine avec $m =13>0$. $g$ est donc croissante sur $\mathbb{R}$.

	 \end{itemize}

 \end{frame}


\begin{frame}On rappelle que $f(x) = -15x-15$ et $g(x) = 13x+1$ et $A(x) = \dfrac{-15x-15}{13x+1}$. Son tableau de signe est alors 

\medskip \hfil
\begin{tikzpicture}[scale = 0.75]
	\tkzTabInit[lgt = 1.5]{$x$/1.25, $f(x)$ / 1, $g(x)$ / 1, $A(x)$/1}{$-\infty$, $-1$, $\dfrac{-1}{13}$, $+\infty$}
	\tkzTabLine{ , +, z, -, t, -, }
	\tkzTabLine{ , -, t, -, z, +, }
	\tkzTabLine{ , -, z, +, d, -, }
	\end{tikzpicture}

 Finalement l'ensemble de solutions de $\dfrac{-15x-15}{13x+1}<0$ est\[S = \left]-\infty;-1\right[\cup\left]\dfrac{-1}{13};+\infty\right[\]

\end{frame}


\begin{frame}
\vspace{-10mm}
	\frametitle{Correction 2}
\vspace*{1cm} 
  On pose $A(x) = \dfrac{5x-2}{-10x-6} = \dfrac{f(x)}{g(x)}$ avec $f(x) = 5x-2$ et $g(x) = -10x-6$.

 On cherche quand le quotient s'annule et les potentielles valeurs interdites. Pour cela, on résout $A(x)=0$ en utilisant la Règle du quotient nul : 
 
 $\begin{array}{crclcrcl} 
 
 	  & \dfrac{5x-2}{-10x-6} & = & 0 \\ 
 	  \Leftrightarrow & 5x-2 & =& 0 & \text{et} & -10x-6\neq & 0 \\ 
 	 \Leftrightarrow & 5x&=&2& \text{et} & -10x & \neq & 6 \\ 
 	 \Leftrightarrow & x&=&\dfrac{2}{5} & \text{et} & x &\neq&\dfrac{-3}{5}
 
 \end{array}$ 
 	 \begin{itemize} 
	\item  $f$ est une fonction affine avec $m =5>0$. $f$ est donc croissante sur $\mathbb{R}$.
	\item $g$ est une fonction affine avec $m =-10<0$. $g$ est donc décroissante sur $\mathbb{R}$.

	 \end{itemize}

 \end{frame}


\begin{frame}On rappelle que $f(x) = 5x-2$ et $g(x) = -10x-6$ et $A(x) = \dfrac{5x-2}{-10x-6}$. Son tableau de signe est alors 

\medskip \hfil
\begin{tikzpicture}[scale = 0.75]
	\tkzTabInit[lgt = 1.5]{$x$/1.25, $f(x)$ / 1, $g(x)$ / 1, $A(x)$/1}{$-\infty$, $\dfrac{-3}{5}$, $\dfrac{2}{5}$, $+\infty$}
	\tkzTabLine{ , -, t, -, z, +, }
	\tkzTabLine{ , +, z, -, t, -, }
	\tkzTabLine{ , -, d, +, z, -, }
	\end{tikzpicture}

 Finalement l'ensemble de solutions de $\dfrac{5x-2}{-10x-6}\geq0$ est\[S = \left]\dfrac{-3}{5};\dfrac{2}{5}\right]\]

\end{frame}


\begin{frame}
\vspace{-10mm}
	\frametitle{Correction 3}
\vspace*{1cm} 
  On pose $A(x) = \dfrac{-4x-8}{-5x+3} = \dfrac{f(x)}{g(x)}$ avec $f(x) = -4x-8$ et $g(x) = -5x+3$.

 On cherche quand le quotient s'annule et les potentielles valeurs interdites. Pour cela, on résout $A(x)=0$ en utilisant la Règle du quotient nul : 
 
 $\begin{array}{crclcrcl} 
 
 	  & \dfrac{-4x-8}{-5x+3} & = & 0 \\ 
 	  \Leftrightarrow & -4x-8 & =& 0 & \text{et} & -5x+3\neq & 0 \\ 
 	 \Leftrightarrow & -4x&=&8& \text{et} & -5x & \neq & -3 \\ 
 	 \Leftrightarrow & x&=&-2 & \text{et} & x &\neq&\dfrac{3}{5}
 
 \end{array}$ 
 	 \begin{itemize} 
	\item  $f$ est une fonction affine avec $m =-4<0$. $f$ est donc décroissante sur $\mathbb{R}$.
	\item $g$ est une fonction affine avec $m =-5<0$. $g$ est donc décroissante sur $\mathbb{R}$.

	 \end{itemize}

 \end{frame}


\begin{frame}On rappelle que $f(x) = -4x-8$ et $g(x) = -5x+3$ et $A(x) = \dfrac{-4x-8}{-5x+3}$. Son tableau de signe est alors 

\medskip \hfil
\begin{tikzpicture}[scale = 0.75]
	\tkzTabInit[lgt = 1.5]{$x$/1.25, $f(x)$ / 1, $g(x)$ / 1, $A(x)$/1}{$-\infty$, $-2$, $\dfrac{3}{5}$, $+\infty$}
	\tkzTabLine{ , +, z, -, t, -, }
	\tkzTabLine{ , +, t, +, z, -, }
	\tkzTabLine{ , +, z, -, d, +, }
	\end{tikzpicture}

 Finalement l'ensemble de solutions de $\dfrac{-4x-8}{-5x+3}>0$ est\[S = \left]-\infty;-2\right[\cup\left]\dfrac{3}{5};+\infty\right[\]

\end{frame}


\begin{frame}
\vspace{-10mm}
	\frametitle{Correction 4}
\vspace*{1cm} 
  On pose $A(x) = \dfrac{3x+7}{-4x-2} = \dfrac{f(x)}{g(x)}$ avec $f(x) = 3x+7$ et $g(x) = -4x-2$.

 On cherche quand le quotient s'annule et les potentielles valeurs interdites. Pour cela, on résout $A(x)=0$ en utilisant la Règle du quotient nul : 
 
 $\begin{array}{crclcrcl} 
 
 	  & \dfrac{3x+7}{-4x-2} & = & 0 \\ 
 	  \Leftrightarrow & 3x+7 & =& 0 & \text{et} & -4x-2\neq & 0 \\ 
 	 \Leftrightarrow & 3x&=&-7& \text{et} & -4x & \neq & 2 \\ 
 	 \Leftrightarrow & x&=&\dfrac{-7}{3} & \text{et} & x &\neq&\dfrac{-1}{2}
 
 \end{array}$ 
 	 \begin{itemize} 
	\item  $f$ est une fonction affine avec $m =3>0$. $f$ est donc croissante sur $\mathbb{R}$.
	\item $g$ est une fonction affine avec $m =-4<0$. $g$ est donc décroissante sur $\mathbb{R}$.

	 \end{itemize}

 \end{frame}


\begin{frame}On rappelle que $f(x) = 3x+7$ et $g(x) = -4x-2$ et $A(x) = \dfrac{3x+7}{-4x-2}$. Son tableau de signe est alors 

\medskip \hfil
\begin{tikzpicture}[scale = 0.75]
	\tkzTabInit[lgt = 1.5]{$x$/1.25, $f(x)$ / 1, $g(x)$ / 1, $A(x)$/1}{$-\infty$, $\dfrac{-7}{3}$, $\dfrac{-1}{2}$, $+\infty$}
	\tkzTabLine{ , -, z, +, t, +, }
	\tkzTabLine{ , +, t, +, z, -, }
	\tkzTabLine{ , -, z, +, d, -, }
	\end{tikzpicture}

 Finalement l'ensemble de solutions de $\dfrac{3x+7}{-4x-2}\geq0$ est\[S = \left[\dfrac{-7}{3};\dfrac{-1}{2}\right[\]

\end{frame}


\begin{frame}
\vspace{-10mm}
	\frametitle{Correction 5}
\vspace*{1cm} 
  On pose $A(x) = \dfrac{-12x+4}{3x+11} = \dfrac{f(x)}{g(x)}$ avec $f(x) = -12x+4$ et $g(x) = 3x+11$.

 On cherche quand le quotient s'annule et les potentielles valeurs interdites. Pour cela, on résout $A(x)=0$ en utilisant la Règle du quotient nul : 
 
 $\begin{array}{crclcrcl} 
 
 	  & \dfrac{-12x+4}{3x+11} & = & 0 \\ 
 	  \Leftrightarrow & -12x+4 & =& 0 & \text{et} & 3x+11\neq & 0 \\ 
 	 \Leftrightarrow & -12x&=&-4& \text{et} & 3x & \neq & -11 \\ 
 	 \Leftrightarrow & x&=&\dfrac{1}{3} & \text{et} & x &\neq&\dfrac{-11}{3}
 
 \end{array}$ 
 	 \begin{itemize} 
	\item  $f$ est une fonction affine avec $m =-12<0$. $f$ est donc décroissante sur $\mathbb{R}$.
	\item $g$ est une fonction affine avec $m =3>0$. $g$ est donc croissante sur $\mathbb{R}$.

	 \end{itemize}

 \end{frame}


\begin{frame}On rappelle que $f(x) = -12x+4$ et $g(x) = 3x+11$ et $A(x) = \dfrac{-12x+4}{3x+11}$. Son tableau de signe est alors 

\medskip \hfil
\begin{tikzpicture}[scale = 0.75]
	\tkzTabInit[lgt = 1.5]{$x$/1.25, $f(x)$ / 1, $g(x)$ / 1, $A(x)$/1}{$-\infty$, $\dfrac{-11}{3}$, $\dfrac{1}{3}$, $+\infty$}
	\tkzTabLine{ , +, t, +, z, -, }
	\tkzTabLine{ , -, z, +, t, +, }
	\tkzTabLine{ , -, d, +, z, -, }
	\end{tikzpicture}

 Finalement l'ensemble de solutions de $\dfrac{-12x+4}{3x+11}\geq0$ est\[S = \left]\dfrac{-11}{3};\dfrac{1}{3}\right]\]

\end{frame}




\end{document}