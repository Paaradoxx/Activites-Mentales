\documentclass[15pt, mathserif]{beamer}

\usepackage[french]{babel}
\usepackage[T1]{fontenc}
\usepackage[utf8]{inputenc}
%\usepackage{esvect}
\usepackage{bm}
\usepackage{eurosym}
\usepackage{tikz}
\usepackage{pgf,tikz,pgfplots}
\pgfplotsset{compat=1.15}
\usepackage{mathrsfs}
\usetikzlibrary{arrows}
\usetikzlibrary{arrows.meta}

\usetikzlibrary{mindmap}
\usepackage{multicol}
\usepackage[tikz]{bclogo}
\usepackage{tkz-tab}
\usepackage{amsmath, tabu}
\usepackage{esvect} %\vv{AB} pour le vecteur AB

\DeclareMathOperator{\e}{e}

%% Tableau

\usepackage{makecell}
\setcellgapes{1pt}
\makegapedcells
\newcolumntype{R}[1]{>{\raggedleft\arraybackslash }b{#1}}
\newcolumntype{L}[1]{>{\raggedright\arraybackslash }b{#1}}
\newcolumntype{C}[1]{>{\centering\arraybackslash }b{#1}}


%pour avoir des parenthèses rondes dans le package fourier
\DeclareSymbolFont{cmoperators}   {OT1}{cmr} {m}{n}
\DeclareSymbolFont{cmlargesymbols}{OMX}{cmex}{m}{n}

\usefonttheme{professionalfonts} %permet d'enlever un bug avec fourier
\usepackage{fourier}
\DeclareMathDelimiter{(}{\mathopen} {cmoperators}{"28}{cmlargesymbols}{"00}
\DeclareMathDelimiter{)}{\mathclose}{cmoperators}{"29}{cmlargesymbols}{"01}

%Graphiques 

\usepackage{pgf,tikz,pgfplots}
\pgfplotsset{compat=1.15}
\usepackage{mathrsfs}
\usetikzlibrary{arrows}
\usetikzlibrary{mindmap}

%ensembles de nbres

\newcommand{\R}{\mathbb{R}}			%permet d'écrire le R "ensemble des réels"'
\newcommand{\N}{\mathbb{N}}			%permet d'écrire le N "ensemble des entiers naturels"
\newcommand{\Z}{\mathbb{Z}}			%permet d'écrire le Z "ensemble des entiers relatifs"
\newcommand{\Prem}{\mathbb{P}}	%permet d'écrire le P "ensemble des nombres premiers" (qui n'a pas marché avec le \P car il existe déjà)
\newcommand{\D}{\mathbb{D}}
\newcommand{\Df}{\mathcal{D}_f}
\newcommand{\Cf}{\mathcal{C}_f}

\newcommand{\Q}{\mathbb{Q}}


\newcommand{\st}[1]{$(#1_n)_{n \in \N}$}

\usetheme{Madrid}
\useoutertheme{miniframes} % Alternatively: miniframes, infolines, split
\useinnertheme{circles}
\definecolor{UBCblue}{rgb}{0.1, 0.25, 0.4} % UBC Blue (primary)
\definecolor{bordeaux}{RGB}{128,0,0}
\usecolortheme[named=UBCblue]{structure}

\usepackage{color} % J'aime bien définir mes couleurs
\definecolor{propcolor}{rgb}{0, 0.5, 1}
\definecolor{thcolor}{rgb}{0.6, 0.07, 0.07}
\colorlet{louis}{blue!70!green!60!white}
\colorlet{sakura}{pink!40!red}

\title{Activités Mentales}
\date{24 Août 2023}

\newcommand{\vco}[2]{\begin{pmatrix} #1 \\ #2 \end{pmatrix}} %Coordonnées de vecteur
\newenvironment{eq}{\begin{cases}\begin{tabu}{ccccc}}{\end{tabu}\end{cases}}
\newenvironment{eql}{\begin{cases}\begin{tabu}{cccccl}}{\end{tabu}\end{cases}}
\newenvironment{eqrl}{\begin{cases}\begin{tabu}{rl}}{\end{tabu}\end{cases}}

\newenvironment{Eq}{\begin{center}\begin{tabular}{rrcl}}{\end{tabular}\end{center}}
\newcommand{\ligneq}[2]{$\Longleftrightarrow$ & $#1$ & $=$ & $#2$ \\}
\newcommand{\Ligneq}[2]{ & $#1$ & $=$ & $#2$ \\}

\newenvironment{RPN}{\begin{center}\begin{tabular}{rrclcrcl}}{\end{tabular}\end{center}}
\newcommand{\Lignerpn}[4]{ & $#1$ & $=$ & $#2$ & ou & $#3$ & $=$ & $#4$ \\}
\newcommand{\lignerpn}[4]{$\Longleftrightarrow$ & $#1$ & $=$ & $#2$ & ou & $#3$ & $=$ & $#4$ \\}

\newenvironment{TRPN}{\begin{center}\begin{tabular}{rrclcrclcrcl}}{\end{tabular}\end{center}}
\newcommand{\Lignetrpn}[6]{ & $#1$ & $=$ & $#2$ & ou & $#3$ & $=$ & $#4$ & ou & $#5$ & $=$ & $#6$ \\}
\newcommand{\lignetrpn}[6]{$\Longleftrightarrow$ & $#1$ & $=$ & $#2$ & ou & $#3$ & $=$ & $#4$ & ou & $#5$ & $=$ & $#6$ \\}
\begin{document}

\begin{frame}
    \titlepage
\end{frame}

\begin{frame} 
	\frametitle{Question 1}
Déterminer la mesure principale de l'angle $\dfrac{39\pi}{7}$ puis le placer sur le cercle trigonométrique.\end{frame}


\begin{frame} 
	\frametitle{Question 2}
Déterminer la mesure principale de l'angle $\dfrac{-63\pi}{8}$ puis le placer sur le cercle trigonométrique.\end{frame}


\begin{frame} 
	\frametitle{Question 3}
Déterminer la mesure principale de l'angle $\dfrac{93\pi}{4}$ puis le placer sur le cercle trigonométrique.\end{frame}


\begin{frame} 
	\frametitle{Question 4}
Déterminer la mesure principale de l'angle $\dfrac{26\pi}{5}$ puis le placer sur le cercle trigonométrique.\end{frame}


\begin{frame} 
	\frametitle{Question 5}
Déterminer la mesure principale de l'angle $\dfrac{73\pi}{7}$ puis le placer sur le cercle trigonométrique.\end{frame}


\begin{frame}
\vspace{-10mm}
	\frametitle{Correction 1}
\begin{minipage}{0.45 \linewidth}
	\begin{align*}
		\dfrac{39\pi}{7} &= \dfrac{39}{7}\times \dfrac{2\pi}{2} \\
		&=\dfrac{39}{14} \times 2 \pi\\
		&=\dfrac{(14\times 2+11) \times 2 \pi}{14}\\
		&=\dfrac{14\times 2 \times 2 \pi}{14}+\dfrac{11\times 2\pi}{14}\\
		&=2\times 2\pi+\dfrac{11\pi}{7}
	\end{align*}
\end{minipage}
\hfil
\begin{minipage}{0.5 \linewidth}
	\begin{tikzpicture}[scale = 0.65]
		\draw[thick] (0,0) circle (2);
		\draw[-{Straight Barb[length = 0.5mm]}] (-2.25,0) -- (2.25, 0);
		\draw[-{Straight Barb[length = 0.5mm]}] (0,-2.25) -- (0, 2.25);
		\begin{scope}[rotate = 51.42]
	\draw[dotted] (0,0) -- (2,0);
	\draw[thick] (1.9, 0) -- (2.1,0);
	\end{scope}

\begin{scope}[rotate = 77.13]
	\draw[dotted] (0,0) -- (2,0);
	\draw[thick] (1.9, 0) -- (2.1,0);
	\end{scope}

\begin{scope}[rotate = 102.84]
	\draw[dotted] (0,0) -- (2,0);
	\draw[thick] (1.9, 0) -- (2.1,0);
	\end{scope}

\begin{scope}[rotate = 128.55]
	\draw[dotted] (0,0) -- (2,0);
	\draw[thick] (1.9, 0) -- (2.1,0);
	\end{scope}

\begin{scope}[rotate = 154.26]
	\draw[dotted] (0,0) -- (2,0);
	\draw[thick] (1.9, 0) -- (2.1,0);
	\end{scope}

\begin{scope}[rotate = 179.97]
	\draw[dotted] (0,0) -- (2,0);
	\draw[thick] (1.9, 0) -- (2.1,0);
	\end{scope}

\begin{scope}[rotate = 205.68]
	\draw[dotted] (0,0) -- (2,0);
	\draw[thick] (1.9, 0) -- (2.1,0);
	\end{scope}

\begin{scope}[rotate = 231.39000000000001]
	\draw[dotted] (0,0) -- (2,0);
	\draw[thick] (1.9, 0) -- (2.1,0);
	\end{scope}

\begin{scope}[rotate = 257.1]
	\draw[dotted] (0,0) -- (2,0);
	\draw[thick] (1.9, 0) -- (2.1,0);
	\end{scope}

\begin{scope}[rotate = 308.52]
	\draw[dotted] (0,0) -- (2,0);
	\draw[thick] (1.9, 0) -- (2.1,0);
	\end{scope}

\begin{scope}[rotate = 334.23]
	\draw[dotted] (0,0) -- (2,0);
	\draw[thick] (1.9, 0) -- (2.1,0);
	\end{scope}

\begin{scope}[rotate = -77.13]
	\draw[dotted, louis, thick] (0,0) -- (2,0);
	\draw[louis, thick] (1.9, 0) -- (2.1,0);
	\draw[louis] (2.3, 0) node [below, right] {$\dfrac{39\pi}{7}=\dfrac{-3\pi}{7}$};
\end{scope}

\draw[dashed, louis, thick] ({2*cos(deg -3/7*pi)},0) -- ({2*cos(deg -3/7*pi)}, {2*sin(deg -3/7*pi)}) -- (0, {2*sin(deg -3/7*pi)});\begin{scope}[rotate = 25.71]
	\draw[thick, dotted, louis] (0,0) -- (2,0);
	\draw[thick, louis] (1.9, 0) -- (2.1,0) node[above right] {$\dfrac{\pi}{7}$};
\end{scope}

\draw[dashed, louis, thick] ({2*cos(deg 1/7*pi)},0) -- ({2*cos(deg 1/7*pi)}, {2*sin(deg 1/7*pi)}) -- (0, {2*sin(deg 1/7*pi)});\end{tikzpicture}
\end{minipage}

Or $\dfrac{11\pi}{7}>\pi$, on fait un tour de moins en retirant $2\pi$: $\dfrac{11\pi}{7}-2\pi = \dfrac{-3\pi}{7}$.

Comme $-\pi <-\dfrac{3\pi}{7}\leq \pi$, la mesure principale de $\dfrac{39\pi}{7}$ est $\dfrac{-3\pi}{7}$.\end{frame}


\begin{frame}
\vspace{-10mm}
	\frametitle{Correction 2}
\begin{minipage}{0.45 \linewidth}
	\begin{align*}
		\dfrac{-63\pi}{8} &= \dfrac{-63}{8}\times \dfrac{2\pi}{2} \\
		&=\dfrac{-63}{16} \times 2 \pi\\
		&=\dfrac{-(16\times 3+15) \times 2 \pi}{16}\\
		&=-\dfrac{16\times 3 \times 2 \pi}{16}-\dfrac{15\times 2\pi}{16}\\
		&=-3\times 2\pi-\dfrac{15\pi}{8}
	\end{align*}
\end{minipage}
\hfil
\begin{minipage}{0.5 \linewidth}
	\begin{tikzpicture}[scale = 0.65]
		\draw[thick] (0,0) circle (2);
		\draw[-{Straight Barb[length = 0.5mm]}] (-2.25,0) -- (2.25, 0);
		\draw[-{Straight Barb[length = 0.5mm]}] (0,-2.25) -- (0, 2.25);
		\begin{scope}[rotate = 45.0]
	\draw[dotted] (0,0) -- (2,0);
	\draw[thick] (1.9, 0) -- (2.1,0);
	\end{scope}

\begin{scope}[rotate = 67.5]
	\draw[dotted] (0,0) -- (2,0);
	\draw[thick] (1.9, 0) -- (2.1,0);
	\end{scope}

\begin{scope}[rotate = 90.0]
	\draw[dotted] (0,0) -- (2,0);
	\draw[thick] (1.9, 0) -- (2.1,0);
	\end{scope}

\begin{scope}[rotate = 112.5]
	\draw[dotted] (0,0) -- (2,0);
	\draw[thick] (1.9, 0) -- (2.1,0);
	\end{scope}

\begin{scope}[rotate = 135.0]
	\draw[dotted] (0,0) -- (2,0);
	\draw[thick] (1.9, 0) -- (2.1,0);
	\end{scope}

\begin{scope}[rotate = 157.5]
	\draw[dotted] (0,0) -- (2,0);
	\draw[thick] (1.9, 0) -- (2.1,0);
	\end{scope}

\begin{scope}[rotate = 202.5]
	\draw[dotted] (0,0) -- (2,0);
	\draw[thick] (1.9, 0) -- (2.1,0);
	\end{scope}

\begin{scope}[rotate = 225.0]
	\draw[dotted] (0,0) -- (2,0);
	\draw[thick] (1.9, 0) -- (2.1,0);
	\end{scope}

\begin{scope}[rotate = 247.5]
	\draw[dotted] (0,0) -- (2,0);
	\draw[thick] (1.9, 0) -- (2.1,0);
	\end{scope}

\begin{scope}[rotate = 270.0]
	\draw[dotted] (0,0) -- (2,0);
	\draw[thick] (1.9, 0) -- (2.1,0);
	\end{scope}

\begin{scope}[rotate = 292.5]
	\draw[dotted] (0,0) -- (2,0);
	\draw[thick] (1.9, 0) -- (2.1,0);
	\end{scope}

\begin{scope}[rotate = 315.0]
	\draw[dotted] (0,0) -- (2,0);
	\draw[thick] (1.9, 0) -- (2.1,0);
	\end{scope}

\begin{scope}[rotate = 337.5]
	\draw[dotted] (0,0) -- (2,0);
	\draw[thick] (1.9, 0) -- (2.1,0);
	\end{scope}

\begin{scope}[rotate = 22.5]
	\draw[dotted, louis, thick] (0,0) -- (2,0);
	\draw[louis, thick] (1.9, 0) -- (2.1,0);
	\draw[louis] (2.3, 0) node [above, right] {$\dfrac{-63\pi}{8}=\dfrac{\pi}{8}$};
\end{scope}

\draw[dashed, louis, thick] ({2*cos(deg 1/8*pi)},0) -- ({2*cos(deg 1/8*pi)}, {2*sin(deg 1/8*pi)}) -- (0, {2*sin(deg 1/8*pi)});\end{tikzpicture}
\end{minipage}

Or $-\dfrac{15\pi}{8}\leq-\pi$, on fait un tour de plus en rajoutant $2\pi$: $-\dfrac{15\pi}{8}+2\pi = \dfrac{\pi}{8}$.

Comme $-\pi <\dfrac{\pi}{8}\leq \pi$, la mesure principale de $\dfrac{-63\pi}{8}$ est $\dfrac{\pi}{8}$.\end{frame}


\begin{frame}
\vspace{-10mm}
	\frametitle{Correction 3}
\begin{minipage}{0.45 \linewidth}
	\begin{align*}
		\dfrac{93\pi}{4} &= \dfrac{93}{4}\times \dfrac{2\pi}{2} \\
		&=\dfrac{93}{8} \times 2 \pi\\
		&=\dfrac{(8\times 11+5) \times 2 \pi}{8}\\
		&=\dfrac{8\times 11 \times 2 \pi}{8}+\dfrac{5\times 2\pi}{8}\\
		&=11\times 2\pi+\dfrac{5\pi}{4}
	\end{align*}
\end{minipage}
\hfil
\begin{minipage}{0.5 \linewidth}
	\begin{tikzpicture}[scale = 0.65]
		\draw[thick] (0,0) circle (2);
		\draw[-{Straight Barb[length = 0.5mm]}] (-2.25,0) -- (2.25, 0);
		\draw[-{Straight Barb[length = 0.5mm]}] (0,-2.25) -- (0, 2.25);
		\begin{scope}[rotate = 90.0]
	\draw[dotted] (0,0) -- (2,0);
	\draw[thick] (1.9, 0) -- (2.1,0);
	\end{scope}

\begin{scope}[rotate = 135.0]
	\draw[dotted] (0,0) -- (2,0);
	\draw[thick] (1.9, 0) -- (2.1,0);
	\end{scope}

\begin{scope}[rotate = 270.0]
	\draw[dotted] (0,0) -- (2,0);
	\draw[thick] (1.9, 0) -- (2.1,0);
	\end{scope}

\begin{scope}[rotate = 315.0]
	\draw[dotted] (0,0) -- (2,0);
	\draw[thick] (1.9, 0) -- (2.1,0);
	\end{scope}

\begin{scope}[rotate = -135.0]
	\draw[dotted, louis, thick] (0,0) -- (2,0);
	\draw[louis, thick] (1.9, 0) -- (2.1,0);
	\draw[louis] (2.3, 0) node [below, left] {$\dfrac{93\pi}{4}=\dfrac{-3\pi}{4}$};
\end{scope}

\draw[dashed, louis, thick] ({2*cos(deg -3/4*pi)},0) -- ({2*cos(deg -3/4*pi)}, {2*sin(deg -3/4*pi)}) -- (0, {2*sin(deg -3/4*pi)});\begin{scope}[rotate = 45.0]
	\draw[thick, dotted, louis] (0,0) -- (2,0);
	\draw[thick, louis] (1.9, 0) -- (2.1,0) node[above right] {$\dfrac{\pi}{4}$};
\end{scope}

\draw[dashed, louis, thick] ({2*cos(deg 1/4*pi)},0) -- ({2*cos(deg 1/4*pi)}, {2*sin(deg 1/4*pi)}) -- (0, {2*sin(deg 1/4*pi)});\end{tikzpicture}
\end{minipage}

Or $\dfrac{5\pi}{4}>\pi$, on fait un tour de moins en retirant $2\pi$: $\dfrac{5\pi}{4}-2\pi = \dfrac{-3\pi}{4}$.

Comme $-\pi <-\dfrac{3\pi}{4}\leq \pi$, la mesure principale de $\dfrac{93\pi}{4}$ est $\dfrac{-3\pi}{4}$.\end{frame}


\begin{frame}
\vspace{-10mm}
	\frametitle{Correction 4}
\begin{minipage}{0.45 \linewidth}
	\begin{align*}
		\dfrac{26\pi}{5} &= \dfrac{26}{5}\times \dfrac{2\pi}{2} \\
		&=\dfrac{26}{10} \times 2 \pi\\
		&=\dfrac{(10\times 2+6) \times 2 \pi}{10}\\
		&=\dfrac{10\times 2 \times 2 \pi}{10}+\dfrac{6\times 2\pi}{10}\\
		&=2\times 2\pi+\dfrac{6\pi}{5}
	\end{align*}
\end{minipage}
\hfil
\begin{minipage}{0.5 \linewidth}
	\begin{tikzpicture}[scale = 0.65]
		\draw[thick] (0,0) circle (2);
		\draw[-{Straight Barb[length = 0.5mm]}] (-2.25,0) -- (2.25, 0);
		\draw[-{Straight Barb[length = 0.5mm]}] (0,-2.25) -- (0, 2.25);
		\begin{scope}[rotate = 72.0]
	\draw[dotted] (0,0) -- (2,0);
	\draw[thick] (1.9, 0) -- (2.1,0);
	\end{scope}

\begin{scope}[rotate = 108.0]
	\draw[dotted] (0,0) -- (2,0);
	\draw[thick] (1.9, 0) -- (2.1,0);
	\end{scope}

\begin{scope}[rotate = 144.0]
	\draw[dotted] (0,0) -- (2,0);
	\draw[thick] (1.9, 0) -- (2.1,0);
	\end{scope}

\begin{scope}[rotate = 252.0]
	\draw[dotted] (0,0) -- (2,0);
	\draw[thick] (1.9, 0) -- (2.1,0);
	\end{scope}

\begin{scope}[rotate = 288.0]
	\draw[dotted] (0,0) -- (2,0);
	\draw[thick] (1.9, 0) -- (2.1,0);
	\end{scope}

\begin{scope}[rotate = 324.0]
	\draw[dotted] (0,0) -- (2,0);
	\draw[thick] (1.9, 0) -- (2.1,0);
	\end{scope}

\begin{scope}[rotate = -144.0]
	\draw[dotted, louis, thick] (0,0) -- (2,0);
	\draw[louis, thick] (1.9, 0) -- (2.1,0);
	\draw[louis] (2.3, 0) node [below, left] {$\dfrac{26\pi}{5}=\dfrac{-4\pi}{5}$};
\end{scope}

\draw[dashed, louis, thick] ({2*cos(deg -4/5*pi)},0) -- ({2*cos(deg -4/5*pi)}, {2*sin(deg -4/5*pi)}) -- (0, {2*sin(deg -4/5*pi)});\begin{scope}[rotate = 36.0]
	\draw[thick, dotted, louis] (0,0) -- (2,0);
	\draw[thick, louis] (1.9, 0) -- (2.1,0) node[above right] {$\dfrac{\pi}{5}$};
\end{scope}

\draw[dashed, louis, thick] ({2*cos(deg 1/5*pi)},0) -- ({2*cos(deg 1/5*pi)}, {2*sin(deg 1/5*pi)}) -- (0, {2*sin(deg 1/5*pi)});\end{tikzpicture}
\end{minipage}

Or $\dfrac{6\pi}{5}>\pi$, on fait un tour de moins en retirant $2\pi$: $\dfrac{6\pi}{5}-2\pi = \dfrac{-4\pi}{5}$.

Comme $-\pi <-\dfrac{4\pi}{5}\leq \pi$, la mesure principale de $\dfrac{26\pi}{5}$ est $\dfrac{-4\pi}{5}$.\end{frame}


\begin{frame}
\vspace{-10mm}
	\frametitle{Correction 5}
\begin{minipage}{0.45 \linewidth}
	\begin{align*}
		\dfrac{73\pi}{7} &= \dfrac{73}{7}\times \dfrac{2\pi}{2} \\
		&=\dfrac{73}{14} \times 2 \pi\\
		&=\dfrac{(14\times 5+3) \times 2 \pi}{14}\\
		&=\dfrac{14\times 5 \times 2 \pi}{14}+\dfrac{3\times 2\pi}{14}\\
		&=5\times 2\pi+\dfrac{3\pi}{7}
	\end{align*}
\end{minipage}
\hfil
\begin{minipage}{0.5 \linewidth}
	\begin{tikzpicture}[scale = 0.65]
		\draw[thick] (0,0) circle (2);
		\draw[-{Straight Barb[length = 0.5mm]}] (-2.25,0) -- (2.25, 0);
		\draw[-{Straight Barb[length = 0.5mm]}] (0,-2.25) -- (0, 2.25);
		\begin{scope}[rotate = 51.42]
	\draw[dotted] (0,0) -- (2,0);
	\draw[thick] (1.9, 0) -- (2.1,0);
	\end{scope}

\begin{scope}[rotate = 102.84]
	\draw[dotted] (0,0) -- (2,0);
	\draw[thick] (1.9, 0) -- (2.1,0);
	\end{scope}

\begin{scope}[rotate = 128.55]
	\draw[dotted] (0,0) -- (2,0);
	\draw[thick] (1.9, 0) -- (2.1,0);
	\end{scope}

\begin{scope}[rotate = 154.26]
	\draw[dotted] (0,0) -- (2,0);
	\draw[thick] (1.9, 0) -- (2.1,0);
	\end{scope}

\begin{scope}[rotate = 179.97]
	\draw[dotted] (0,0) -- (2,0);
	\draw[thick] (1.9, 0) -- (2.1,0);
	\end{scope}

\begin{scope}[rotate = 205.68]
	\draw[dotted] (0,0) -- (2,0);
	\draw[thick] (1.9, 0) -- (2.1,0);
	\end{scope}

\begin{scope}[rotate = 231.39000000000001]
	\draw[dotted] (0,0) -- (2,0);
	\draw[thick] (1.9, 0) -- (2.1,0);
	\end{scope}

\begin{scope}[rotate = 257.1]
	\draw[dotted] (0,0) -- (2,0);
	\draw[thick] (1.9, 0) -- (2.1,0);
	\end{scope}

\begin{scope}[rotate = 282.81]
	\draw[dotted] (0,0) -- (2,0);
	\draw[thick] (1.9, 0) -- (2.1,0);
	\end{scope}

\begin{scope}[rotate = 308.52]
	\draw[dotted] (0,0) -- (2,0);
	\draw[thick] (1.9, 0) -- (2.1,0);
	\end{scope}

\begin{scope}[rotate = 334.23]
	\draw[dotted] (0,0) -- (2,0);
	\draw[thick] (1.9, 0) -- (2.1,0);
	\end{scope}

\begin{scope}[rotate = 77.13]
	\draw[dotted, louis, thick] (0,0) -- (2,0);
	\draw[louis, thick] (1.9, 0) -- (2.1,0);
	\draw[louis] (2.3, 0) node [above, right] {$\dfrac{73\pi}{7}=\dfrac{3\pi}{7}$};
\end{scope}

\draw[dashed, louis, thick] ({2*cos(deg 3/7*pi)},0) -- ({2*cos(deg 3/7*pi)}, {2*sin(deg 3/7*pi)}) -- (0, {2*sin(deg 3/7*pi)});\begin{scope}[rotate = 25.71]
	\draw[thick, dotted, louis] (0,0) -- (2,0);
	\draw[thick, louis] (1.9, 0) -- (2.1,0) node[above right] {$\dfrac{\pi}{7}$};
\end{scope}

\draw[dashed, louis, thick] ({2*cos(deg 1/7*pi)},0) -- ({2*cos(deg 1/7*pi)}, {2*sin(deg 1/7*pi)}) -- (0, {2*sin(deg 1/7*pi)});\end{tikzpicture}
\end{minipage}

Comme $-\pi < \dfrac{3\pi}{7}\leq \pi$, la mesure principale de $\dfrac{73\pi}{7}$ est $\dfrac{-3\pi}{7}$.\end{frame}




\end{document}