\documentclass[15pt, mathserif]{beamer}

\usepackage[french]{babel}
\usepackage[T1]{fontenc}
\usepackage[utf8]{inputenc}
%\usepackage{esvect}
\usepackage{bm}
\usepackage{eurosym}
\usepackage{tikz}
\usepackage{pgf,tikz,pgfplots}
\pgfplotsset{compat=1.15}
\usepackage{mathrsfs}
\usetikzlibrary{arrows}
\usetikzlibrary{arrows.meta}

\usetikzlibrary{mindmap}
\usepackage{multicol}
\usepackage[tikz]{bclogo}
\usepackage{tkz-tab}
\usepackage{amsmath, tabu}
\usepackage{esvect} %\vv{AB} pour le vecteur AB

\DeclareMathOperator{\e}{e}

%% Tableau

\usepackage{makecell}
\setcellgapes{1pt}
\makegapedcells
\newcolumntype{R}[1]{>{\raggedleft\arraybackslash }b{#1}}
\newcolumntype{L}[1]{>{\raggedright\arraybackslash }b{#1}}
\newcolumntype{C}[1]{>{\centering\arraybackslash }b{#1}}


%pour avoir des parenthèses rondes dans le package fourier
\DeclareSymbolFont{cmoperators}   {OT1}{cmr} {m}{n}
\DeclareSymbolFont{cmlargesymbols}{OMX}{cmex}{m}{n}

\usefonttheme{professionalfonts} %permet d'enlever un bug avec fourier
\usepackage{fourier}
\DeclareMathDelimiter{(}{\mathopen} {cmoperators}{"28}{cmlargesymbols}{"00}
\DeclareMathDelimiter{)}{\mathclose}{cmoperators}{"29}{cmlargesymbols}{"01}

%Graphiques 

\usepackage{pgf,tikz,pgfplots}
\pgfplotsset{compat=1.15}
\usepackage{mathrsfs}
\usetikzlibrary{arrows}
\usetikzlibrary{mindmap}

%ensembles de nbres

\newcommand{\R}{\mathbb{R}}			%permet d'écrire le R "ensemble des réels"'
\newcommand{\N}{\mathbb{N}}			%permet d'écrire le N "ensemble des entiers naturels"
\newcommand{\Z}{\mathbb{Z}}			%permet d'écrire le Z "ensemble des entiers relatifs"
\newcommand{\Prem}{\mathbb{P}}	%permet d'écrire le P "ensemble des nombres premiers" (qui n'a pas marché avec le \P car il existe déjà)
\newcommand{\D}{\mathbb{D}}
\newcommand{\Df}{\mathcal{D}_f}
\newcommand{\Cf}{\mathcal{C}_f}

\newcommand{\Q}{\mathbb{Q}}


\newcommand{\st}[1]{$(#1_n)_{n \in \N}$}

\usetheme{Madrid}
\useoutertheme{miniframes} % Alternatively: miniframes, infolines, split
\useinnertheme{circles}
\definecolor{UBCblue}{rgb}{0.1, 0.25, 0.4} % UBC Blue (primary)
\definecolor{bordeaux}{RGB}{128,0,0}
\usecolortheme[named=UBCblue]{structure}

\usepackage{color} % J'aime bien définir mes couleurs
\definecolor{propcolor}{rgb}{0, 0.5, 1}
\definecolor{thcolor}{rgb}{0.6, 0.07, 0.07}
\colorlet{louis}{blue!70!green!60!white}
\colorlet{sakura}{pink!40!red}

\title{Activités Mentales}
\date{24 Août 2023}

\newcommand{\vco}[2]{\begin{pmatrix} #1 \\ #2 \end{pmatrix}} %Coordonnées de vecteur
\newenvironment{eq}{\begin{cases}\begin{tabu}{ccccc}}{\end{tabu}\end{cases}}
\newenvironment{eql}{\begin{cases}\begin{tabu}{cccccl}}{\end{tabu}\end{cases}}
\newenvironment{eqrl}{\begin{cases}\begin{tabu}{rl}}{\end{tabu}\end{cases}}

\newenvironment{Eq}{\begin{center}\begin{tabular}{rrcl}}{\end{tabular}\end{center}}
\newcommand{\ligneq}[2]{$\Longleftrightarrow$ & $#1$ & $=$ & $#2$ \\}
\newcommand{\Ligneq}[2]{ & $#1$ & $=$ & $#2$ \\}

\newenvironment{RPN}{\begin{center}\begin{tabular}{rrclcrcl}}{\end{tabular}\end{center}}
\newcommand{\Lignerpn}[4]{ & $#1$ & $=$ & $#2$ & ou & $#3$ & $=$ & $#4$ \\}
\newcommand{\lignerpn}[4]{$\Longleftrightarrow$ & $#1$ & $=$ & $#2$ & ou & $#3$ & $=$ & $#4$ \\}

\newenvironment{TRPN}{\begin{center}\begin{tabular}{rrclcrclcrcl}}{\end{tabular}\end{center}}
\newcommand{\Lignetrpn}[6]{ & $#1$ & $=$ & $#2$ & ou & $#3$ & $=$ & $#4$ & ou & $#5$ & $=$ & $#6$ \\}
\newcommand{\lignetrpn}[6]{$\Longleftrightarrow$ & $#1$ & $=$ & $#2$ & ou & $#3$ & $=$ & $#4$ & ou & $#5$ & $=$ & $#6$ \\}
\begin{document}

\begin{frame}
    \titlepage
\end{frame}

\begin{frame} 
	\frametitle{Question 1}
On considère la suite définie sur $\N$ par $u_n=\dfrac{-n+1}{n+5}$. 
 
 Exprimer $u_{n+1}$ en fonction de $n$.\end{frame}


\begin{frame} 
	\frametitle{Question 2}
On considère la suite définie sur $\N$ par $u_n=\dfrac{2n^2+4}{7n^2+10}$. 
 
 Exprimer $u_{n+1}$ en fonction de $n$.\end{frame}


\begin{frame} 
	\frametitle{Question 3}
On considère la suite définie sur $\N$ par $u_n=-6n^2-3n-10$. 
 
 Exprimer $u_{n+1}$ en fonction de $n$.\end{frame}


\begin{frame} 
	\frametitle{Question 4}
On considère la suite définie sur $\N$ par $u_n=\dfrac{-10n+5}{7n+6}$. 
 
 Exprimer $u_{n+1}$ en fonction de $n$.\end{frame}


\begin{frame} 
	\frametitle{Question 5}
On considère la suite définie sur $\N$ par $u_n=\dfrac{n-6}{6n+3}$. 
 
 Exprimer $u_{n+1}$ en fonction de $n$.\end{frame}


\begin{frame}
\vspace{-10mm}
	\frametitle{Correction 1}
Comme $u_{\textcolor{blue}{n}}=\dfrac{-\textcolor{blue}{n}+1}{\textcolor{blue}{n}+5}$. Alors on a \begin{align*} 
 u_{\textcolor{blue}{n+1}} &=\dfrac{-\textcolor{blue}{(n+1)}+1}{\textcolor{blue}{(n+1)}+5} \\ 
 &= \dfrac{-n-1+1}{n+1+5} \\ 
 &= \dfrac{-n}{n+6}
 \end{align*}\end{frame}


\begin{frame}
\vspace{-10mm}
	\frametitle{Correction 2}
Comme $u_{\textcolor{blue}{n}}=\dfrac{2\textcolor{blue}{n}^2+4}{7\textcolor{blue}{n}^2+10}$. Alors on a \begin{align*} 
 u_{\textcolor{blue}{n+1}} &=\dfrac{2\textcolor{blue}{(n+1)}^2+4}{7\textcolor{blue}{(n+1)}^2+10} \\ 
 &= \dfrac{2(n^2+2n+1)+4}{7(n^2+2n+1)+10} \\ 
 &= \dfrac{2n^2+4n+6}{7n^2+14n+17}
 \end{align*}\end{frame}


\begin{frame}
\vspace{-10mm}
	\frametitle{Correction 3}
Comme $u_{\textcolor{blue}{n}}=-6{\textcolor{blue}{n}}^2-3{\textcolor{blue}{n}}-10$. Alors on a \begin{align*} 
 u_{{\textcolor{blue}{n+1}}} &=-6{\textcolor{blue}{(n+1)}}^2-3{\textcolor{blue}{(n+1)}}-10 \\ 
 &= -6(n^2+2n+1)-3n-3-10 \\ 
 &= -6n^2 -12n -6-3n-13\\ 
 &= -6n^2 -15n -19
 \end{align*}\end{frame}


\begin{frame}
\vspace{-10mm}
	\frametitle{Correction 4}
Comme $u_{\textcolor{blue}{n}}=\dfrac{-10\textcolor{blue}{n}+5}{7\textcolor{blue}{n}+6}$. Alors on a \begin{align*} 
 u_{\textcolor{blue}{n+1}} &=\dfrac{-10\textcolor{blue}{(n+1)}+5}{7\textcolor{blue}{(n+1)}+6} \\ 
 &= \dfrac{-10n-10+5}{7n+7+6} \\ 
 &= \dfrac{-10n-5}{7n+13}
 \end{align*}\end{frame}


\begin{frame}
\vspace{-10mm}
	\frametitle{Correction 5}
Comme $u_{\textcolor{blue}{n}}=\dfrac{\textcolor{blue}{n}-6}{6\textcolor{blue}{n}+3}$. Alors on a \begin{align*} 
 u_{\textcolor{blue}{n+1}} &=\dfrac{\textcolor{blue}{(n+1)}-6}{6\textcolor{blue}{(n+1)}+3} \\ 
 &= \dfrac{n+1-6}{6n+6+3} \\ 
 &= \dfrac{n-5}{6n+9}
 \end{align*}\end{frame}




\end{document}