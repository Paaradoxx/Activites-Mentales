\documentclass[15pt, mathserif]{beamer}

\usepackage[french]{babel}
\usepackage[T1]{fontenc}
\usepackage[utf8]{inputenc}
%\usepackage{esvect}
\usepackage{bm}
\usepackage{eurosym}
\usepackage{tikz}
\usepackage{pgf,tikz,pgfplots}
\pgfplotsset{compat=1.15}
\usepackage{mathrsfs}
\usetikzlibrary{arrows}
\usetikzlibrary{arrows.meta}

\usetikzlibrary{mindmap}
\usepackage{multicol}
\usepackage[tikz]{bclogo}
\usepackage{tkz-tab}
\usepackage{amsmath, tabu}
\usepackage{esvect} %\vv{AB} pour le vecteur AB

\DeclareMathOperator{\e}{e}

%% Tableau

\usepackage{makecell}
\setcellgapes{1pt}
\makegapedcells
\newcolumntype{R}[1]{>{\raggedleft\arraybackslash }b{#1}}
\newcolumntype{L}[1]{>{\raggedright\arraybackslash }b{#1}}
\newcolumntype{C}[1]{>{\centering\arraybackslash }b{#1}}


%pour avoir des parenthèses rondes dans le package fourier
\DeclareSymbolFont{cmoperators}   {OT1}{cmr} {m}{n}
\DeclareSymbolFont{cmlargesymbols}{OMX}{cmex}{m}{n}

\usefonttheme{professionalfonts} %permet d'enlever un bug avec fourier
\usepackage{fourier}
\DeclareMathDelimiter{(}{\mathopen} {cmoperators}{"28}{cmlargesymbols}{"00}
\DeclareMathDelimiter{)}{\mathclose}{cmoperators}{"29}{cmlargesymbols}{"01}

%Graphiques 

\usepackage{pgf,tikz,pgfplots}
\pgfplotsset{compat=1.15}
\usepackage{mathrsfs}
\usetikzlibrary{arrows}
\usetikzlibrary{mindmap}

%ensembles de nbres

\newcommand{\R}{\mathbb{R}}			%permet d'écrire le R "ensemble des réels"'
\newcommand{\N}{\mathbb{N}}			%permet d'écrire le N "ensemble des entiers naturels"
\newcommand{\Z}{\mathbb{Z}}			%permet d'écrire le Z "ensemble des entiers relatifs"
\newcommand{\Prem}{\mathbb{P}}	%permet d'écrire le P "ensemble des nombres premiers" (qui n'a pas marché avec le \P car il existe déjà)
\newcommand{\D}{\mathbb{D}}
\newcommand{\Df}{\mathcal{D}_f}
\newcommand{\Cf}{\mathcal{C}_f}

\newcommand{\Q}{\mathbb{Q}}


\newcommand{\st}[1]{$(#1_n)_{n \in \N}$}

\usetheme{Madrid}
\useoutertheme{miniframes} % Alternatively: miniframes, infolines, split
\useinnertheme{circles}
\definecolor{UBCblue}{rgb}{0.1, 0.25, 0.4} % UBC Blue (primary)
\definecolor{bordeaux}{RGB}{128,0,0}
\usecolortheme[named=UBCblue]{structure}

\usepackage{color} % J'aime bien définir mes couleurs
\definecolor{propcolor}{rgb}{0, 0.5, 1}
\definecolor{thcolor}{rgb}{0.6, 0.07, 0.07}
\colorlet{louis}{blue!70!green!60!white}
\colorlet{sakura}{pink!40!red}

\title{Activités Mentales}
\date{24 Août 2023}

\newcommand{\vco}[2]{\begin{pmatrix} #1 \\ #2 \end{pmatrix}} %Coordonnées de vecteur
\newenvironment{eq}{\begin{cases}\begin{tabu}{ccccc}}{\end{tabu}\end{cases}}
\newenvironment{eql}{\begin{cases}\begin{tabu}{cccccl}}{\end{tabu}\end{cases}}
\newenvironment{eqrl}{\begin{cases}\begin{tabu}{rl}}{\end{tabu}\end{cases}}

\newenvironment{Eq}{\begin{center}\begin{tabular}{rrcl}}{\end{tabular}\end{center}}
\newcommand{\ligneq}[2]{$\Longleftrightarrow$ & $#1$ & $=$ & $#2$ \\}
\newcommand{\Ligneq}[2]{ & $#1$ & $=$ & $#2$ \\}

\newenvironment{RPN}{\begin{center}\begin{tabular}{rrclcrcl}}{\end{tabular}\end{center}}
\newcommand{\Lignerpn}[4]{ & $#1$ & $=$ & $#2$ & ou & $#3$ & $=$ & $#4$ \\}
\newcommand{\lignerpn}[4]{$\Longleftrightarrow$ & $#1$ & $=$ & $#2$ & ou & $#3$ & $=$ & $#4$ \\}

\newenvironment{TRPN}{\begin{center}\begin{tabular}{rrclcrclcrcl}}{\end{tabular}\end{center}}
\newcommand{\Lignetrpn}[6]{ & $#1$ & $=$ & $#2$ & ou & $#3$ & $=$ & $#4$ & ou & $#5$ & $=$ & $#6$ \\}
\newcommand{\lignetrpn}[6]{$\Longleftrightarrow$ & $#1$ & $=$ & $#2$ & ou & $#3$ & $=$ & $#4$ & ou & $#5$ & $=$ & $#6$ \\}
\begin{document}

\begin{frame}
    \titlepage
\end{frame}

\begin{frame} 
	\frametitle{Question 1}
	Résoudre dans $\mathbb{R}$ l'équation suivante: \[(E):~ 8x-4=8x-7\]\end{frame}


\begin{frame} 
	\frametitle{Question 2}
	Résoudre dans $\mathbb{R}$ l'équation suivante: \[(E):~ -8x+4=-4x-2\]\end{frame}


\begin{frame} 
	\frametitle{Question 3}
	Résoudre dans $\mathbb{R}$ l'équation suivante: \[(E):~ 7x+6=-7x+4\]\end{frame}


\begin{frame} 
	\frametitle{Question 4}
	Résoudre dans $\mathbb{R}$ l'équation suivante: \[(E):~ -7x+6=-10x-2\]\end{frame}


\begin{frame} 
	\frametitle{Question 5}
	Résoudre dans $\mathbb{R}$ l'équation suivante: \[(E):~ 3x+6=7x+1\]\end{frame}


\begin{frame}
\vspace{-10mm}
	\frametitle{Correction 1}
	\begin{align*} (E)& \Leftrightarrow 8x-4=8x-7\\
		&\Leftrightarrow 8x-4-8x=8x-7-8x\\
		&\Leftrightarrow -4=-7
	\end{align*}
	L'équation n'est jamais vérifiée.

Finalement l'ensemble des solutions de $(E)$ est $S = \emptyset$.
\end{frame}


\begin{frame}
\vspace{-10mm}
	\frametitle{Correction 2}
	\begin{align*} (E)& \Leftrightarrow -8x+4=-4x-2\\
		&\Leftrightarrow -8x+4+4x=-4x-2+4x\\
		&\Leftrightarrow -4x+4=-2\\
		&\Leftrightarrow -4x+4-4=-2-4\\
		&\Leftrightarrow -4x=-6\\
		&\Leftrightarrow \dfrac{-4x}{-4}=\dfrac{-6}{-4} \\
		&\Leftrightarrow x= \dfrac{3}{2}
	\end{align*}
	Finalement l'ensemble des solutions de $(E)$ est $S = \left\{\dfrac{3}{2}\right\}$.
\end{frame}


\begin{frame}
\vspace{-10mm}
	\frametitle{Correction 3}
	\begin{align*} (E)& \Leftrightarrow 7x+6=-7x+4\\
		&\Leftrightarrow 7x+6+7x=-7x+4+7x\\
		&\Leftrightarrow 14x+6=4\\
		&\Leftrightarrow 14x+6-6=4-6\\
		&\Leftrightarrow 14x=-2\\
		&\Leftrightarrow \dfrac{14x}{14}=\dfrac{-2}{14} \\
		&\Leftrightarrow x= \dfrac{-1}{7}
	\end{align*}
	Finalement l'ensemble des solutions de $(E)$ est $S = \left\{\dfrac{-1}{7}\right\}$.
\end{frame}


\begin{frame}
\vspace{-10mm}
	\frametitle{Correction 4}
	\begin{align*} (E)& \Leftrightarrow -7x+6=-10x-2\\
		&\Leftrightarrow -7x+6+10x=-10x-2+10x\\
		&\Leftrightarrow 3x+6=-2\\
		&\Leftrightarrow 3x+6-6=-2-6\\
		&\Leftrightarrow 3x=-8\\
		&\Leftrightarrow \dfrac{3x}{3}=\dfrac{-8}{3} \\
		&\Leftrightarrow x= \dfrac{-8}{3}
	\end{align*}
	Finalement l'ensemble des solutions de $(E)$ est $S = \left\{\dfrac{-8}{3}\right\}$.
\end{frame}


\begin{frame}
\vspace{-10mm}
	\frametitle{Correction 5}
	\begin{align*} (E)& \Leftrightarrow 3x+6=7x+1\\
		&\Leftrightarrow 3x+6-7x=7x+1-7x\\
		&\Leftrightarrow -4x+6=1\\
		&\Leftrightarrow -4x+6-6=1-6\\
		&\Leftrightarrow -4x=-5\\
		&\Leftrightarrow \dfrac{-4x}{-4}=\dfrac{-5}{-4} \\
		&\Leftrightarrow x= \dfrac{5}{4}
	\end{align*}
	Finalement l'ensemble des solutions de $(E)$ est $S = \left\{\dfrac{5}{4}\right\}$.
\end{frame}




\end{document}