\documentclass[15pt, mathserif]{beamer}

\usepackage[french]{babel}
\usepackage[T1]{fontenc}
\usepackage[utf8]{inputenc}
%\usepackage{esvect}
\usepackage{bm}
\usepackage{eurosym}
\usepackage{tikz}
\usepackage{pgf,tikz,pgfplots}
\pgfplotsset{compat=1.15}
\usepackage{mathrsfs}
\usetikzlibrary{arrows}
\usetikzlibrary{arrows.meta}

\usetikzlibrary{mindmap}
\usepackage{multicol}
\usepackage[tikz]{bclogo}
\usepackage{tkz-tab}
\usepackage{amsmath, tabu}
\usepackage{esvect} %\vv{AB} pour le vecteur AB

\DeclareMathOperator{\e}{e}

%% Tableau

\usepackage{makecell}
\setcellgapes{1pt}
\makegapedcells
\newcolumntype{R}[1]{>{\raggedleft\arraybackslash }b{#1}}
\newcolumntype{L}[1]{>{\raggedright\arraybackslash }b{#1}}
\newcolumntype{C}[1]{>{\centering\arraybackslash }b{#1}}


%pour avoir des parenthèses rondes dans le package fourier
\DeclareSymbolFont{cmoperators}   {OT1}{cmr} {m}{n}
\DeclareSymbolFont{cmlargesymbols}{OMX}{cmex}{m}{n}

\usefonttheme{professionalfonts} %permet d'enlever un bug avec fourier
\usepackage{fourier}
\DeclareMathDelimiter{(}{\mathopen} {cmoperators}{"28}{cmlargesymbols}{"00}
\DeclareMathDelimiter{)}{\mathclose}{cmoperators}{"29}{cmlargesymbols}{"01}

%Graphiques 

\usepackage{pgf,tikz,pgfplots}
\pgfplotsset{compat=1.15}
\usepackage{mathrsfs}
\usetikzlibrary{arrows}
\usetikzlibrary{mindmap}

%ensembles de nbres

\newcommand{\R}{\mathbb{R}}			%permet d'écrire le R "ensemble des réels"'
\newcommand{\N}{\mathbb{N}}			%permet d'écrire le N "ensemble des entiers naturels"
\newcommand{\Z}{\mathbb{Z}}			%permet d'écrire le Z "ensemble des entiers relatifs"
\newcommand{\Prem}{\mathbb{P}}	%permet d'écrire le P "ensemble des nombres premiers" (qui n'a pas marché avec le \P car il existe déjà)
\newcommand{\D}{\mathbb{D}}
\newcommand{\Df}{\mathcal{D}_f}
\newcommand{\Cf}{\mathcal{C}_f}

\newcommand{\Q}{\mathbb{Q}}


\newcommand{\st}[1]{$(#1_n)_{n \in \N}$}

\usetheme{Madrid}
\useoutertheme{miniframes} % Alternatively: miniframes, infolines, split
\useinnertheme{circles}
\definecolor{UBCblue}{rgb}{0.1, 0.25, 0.4} % UBC Blue (primary)
\definecolor{bordeaux}{RGB}{128,0,0}
\usecolortheme[named=UBCblue]{structure}

\usepackage{color} % J'aime bien définir mes couleurs
\definecolor{propcolor}{rgb}{0, 0.5, 1}
\definecolor{thcolor}{rgb}{0.6, 0.07, 0.07}
\colorlet{louis}{blue!70!green!60!white}
\colorlet{sakura}{pink!40!red}

\title{Activités Mentales}
\date{24 Août 2023}

\newcommand{\vco}[2]{\begin{pmatrix} #1 \\ #2 \end{pmatrix}} %Coordonnées de vecteur
\newenvironment{eq}{\begin{cases}\begin{tabu}{ccccc}}{\end{tabu}\end{cases}}
\newenvironment{eql}{\begin{cases}\begin{tabu}{cccccl}}{\end{tabu}\end{cases}}
\newenvironment{eqrl}{\begin{cases}\begin{tabu}{rl}}{\end{tabu}\end{cases}}

\newenvironment{Eq}{\begin{center}\begin{tabular}{rrcl}}{\end{tabular}\end{center}}
\newcommand{\ligneq}[2]{$\Longleftrightarrow$ & $#1$ & $=$ & $#2$ \\}
\newcommand{\Ligneq}[2]{ & $#1$ & $=$ & $#2$ \\}

\newenvironment{RPN}{\begin{center}\begin{tabular}{rrclcrcl}}{\end{tabular}\end{center}}
\newcommand{\Lignerpn}[4]{ & $#1$ & $=$ & $#2$ & ou & $#3$ & $=$ & $#4$ \\}
\newcommand{\lignerpn}[4]{$\Longleftrightarrow$ & $#1$ & $=$ & $#2$ & ou & $#3$ & $=$ & $#4$ \\}

\newenvironment{TRPN}{\begin{center}\begin{tabular}{rrclcrclcrcl}}{\end{tabular}\end{center}}
\newcommand{\Lignetrpn}[6]{ & $#1$ & $=$ & $#2$ & ou & $#3$ & $=$ & $#4$ & ou & $#5$ & $=$ & $#6$ \\}
\newcommand{\lignetrpn}[6]{$\Longleftrightarrow$ & $#1$ & $=$ & $#2$ & ou & $#3$ & $=$ & $#4$ & ou & $#5$ & $=$ & $#6$ \\}
\begin{document}

\begin{frame}
    \titlepage
\end{frame}

\begin{frame} 
	\frametitle{Question 1}
Soit $(u_n)_{n\in\mathbb{N}}$ une suite arithmétique de raison $r = -6$ et de premier terme $u_1 = -2$.

Calculer $\displaystyle\sum_{k=27}^{39}u_k$.\end{frame}


\begin{frame} 
	\frametitle{Question 2}
Soit $(u_n)_{n\in\mathbb{N}}$ une suite arithmétique de raison $r = 13$ et de premier terme $u_1 = -15$.

Calculer $\displaystyle\sum_{k=35}^{75}u_k$.\end{frame}


\begin{frame} 
	\frametitle{Question 3}
Soit $(u_n)_{n\in\mathbb{N}}$ une suite arithmétique de raison $r = -3$ et de premier terme $u_3 = 25$.

Calculer $\displaystyle\sum_{k=14}^{43}u_k$.\end{frame}


\begin{frame} 
	\frametitle{Question 4}
Soit $(u_n)_{n\in\mathbb{N}}$ une suite arithmétique de raison $r = 22$ et de premier terme $u_0 = -28$.

Calculer $\displaystyle\sum_{k=28}^{68}u_k$.\end{frame}


\begin{frame} 
	\frametitle{Question 5}
Soit $(u_n)_{n\in\mathbb{N}}$ une suite arithmétique de raison $r = -6$ et de premier terme $u_3 = -19$.

Calculer $\displaystyle\sum_{k=31}^{61}u_k$.\end{frame}


\begin{frame}
\vspace{-10mm}
	\frametitle{Correction 1}
Comme $(u_n)_{n\in\mathbb{N}}$ est une suite arithmétique de raison $r = -6$ et de premier terme $u_1=-2$.

 On a $u_{27}= u_1+(27-1)\times r = -2+26\times\left(-6\right) = -158$ 

 et $u_{39} = u_0 + 39\times r = -2(39-1)\times\left(-6\right) = -230$.

On a alors

\begin{align*}\displaystyle\sum_{k=27}^{39} u_k &= (39-27+1) \times \dfrac{u_{27}+u_{39}}{2}\\
	&=13\times \dfrac{-164+\left(-236\right)}{2}\\
	&=-2600
\end{align*}\end{frame}


\begin{frame}
\vspace{-10mm}
	\frametitle{Correction 2}
Comme $(u_n)_{n\in\mathbb{N}}$ est une suite arithmétique de raison $r = 13$ et de premier terme $u_1=-15$.

 On a $u_{35}= u_1+(35-1)\times r = -15+34\times13 = 427$ 

 et $u_{75} = u_0 + 75\times r = -15(75-1)\times13 = 947$.

On a alors

\begin{align*}\displaystyle\sum_{k=35}^{75} u_k &= (75-35+1) \times \dfrac{u_{35}+u_{75}}{2}\\
	&=41\times \dfrac{440+960}{2}\\
	&=28700
\end{align*}\end{frame}


\begin{frame}
\vspace{-10mm}
	\frametitle{Correction 3}
Comme $(u_n)_{n\in\mathbb{N}}$ est une suite arithmétique de raison $r = -3$ et de premier terme $u_3=25$.

 On a $u_{14}= u_3+(14-3)\times r = 25+11\times\left(-3\right) = -8$ 

 et $u_{43} = u_0 + 43\times r = 25(43-3)\times\left(-3\right) = -95$.

On a alors

\begin{align*}\displaystyle\sum_{k=14}^{43} u_k &= (43-14+1) \times \dfrac{u_{14}+u_{43}}{2}\\
	&=30\times \dfrac{-17+\left(-104\right)}{2}\\
	&=-1815
\end{align*}\end{frame}


\begin{frame}
\vspace{-10mm}
	\frametitle{Correction 4}
Comme $(u_n)_{n\in\mathbb{N}}$ est une suite arithmétique de raison $r = 22$ et de premier terme $u_0=-28$.

 On a $u_{28}= u_0 + 28\times r = -28+28\times22 = 588$ 

 et $u_{68} = u_0 + 68\times r = -28+68\times22 = 1468$.

On a alors

\begin{align*}\displaystyle\sum_{k=28}^{68} u_k &= (68-28+1) \times \dfrac{u_{28}+u_{68}}{2}\\
	&=41\times \dfrac{588+1468}{2}\\
	&=42148
\end{align*}\end{frame}


\begin{frame}
\vspace{-10mm}
	\frametitle{Correction 5}
Comme $(u_n)_{n\in\mathbb{N}}$ est une suite arithmétique de raison $r = -6$ et de premier terme $u_3=-19$.

 On a $u_{31}= u_3+(31-3)\times r = -19+28\times\left(-6\right) = -187$ 

 et $u_{61} = u_0 + 61\times r = -19(61-3)\times\left(-6\right) = -367$.

On a alors

\begin{align*}\displaystyle\sum_{k=31}^{61} u_k &= (61-31+1) \times \dfrac{u_{31}+u_{61}}{2}\\
	&=31\times \dfrac{-205+\left(-385\right)}{2}\\
	&=-9145
\end{align*}\end{frame}




\end{document}