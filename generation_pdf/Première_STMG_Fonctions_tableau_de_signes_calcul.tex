\documentclass[15pt, mathserif]{beamer}

\usepackage[french]{babel}
\usepackage[T1]{fontenc}
\usepackage[utf8]{inputenc}
%\usepackage{esvect}
\usepackage{bm}
\usepackage{eurosym}
\usepackage{tikz}
\usepackage{pgf,tikz,pgfplots}
\pgfplotsset{compat=1.15}
\usepackage{mathrsfs}
\usetikzlibrary{arrows}
\usetikzlibrary{arrows.meta}

\usetikzlibrary{mindmap}
\usepackage{multicol}
\usepackage[tikz]{bclogo}
\usepackage{tkz-tab}
\usepackage{amsmath, tabu}
\usepackage{esvect} %\vv{AB} pour le vecteur AB

\DeclareMathOperator{\e}{e}

%% Tableau

\usepackage{makecell}
\setcellgapes{1pt}
\makegapedcells
\newcolumntype{R}[1]{>{\raggedleft\arraybackslash }b{#1}}
\newcolumntype{L}[1]{>{\raggedright\arraybackslash }b{#1}}
\newcolumntype{C}[1]{>{\centering\arraybackslash }b{#1}}


%pour avoir des parenthèses rondes dans le package fourier
\DeclareSymbolFont{cmoperators}   {OT1}{cmr} {m}{n}
\DeclareSymbolFont{cmlargesymbols}{OMX}{cmex}{m}{n}

\usefonttheme{professionalfonts} %permet d'enlever un bug avec fourier
\usepackage{fourier}
\DeclareMathDelimiter{(}{\mathopen} {cmoperators}{"28}{cmlargesymbols}{"00}
\DeclareMathDelimiter{)}{\mathclose}{cmoperators}{"29}{cmlargesymbols}{"01}

%Graphiques 

\usepackage{pgf,tikz,pgfplots}
\pgfplotsset{compat=1.15}
\usepackage{mathrsfs}
\usetikzlibrary{arrows}
\usetikzlibrary{mindmap}

%ensembles de nbres

\newcommand{\R}{\mathbb{R}}			%permet d'écrire le R "ensemble des réels"'
\newcommand{\N}{\mathbb{N}}			%permet d'écrire le N "ensemble des entiers naturels"
\newcommand{\Z}{\mathbb{Z}}			%permet d'écrire le Z "ensemble des entiers relatifs"
\newcommand{\Prem}{\mathbb{P}}	%permet d'écrire le P "ensemble des nombres premiers" (qui n'a pas marché avec le \P car il existe déjà)
\newcommand{\D}{\mathbb{D}}
\newcommand{\Df}{\mathcal{D}_f}
\newcommand{\Cf}{\mathcal{C}_f}

\newcommand{\Q}{\mathbb{Q}}


\newcommand{\st}[1]{$(#1_n)_{n \in \N}$}

\usetheme{Madrid}
\useoutertheme{miniframes} % Alternatively: miniframes, infolines, split
\useinnertheme{circles}
\definecolor{UBCblue}{rgb}{0.1, 0.25, 0.4} % UBC Blue (primary)
\definecolor{bordeaux}{RGB}{128,0,0}
\usecolortheme[named=UBCblue]{structure}

\usepackage{color} % J'aime bien définir mes couleurs
\definecolor{propcolor}{rgb}{0, 0.5, 1}
\definecolor{thcolor}{rgb}{0.6, 0.07, 0.07}
\colorlet{louis}{blue!70!green!60!white}
\colorlet{sakura}{pink!40!red}

\title{Activités Mentales}
\date{24 Août 2023}

\newcommand{\vco}[2]{\begin{pmatrix} #1 \\ #2 \end{pmatrix}} %Coordonnées de vecteur
\newenvironment{eq}{\begin{cases}\begin{tabu}{ccccc}}{\end{tabu}\end{cases}}
\newenvironment{eql}{\begin{cases}\begin{tabu}{cccccl}}{\end{tabu}\end{cases}}
\newenvironment{eqrl}{\begin{cases}\begin{tabu}{rl}}{\end{tabu}\end{cases}}

\newenvironment{Eq}{\begin{center}\begin{tabular}{rrcl}}{\end{tabular}\end{center}}
\newcommand{\ligneq}[2]{$\Longleftrightarrow$ & $#1$ & $=$ & $#2$ \\}
\newcommand{\Ligneq}[2]{ & $#1$ & $=$ & $#2$ \\}

\newenvironment{RPN}{\begin{center}\begin{tabular}{rrclcrcl}}{\end{tabular}\end{center}}
\newcommand{\Lignerpn}[4]{ & $#1$ & $=$ & $#2$ & ou & $#3$ & $=$ & $#4$ \\}
\newcommand{\lignerpn}[4]{$\Longleftrightarrow$ & $#1$ & $=$ & $#2$ & ou & $#3$ & $=$ & $#4$ \\}

\newenvironment{TRPN}{\begin{center}\begin{tabular}{rrclcrclcrcl}}{\end{tabular}\end{center}}
\newcommand{\Lignetrpn}[6]{ & $#1$ & $=$ & $#2$ & ou & $#3$ & $=$ & $#4$ & ou & $#5$ & $=$ & $#6$ \\}
\newcommand{\lignetrpn}[6]{$\Longleftrightarrow$ & $#1$ & $=$ & $#2$ & ou & $#3$ & $=$ & $#4$ & ou & $#5$ & $=$ & $#6$ \\}
\begin{document}

\begin{frame}
    \titlepage
\end{frame}

\begin{frame} 
	\frametitle{Question 1}
Résoudre l'inéquation \[-13(4x-3)(-5x-10)>0\]\end{frame}


\begin{frame} 
	\frametitle{Question 2}
Résoudre l'inéquation \[13(10x+11)(-3x-7)<0\]\end{frame}


\begin{frame} 
	\frametitle{Question 3}
Résoudre l'inéquation \[-5(-3x-11)(8x-2)<0\]\end{frame}


\begin{frame} 
	\frametitle{Question 4}
Résoudre l'inéquation \[9(-7x-15)(4x-3)\leq0\]\end{frame}


\begin{frame} 
	\frametitle{Question 5}
Résoudre l'inéquation \[11(11x+6)(-2x+3)\geq0\]\end{frame}


\begin{frame}
\vspace{-10mm}
	\frametitle{Correction 1}
On pose $A(x) = -13(4x-3)(-5x-10) = -13\times f(x) \times g(x)$ avec $f(x) = 4x-3$ et $g(x) = -5x-10$.

\begin{itemize}
	\item $f$ est une fonction affine avec $m =4>0$. $f$ est donc croissante sur $\mathbb{R}$.

	 De plus $f(x) = 0 \Leftrightarrow x = \dfrac{3}{4}$.
	\item $g$ est une fonction affine avec $m =-5<0$. $g$ est donc décroissante sur $\mathbb{R}$.

	 De plus $g(x) = 0 \Leftrightarrow x = -2$.
\end{itemize}

 \end{frame}


\begin{frame}On rappelle que $f(x) = 4x-3$ et $g(x) = -5x-10$ et $A(x) = -13(4x-3)(-5x-10)$. Son tableau de signe est alors 

\medskip \hfil
\begin{tikzpicture}[scale = 0.75]
	\tkzTabInit[lgt = 1.5]{$x$/1.25, $-13$/ 1, $f(x)$ / 1, $g(x)$ / 1, $A(x)$/1}{$-\infty$, $-2$, $\dfrac{3}{4}$, $+\infty$}
	\tkzTabLine{ , -, t, -, t, -, }
	\tkzTabLine{ , -, t, -, z, +, }
	\tkzTabLine{ , +, z, -, t, -, }
	\tkzTabLine{ , +, z, -, z, +, }
	\end{tikzpicture}

 Finalement l'ensemble de solutions de $-13(4x-3)(-5x-10)>0$ est\[S = \left]-\infty;-2\right[\cup\left]\dfrac{3}{4};+\infty\right[\]

\end{frame}


\begin{frame}
\vspace{-10mm}
	\frametitle{Correction 2}
On pose $A(x) = 13(10x+11)(-3x-7) = 13\times f(x) \times g(x)$ avec $f(x) = 10x+11$ et $g(x) = -3x-7$.

\begin{itemize}
	\item $f$ est une fonction affine avec $m =10>0$. $f$ est donc croissante sur $\mathbb{R}$.

	 De plus $f(x) = 0 \Leftrightarrow x = \dfrac{-11}{10}$.
	\item $g$ est une fonction affine avec $m =-3<0$. $g$ est donc décroissante sur $\mathbb{R}$.

	 De plus $g(x) = 0 \Leftrightarrow x = \dfrac{-7}{3}$.
\end{itemize}

 \end{frame}


\begin{frame}On rappelle que $f(x) = 10x+11$ et $g(x) = -3x-7$ et $A(x) = 13(10x+11)(-3x-7)$. Son tableau de signe est alors 

\medskip \hfil
\begin{tikzpicture}[scale = 0.75]
	\tkzTabInit[lgt = 1.5]{$x$/1.25, $13$/ 1, $f(x)$ / 1, $g(x)$ / 1, $A(x)$/1}{$-\infty$, $\dfrac{-7}{3}$, $\dfrac{-11}{10}$, $+\infty$}
	\tkzTabLine{ , +, t, +, t, +, }
	\tkzTabLine{ , -, t, -, z, +, }
	\tkzTabLine{ , +, z, -, +, -, }
	\tkzTabLine{ , -, z, +, z, -, }
	\end{tikzpicture}

 Finalement l'ensemble de solutions de $13(10x+11)(-3x-7)<0$ est\[S = \left]\dfrac{-7}{3};\dfrac{-11}{10}\right[\]

\end{frame}


\begin{frame}
\vspace{-10mm}
	\frametitle{Correction 3}
On pose $A(x) = -5(-3x-11)(8x-2) = -5\times f(x) \times g(x)$ avec $f(x) = -3x-11$ et $g(x) = 8x-2$.

\begin{itemize}
	\item $f$ est une fonction affine avec $m =-3<0$. $f$ est donc décroissante sur $\mathbb{R}$.

	 De plus $f(x) = 0 \Leftrightarrow x = \dfrac{-11}{3}$.
	\item $g$ est une fonction affine avec $m =8>0$. $g$ est donc croissante sur $\mathbb{R}$.

	 De plus $g(x) = 0 \Leftrightarrow x = \dfrac{1}{4}$.
\end{itemize}

 \end{frame}


\begin{frame}On rappelle que $f(x) = -3x-11$ et $g(x) = 8x-2$ et $A(x) = -5(-3x-11)(8x-2)$. Son tableau de signe est alors 

\medskip \hfil
\begin{tikzpicture}[scale = 0.75]
	\tkzTabInit[lgt = 1.5]{$x$/1.25, $-5$/ 1, $f(x)$ / 1, $g(x)$ / 1, $A(x)$/1}{$-\infty$, $\dfrac{-11}{3}$, $\dfrac{1}{4}$, $+\infty$}
	\tkzTabLine{ , -, t, -, t, -, }
	\tkzTabLine{ , +, z, -, t, -, }
	\tkzTabLine{ , -, t, -, z, +, }
	\tkzTabLine{ , +, z, -, z, +, }
	\end{tikzpicture}

 Finalement l'ensemble de solutions de $-5(-3x-11)(8x-2)<0$ est\[S = \left]\dfrac{-11}{3};\dfrac{1}{4}\right[\]

\end{frame}


\begin{frame}
\vspace{-10mm}
	\frametitle{Correction 4}
On pose $A(x) = 9(-7x-15)(4x-3) = 9\times f(x) \times g(x)$ avec $f(x) = -7x-15$ et $g(x) = 4x-3$.

\begin{itemize}
	\item $f$ est une fonction affine avec $m =-7<0$. $f$ est donc décroissante sur $\mathbb{R}$.

	 De plus $f(x) = 0 \Leftrightarrow x = \dfrac{-15}{7}$.
	\item $g$ est une fonction affine avec $m =4>0$. $g$ est donc croissante sur $\mathbb{R}$.

	 De plus $g(x) = 0 \Leftrightarrow x = \dfrac{3}{4}$.
\end{itemize}

 \end{frame}


\begin{frame}On rappelle que $f(x) = -7x-15$ et $g(x) = 4x-3$ et $A(x) = 9(-7x-15)(4x-3)$. Son tableau de signe est alors 

\medskip \hfil
\begin{tikzpicture}[scale = 0.75]
	\tkzTabInit[lgt = 1.5]{$x$/1.25, $9$/ 1, $f(x)$ / 1, $g(x)$ / 1, $A(x)$/1}{$-\infty$, $\dfrac{-15}{7}$, $\dfrac{3}{4}$, $+\infty$}
	\tkzTabLine{ , +, t, +, t, +, }
	\tkzTabLine{ , +, z, -, t, -, }
	\tkzTabLine{ , -, t, -, z, +, }
	\tkzTabLine{ , -, z, +, z, -, }
	\end{tikzpicture}

 Finalement l'ensemble de solutions de $9(-7x-15)(4x-3)\leq0$ est\[S = \left]-\infty;\dfrac{-15}{7}\right]\cup\left[\dfrac{3}{4};+\infty\right[\]

\end{frame}


\begin{frame}
\vspace{-10mm}
	\frametitle{Correction 5}
On pose $A(x) = 11(11x+6)(-2x+3) = 11\times f(x) \times g(x)$ avec $f(x) = 11x+6$ et $g(x) = -2x+3$.

\begin{itemize}
	\item $f$ est une fonction affine avec $m =11>0$. $f$ est donc croissante sur $\mathbb{R}$.

	 De plus $f(x) = 0 \Leftrightarrow x = \dfrac{-6}{11}$.
	\item $g$ est une fonction affine avec $m =-2<0$. $g$ est donc décroissante sur $\mathbb{R}$.

	 De plus $g(x) = 0 \Leftrightarrow x = \dfrac{3}{2}$.
\end{itemize}

 \end{frame}


\begin{frame}On rappelle que $f(x) = 11x+6$ et $g(x) = -2x+3$ et $A(x) = 11(11x+6)(-2x+3)$. Son tableau de signe est alors 

\medskip \hfil
\begin{tikzpicture}[scale = 0.75]
	\tkzTabInit[lgt = 1.5]{$x$/1.25, $11$/ 1, $f(x)$ / 1, $g(x)$ / 1, $A(x)$/1}{$-\infty$, $\dfrac{-6}{11}$, $\dfrac{3}{2}$, $+\infty$}
	\tkzTabLine{ , +, t, +, t, +, }
	\tkzTabLine{ , -, z, +, t, +, }
	\tkzTabLine{ , +, t, +, z, -, }
	\tkzTabLine{ , -, z, +, z, -, }
	\end{tikzpicture}

 Finalement l'ensemble de solutions de $11(11x+6)(-2x+3)\geq0$ est\[S = \left[\dfrac{-6}{11};\dfrac{3}{2}\right]\]

\end{frame}




\end{document}