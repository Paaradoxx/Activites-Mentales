\documentclass[15pt, mathserif]{beamer}

\usepackage[french]{babel}
\usepackage[T1]{fontenc}
\usepackage[utf8]{inputenc}
%\usepackage{esvect}
\usepackage{bm}
\usepackage{eurosym}
\usepackage{tikz}
\usepackage{pgf,tikz,pgfplots}
\pgfplotsset{compat=1.15}
\usepackage{mathrsfs}
\usetikzlibrary{arrows}
\usetikzlibrary{arrows.meta}

\usetikzlibrary{mindmap}
\usepackage{multicol}
\usepackage[tikz]{bclogo}
\usepackage{tkz-tab}
\usepackage{amsmath, tabu}
\usepackage{esvect} %\vv{AB} pour le vecteur AB

\DeclareMathOperator{\e}{e}

%% Tableau

\usepackage{makecell}
\setcellgapes{1pt}
\makegapedcells
\newcolumntype{R}[1]{>{\raggedleft\arraybackslash }b{#1}}
\newcolumntype{L}[1]{>{\raggedright\arraybackslash }b{#1}}
\newcolumntype{C}[1]{>{\centering\arraybackslash }b{#1}}


%pour avoir des parenthèses rondes dans le package fourier
\DeclareSymbolFont{cmoperators}   {OT1}{cmr} {m}{n}
\DeclareSymbolFont{cmlargesymbols}{OMX}{cmex}{m}{n}

\usefonttheme{professionalfonts} %permet d'enlever un bug avec fourier
\usepackage{fourier}
\DeclareMathDelimiter{(}{\mathopen} {cmoperators}{"28}{cmlargesymbols}{"00}
\DeclareMathDelimiter{)}{\mathclose}{cmoperators}{"29}{cmlargesymbols}{"01}

%Graphiques 

\usepackage{pgf,tikz,pgfplots}
\pgfplotsset{compat=1.15}
\usepackage{mathrsfs}
\usetikzlibrary{arrows}
\usetikzlibrary{mindmap}

%ensembles de nbres

\newcommand{\R}{\mathbb{R}}			%permet d'écrire le R "ensemble des réels"'
\newcommand{\N}{\mathbb{N}}			%permet d'écrire le N "ensemble des entiers naturels"
\newcommand{\Z}{\mathbb{Z}}			%permet d'écrire le Z "ensemble des entiers relatifs"
\newcommand{\Prem}{\mathbb{P}}	%permet d'écrire le P "ensemble des nombres premiers" (qui n'a pas marché avec le \P car il existe déjà)
\newcommand{\D}{\mathbb{D}}
\newcommand{\Df}{\mathcal{D}_f}
\newcommand{\Cf}{\mathcal{C}_f}

\newcommand{\Q}{\mathbb{Q}}


\newcommand{\st}[1]{$(#1_n)_{n \in \N}$}

\usetheme{Madrid}
\useoutertheme{miniframes} % Alternatively: miniframes, infolines, split
\useinnertheme{circles}
\definecolor{UBCblue}{rgb}{0.1, 0.25, 0.4} % UBC Blue (primary)
\definecolor{bordeaux}{RGB}{128,0,0}
\usecolortheme[named=UBCblue]{structure}

\usepackage{color} % J'aime bien définir mes couleurs
\definecolor{propcolor}{rgb}{0, 0.5, 1}
\definecolor{thcolor}{rgb}{0.6, 0.07, 0.07}
\colorlet{louis}{blue!70!green!60!white}
\colorlet{sakura}{pink!40!red}

\title{Activités Mentales}
\date{24 Août 2023}

\newcommand{\vco}[2]{\begin{pmatrix} #1 \\ #2 \end{pmatrix}} %Coordonnées de vecteur
\newenvironment{eq}{\begin{cases}\begin{tabu}{ccccc}}{\end{tabu}\end{cases}}
\newenvironment{eql}{\begin{cases}\begin{tabu}{cccccl}}{\end{tabu}\end{cases}}
\newenvironment{eqrl}{\begin{cases}\begin{tabu}{rl}}{\end{tabu}\end{cases}}

\newenvironment{Eq}{\begin{center}\begin{tabular}{rrcl}}{\end{tabular}\end{center}}
\newcommand{\ligneq}[2]{$\Longleftrightarrow$ & $#1$ & $=$ & $#2$ \\}
\newcommand{\Ligneq}[2]{ & $#1$ & $=$ & $#2$ \\}

\newenvironment{RPN}{\begin{center}\begin{tabular}{rrclcrcl}}{\end{tabular}\end{center}}
\newcommand{\Lignerpn}[4]{ & $#1$ & $=$ & $#2$ & ou & $#3$ & $=$ & $#4$ \\}
\newcommand{\lignerpn}[4]{$\Longleftrightarrow$ & $#1$ & $=$ & $#2$ & ou & $#3$ & $=$ & $#4$ \\}

\newenvironment{TRPN}{\begin{center}\begin{tabular}{rrclcrclcrcl}}{\end{tabular}\end{center}}
\newcommand{\Lignetrpn}[6]{ & $#1$ & $=$ & $#2$ & ou & $#3$ & $=$ & $#4$ & ou & $#5$ & $=$ & $#6$ \\}
\newcommand{\lignetrpn}[6]{$\Longleftrightarrow$ & $#1$ & $=$ & $#2$ & ou & $#3$ & $=$ & $#4$ & ou & $#5$ & $=$ & $#6$ \\}
\begin{document}

\begin{frame}
    \titlepage
\end{frame}

\begin{frame} 
	\frametitle{Question 1}
Résoudre dans $\mathbb{R}$ l'équation $f(x)=-0.8$ \begin{tikzpicture}[line cap=round,line join=round,>=triangle 45,x=1cm,y=2cm] 
 \begin{axis}[x=2cm,y=2cm,axis lines=middle,ymajorgrids=true,xmajorgrids=true,xmin=-2.777189991993057,xmax=2.8528753556051187,ymin=-1.6389406056750524,ymax=1.3869873255497247,xtick={-2.5,-2,...,2.5},ytick={-1.6,-1.4000000000000001,...,1.2000000000000002},] 
 \clip(-2.777189991993057,-1.6389406056750524) rectangle (2.8528753556051187,1.3869873255497247); 
 \draw[line width=1pt,smooth,samples=100,domain=-2.777189991993057:2.8528753556051196] plot(\x,{0.1*((\x)-1)*((\x)-1.5)*((\x))-0.8}); 
 \begin{scriptsize} 
 \end{scriptsize} 
 \end{axis} 
 \end{tikzpicture}\end{frame}


\begin{frame} 
	\frametitle{Question 2}
Résoudre dans $\mathbb{R}$ l'équation $f(x)=0.4$ \begin{tikzpicture}[line cap=round,line join=round,>=triangle 45,x=1cm,y=2cm] 
 \begin{axis}[x=2cm,y=2cm,axis lines=middle,ymajorgrids=true,xmajorgrids=true,xmin=-2.777189991993057,xmax=2.8528753556051187,ymin=-1.6389406056750524,ymax=1.3869873255497247,xtick={-2.5,-2,...,2.5},ytick={-1.6,-1.4000000000000001,...,1.2000000000000002},] 
 \clip(-2.777189991993057,-1.6389406056750524) rectangle (2.8528753556051187,1.3869873255497247); 
 \draw[line width=1pt,smooth,samples=100,domain=-2.777189991993057:2.8528753556051196] plot(\x,{0.1*((\x)+2.5)*((\x))+0.4}); 
 \begin{scriptsize} 
 \end{scriptsize} 
 \end{axis} 
 \end{tikzpicture}\end{frame}


\begin{frame} 
	\frametitle{Question 3}
Résoudre dans $\mathbb{R}$ l'équation $f(x)=-0.6$ \begin{tikzpicture}[line cap=round,line join=round,>=triangle 45,x=1cm,y=2cm] 
 \begin{axis}[x=2cm,y=2cm,axis lines=middle,ymajorgrids=true,xmajorgrids=true,xmin=-2.777189991993057,xmax=2.8528753556051187,ymin=-1.6389406056750524,ymax=1.3869873255497247,xtick={-2.5,-2,...,2.5},ytick={-1.6,-1.4000000000000001,...,1.2000000000000002},] 
 \clip(-2.777189991993057,-1.6389406056750524) rectangle (2.8528753556051187,1.3869873255497247); 
 \draw[line width=1pt,smooth,samples=100,domain=-2.777189991993057:2.8528753556051196] plot(\x,{0.1*((\x)+2)*((\x)-1.5)*((\x))-0.6}); 
 \begin{scriptsize} 
 \end{scriptsize} 
 \end{axis} 
 \end{tikzpicture}\end{frame}


\begin{frame} 
	\frametitle{Question 4}
Résoudre dans $\mathbb{R}$ l'équation $f(x)=0.4$ \begin{tikzpicture}[line cap=round,line join=round,>=triangle 45,x=1cm,y=2cm] 
 \begin{axis}[x=2cm,y=2cm,axis lines=middle,ymajorgrids=true,xmajorgrids=true,xmin=-2.777189991993057,xmax=2.8528753556051187,ymin=-1.6389406056750524,ymax=1.3869873255497247,xtick={-2.5,-2,...,2.5},ytick={-1.6,-1.4000000000000001,...,1.2000000000000002},] 
 \clip(-2.777189991993057,-1.6389406056750524) rectangle (2.8528753556051187,1.3869873255497247); 
 \draw[line width=1pt,smooth,samples=100,domain=-2.777189991993057:2.8528753556051196] plot(\x,{0.1*((\x)+2)*((\x))+0.4}); 
 \begin{scriptsize} 
 \end{scriptsize} 
 \end{axis} 
 \end{tikzpicture}\end{frame}


\begin{frame} 
	\frametitle{Question 5}
Résoudre dans $\mathbb{R}$ l'équation $f(x)=-0.8$ \begin{tikzpicture}[line cap=round,line join=round,>=triangle 45,x=1cm,y=2cm] 
 \begin{axis}[x=2cm,y=2cm,axis lines=middle,ymajorgrids=true,xmajorgrids=true,xmin=-2.777189991993057,xmax=2.8528753556051187,ymin=-1.6389406056750524,ymax=1.3869873255497247,xtick={-2.5,-2,...,2.5},ytick={-1.6,-1.4000000000000001,...,1.2000000000000002},] 
 \clip(-2.777189991993057,-1.6389406056750524) rectangle (2.8528753556051187,1.3869873255497247); 
 \draw[line width=1pt,smooth,samples=100,domain=-2.777189991993057:2.8528753556051196] plot(\x,{0.1*((\x)+2)*((\x))-0.8}); 
 \begin{scriptsize} 
 \end{scriptsize} 
 \end{axis} 
 \end{tikzpicture}\end{frame}


\begin{frame}
\vspace{-10mm}
	\frametitle{Correction 1}
\vspace*{1cm} Résoudre dans $\mathbb{R}$ l'équation $f(x)=-0.8$ \definecolor{ccqqqq}{rgb}{0.8,0,0} 
\begin{tikzpicture}[line cap=round,line join=round,>=triangle 45,x=1cm,y=2cm] 
 \begin{axis}[x=2cm,y=2cm,axis lines=middle,ymajorgrids=true,xmajorgrids=true,xmin=-2.777189991993057,xmax=2.8528753556051187,ymin=-1.6389406056750524,ymax=1.3869873255497247,xtick={-2.5,-2,...,2.5},ytick={-1.6,-1.4000000000000001,...,1.2000000000000002},] 
 \clip(-2.777189991993057,-1.6389406056750524) rectangle (2.8528753556051187,1.3869873255497247); 
 \draw[line width=1pt,smooth,samples=100,domain=-2.777189991993057:2.8528753556051196] plot(\x,{0.1*((\x)-1)*((\x)-1.5)*((\x))-0.8}); 
 \draw [line width=1pt,dash pattern=on 1pt off 1pt,color=ccqqqq,domain=-2.7:2.8] plot(\x,{(-0.8-0*\x)/1});
 \draw [line width=1pt,dash pattern=on 1pt off 1pt,color=ccqqqq] (1,0)-- (1,-0.8);
 \draw [line width=1pt,dash pattern=on 1pt off 1pt,color=ccqqqq] (1.5,0)-- (1.5,-0.8);
 \draw [line width=1pt,dash pattern=on 1pt off 1pt,color=ccqqqq] (0,0)-- (0,-0.8); 
 \begin{scriptsize} 
 \end{scriptsize} 
 \end{axis} 
 \end{tikzpicture} 
 L'ensemble des solutions de l'équation $f(x)=-0.8$ est $\mathcal{S}= \{ 1;1.5;0 \}$\end{frame}


\begin{frame}
\vspace{-10mm}
	\frametitle{Correction 2}
\vspace*{1cm} Résoudre dans $\mathbb{R}$ l'équation $f(x)=0.4$ \definecolor{ccqqqq}{rgb}{0.8,0,0} 
\begin{tikzpicture}[line cap=round,line join=round,>=triangle 45,x=1cm,y=2cm] 
 \begin{axis}[x=2cm,y=2cm,axis lines=middle,ymajorgrids=true,xmajorgrids=true,xmin=-2.777189991993057,xmax=2.8528753556051187,ymin=-1.6389406056750524,ymax=1.3869873255497247,xtick={-2.5,-2,...,2.5},ytick={-1.6,-1.4000000000000001,...,1.2000000000000002},] 
 \clip(-2.777189991993057,-1.6389406056750524) rectangle (2.8528753556051187,1.3869873255497247); 
 \draw[line width=1pt,smooth,samples=100,domain=-2.777189991993057:2.8528753556051196] plot(\x,{0.1*((\x)+2.5)*((\x))+0.4}); 
 \draw [line width=1pt,dash pattern=on 1pt off 1pt,color=ccqqqq,domain=-2.7:2.8] plot(\x,{(0.4-0*\x)/1});
 \draw [line width=1pt,dash pattern=on 1pt off 1pt,color=ccqqqq] (-2.5,0)-- (-2.5,0.4);
 \draw [line width=1pt,dash pattern=on 1pt off 1pt,color=ccqqqq] (0,0)-- (0,0.4); 
 \begin{scriptsize} 
 \end{scriptsize} 
 \end{axis} 
 \end{tikzpicture} 
 L'ensemble des solutions de l'équation $f(x)=0.4$ est $\mathcal{S}= \{ -2.5;0 \}$\end{frame}


\begin{frame}
\vspace{-10mm}
	\frametitle{Correction 3}
\vspace*{1cm} Résoudre dans $\mathbb{R}$ l'équation $f(x)=-0.6$ \definecolor{ccqqqq}{rgb}{0.8,0,0} 
\begin{tikzpicture}[line cap=round,line join=round,>=triangle 45,x=1cm,y=2cm] 
 \begin{axis}[x=2cm,y=2cm,axis lines=middle,ymajorgrids=true,xmajorgrids=true,xmin=-2.777189991993057,xmax=2.8528753556051187,ymin=-1.6389406056750524,ymax=1.3869873255497247,xtick={-2.5,-2,...,2.5},ytick={-1.6,-1.4000000000000001,...,1.2000000000000002},] 
 \clip(-2.777189991993057,-1.6389406056750524) rectangle (2.8528753556051187,1.3869873255497247); 
 \draw[line width=1pt,smooth,samples=100,domain=-2.777189991993057:2.8528753556051196] plot(\x,{0.1*((\x)+2)*((\x)-1.5)*((\x))-0.6}); 
 \draw [line width=1pt,dash pattern=on 1pt off 1pt,color=ccqqqq,domain=-2.7:2.8] plot(\x,{(-0.6-0*\x)/1});
 \draw [line width=1pt,dash pattern=on 1pt off 1pt,color=ccqqqq] (-2,0)-- (-2,-0.6);
 \draw [line width=1pt,dash pattern=on 1pt off 1pt,color=ccqqqq] (1.5,0)-- (1.5,-0.6);
 \draw [line width=1pt,dash pattern=on 1pt off 1pt,color=ccqqqq] (0,0)-- (0,-0.6); 
 \begin{scriptsize} 
 \end{scriptsize} 
 \end{axis} 
 \end{tikzpicture} 
 L'ensemble des solutions de l'équation $f(x)=-0.6$ est $\mathcal{S}= \{ -2;1.5;0 \}$\end{frame}


\begin{frame}
\vspace{-10mm}
	\frametitle{Correction 4}
\vspace*{1cm} Résoudre dans $\mathbb{R}$ l'équation $f(x)=0.4$ \definecolor{ccqqqq}{rgb}{0.8,0,0} 
\begin{tikzpicture}[line cap=round,line join=round,>=triangle 45,x=1cm,y=2cm] 
 \begin{axis}[x=2cm,y=2cm,axis lines=middle,ymajorgrids=true,xmajorgrids=true,xmin=-2.777189991993057,xmax=2.8528753556051187,ymin=-1.6389406056750524,ymax=1.3869873255497247,xtick={-2.5,-2,...,2.5},ytick={-1.6,-1.4000000000000001,...,1.2000000000000002},] 
 \clip(-2.777189991993057,-1.6389406056750524) rectangle (2.8528753556051187,1.3869873255497247); 
 \draw[line width=1pt,smooth,samples=100,domain=-2.777189991993057:2.8528753556051196] plot(\x,{0.1*((\x)+2)*((\x))+0.4}); 
 \draw [line width=1pt,dash pattern=on 1pt off 1pt,color=ccqqqq,domain=-2.7:2.8] plot(\x,{(0.4-0*\x)/1});
 \draw [line width=1pt,dash pattern=on 1pt off 1pt,color=ccqqqq] (-2,0)-- (-2,0.4);
 \draw [line width=1pt,dash pattern=on 1pt off 1pt,color=ccqqqq] (0,0)-- (0,0.4); 
 \begin{scriptsize} 
 \end{scriptsize} 
 \end{axis} 
 \end{tikzpicture} 
 L'ensemble des solutions de l'équation $f(x)=0.4$ est $\mathcal{S}= \{ -2;0 \}$\end{frame}


\begin{frame}
\vspace{-10mm}
	\frametitle{Correction 5}
\vspace*{1cm} Résoudre dans $\mathbb{R}$ l'équation $f(x)=-0.8$ \definecolor{ccqqqq}{rgb}{0.8,0,0} 
\begin{tikzpicture}[line cap=round,line join=round,>=triangle 45,x=1cm,y=2cm] 
 \begin{axis}[x=2cm,y=2cm,axis lines=middle,ymajorgrids=true,xmajorgrids=true,xmin=-2.777189991993057,xmax=2.8528753556051187,ymin=-1.6389406056750524,ymax=1.3869873255497247,xtick={-2.5,-2,...,2.5},ytick={-1.6,-1.4000000000000001,...,1.2000000000000002},] 
 \clip(-2.777189991993057,-1.6389406056750524) rectangle (2.8528753556051187,1.3869873255497247); 
 \draw[line width=1pt,smooth,samples=100,domain=-2.777189991993057:2.8528753556051196] plot(\x,{0.1*((\x)+2)*((\x))-0.8}); 
 \draw [line width=1pt,dash pattern=on 1pt off 1pt,color=ccqqqq,domain=-2.7:2.8] plot(\x,{(-0.8-0*\x)/1});
 \draw [line width=1pt,dash pattern=on 1pt off 1pt,color=ccqqqq] (-2,0)-- (-2,-0.8);
 \draw [line width=1pt,dash pattern=on 1pt off 1pt,color=ccqqqq] (0,0)-- (0,-0.8); 
 \begin{scriptsize} 
 \end{scriptsize} 
 \end{axis} 
 \end{tikzpicture} 
 L'ensemble des solutions de l'équation $f(x)=-0.8$ est $\mathcal{S}= \{ -2;0 \}$\end{frame}




\end{document}