\documentclass[15pt, mathserif]{beamer}

\usepackage[french]{babel}
\usepackage[T1]{fontenc}
\usepackage[utf8]{inputenc}
%\usepackage{esvect}
\usepackage{bm}
\usepackage{eurosym}
\usepackage{tikz}
\usepackage{pgf,tikz,pgfplots}
\pgfplotsset{compat=1.15}
\usepackage{mathrsfs}
\usetikzlibrary{arrows}
\usetikzlibrary{arrows.meta}

\usetikzlibrary{mindmap}
\usepackage{multicol}
\usepackage[tikz]{bclogo}
\usepackage{tkz-tab}
\usepackage{amsmath, tabu}
\usepackage{esvect} %\vv{AB} pour le vecteur AB

\DeclareMathOperator{\e}{e}

%% Tableau

\usepackage{makecell}
\setcellgapes{1pt}
\makegapedcells
\newcolumntype{R}[1]{>{\raggedleft\arraybackslash }b{#1}}
\newcolumntype{L}[1]{>{\raggedright\arraybackslash }b{#1}}
\newcolumntype{C}[1]{>{\centering\arraybackslash }b{#1}}


%pour avoir des parenthèses rondes dans le package fourier
\DeclareSymbolFont{cmoperators}   {OT1}{cmr} {m}{n}
\DeclareSymbolFont{cmlargesymbols}{OMX}{cmex}{m}{n}

\usefonttheme{professionalfonts} %permet d'enlever un bug avec fourier
\usepackage{fourier}
\DeclareMathDelimiter{(}{\mathopen} {cmoperators}{"28}{cmlargesymbols}{"00}
\DeclareMathDelimiter{)}{\mathclose}{cmoperators}{"29}{cmlargesymbols}{"01}

%Graphiques 

\usepackage{pgf,tikz,pgfplots}
\pgfplotsset{compat=1.15}
\usepackage{mathrsfs}
\usetikzlibrary{arrows}
\usetikzlibrary{mindmap}

%ensembles de nbres

\newcommand{\R}{\mathbb{R}}			%permet d'écrire le R "ensemble des réels"'
\newcommand{\N}{\mathbb{N}}			%permet d'écrire le N "ensemble des entiers naturels"
\newcommand{\Z}{\mathbb{Z}}			%permet d'écrire le Z "ensemble des entiers relatifs"
\newcommand{\Prem}{\mathbb{P}}	%permet d'écrire le P "ensemble des nombres premiers" (qui n'a pas marché avec le \P car il existe déjà)
\newcommand{\D}{\mathbb{D}}
\newcommand{\Df}{\mathcal{D}_f}
\newcommand{\Cf}{\mathcal{C}_f}

\newcommand{\Q}{\mathbb{Q}}


\newcommand{\st}[1]{$(#1_n)_{n \in \N}$}

\usetheme{Madrid}
\useoutertheme{miniframes} % Alternatively: miniframes, infolines, split
\useinnertheme{circles}
\definecolor{UBCblue}{rgb}{0.1, 0.25, 0.4} % UBC Blue (primary)
\definecolor{bordeaux}{RGB}{128,0,0}
\usecolortheme[named=UBCblue]{structure}

\usepackage{color} % J'aime bien définir mes couleurs
\definecolor{propcolor}{rgb}{0, 0.5, 1}
\definecolor{thcolor}{rgb}{0.6, 0.07, 0.07}
\colorlet{louis}{blue!70!green!60!white}
\colorlet{sakura}{pink!40!red}

\title{Activités Mentales}
\date{24 Août 2023}

\newcommand{\vco}[2]{\begin{pmatrix} #1 \\ #2 \end{pmatrix}} %Coordonnées de vecteur
\newenvironment{eq}{\begin{cases}\begin{tabu}{ccccc}}{\end{tabu}\end{cases}}
\newenvironment{eql}{\begin{cases}\begin{tabu}{cccccl}}{\end{tabu}\end{cases}}
\newenvironment{eqrl}{\begin{cases}\begin{tabu}{rl}}{\end{tabu}\end{cases}}

\newenvironment{Eq}{\begin{center}\begin{tabular}{rrcl}}{\end{tabular}\end{center}}
\newcommand{\ligneq}[2]{$\Longleftrightarrow$ & $#1$ & $=$ & $#2$ \\}
\newcommand{\Ligneq}[2]{ & $#1$ & $=$ & $#2$ \\}

\newenvironment{RPN}{\begin{center}\begin{tabular}{rrclcrcl}}{\end{tabular}\end{center}}
\newcommand{\Lignerpn}[4]{ & $#1$ & $=$ & $#2$ & ou & $#3$ & $=$ & $#4$ \\}
\newcommand{\lignerpn}[4]{$\Longleftrightarrow$ & $#1$ & $=$ & $#2$ & ou & $#3$ & $=$ & $#4$ \\}

\newenvironment{TRPN}{\begin{center}\begin{tabular}{rrclcrclcrcl}}{\end{tabular}\end{center}}
\newcommand{\Lignetrpn}[6]{ & $#1$ & $=$ & $#2$ & ou & $#3$ & $=$ & $#4$ & ou & $#5$ & $=$ & $#6$ \\}
\newcommand{\lignetrpn}[6]{$\Longleftrightarrow$ & $#1$ & $=$ & $#2$ & ou & $#3$ & $=$ & $#4$ & ou & $#5$ & $=$ & $#6$ \\}
\begin{document}

\begin{frame}
    \titlepage
\end{frame}

\begin{frame} 
	\frametitle{Question 1}
On considère la suite $(u_n)_{n\in\mathbb{N}}$ définie par la relation de récurrence suivante:\[\begin{cases} u_{n+1} = -0.16u_n^2+1.16u_n\\[1em] u_0 = 0.54\end{cases}.\]

On pose pour tout $x \in [0;1], f(x) = -0.16x^2+1.16x$.

\begin{enumerate}
	\item Étudier les variations de $f$ sur $[0;1]$.
	\item Démontrer que pour tout $n \in \mathbb{N}, ~0 \leq u_n \leq 1$.
	\item Démontrer que $(u_n)_{n\in\mathbb{N}}$ est décroissante.
	\item En déduire finalement que $(u_n)_{n\in\mathbb{N}}$ est une suite convergente.
\end{enumerate}
\end{frame}


\begin{frame} 
	\frametitle{Question 2}
On considère la suite $(u_n)_{n\in\mathbb{N}}$ définie par la relation de récurrence suivante:\[\begin{cases} u_{n+1} = -0.32u_n^2+1.32u_n\\[1em] u_0 = 0.69\end{cases}.\]

On pose pour tout $x \in [0;1], f(x) = -0.32x^2+1.32x$.

\begin{enumerate}
	\item Étudier les variations de $f$ sur $[0;1]$.
	\item Démontrer que pour tout $n \in \mathbb{N}, ~0 \leq u_n \leq 1$.
	\item Démontrer que $(u_n)_{n\in\mathbb{N}}$ est décroissante.
	\item En déduire finalement que $(u_n)_{n\in\mathbb{N}}$ est une suite convergente.
\end{enumerate}
\end{frame}


\begin{frame} 
	\frametitle{Question 3}
On considère la suite $(u_n)_{n\in\mathbb{N}}$ définie par la relation de récurrence suivante:\[\begin{cases} u_{n+1} = -0.73u_n^2+1.73u_n\\[1em] u_0 = 0.52\end{cases}.\]

On pose pour tout $x \in [0;1], f(x) = -0.73x^2+1.73x$.

\begin{enumerate}
	\item Étudier les variations de $f$ sur $[0;1]$.
	\item Démontrer que pour tout $n \in \mathbb{N}, ~0 \leq u_n \leq 1$.
	\item Démontrer que $(u_n)_{n\in\mathbb{N}}$ est décroissante.
	\item En déduire finalement que $(u_n)_{n\in\mathbb{N}}$ est une suite convergente.
\end{enumerate}
\end{frame}


\begin{frame} 
	\frametitle{Question 4}
On considère la suite $(u_n)_{n\in\mathbb{N}}$ définie par la relation de récurrence suivante:\[\begin{cases} u_{n+1} = u_n^2u_n\\[1em] u_0 = 0.21\end{cases}.\]

On pose pour tout $x \in [0;1], f(x) = x^2x$.

\begin{enumerate}
	\item Étudier les variations de $f$ sur $[0;1]$.
	\item Démontrer que pour tout $n \in \mathbb{N}, ~0 \leq u_n \leq 1$.
	\item Démontrer que $(u_n)_{n\in\mathbb{N}}$ est croissante.
	\item En déduire finalement que $(u_n)_{n\in\mathbb{N}}$ est une suite convergente.
\end{enumerate}
\end{frame}


\begin{frame} 
	\frametitle{Question 5}
On considère la suite $(u_n)_{n\in\mathbb{N}}$ définie par la relation de récurrence suivante:\[\begin{cases} u_{n+1} = 0.21u_n^2+0.79u_n\\[1em] u_0 = 0.5\end{cases}.\]

On pose pour tout $x \in [0;1], f(x) = 0.21x^2+0.79x$.

\begin{enumerate}
	\item Étudier les variations de $f$ sur $[0;1]$.
	\item Démontrer que pour tout $n \in \mathbb{N}, ~0 \leq u_n \leq 1$.
	\item Démontrer que $(u_n)_{n\in\mathbb{N}}$ est croissante.
	\item En déduire finalement que $(u_n)_{n\in\mathbb{N}}$ est une suite convergente.
\end{enumerate}
\end{frame}


\begin{frame}
\vspace{-10mm}
	\frametitle{Correction 1}
On a $\begin{cases} u_{n+1} = -0.16u_n^2+1.16u_n\\[1em] u_0 = 0.54\end{cases}$ et $f(x) = -0.16x^2+1.16x$.

\begin{enumerate}
	\item $f$ est une fonction définie et dérivable sur $[0;1]$. On a \[\forall x \in [0;1], ~ f'(x) = -0.32x+1.16\]

 Or $f'(x) \geq 0$ sur $[0;1]$ (étude du signe à réaliser). 

La fonction $f$ est donc croissante sur $[0;1]$.

\end{enumerate}
\end{frame}


\begin{frame}On a $\begin{cases} u_{n+1} = -0.16u_n^2+1.16u_n\\[1em] u_0 = 0.54\end{cases}$ et $f(x) = -0.16x^2+1.16x$.

\begin{enumerate}\setcounter{enumi}{1}

	\item On pose pour tout entier $n \in \mathbb{N}$ l'hypothèse de récurrence $H_n:~"0 \leq u_n \leq 1"$.

\medskip

Initialisation: On a $u_0 =0.54 \in [0;1]$, donc $H_0$ est vraie.

\medskip

Hérédité: Supposons $H_k$ vraie pour \textbf\underline{{$k$ fixé}} et montrons que $H_{k+1}$ est vraie. C'est-à-dire, montrons que si $0 \leq u_k \leq 1$ alors $0 \leq u_{k+1} \leq 1$.

 Par croissance de la fonction $f$ sur l'intervalle $[0;1]$, on a \[0\leq u_k \leq 1 \quad \Rightarrow \quad f(0) \leq f(u_k) \leq f(1).\]

 Or $f(0) = 0$, $f(1) = -0.16\times 1^2+1.16= 1$ et $f(u_k) = u_{k+1}$.

 Finalement $0\leq u_k \leq 1 \Rightarrow 0 \leq u_{k+1} \leq 1$. Donc $H_{k} \Rightarrow H_{k+1}$.

 On a finalement démontré par récurrence que pour tout $n \in\mathbb{N},~ 0 \leq u_n \leq 1$.

\end{enumerate}
\end{frame}


\begin{frame}On a $\begin{cases} u_{n+1} = -0.16u_n^2+1.16u_n\\[1em] u_0 = 0.54\end{cases}$ et $f(x) = -0.16x^2+1.16x$.

\begin{enumerate}\setcounter{enumi}{2}

	\item On pose pour tout entier $n \in \mathbb{N}$ l'hypothèse de récurrence $H_n:~" u_n \leq u_{n+1}"$.

\medskip

Initialisation: On a $u_0 =0.54$ et $u_1 = -0.16\times0.54^2+1.16\times0.54 = 0.5797439999999999\geq 0.54$ , donc $H_0$ est vraie.

\medskip

Hérédité: Supposons $H_k$ vraie pour \textbf\underline{{$k$ fixé}} et montrons que $H_{k+1}$ est vraie. C'est-à-dire, montrons que si $u_k \leq  u_{k+1}$ alors $u_{k+1} \leq  u_{k+2}$.

 Par croissance de la fonction $f$ sur l'intervalle $[0;1]$ et puisque les termes en jeu sont dans $[0;1]$, on a \[ u_k \leq u_{k+1} \quad \Rightarrow \quad f(u_k) \leq  f(u_{k+1}) \quad \Rightarrow \quad  u_{k+1}\leq u_{k+2}. \]

Donc $H_{k} \Rightarrow H_{k+1}$.

 On a finalement démontré par récurrence que pour tout $n \in\mathbb{N},~ u_n \leq u_{n+1}$ et donc $(u_n)_{n\in\mathbb{N}}$ est  croissante .

\end{enumerate}
\end{frame}


\begin{frame}

\begin{enumerate}\setcounter{enumi}{3}

	\item D'après les questions $3)$ et $2)$ , $(u_n)_{n\in\mathbb{N}}$ est croissante et majorée. Donc d'après le théorème de convergence monotone, $(u_n)_{n\in\mathbb{N}}$ converge.

\end{enumerate}\end{frame}


\begin{frame}
\vspace{-10mm}
	\frametitle{Correction 2}
On a $\begin{cases} u_{n+1} = -0.32u_n^2+1.32u_n\\[1em] u_0 = 0.69\end{cases}$ et $f(x) = -0.32x^2+1.32x$.

\begin{enumerate}
	\item $f$ est une fonction définie et dérivable sur $[0;1]$. On a \[\forall x \in [0;1], ~ f'(x) = -0.64x+1.32\]

 Or $f'(x) \geq 0$ sur $[0;1]$ (étude du signe à réaliser). 

La fonction $f$ est donc croissante sur $[0;1]$.

\end{enumerate}
\end{frame}


\begin{frame}On a $\begin{cases} u_{n+1} = -0.32u_n^2+1.32u_n\\[1em] u_0 = 0.69\end{cases}$ et $f(x) = -0.32x^2+1.32x$.

\begin{enumerate}\setcounter{enumi}{1}

	\item On pose pour tout entier $n \in \mathbb{N}$ l'hypothèse de récurrence $H_n:~"0 \leq u_n \leq 1"$.

\medskip

Initialisation: On a $u_0 =0.69 \in [0;1]$, donc $H_0$ est vraie.

\medskip

Hérédité: Supposons $H_k$ vraie pour \textbf\underline{{$k$ fixé}} et montrons que $H_{k+1}$ est vraie. C'est-à-dire, montrons que si $0 \leq u_k \leq 1$ alors $0 \leq u_{k+1} \leq 1$.

 Par croissance de la fonction $f$ sur l'intervalle $[0;1]$, on a \[0\leq u_k \leq 1 \quad \Rightarrow \quad f(0) \leq f(u_k) \leq f(1).\]

 Or $f(0) = 0$, $f(1) = -0.32\times 1^2+1.32= 1$ et $f(u_k) = u_{k+1}$.

 Finalement $0\leq u_k \leq 1 \Rightarrow 0 \leq u_{k+1} \leq 1$. Donc $H_{k} \Rightarrow H_{k+1}$.

 On a finalement démontré par récurrence que pour tout $n \in\mathbb{N},~ 0 \leq u_n \leq 1$.

\end{enumerate}
\end{frame}


\begin{frame}On a $\begin{cases} u_{n+1} = -0.32u_n^2+1.32u_n\\[1em] u_0 = 0.69\end{cases}$ et $f(x) = -0.32x^2+1.32x$.

\begin{enumerate}\setcounter{enumi}{2}

	\item On pose pour tout entier $n \in \mathbb{N}$ l'hypothèse de récurrence $H_n:~" u_n \leq u_{n+1}"$.

\medskip

Initialisation: On a $u_0 =0.69$ et $u_1 = -0.32\times0.69^2+1.32\times0.69 = 0.758448\geq 0.69$ , donc $H_0$ est vraie.

\medskip

Hérédité: Supposons $H_k$ vraie pour \textbf\underline{{$k$ fixé}} et montrons que $H_{k+1}$ est vraie. C'est-à-dire, montrons que si $u_k \leq  u_{k+1}$ alors $u_{k+1} \leq  u_{k+2}$.

 Par croissance de la fonction $f$ sur l'intervalle $[0;1]$ et puisque les termes en jeu sont dans $[0;1]$, on a \[ u_k \leq u_{k+1} \quad \Rightarrow \quad f(u_k) \leq  f(u_{k+1}) \quad \Rightarrow \quad  u_{k+1}\leq u_{k+2}. \]

Donc $H_{k} \Rightarrow H_{k+1}$.

 On a finalement démontré par récurrence que pour tout $n \in\mathbb{N},~ u_n \leq u_{n+1}$ et donc $(u_n)_{n\in\mathbb{N}}$ est  croissante .

\end{enumerate}
\end{frame}


\begin{frame}

\begin{enumerate}\setcounter{enumi}{3}

	\item D'après les questions $3)$ et $2)$ , $(u_n)_{n\in\mathbb{N}}$ est croissante et majorée. Donc d'après le théorème de convergence monotone, $(u_n)_{n\in\mathbb{N}}$ converge.

\end{enumerate}\end{frame}


\begin{frame}
\vspace{-10mm}
	\frametitle{Correction 3}
On a $\begin{cases} u_{n+1} = -0.73u_n^2+1.73u_n\\[1em] u_0 = 0.52\end{cases}$ et $f(x) = -0.73x^2+1.73x$.

\begin{enumerate}
	\item $f$ est une fonction définie et dérivable sur $[0;1]$. On a \[\forall x \in [0;1], ~ f'(x) = -1.46x+1.73\]

 Or $f'(x) \geq 0$ sur $[0;1]$ (étude du signe à réaliser). 

La fonction $f$ est donc croissante sur $[0;1]$.

\end{enumerate}
\end{frame}


\begin{frame}On a $\begin{cases} u_{n+1} = -0.73u_n^2+1.73u_n\\[1em] u_0 = 0.52\end{cases}$ et $f(x) = -0.73x^2+1.73x$.

\begin{enumerate}\setcounter{enumi}{1}

	\item On pose pour tout entier $n \in \mathbb{N}$ l'hypothèse de récurrence $H_n:~"0 \leq u_n \leq 1"$.

\medskip

Initialisation: On a $u_0 =0.52 \in [0;1]$, donc $H_0$ est vraie.

\medskip

Hérédité: Supposons $H_k$ vraie pour \textbf\underline{{$k$ fixé}} et montrons que $H_{k+1}$ est vraie. C'est-à-dire, montrons que si $0 \leq u_k \leq 1$ alors $0 \leq u_{k+1} \leq 1$.

 Par croissance de la fonction $f$ sur l'intervalle $[0;1]$, on a \[0\leq u_k \leq 1 \quad \Rightarrow \quad f(0) \leq f(u_k) \leq f(1).\]

 Or $f(0) = 0$, $f(1) = -0.73\times 1^2+1.73= 1$ et $f(u_k) = u_{k+1}$.

 Finalement $0\leq u_k \leq 1 \Rightarrow 0 \leq u_{k+1} \leq 1$. Donc $H_{k} \Rightarrow H_{k+1}$.

 On a finalement démontré par récurrence que pour tout $n \in\mathbb{N},~ 0 \leq u_n \leq 1$.

\end{enumerate}
\end{frame}


\begin{frame}On a $\begin{cases} u_{n+1} = -0.73u_n^2+1.73u_n\\[1em] u_0 = 0.52\end{cases}$ et $f(x) = -0.73x^2+1.73x$.

\begin{enumerate}\setcounter{enumi}{2}

	\item On pose pour tout entier $n \in \mathbb{N}$ l'hypothèse de récurrence $H_n:~" u_n \leq u_{n+1}"$.

\medskip

Initialisation: On a $u_0 =0.52$ et $u_1 = -0.73\times0.52^2+1.73\times0.52 = 0.702208\geq 0.52$ , donc $H_0$ est vraie.

\medskip

Hérédité: Supposons $H_k$ vraie pour \textbf\underline{{$k$ fixé}} et montrons que $H_{k+1}$ est vraie. C'est-à-dire, montrons que si $u_k \leq  u_{k+1}$ alors $u_{k+1} \leq  u_{k+2}$.

 Par croissance de la fonction $f$ sur l'intervalle $[0;1]$ et puisque les termes en jeu sont dans $[0;1]$, on a \[ u_k \leq u_{k+1} \quad \Rightarrow \quad f(u_k) \leq  f(u_{k+1}) \quad \Rightarrow \quad  u_{k+1}\leq u_{k+2}. \]

Donc $H_{k} \Rightarrow H_{k+1}$.

 On a finalement démontré par récurrence que pour tout $n \in\mathbb{N},~ u_n \leq u_{n+1}$ et donc $(u_n)_{n\in\mathbb{N}}$ est  croissante .

\end{enumerate}
\end{frame}


\begin{frame}

\begin{enumerate}\setcounter{enumi}{3}

	\item D'après les questions $3)$ et $2)$ , $(u_n)_{n\in\mathbb{N}}$ est croissante et majorée. Donc d'après le théorème de convergence monotone, $(u_n)_{n\in\mathbb{N}}$ converge.

\end{enumerate}\end{frame}


\begin{frame}
\vspace{-10mm}
	\frametitle{Correction 4}
On a $\begin{cases} u_{n+1} = u_n^2u_n\\[1em] u_0 = 0.21\end{cases}$ et $f(x) = x^2x$.

\begin{enumerate}
	\item $f$ est une fonction définie et dérivable sur $[0;1]$. On a \[\forall x \in [0;1], ~ f'(x) = 2.0x\]

 Or $f'(x) \geq 0$ sur $[0;1]$ (étude du signe à réaliser). 

La fonction $f$ est donc croissante sur $[0;1]$.

\end{enumerate}
\end{frame}


\begin{frame}On a $\begin{cases} u_{n+1} = u_n^2u_n\\[1em] u_0 = 0.21\end{cases}$ et $f(x) = x^2x$.

\begin{enumerate}\setcounter{enumi}{1}

	\item On pose pour tout entier $n \in \mathbb{N}$ l'hypothèse de récurrence $H_n:~"0 \leq u_n \leq 1"$.

\medskip

Initialisation: On a $u_0 =0.21 \in [0;1]$, donc $H_0$ est vraie.

\medskip

Hérédité: Supposons $H_k$ vraie pour \textbf\underline{{$k$ fixé}} et montrons que $H_{k+1}$ est vraie. C'est-à-dire, montrons que si $0 \leq u_k \leq 1$ alors $0 \leq u_{k+1} \leq 1$.

 Par croissance de la fonction $f$ sur l'intervalle $[0;1]$, on a \[0\leq u_k \leq 1 \quad \Rightarrow \quad f(0) \leq f(u_k) \leq f(1).\]

 Or $f(0) = 0$, $f(1) = 1.0\times 1^2= 1$ et $f(u_k) = u_{k+1}$.

 Finalement $0\leq u_k \leq 1 \Rightarrow 0 \leq u_{k+1} \leq 1$. Donc $H_{k} \Rightarrow H_{k+1}$.

 On a finalement démontré par récurrence que pour tout $n \in\mathbb{N},~ 0 \leq u_n \leq 1$.

\end{enumerate}
\end{frame}


\begin{frame}On a $\begin{cases} u_{n+1} = u_n^2u_n\\[1em] u_0 = 0.21\end{cases}$ et $f(x) = x^2x$.

\begin{enumerate}\setcounter{enumi}{2}

	\item On pose pour tout entier $n \in \mathbb{N}$ l'hypothèse de récurrence $H_n:~" u_n \geq u_{n+1}"$.

\medskip

Initialisation: On a $u_0 =0.21$ et $u_1 = 1.0\times0.21^2+0.0\times0.21 = 0.04409999999999999\leq 0.21$ , donc $H_0$ est vraie.

\medskip

Hérédité: Supposons $H_k$ vraie pour \textbf\underline{{$k$ fixé}} et montrons que $H_{k+1}$ est vraie. C'est-à-dire, montrons que si $u_k \geq  u_{k+1}$ alors $u_{k+1} \geq  u_{k+2}$.

 Par croissance de la fonction $f$ sur l'intervalle $[0;1]$ et puisque les termes en jeu sont dans $[0;1]$, on a \[ u_k \geq u_{k+1} \quad \Rightarrow \quad f(u_k) \geq  f(u_{k+1}) \quad \Rightarrow \quad  u_{k+1}\geq u_{k+2}. \]

Donc $H_{k} \Rightarrow H_{k+1}$.

 On a finalement démontré par récurrence que pour tout $n \in\mathbb{N},~ u_n \geq u_{n+1}$ et donc $(u_n)_{n\in\mathbb{N}}$ est décroissante .

\end{enumerate}
\end{frame}


\begin{frame}

\begin{enumerate}\setcounter{enumi}{3}

	\item D'après les questions $3)$ et $2)$ , $(u_n)_{n\in\mathbb{N}}$ est décroissante et minorée. Donc d'après le théorème de convergence monotone, $(u_n)_{n\in\mathbb{N}}$ converge.

\end{enumerate}\end{frame}


\begin{frame}
\vspace{-10mm}
	\frametitle{Correction 5}
On a $\begin{cases} u_{n+1} = 0.21u_n^2+0.79u_n\\[1em] u_0 = 0.5\end{cases}$ et $f(x) = 0.21x^2+0.79x$.

\begin{enumerate}
	\item $f$ est une fonction définie et dérivable sur $[0;1]$. On a \[\forall x \in [0;1], ~ f'(x) = 0.42x+0.79\]

 Or $f'(x) \geq 0$ sur $[0;1]$ (étude du signe à réaliser). 

La fonction $f$ est donc croissante sur $[0;1]$.

\end{enumerate}
\end{frame}


\begin{frame}On a $\begin{cases} u_{n+1} = 0.21u_n^2+0.79u_n\\[1em] u_0 = 0.5\end{cases}$ et $f(x) = 0.21x^2+0.79x$.

\begin{enumerate}\setcounter{enumi}{1}

	\item On pose pour tout entier $n \in \mathbb{N}$ l'hypothèse de récurrence $H_n:~"0 \leq u_n \leq 1"$.

\medskip

Initialisation: On a $u_0 =0.5 \in [0;1]$, donc $H_0$ est vraie.

\medskip

Hérédité: Supposons $H_k$ vraie pour \textbf\underline{{$k$ fixé}} et montrons que $H_{k+1}$ est vraie. C'est-à-dire, montrons que si $0 \leq u_k \leq 1$ alors $0 \leq u_{k+1} \leq 1$.

 Par croissance de la fonction $f$ sur l'intervalle $[0;1]$, on a \[0\leq u_k \leq 1 \quad \Rightarrow \quad f(0) \leq f(u_k) \leq f(1).\]

 Or $f(0) = 0$, $f(1) = 0.21\times 1^2+0.79= 1$ et $f(u_k) = u_{k+1}$.

 Finalement $0\leq u_k \leq 1 \Rightarrow 0 \leq u_{k+1} \leq 1$. Donc $H_{k} \Rightarrow H_{k+1}$.

 On a finalement démontré par récurrence que pour tout $n \in\mathbb{N},~ 0 \leq u_n \leq 1$.

\end{enumerate}
\end{frame}


\begin{frame}On a $\begin{cases} u_{n+1} = 0.21u_n^2+0.79u_n\\[1em] u_0 = 0.5\end{cases}$ et $f(x) = 0.21x^2+0.79x$.

\begin{enumerate}\setcounter{enumi}{2}

	\item On pose pour tout entier $n \in \mathbb{N}$ l'hypothèse de récurrence $H_n:~" u_n \geq u_{n+1}"$.

\medskip

Initialisation: On a $u_0 =0.5$ et $u_1 = 0.21\times0.5^2+0.79\times0.5 = 0.4475\leq 0.5$ , donc $H_0$ est vraie.

\medskip

Hérédité: Supposons $H_k$ vraie pour \textbf\underline{{$k$ fixé}} et montrons que $H_{k+1}$ est vraie. C'est-à-dire, montrons que si $u_k \geq  u_{k+1}$ alors $u_{k+1} \geq  u_{k+2}$.

 Par croissance de la fonction $f$ sur l'intervalle $[0;1]$ et puisque les termes en jeu sont dans $[0;1]$, on a \[ u_k \geq u_{k+1} \quad \Rightarrow \quad f(u_k) \geq  f(u_{k+1}) \quad \Rightarrow \quad  u_{k+1}\geq u_{k+2}. \]

Donc $H_{k} \Rightarrow H_{k+1}$.

 On a finalement démontré par récurrence que pour tout $n \in\mathbb{N},~ u_n \geq u_{n+1}$ et donc $(u_n)_{n\in\mathbb{N}}$ est décroissante .

\end{enumerate}
\end{frame}


\begin{frame}

\begin{enumerate}\setcounter{enumi}{3}

	\item D'après les questions $3)$ et $2)$ , $(u_n)_{n\in\mathbb{N}}$ est décroissante et minorée. Donc d'après le théorème de convergence monotone, $(u_n)_{n\in\mathbb{N}}$ converge.

\end{enumerate}\end{frame}




\end{document}