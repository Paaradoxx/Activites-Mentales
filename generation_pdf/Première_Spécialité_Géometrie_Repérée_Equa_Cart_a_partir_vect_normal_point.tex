\documentclass[15pt, mathserif]{beamer}

\usepackage[french]{babel}
\usepackage[T1]{fontenc}
\usepackage[utf8]{inputenc}
%\usepackage{esvect}
\usepackage{bm}
\usepackage{eurosym}
\usepackage{tikz}
\usepackage{pgf,tikz,pgfplots}
\pgfplotsset{compat=1.15}
\usepackage{mathrsfs}
\usetikzlibrary{arrows}
\usetikzlibrary{arrows.meta}

\usetikzlibrary{mindmap}
\usepackage{multicol}
\usepackage[tikz]{bclogo}
\usepackage{tkz-tab}
\usepackage{amsmath, tabu}
\usepackage{esvect} %\vv{AB} pour le vecteur AB

\DeclareMathOperator{\e}{e}

%% Tableau

\usepackage{makecell}
\setcellgapes{1pt}
\makegapedcells
\newcolumntype{R}[1]{>{\raggedleft\arraybackslash }b{#1}}
\newcolumntype{L}[1]{>{\raggedright\arraybackslash }b{#1}}
\newcolumntype{C}[1]{>{\centering\arraybackslash }b{#1}}


%pour avoir des parenthèses rondes dans le package fourier
\DeclareSymbolFont{cmoperators}   {OT1}{cmr} {m}{n}
\DeclareSymbolFont{cmlargesymbols}{OMX}{cmex}{m}{n}

\usefonttheme{professionalfonts} %permet d'enlever un bug avec fourier
\usepackage{fourier}
\DeclareMathDelimiter{(}{\mathopen} {cmoperators}{"28}{cmlargesymbols}{"00}
\DeclareMathDelimiter{)}{\mathclose}{cmoperators}{"29}{cmlargesymbols}{"01}

%Graphiques 

\usepackage{pgf,tikz,pgfplots}
\pgfplotsset{compat=1.15}
\usepackage{mathrsfs}
\usetikzlibrary{arrows}
\usetikzlibrary{mindmap}

%ensembles de nbres

\newcommand{\R}{\mathbb{R}}			%permet d'écrire le R "ensemble des réels"'
\newcommand{\N}{\mathbb{N}}			%permet d'écrire le N "ensemble des entiers naturels"
\newcommand{\Z}{\mathbb{Z}}			%permet d'écrire le Z "ensemble des entiers relatifs"
\newcommand{\Prem}{\mathbb{P}}	%permet d'écrire le P "ensemble des nombres premiers" (qui n'a pas marché avec le \P car il existe déjà)
\newcommand{\D}{\mathbb{D}}
\newcommand{\Df}{\mathcal{D}_f}
\newcommand{\Cf}{\mathcal{C}_f}

\newcommand{\Q}{\mathbb{Q}}


\newcommand{\st}[1]{$(#1_n)_{n \in \N}$}

\usetheme{Madrid}
\useoutertheme{miniframes} % Alternatively: miniframes, infolines, split
\useinnertheme{circles}
\definecolor{UBCblue}{rgb}{0.1, 0.25, 0.4} % UBC Blue (primary)
\definecolor{bordeaux}{RGB}{128,0,0}
\usecolortheme[named=UBCblue]{structure}

\usepackage{color} % J'aime bien définir mes couleurs
\definecolor{propcolor}{rgb}{0, 0.5, 1}
\definecolor{thcolor}{rgb}{0.6, 0.07, 0.07}
\colorlet{louis}{blue!70!green!60!white}
\colorlet{sakura}{pink!40!red}

\title{Activités Mentales}
\date{24 Août 2023}

\newcommand{\vco}[2]{\begin{pmatrix} #1 \\ #2 \end{pmatrix}} %Coordonnées de vecteur
\newenvironment{eq}{\begin{cases}\begin{tabu}{ccccc}}{\end{tabu}\end{cases}}
\newenvironment{eql}{\begin{cases}\begin{tabu}{cccccl}}{\end{tabu}\end{cases}}
\newenvironment{eqrl}{\begin{cases}\begin{tabu}{rl}}{\end{tabu}\end{cases}}

\newenvironment{Eq}{\begin{center}\begin{tabular}{rrcl}}{\end{tabular}\end{center}}
\newcommand{\ligneq}[2]{$\Longleftrightarrow$ & $#1$ & $=$ & $#2$ \\}
\newcommand{\Ligneq}[2]{ & $#1$ & $=$ & $#2$ \\}

\newenvironment{RPN}{\begin{center}\begin{tabular}{rrclcrcl}}{\end{tabular}\end{center}}
\newcommand{\Lignerpn}[4]{ & $#1$ & $=$ & $#2$ & ou & $#3$ & $=$ & $#4$ \\}
\newcommand{\lignerpn}[4]{$\Longleftrightarrow$ & $#1$ & $=$ & $#2$ & ou & $#3$ & $=$ & $#4$ \\}

\newenvironment{TRPN}{\begin{center}\begin{tabular}{rrclcrclcrcl}}{\end{tabular}\end{center}}
\newcommand{\Lignetrpn}[6]{ & $#1$ & $=$ & $#2$ & ou & $#3$ & $=$ & $#4$ & ou & $#5$ & $=$ & $#6$ \\}
\newcommand{\lignetrpn}[6]{$\Longleftrightarrow$ & $#1$ & $=$ & $#2$ & ou & $#3$ & $=$ & $#4$ & ou & $#5$ & $=$ & $#6$ \\}
\begin{document}

\begin{frame}
    \titlepage
\end{frame}

\begin{frame} 
	\frametitle{Question 1}
Considérons un point $A(-6;3)$ et $\vec{n} \begin{pmatrix}
  8\\ 
 8 
 \end{pmatrix}$. 
 
  Quel est l'ensemble des points $M(x;y)$ tels que $\overrightarrow{AM} \cdot \vec{n}=0$.\end{frame}


\begin{frame} 
	\frametitle{Question 2}
Considérons un point $A(-6;7)$ et $\vec{n} \begin{pmatrix}
  7\\ 
 4 
 \end{pmatrix}$. 
 
  Quel est l'ensemble des points $M(x;y)$ tels que $\overrightarrow{AM} \cdot \vec{n}=0$.\end{frame}


\begin{frame} 
	\frametitle{Question 3}
Considérons un point $A(-5;5)$ et $\vec{n} \begin{pmatrix}
  -6\\ 
 7 
 \end{pmatrix}$. 
 
  Quel est l'ensemble des points $M(x;y)$ tels que $\overrightarrow{AM} \cdot \vec{n}=0$.\end{frame}


\begin{frame} 
	\frametitle{Question 4}
Considérons un point $A(-5;6)$ et $\vec{n} \begin{pmatrix}
  -4\\ 
 1 
 \end{pmatrix}$. 
 
  Quel est l'ensemble des points $M(x;y)$ tels que $\overrightarrow{AM} \cdot \vec{n}=0$.\end{frame}


\begin{frame} 
	\frametitle{Question 5}
Considérons un point $A(-8;10)$ et $\vec{n} \begin{pmatrix}
  10\\ 
 2 
 \end{pmatrix}$. 
 
  Quel est l'ensemble des points $M(x;y)$ tels que $\overrightarrow{AM} \cdot \vec{n}=0$.\end{frame}


\begin{frame}
\vspace{-10mm}
	\frametitle{Correction 1}
\vspace{0.5cm} 
 Considérons un point $A(-6;3)$ et $\vec{n} \begin{pmatrix}
  8\\ 
 8 
 \end{pmatrix}$. 
 
 Quel est l'ensemble des points $M(x;y)$ tels que $\overrightarrow{AM} \cdot \vec{n}=0$?
 
 Le vecteur $\overrightarrow{AM}$ a pour coordonnées $\begin{pmatrix} 
 x_M-x_A \\ 
  y_M-y_A 
 \end{pmatrix}=\begin{pmatrix} x+6 \\ 
 y -3 
 \end{pmatrix}$. 
 
 \begin{Eq} 
 \Ligneq{\overrightarrow{AM} \cdot \vec{n}}{0} 
 \ligneq{(x+6) \times 8 + (y-3) \times8}{0} 
 \ligneq{8x+48+8y-24}{0} 
 \ligneq{8x+8y+24}{0} 
 \end{Eq} 
 
 On obtient une équation cartésienne : l'ensemble des points $M$ tels que $\overrightarrow{AM}\cdot \vec{n}=0$ est donc une droite $d$ d'équation cartésienne $8x+8y+24=0$ telle que $a=8$, $b=8$ et $c=24$. \end{frame}


\begin{frame}
\vspace{-10mm}
	\frametitle{Correction 2}
\vspace{0.5cm} 
 Considérons un point $A(-6;7)$ et $\vec{n} \begin{pmatrix}
  7\\ 
 4 
 \end{pmatrix}$. 
 
 Quel est l'ensemble des points $M(x;y)$ tels que $\overrightarrow{AM} \cdot \vec{n}=0$?
 
 Le vecteur $\overrightarrow{AM}$ a pour coordonnées $\begin{pmatrix} 
 x_M-x_A \\ 
  y_M-y_A 
 \end{pmatrix}=\begin{pmatrix} x+6 \\ 
 y -7 
 \end{pmatrix}$. 
 
 \begin{Eq} 
 \Ligneq{\overrightarrow{AM} \cdot \vec{n}}{0} 
 \ligneq{(x+6) \times 7 + (y-7) \times4}{0} 
 \ligneq{7x+42+4y-28}{0} 
 \ligneq{7x+4y+14}{0} 
 \end{Eq} 
 
 On obtient une équation cartésienne : l'ensemble des points $M$ tels que $\overrightarrow{AM}\cdot \vec{n}=0$ est donc une droite $d$ d'équation cartésienne $7x+4y+14=0$ telle que $a=7$, $b=4$ et $c=14$. \end{frame}


\begin{frame}
\vspace{-10mm}
	\frametitle{Correction 3}
\vspace{0.5cm} 
 Considérons un point $A(-5;5)$ et $\vec{n} \begin{pmatrix}
  -6\\ 
 7 
 \end{pmatrix}$. 
 
 Quel est l'ensemble des points $M(x;y)$ tels que $\overrightarrow{AM} \cdot \vec{n}=0$?
 
 Le vecteur $\overrightarrow{AM}$ a pour coordonnées $\begin{pmatrix} 
 x_M-x_A \\ 
  y_M-y_A 
 \end{pmatrix}=\begin{pmatrix} x+5 \\ 
 y -5 
 \end{pmatrix}$. 
 
 \begin{Eq} 
 \Ligneq{\overrightarrow{AM} \cdot \vec{n}}{0} 
 \ligneq{(x+5) \times \left(-6\right) + (y-5) \times7}{0} 
 \ligneq{-6x-30+7y-35}{0} 
 \ligneq{-6x+7y-65}{0} 
 \end{Eq} 
 
 On obtient une équation cartésienne : l'ensemble des points $M$ tels que $\overrightarrow{AM}\cdot \vec{n}=0$ est donc une droite $d$ d'équation cartésienne $-6x+7y-65=0$ telle que $a=-6$, $b=7$ et $c=-65$. \end{frame}


\begin{frame}
\vspace{-10mm}
	\frametitle{Correction 4}
\vspace{0.5cm} 
 Considérons un point $A(-5;6)$ et $\vec{n} \begin{pmatrix}
  -4\\ 
 1 
 \end{pmatrix}$. 
 
 Quel est l'ensemble des points $M(x;y)$ tels que $\overrightarrow{AM} \cdot \vec{n}=0$?
 
 Le vecteur $\overrightarrow{AM}$ a pour coordonnées $\begin{pmatrix} 
 x_M-x_A \\ 
  y_M-y_A 
 \end{pmatrix}=\begin{pmatrix} x+5 \\ 
 y -6 
 \end{pmatrix}$. 
 
 \begin{Eq} 
 \Ligneq{\overrightarrow{AM} \cdot \vec{n}}{0} 
 \ligneq{(x+5) \times \left(-4\right) + (y-6) \times1}{0} 
 \ligneq{-4x-20+y-6}{0} 
 \ligneq{-4x+y-26}{0} 
 \end{Eq} 
 
 On obtient une équation cartésienne : l'ensemble des points $M$ tels que $\overrightarrow{AM}\cdot \vec{n}=0$ est donc une droite $d$ d'équation cartésienne $-4x+y-26=0$ telle que $a=-4$, $b=1$ et $c=-26$. \end{frame}


\begin{frame}
\vspace{-10mm}
	\frametitle{Correction 5}
\vspace{0.5cm} 
 Considérons un point $A(-8;10)$ et $\vec{n} \begin{pmatrix}
  10\\ 
 2 
 \end{pmatrix}$. 
 
 Quel est l'ensemble des points $M(x;y)$ tels que $\overrightarrow{AM} \cdot \vec{n}=0$?
 
 Le vecteur $\overrightarrow{AM}$ a pour coordonnées $\begin{pmatrix} 
 x_M-x_A \\ 
  y_M-y_A 
 \end{pmatrix}=\begin{pmatrix} x+8 \\ 
 y -10 
 \end{pmatrix}$. 
 
 \begin{Eq} 
 \Ligneq{\overrightarrow{AM} \cdot \vec{n}}{0} 
 \ligneq{(x+8) \times 10 + (y-10) \times2}{0} 
 \ligneq{10x+80+2y-20}{0} 
 \ligneq{10x+2y+60}{0} 
 \end{Eq} 
 
 On obtient une équation cartésienne : l'ensemble des points $M$ tels que $\overrightarrow{AM}\cdot \vec{n}=0$ est donc une droite $d$ d'équation cartésienne $10x+2y+60=0$ telle que $a=10$, $b=2$ et $c=60$. \end{frame}




\end{document}