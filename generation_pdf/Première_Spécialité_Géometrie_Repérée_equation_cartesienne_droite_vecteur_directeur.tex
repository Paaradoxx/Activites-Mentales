\documentclass[15pt, mathserif]{beamer}

\usepackage[french]{babel}
\usepackage[T1]{fontenc}
\usepackage[utf8]{inputenc}
%\usepackage{esvect}
\usepackage{bm}
\usepackage{eurosym}
\usepackage{tikz}
\usepackage{pgf,tikz,pgfplots}
\pgfplotsset{compat=1.15}
\usepackage{mathrsfs}
\usetikzlibrary{arrows}
\usetikzlibrary{arrows.meta}

\usetikzlibrary{mindmap}
\usepackage{multicol}
\usepackage[tikz]{bclogo}
\usepackage{tkz-tab}
\usepackage{amsmath, tabu}
\usepackage{esvect} %\vv{AB} pour le vecteur AB

\DeclareMathOperator{\e}{e}

%% Tableau

\usepackage{makecell}
\setcellgapes{1pt}
\makegapedcells
\newcolumntype{R}[1]{>{\raggedleft\arraybackslash }b{#1}}
\newcolumntype{L}[1]{>{\raggedright\arraybackslash }b{#1}}
\newcolumntype{C}[1]{>{\centering\arraybackslash }b{#1}}


%pour avoir des parenthèses rondes dans le package fourier
\DeclareSymbolFont{cmoperators}   {OT1}{cmr} {m}{n}
\DeclareSymbolFont{cmlargesymbols}{OMX}{cmex}{m}{n}

\usefonttheme{professionalfonts} %permet d'enlever un bug avec fourier
\usepackage{fourier}
\DeclareMathDelimiter{(}{\mathopen} {cmoperators}{"28}{cmlargesymbols}{"00}
\DeclareMathDelimiter{)}{\mathclose}{cmoperators}{"29}{cmlargesymbols}{"01}

%Graphiques 

\usepackage{pgf,tikz,pgfplots}
\pgfplotsset{compat=1.15}
\usepackage{mathrsfs}
\usetikzlibrary{arrows}
\usetikzlibrary{mindmap}

%ensembles de nbres

\newcommand{\R}{\mathbb{R}}			%permet d'écrire le R "ensemble des réels"'
\newcommand{\N}{\mathbb{N}}			%permet d'écrire le N "ensemble des entiers naturels"
\newcommand{\Z}{\mathbb{Z}}			%permet d'écrire le Z "ensemble des entiers relatifs"
\newcommand{\Prem}{\mathbb{P}}	%permet d'écrire le P "ensemble des nombres premiers" (qui n'a pas marché avec le \P car il existe déjà)
\newcommand{\D}{\mathbb{D}}
\newcommand{\Df}{\mathcal{D}_f}
\newcommand{\Cf}{\mathcal{C}_f}

\newcommand{\Q}{\mathbb{Q}}


\newcommand{\st}[1]{$(#1_n)_{n \in \N}$}

\usetheme{Madrid}
\useoutertheme{miniframes} % Alternatively: miniframes, infolines, split
\useinnertheme{circles}
\definecolor{UBCblue}{rgb}{0.1, 0.25, 0.4} % UBC Blue (primary)
\definecolor{bordeaux}{RGB}{128,0,0}
\usecolortheme[named=UBCblue]{structure}

\usepackage{color} % J'aime bien définir mes couleurs
\definecolor{propcolor}{rgb}{0, 0.5, 1}
\definecolor{thcolor}{rgb}{0.6, 0.07, 0.07}
\colorlet{louis}{blue!70!green!60!white}
\colorlet{sakura}{pink!40!red}

\title{Activités Mentales}
\date{24 Août 2023}

\newcommand{\vco}[2]{\begin{pmatrix} #1 \\ #2 \end{pmatrix}} %Coordonnées de vecteur
\newenvironment{eq}{\begin{cases}\begin{tabu}{ccccc}}{\end{tabu}\end{cases}}
\newenvironment{eql}{\begin{cases}\begin{tabu}{cccccl}}{\end{tabu}\end{cases}}
\newenvironment{eqrl}{\begin{cases}\begin{tabu}{rl}}{\end{tabu}\end{cases}}

\newenvironment{Eq}{\begin{center}\begin{tabular}{rrcl}}{\end{tabular}\end{center}}
\newcommand{\ligneq}[2]{$\Longleftrightarrow$ & $#1$ & $=$ & $#2$ \\}
\newcommand{\Ligneq}[2]{ & $#1$ & $=$ & $#2$ \\}

\newenvironment{RPN}{\begin{center}\begin{tabular}{rrclcrcl}}{\end{tabular}\end{center}}
\newcommand{\Lignerpn}[4]{ & $#1$ & $=$ & $#2$ & ou & $#3$ & $=$ & $#4$ \\}
\newcommand{\lignerpn}[4]{$\Longleftrightarrow$ & $#1$ & $=$ & $#2$ & ou & $#3$ & $=$ & $#4$ \\}

\newenvironment{TRPN}{\begin{center}\begin{tabular}{rrclcrclcrcl}}{\end{tabular}\end{center}}
\newcommand{\Lignetrpn}[6]{ & $#1$ & $=$ & $#2$ & ou & $#3$ & $=$ & $#4$ & ou & $#5$ & $=$ & $#6$ \\}
\newcommand{\lignetrpn}[6]{$\Longleftrightarrow$ & $#1$ & $=$ & $#2$ & ou & $#3$ & $=$ & $#4$ & ou & $#5$ & $=$ & $#6$ \\}
\begin{document}

\begin{frame}
    \titlepage
\end{frame}

\begin{frame} 
	\frametitle{Question 1}
On considère le point $M\left(-2~;~8\right)$ et le vecteur $\vv{u}\vco{-1}{-7}$.

\bigskip

Déterminer une équation cartésienne de la droite passant par $M$ et de vecteur directeur $\vv{u}$\end{frame}


\begin{frame} 
	\frametitle{Question 2}
On considère le point $M\left(-2~;~-6\right)$ et le vecteur $\vv{u}\vco{-2}{7}$.

\bigskip

Déterminer une équation cartésienne de la droite passant par $M$ et de vecteur directeur $\vv{u}$\end{frame}


\begin{frame} 
	\frametitle{Question 3}
On considère le point $M\left(-10~;~-2\right)$ et le vecteur $\vv{u}\vco{5}{-1}$.

\bigskip

Déterminer une équation cartésienne de la droite passant par $M$ et de vecteur directeur $\vv{u}$\end{frame}


\begin{frame} 
	\frametitle{Question 4}
On considère le point $M\left(-8~;~-7\right)$ et le vecteur $\vv{u}\vco{-10}{3}$.

\bigskip

Déterminer une équation cartésienne de la droite passant par $M$ et de vecteur directeur $\vv{u}$\end{frame}


\begin{frame} 
	\frametitle{Question 5}
On considère le point $M\left(-10~;~-7\right)$ et le vecteur $\vv{u}\vco{2}{-5}$.

\bigskip

Déterminer une équation cartésienne de la droite passant par $M$ et de vecteur directeur $\vv{u}$\end{frame}


\begin{frame}
\vspace{-10mm}
	\frametitle{Correction 1}
\vspace*{2em}
$d$ est de vecteur directeur $\vv{u}\vco{-1}{-7}$ et passant par $M\left(-2~;~8\right)$.

Une équation cartésienne de la droite est de la forme $ax+by+c=0$.

Comme  $\vv{u}\vco{-1}{-7}$ est un vecteur directeur de $d$, il est de la forme 

\smallskip

\hfil $\vco{-1}{-7}=\vco{-b}{a} \Leftrightarrow \begin{cases} -1& = -b \\ -7&=a \end{cases} \Leftrightarrow \begin{cases} b &= 1\\ a &=-7\end{cases}.$\smallskip

 L'équation est alors de la forme $-7x+y + c = 0$. Or \[M(-2~;~8) \in d \Leftrightarrow -7\times \left(-2\right)+8+c=0 \Leftrightarrow 22+c = 0 \Leftrightarrow c = -22.\] Finalement une équation cartésienne de la droite passant par $M\left(-2~;~8\right)$ et de vecteur directeur $\vv{u}\vco{-1}{-7}$ est $d:~-7x+y-22=0.$\end{frame}


\begin{frame}
\vspace{-10mm}
	\frametitle{Correction 2}
\vspace*{2em}
$d$ est de vecteur directeur $\vv{u}\vco{-2}{7}$ et passant par $M\left(-2~;~-6\right)$.

Une équation cartésienne de la droite est de la forme $ax+by+c=0$.

Comme  $\vv{u}\vco{-2}{7}$ est un vecteur directeur de $d$, il est de la forme 

\smallskip

\hfil $\vco{-2}{7}=\vco{-b}{a} \Leftrightarrow \begin{cases} -2& = -b \\ 7&=a \end{cases} \Leftrightarrow \begin{cases} b &= 2\\ a &=7\end{cases}.$\smallskip

 L'équation est alors de la forme $7x+2y + c = 0$. Or \[M(-2~;~-6) \in d \Leftrightarrow 7\times \left(-2\right)+2\times \left(-6\right)+c=0 \Leftrightarrow -26+c = 0 \Leftrightarrow c = 26.\] Finalement une équation cartésienne de la droite passant par $M\left(-2~;~-6\right)$ et de vecteur directeur $\vv{u}\vco{-2}{7}$ est $d:~7x+2y+26=0.$\end{frame}


\begin{frame}
\vspace{-10mm}
	\frametitle{Correction 3}
\vspace*{2em}
$d$ est de vecteur directeur $\vv{u}\vco{5}{-1}$ et passant par $M\left(-10~;~-2\right)$.

Une équation cartésienne de la droite est de la forme $ax+by+c=0$.

Comme  $\vv{u}\vco{5}{-1}$ est un vecteur directeur de $d$, il est de la forme 

\smallskip

\hfil $\vco{5}{-1}=\vco{-b}{a} \Leftrightarrow \begin{cases} 5& = -b \\ -1&=a \end{cases} \Leftrightarrow \begin{cases} b &= -5\\ a &=-1\end{cases}.$\smallskip

 L'équation est alors de la forme $-x-5y + c = 0$. Or \[M(-10~;~-2) \in d \Leftrightarrow -\left(-10\right)-5\times \left(-2\right)+c=0 \Leftrightarrow 20+c = 0 \Leftrightarrow c = -20.\] Finalement une équation cartésienne de la droite passant par $M\left(-10~;~-2\right)$ et de vecteur directeur $\vv{u}\vco{5}{-1}$ est $d:~-x-5y-20=0.$\end{frame}


\begin{frame}
\vspace{-10mm}
	\frametitle{Correction 4}
\vspace*{2em}
$d$ est de vecteur directeur $\vv{u}\vco{-10}{3}$ et passant par $M\left(-8~;~-7\right)$.

Une équation cartésienne de la droite est de la forme $ax+by+c=0$.

Comme  $\vv{u}\vco{-10}{3}$ est un vecteur directeur de $d$, il est de la forme 

\smallskip

\hfil $\vco{-10}{3}=\vco{-b}{a} \Leftrightarrow \begin{cases} -10& = -b \\ 3&=a \end{cases} \Leftrightarrow \begin{cases} b &= 10\\ a &=3\end{cases}.$\smallskip

 L'équation est alors de la forme $3x+10y + c = 0$. Or \[M(-8~;~-7) \in d \Leftrightarrow 3\times \left(-8\right)+10\times \left(-7\right)+c=0 \Leftrightarrow -94+c = 0 \Leftrightarrow c = 94.\] Finalement une équation cartésienne de la droite passant par $M\left(-8~;~-7\right)$ et de vecteur directeur $\vv{u}\vco{-10}{3}$ est $d:~3x+10y+94=0.$\end{frame}


\begin{frame}
\vspace{-10mm}
	\frametitle{Correction 5}
\vspace*{2em}
$d$ est de vecteur directeur $\vv{u}\vco{2}{-5}$ et passant par $M\left(-10~;~-7\right)$.

Une équation cartésienne de la droite est de la forme $ax+by+c=0$.

Comme  $\vv{u}\vco{2}{-5}$ est un vecteur directeur de $d$, il est de la forme 

\smallskip

\hfil $\vco{2}{-5}=\vco{-b}{a} \Leftrightarrow \begin{cases} 2& = -b \\ -5&=a \end{cases} \Leftrightarrow \begin{cases} b &= -2\\ a &=-5\end{cases}.$\smallskip

 L'équation est alors de la forme $-5x-2y + c = 0$. Or \[M(-10~;~-7) \in d \Leftrightarrow -5\times \left(-10\right)-2\times \left(-7\right)+c=0 \Leftrightarrow 64+c = 0 \Leftrightarrow c = -64.\] Finalement une équation cartésienne de la droite passant par $M\left(-10~;~-7\right)$ et de vecteur directeur $\vv{u}\vco{2}{-5}$ est $d:~-5x-2y-64=0.$\end{frame}




\end{document}