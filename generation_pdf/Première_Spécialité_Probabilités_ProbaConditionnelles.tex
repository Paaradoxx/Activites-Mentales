\documentclass[15pt, mathserif]{beamer}

\usepackage[french]{babel}
\usepackage[T1]{fontenc}
\usepackage[utf8]{inputenc}
%\usepackage{esvect}
\usepackage{bm}
\usepackage{eurosym}
\usepackage{tikz}
\usepackage{pgf,tikz,pgfplots}
\pgfplotsset{compat=1.15}
\usepackage{mathrsfs}
\usetikzlibrary{arrows}
\usetikzlibrary{arrows.meta}

\usetikzlibrary{mindmap}
\usepackage{multicol}
\usepackage[tikz]{bclogo}
\usepackage{tkz-tab}
\usepackage{amsmath, tabu}
\usepackage{esvect} %\vv{AB} pour le vecteur AB

\DeclareMathOperator{\e}{e}

%% Tableau

\usepackage{makecell}
\setcellgapes{1pt}
\makegapedcells
\newcolumntype{R}[1]{>{\raggedleft\arraybackslash }b{#1}}
\newcolumntype{L}[1]{>{\raggedright\arraybackslash }b{#1}}
\newcolumntype{C}[1]{>{\centering\arraybackslash }b{#1}}


%pour avoir des parenthèses rondes dans le package fourier
\DeclareSymbolFont{cmoperators}   {OT1}{cmr} {m}{n}
\DeclareSymbolFont{cmlargesymbols}{OMX}{cmex}{m}{n}

\usefonttheme{professionalfonts} %permet d'enlever un bug avec fourier
\usepackage{fourier}
\DeclareMathDelimiter{(}{\mathopen} {cmoperators}{"28}{cmlargesymbols}{"00}
\DeclareMathDelimiter{)}{\mathclose}{cmoperators}{"29}{cmlargesymbols}{"01}

%Graphiques 

\usepackage{pgf,tikz,pgfplots}
\pgfplotsset{compat=1.15}
\usepackage{mathrsfs}
\usetikzlibrary{arrows}
\usetikzlibrary{mindmap}

%ensembles de nbres

\newcommand{\R}{\mathbb{R}}			%permet d'écrire le R "ensemble des réels"'
\newcommand{\N}{\mathbb{N}}			%permet d'écrire le N "ensemble des entiers naturels"
\newcommand{\Z}{\mathbb{Z}}			%permet d'écrire le Z "ensemble des entiers relatifs"
\newcommand{\Prem}{\mathbb{P}}	%permet d'écrire le P "ensemble des nombres premiers" (qui n'a pas marché avec le \P car il existe déjà)
\newcommand{\D}{\mathbb{D}}
\newcommand{\Df}{\mathcal{D}_f}
\newcommand{\Cf}{\mathcal{C}_f}

\newcommand{\Q}{\mathbb{Q}}


\newcommand{\st}[1]{$(#1_n)_{n \in \N}$}

\usetheme{Madrid}
\useoutertheme{miniframes} % Alternatively: miniframes, infolines, split
\useinnertheme{circles}
\definecolor{UBCblue}{rgb}{0.1, 0.25, 0.4} % UBC Blue (primary)
\definecolor{bordeaux}{RGB}{128,0,0}
\usecolortheme[named=UBCblue]{structure}

\usepackage{color} % J'aime bien définir mes couleurs
\definecolor{propcolor}{rgb}{0, 0.5, 1}
\definecolor{thcolor}{rgb}{0.6, 0.07, 0.07}
\colorlet{louis}{blue!70!green!60!white}
\colorlet{sakura}{pink!40!red}

\title{Activités Mentales}
\date{24 Août 2023}

\newcommand{\vco}[2]{\begin{pmatrix} #1 \\ #2 \end{pmatrix}} %Coordonnées de vecteur
\newenvironment{eq}{\begin{cases}\begin{tabu}{ccccc}}{\end{tabu}\end{cases}}
\newenvironment{eql}{\begin{cases}\begin{tabu}{cccccl}}{\end{tabu}\end{cases}}
\newenvironment{eqrl}{\begin{cases}\begin{tabu}{rl}}{\end{tabu}\end{cases}}

\newenvironment{Eq}{\begin{center}\begin{tabular}{rrcl}}{\end{tabular}\end{center}}
\newcommand{\ligneq}[2]{$\Longleftrightarrow$ & $#1$ & $=$ & $#2$ \\}
\newcommand{\Ligneq}[2]{ & $#1$ & $=$ & $#2$ \\}

\newenvironment{RPN}{\begin{center}\begin{tabular}{rrclcrcl}}{\end{tabular}\end{center}}
\newcommand{\Lignerpn}[4]{ & $#1$ & $=$ & $#2$ & ou & $#3$ & $=$ & $#4$ \\}
\newcommand{\lignerpn}[4]{$\Longleftrightarrow$ & $#1$ & $=$ & $#2$ & ou & $#3$ & $=$ & $#4$ \\}

\newenvironment{TRPN}{\begin{center}\begin{tabular}{rrclcrclcrcl}}{\end{tabular}\end{center}}
\newcommand{\Lignetrpn}[6]{ & $#1$ & $=$ & $#2$ & ou & $#3$ & $=$ & $#4$ & ou & $#5$ & $=$ & $#6$ \\}
\newcommand{\lignetrpn}[6]{$\Longleftrightarrow$ & $#1$ & $=$ & $#2$ & ou & $#3$ & $=$ & $#4$ & ou & $#5$ & $=$ & $#6$ \\}
\begin{document}

\begin{frame}
    \titlepage
\end{frame}

\begin{frame} 
	\frametitle{Question 1}
On étudie dans cette exercice deux évènements K et E. On sait que : 
 
 \begin{itemize} 
 	 \item La probabilité de ne pas avoir l'évènement K est 2/5
 	  \item La probabilité d'avoir l'évènement E sachant qu'on a l'évènement K est 2/5
 	  \item La probabilité d'avoir l'évènement E sachant que l'évènement K n'a pas eu lieu est 3/5
 \end{itemize} 
 
 \begin{enumerate} 
 	 \item Construire un arbre pondéré de la situation. 
 	 \item Calculer $\Prem(K\cap E)$
 	 \item Donner la probabilité d'avoir l'évènement E. 
 \item Calculer $\Prem_{E}(K)$. 
 \end{enumerate} \end{frame}


\begin{frame} 
	\frametitle{Question 2}
On étudie dans cette exercice deux évènements R et P. On sait que : 
 
 \begin{itemize} 
 	 \item La probabilité de ne pas avoir l'évènement R est 3/10 
 	 \item La probabilité de ne pas avoir l'évènement P sachant qu'on a l'évènement R est 1/5 
 	 \item La probabilité de ne pas avoir l'évènement P sachant que l'évènement R n'a pas eu lieu est 1/2
 \end{itemize} 
 
 \begin{enumerate} 
 	 \item Construire un arbre pondéré de la situation. 
 	 \item Calculer $\Prem(\overline{R} \cap \overline{P})$
 	 \item Donner la probabilité de ne pas avoir l'évènement P. 
 \item Calculer $\Prem_{\overline{P}}(R)$. 
 \end{enumerate} \end{frame}


\begin{frame} 
	\frametitle{Question 3}
On étudie dans cette exercice deux évènements W et G. On sait que : 
 
 \begin{itemize} 
 	 \item La probabilité de ne pas avoir l'évènement W est 3/5 
 	 \item La probabilité de ne pas avoir l'évènement G sachant qu'on a l'évènement W est 1/5 
 	 \item La probabilité de ne pas avoir l'évènement G sachant que l'évènement W n'a pas eu lieu est 7/10
 \end{itemize} 
 
 \begin{enumerate} 
 	 \item Construire un arbre pondéré de la situation. 
 	 \item Calculer $\Prem(\overline{W} \cap G)$
 	 \item Donner la probabilité d'avoir l'évènement G. 
 \item Calculer $\Prem_{G}(W)$. 
 \end{enumerate} \end{frame}


\begin{frame} 
	\frametitle{Question 4}
On étudie dans cette exercice deux évènements E et C. On sait que : 
 
 \begin{itemize} 
 	 \item La probabilité de ne pas avoir l'évènement E est 1/5
 	  \item La probabilité de ne pas avoir l'évènement C sachant qu'on a l'évènement E est 7/10 
 	 \item La probabilité d'avoir l'évènement C sachant que l'évènement E n'a pas eu lieu est 1/5
 \end{itemize} 
 
 \begin{enumerate} 
 	 \item Construire un arbre pondéré de la situation. 
 	 \item Calculer $\Prem(\overline{E} \cap \overline{C})$
 	 \item Donner la probabilité de ne pas avoir l'évènement C. 
 \item Calculer $\Prem_{\overline{C}}(E)$. 
 \end{enumerate} \end{frame}


\begin{frame} 
	\frametitle{Question 5}
On étudie dans cette exercice deux évènements O et I. On sait que : 
 
 \begin{itemize} 
 	 \item La probabilité d'avoir l'évènement O est 3/10 
 	 \item La probabilité d'avoir l'évènement I sachant qu'on a l'évènement O est 7/10 
 	 \item La probabilité de ne pas avoir l'évènement I sachant que l'évènement O n'a pas eu lieu est 2/5
 \end{itemize} 
 
 \begin{enumerate} 
 	 \item Construire un arbre pondéré de la situation. 
 	 \item Calculer $\Prem(O\cap \overline{I})$
 	 \item Donner la probabilité de ne pas avoir l'évènement I. 
 \item Calculer $\Prem_{\overline{I}}(O)$. 
 \end{enumerate} \end{frame}


\begin{frame}
\vspace{-10mm}
	\frametitle{Correction 1}
\tikzstyle{level 1}=[level distance=4cm, sibling distance=2cm] 
 \tikzstyle{level 2}=[level distance=4cm, sibling distance=1.5cm] 
 \tikzstyle{bag} = [text width=3em, text centered] 
 
 \begin{enumerate} \item ~ 
 
 	 \begin{tikzpicture}[grow=right, sloped] 
 	 \node[bag]{} 
 	 child{ 
 	 	 node[bag]{$\overline{K}$} 
 	 	 child { 
 	 	 	  node[ label=right: 
 	 	 	 	 {$\overline{E}$}] {} 
 	 	 	  edge from parent 
 	 	 	 node[above] {} 
 	 	 	 node[below]  {$2/5$} 
 	 	 	 } 
 	 	 child { 
 	 	 	 node[ label=right:
 	 	 	 {$E$}] {} 
 	 	 	 edge from parent 
 	 	 	 node[above] {$3/5$} 
 	 	 	 node[below]  {} 
 	 	 	 } 
 	 	 	 edge from parent 
 	 	 	  node[below] {$2/5$} 
 	 	 	 node[above]  {} 
 	 } 
 	 	   child { 
 	 	 	 node[bag] {$K$} 
 	 	 	 child { 
 	 	 	 	 node[ label=right: 
 	 	 	 	  {$\overline{E}$}] {} 
 	 	 	 	  edge from parent 
 	 	 	 node[above] {}
 	 	 	 	 node[below]  {$3/5$} 
 	 	 	 } 
 	 	 	 child { 
 	 	 	 node[ label=right: 
 	 	 	 	 {$E$}] {} 
 	 	 	 	 edge from parent 
 	 	 	 	 node[above] {$2/5$} 
 	 	 	 	 node[below]  {} 
 	 	 	 } 
 	 	 	 edge from parent 
 	 	 	 node[above] {$3/5$} 
 	 	 	 node[above]  {} 
 	 }; 
 	 \end{tikzpicture} 
 	 \item $\Prem(K\cap E)=\Prem(K) \times \Prem_{K}(E)= 3/5\times 2/5= 6/25$\end{enumerate} 
 
 \end{frame} 
 
 \begin{frame}  
 \begin{enumerate} \setcounter{enumi}{2}  
 	 \item On cherche $\Prem(E)$. D'après la formule des probabilités totales car $K$ et $\overline{K}$ forment une partition de l'univers, on a \begin{align*} \Prem(E) &= \Prem(E \cap K) + \Prem(E \cap \overline{K}) \\ 
 &= 2/5\times 3/5+ 3/5\times 2/5 \\ 
 &= 12/25
 \end{align*}
  
 \item $\Prem_{E}(K)=\dfrac{\Prem(K\cap E)}{\Prem(E)} \simeq 1/2$.
 \end{enumerate}\end{frame}


\begin{frame}
\vspace{-10mm}
	\frametitle{Correction 2}
\tikzstyle{level 1}=[level distance=4cm, sibling distance=2cm] 
 \tikzstyle{level 2}=[level distance=4cm, sibling distance=1.5cm] 
 \tikzstyle{bag} = [text width=3em, text centered] 
 
 \begin{enumerate} \item ~ 
 
 	 \begin{tikzpicture}[grow=right, sloped] 
 	 \node[bag]{} 
 	 child{ 
 	 	 node[bag]{$\overline{R}$} 
 	 	 child { 
 	 	 	  node[ label=right: 
 	 	 	 	 {$\overline{P}$}] {} 
 	 	 	  edge from parent 
 	 	 	 node[above] {} 
 	 	 	 node[below]  {$1/2$} 
 	 	 	 } 
 	 	 child { 
 	 	 	 node[ label=right:
 	 	 	 {$P$}] {} 
 	 	 	 edge from parent 
 	 	 	 node[above] {$1/2$} 
 	 	 	 node[below]  {} 
 	 	 	 } 
 	 	 	 edge from parent 
 	 	 	  node[below] {$3/10$} 
 	 	 	 node[above]  {} 
 	 } 
 	 	   child { 
 	 	 	 node[bag] {$R$} 
 	 	 	 child { 
 	 	 	 	 node[ label=right: 
 	 	 	 	  {$\overline{P}$}] {} 
 	 	 	 	  edge from parent 
 	 	 	 node[above] {}
 	 	 	 	 node[below]  {$1/5$} 
 	 	 	 } 
 	 	 	 child { 
 	 	 	 node[ label=right: 
 	 	 	 	 {$P$}] {} 
 	 	 	 	 edge from parent 
 	 	 	 	 node[above] {$4/5$} 
 	 	 	 	 node[below]  {} 
 	 	 	 } 
 	 	 	 edge from parent 
 	 	 	 node[above] {$7/10$} 
 	 	 	 node[above]  {} 
 	 }; 
 	 \end{tikzpicture} 
 	 \item $\Prem(\overline{R} \cap \overline{P})=\Prem(\overline{R}) \times \Prem_{\overline{R}}(\overline{P})= 3/10\times 1/2= 3/50$\end{enumerate} 
 
 \end{frame} 
 
 \begin{frame}  
 \begin{enumerate} \setcounter{enumi}{2}  
 	 \item On cherche $\Prem(\overline{P})$. D'après la formule des probabilités totales car $R$ et $\overline{R}$ forment une partition de l'univers, on a \begin{align*} \Prem(\overline{P}) &= \Prem(\overline{P } \cap R) + \Prem(\overline{P } \cap \overline{R}) \\ 
 &= 1/5\times 7/10+ 1/2\times 3/10 \\ 
 &= 29/100
 \end{align*}
  
 \item $\Prem_{\overline{P}}(R)=\dfrac{\Prem(R\cap \overline{P})}{\Prem(\overline{P})} \simeq 483/1000$.
 \end{enumerate}\end{frame}


\begin{frame}
\vspace{-10mm}
	\frametitle{Correction 3}
\tikzstyle{level 1}=[level distance=4cm, sibling distance=2cm] 
 \tikzstyle{level 2}=[level distance=4cm, sibling distance=1.5cm] 
 \tikzstyle{bag} = [text width=3em, text centered] 
 
 \begin{enumerate} \item ~ 
 
 	 \begin{tikzpicture}[grow=right, sloped] 
 	 \node[bag]{} 
 	 child{ 
 	 	 node[bag]{$\overline{W}$} 
 	 	 child { 
 	 	 	  node[ label=right: 
 	 	 	 	 {$\overline{G}$}] {} 
 	 	 	  edge from parent 
 	 	 	 node[above] {} 
 	 	 	 node[below]  {$7/10$} 
 	 	 	 } 
 	 	 child { 
 	 	 	 node[ label=right:
 	 	 	 {$G$}] {} 
 	 	 	 edge from parent 
 	 	 	 node[above] {$3/10$} 
 	 	 	 node[below]  {} 
 	 	 	 } 
 	 	 	 edge from parent 
 	 	 	  node[below] {$3/5$} 
 	 	 	 node[above]  {} 
 	 } 
 	 	   child { 
 	 	 	 node[bag] {$W$} 
 	 	 	 child { 
 	 	 	 	 node[ label=right: 
 	 	 	 	  {$\overline{G}$}] {} 
 	 	 	 	  edge from parent 
 	 	 	 node[above] {}
 	 	 	 	 node[below]  {$1/5$} 
 	 	 	 } 
 	 	 	 child { 
 	 	 	 node[ label=right: 
 	 	 	 	 {$G$}] {} 
 	 	 	 	 edge from parent 
 	 	 	 	 node[above] {$4/5$} 
 	 	 	 	 node[below]  {} 
 	 	 	 } 
 	 	 	 edge from parent 
 	 	 	 node[above] {$2/5$} 
 	 	 	 node[above]  {} 
 	 }; 
 	 \end{tikzpicture} 
 	 \item $\Prem(\overline{W} \cap G)=\Prem(\overline{W}) \times \Prem_{\overline{W}}(G)= 3/5\times 3/10= 12/25$\end{enumerate} 
 
 \end{frame} 
 
 \begin{frame}  
 \begin{enumerate} \setcounter{enumi}{2}  
 	 \item On cherche $\Prem(G)$. D'après la formule des probabilités totales car $W$ et $\overline{W}$ forment une partition de l'univers, on a \begin{align*} \Prem(G) &= \Prem(G \cap W) + \Prem(G \cap \overline{W}) \\ 
 &= 4/5\times 2/5+ 3/10\times 3/5 \\ 
 &= 1/2
 \end{align*}
  
 \item $\Prem_{G}(W)=\dfrac{\Prem(W\cap G)}{\Prem(G)} \simeq 16/25$.
 \end{enumerate}\end{frame}


\begin{frame}
\vspace{-10mm}
	\frametitle{Correction 4}
\tikzstyle{level 1}=[level distance=4cm, sibling distance=2cm] 
 \tikzstyle{level 2}=[level distance=4cm, sibling distance=1.5cm] 
 \tikzstyle{bag} = [text width=3em, text centered] 
 
 \begin{enumerate} \item ~ 
 
 	 \begin{tikzpicture}[grow=right, sloped] 
 	 \node[bag]{} 
 	 child{ 
 	 	 node[bag]{$\overline{E}$} 
 	 	 child { 
 	 	 	  node[ label=right: 
 	 	 	 	 {$\overline{C}$}] {} 
 	 	 	  edge from parent 
 	 	 	 node[above] {} 
 	 	 	 node[below]  {$4/5$} 
 	 	 	 } 
 	 	 child { 
 	 	 	 node[ label=right:
 	 	 	 {$C$}] {} 
 	 	 	 edge from parent 
 	 	 	 node[above] {$1/5$} 
 	 	 	 node[below]  {} 
 	 	 	 } 
 	 	 	 edge from parent 
 	 	 	  node[below] {$1/5$} 
 	 	 	 node[above]  {} 
 	 } 
 	 	   child { 
 	 	 	 node[bag] {$E$} 
 	 	 	 child { 
 	 	 	 	 node[ label=right: 
 	 	 	 	  {$\overline{C}$}] {} 
 	 	 	 	  edge from parent 
 	 	 	 node[above] {}
 	 	 	 	 node[below]  {$7/10$} 
 	 	 	 } 
 	 	 	 child { 
 	 	 	 node[ label=right: 
 	 	 	 	 {$C$}] {} 
 	 	 	 	 edge from parent 
 	 	 	 	 node[above] {$3/10$} 
 	 	 	 	 node[below]  {} 
 	 	 	 } 
 	 	 	 edge from parent 
 	 	 	 node[above] {$4/5$} 
 	 	 	 node[above]  {} 
 	 }; 
 	 \end{tikzpicture} 
 	 \item $\Prem(\overline{E} \cap \overline{C})=\Prem(\overline{E}) \times \Prem_{\overline{E}}(\overline{C})= 1/5\times 4/5= 7/50$\end{enumerate} 
 
 \end{frame} 
 
 \begin{frame}  
 \begin{enumerate} \setcounter{enumi}{2}  
 	 \item On cherche $\Prem(\overline{C})$. D'après la formule des probabilités totales car $E$ et $\overline{E}$ forment une partition de l'univers, on a \begin{align*} \Prem(\overline{C}) &= \Prem(\overline{C } \cap E) + \Prem(\overline{C } \cap \overline{E}) \\ 
 &= 7/10\times 4/5+ 4/5\times 1/5 \\ 
 &= 18/25
 \end{align*}
  
 \item $\Prem_{\overline{C}}(E)=\dfrac{\Prem(E\cap \overline{C})}{\Prem(\overline{C})} \simeq 389/500$.
 \end{enumerate}\end{frame}


\begin{frame}
\vspace{-10mm}
	\frametitle{Correction 5}
\tikzstyle{level 1}=[level distance=4cm, sibling distance=2cm] 
 \tikzstyle{level 2}=[level distance=4cm, sibling distance=1.5cm] 
 \tikzstyle{bag} = [text width=3em, text centered] 
 
 \begin{enumerate} \item ~ 
 
 	 \begin{tikzpicture}[grow=right, sloped] 
 	 \node[bag]{} 
 	 child{ 
 	 	 node[bag]{$\overline{O}$} 
 	 	 child { 
 	 	 	  node[ label=right: 
 	 	 	 	 {$\overline{I}$}] {} 
 	 	 	  edge from parent 
 	 	 	 node[above] {} 
 	 	 	 node[below]  {$2/5$} 
 	 	 	 } 
 	 	 child { 
 	 	 	 node[ label=right:
 	 	 	 {$I$}] {} 
 	 	 	 edge from parent 
 	 	 	 node[above] {$3/5$} 
 	 	 	 node[below]  {} 
 	 	 	 } 
 	 	 	 edge from parent 
 	 	 	  node[below] {$7/10$} 
 	 	 	 node[above]  {} 
 	 } 
 	 	   child { 
 	 	 	 node[bag] {$O$} 
 	 	 	 child { 
 	 	 	 	 node[ label=right: 
 	 	 	 	  {$\overline{I}$}] {} 
 	 	 	 	  edge from parent 
 	 	 	 node[above] {}
 	 	 	 	 node[below]  {$3/10$} 
 	 	 	 } 
 	 	 	 child { 
 	 	 	 node[ label=right: 
 	 	 	 	 {$I$}] {} 
 	 	 	 	 edge from parent 
 	 	 	 	 node[above] {$7/10$} 
 	 	 	 	 node[below]  {} 
 	 	 	 } 
 	 	 	 edge from parent 
 	 	 	 node[above] {$3/10$} 
 	 	 	 node[above]  {} 
 	 }; 
 	 \end{tikzpicture} 
 	 \item Calculer $\Prem(O\cap \overline{I})=\Prem(O) \times \Prem_{O}(\overline{I})= 3/10\times 3/10= 9/100$\end{enumerate} 
 
 \end{frame} 
 
 \begin{frame}  
 \begin{enumerate} \setcounter{enumi}{2}  
 	 \item On cherche $\Prem(\overline{I})$. D'après la formule des probabilités totales car $O$ et $\overline{O}$ forment une partition de l'univers, on a \begin{align*} \Prem(\overline{I}) &= \Prem(\overline{I } \cap O) + \Prem(\overline{I } \cap \overline{O}) \\ 
 &= 3/10\times 3/10+ 2/5\times 7/10 \\ 
 &= 37/100
 \end{align*}
  
 \item $\Prem_{\overline{I}}(O)=\dfrac{\Prem(O\cap \overline{I})}{\Prem(\overline{I})} \simeq 243/1000$.
 \end{enumerate}\end{frame}




\end{document}