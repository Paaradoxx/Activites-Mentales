\documentclass[15pt, mathserif]{beamer}

\usepackage[french]{babel}
\usepackage[T1]{fontenc}
\usepackage[utf8]{inputenc}
%\usepackage{esvect}
\usepackage{bm}
\usepackage{eurosym}
\usepackage{tikz}
\usepackage{pgf,tikz,pgfplots}
\pgfplotsset{compat=1.15}
\usepackage{mathrsfs}
\usetikzlibrary{arrows}
\usetikzlibrary{arrows.meta}

\usetikzlibrary{mindmap}
\usepackage{multicol}
\usepackage[tikz]{bclogo}
\usepackage{tkz-tab}
\usepackage{amsmath, tabu}
\usepackage{esvect} %\vv{AB} pour le vecteur AB

\DeclareMathOperator{\e}{e}

%% Tableau

\usepackage{makecell}
\setcellgapes{1pt}
\makegapedcells
\newcolumntype{R}[1]{>{\raggedleft\arraybackslash }b{#1}}
\newcolumntype{L}[1]{>{\raggedright\arraybackslash }b{#1}}
\newcolumntype{C}[1]{>{\centering\arraybackslash }b{#1}}


%pour avoir des parenthèses rondes dans le package fourier
\DeclareSymbolFont{cmoperators}   {OT1}{cmr} {m}{n}
\DeclareSymbolFont{cmlargesymbols}{OMX}{cmex}{m}{n}

\usefonttheme{professionalfonts} %permet d'enlever un bug avec fourier
\usepackage{fourier}
\DeclareMathDelimiter{(}{\mathopen} {cmoperators}{"28}{cmlargesymbols}{"00}
\DeclareMathDelimiter{)}{\mathclose}{cmoperators}{"29}{cmlargesymbols}{"01}

%Graphiques 

\usepackage{pgf,tikz,pgfplots}
\pgfplotsset{compat=1.15}
\usepackage{mathrsfs}
\usetikzlibrary{arrows}
\usetikzlibrary{mindmap}

%ensembles de nbres

\newcommand{\R}{\mathbb{R}}			%permet d'écrire le R "ensemble des réels"'
\newcommand{\N}{\mathbb{N}}			%permet d'écrire le N "ensemble des entiers naturels"
\newcommand{\Z}{\mathbb{Z}}			%permet d'écrire le Z "ensemble des entiers relatifs"
\newcommand{\Prem}{\mathbb{P}}	%permet d'écrire le P "ensemble des nombres premiers" (qui n'a pas marché avec le \P car il existe déjà)
\newcommand{\D}{\mathbb{D}}
\newcommand{\Df}{\mathcal{D}_f}
\newcommand{\Cf}{\mathcal{C}_f}

\newcommand{\Q}{\mathbb{Q}}


\newcommand{\st}[1]{$(#1_n)_{n \in \N}$}

\usetheme{Madrid}
\useoutertheme{miniframes} % Alternatively: miniframes, infolines, split
\useinnertheme{circles}
\definecolor{UBCblue}{rgb}{0.1, 0.25, 0.4} % UBC Blue (primary)
\definecolor{bordeaux}{RGB}{128,0,0}
\usecolortheme[named=UBCblue]{structure}

\usepackage{color} % J'aime bien définir mes couleurs
\definecolor{propcolor}{rgb}{0, 0.5, 1}
\definecolor{thcolor}{rgb}{0.6, 0.07, 0.07}
\colorlet{louis}{blue!70!green!60!white}
\colorlet{sakura}{pink!40!red}

\title{Activités Mentales}
\date{24 Août 2023}

\newcommand{\vco}[2]{\begin{pmatrix} #1 \\ #2 \end{pmatrix}} %Coordonnées de vecteur
\newenvironment{eq}{\begin{cases}\begin{tabu}{ccccc}}{\end{tabu}\end{cases}}
\newenvironment{eql}{\begin{cases}\begin{tabu}{cccccl}}{\end{tabu}\end{cases}}
\newenvironment{eqrl}{\begin{cases}\begin{tabu}{rl}}{\end{tabu}\end{cases}}

\newenvironment{Eq}{\begin{center}\begin{tabular}{rrcl}}{\end{tabular}\end{center}}
\newcommand{\ligneq}[2]{$\Longleftrightarrow$ & $#1$ & $=$ & $#2$ \\}
\newcommand{\Ligneq}[2]{ & $#1$ & $=$ & $#2$ \\}

\newenvironment{RPN}{\begin{center}\begin{tabular}{rrclcrcl}}{\end{tabular}\end{center}}
\newcommand{\Lignerpn}[4]{ & $#1$ & $=$ & $#2$ & ou & $#3$ & $=$ & $#4$ \\}
\newcommand{\lignerpn}[4]{$\Longleftrightarrow$ & $#1$ & $=$ & $#2$ & ou & $#3$ & $=$ & $#4$ \\}

\newenvironment{TRPN}{\begin{center}\begin{tabular}{rrclcrclcrcl}}{\end{tabular}\end{center}}
\newcommand{\Lignetrpn}[6]{ & $#1$ & $=$ & $#2$ & ou & $#3$ & $=$ & $#4$ & ou & $#5$ & $=$ & $#6$ \\}
\newcommand{\lignetrpn}[6]{$\Longleftrightarrow$ & $#1$ & $=$ & $#2$ & ou & $#3$ & $=$ & $#4$ & ou & $#5$ & $=$ & $#6$ \\}
\begin{document}

\begin{frame}
    \titlepage
\end{frame}

\begin{frame} 
	\frametitle{Question 1}
 Un vendeur reçoit chaque année une prime de  2000 \euro{} qu'il place systématiquement, toujours à un taux annuel de 5 \% . 
 \begin{enumerate} 
 	 \item À combien s'élèvera le capital au bout de 1 an ? 2ans ?  
 	 \item On considère la suite \st{u} qui représente le capital au bout de $n$ années. Exprimer $u_{n+1}$ en fonction de $u_n$. 
 	 \item Quelle est la nature de la suite \st{u} ? 
 	 \item  À combien s'élèvera le capital au bout de 10 ans ? 
 \end{enumerate} \end{frame}


\begin{frame} 
	\frametitle{Question 2}
 Un vendeur reçoit chaque année une prime de  1700 \euro{} qu'il place systématiquement, toujours à un taux annuel de 5 \% . 
 \begin{enumerate} 
 	 \item À combien s'élèvera le capital au bout de 1 an ? 2ans ?  
 	 \item On considère la suite \st{u} qui représente le capital au bout de $n$ années. Exprimer $u_{n+1}$ en fonction de $u_n$. 
 	 \item Quelle est la nature de la suite \st{u} ? 
 	 \item  À combien s'élèvera le capital au bout de 10 ans ? 
 \end{enumerate} \end{frame}


\begin{frame} 
	\frametitle{Question 3}
 Une société du secteur des nouvelles technologies prévoit une augmentation de son chiffre d'affaire de 16 \% . La première année, leur chiffre d'affaire était de  250000 habitants. 
 \begin{enumerate} 
 	 \item Quelle sera le chiffre d'affaire la première année ? La deuxième année ? 
 	 \item On considère la suite \st{u} qui représente le chiffre d'affaire de l'entreprise au bout de $n$ années. Exprimer $u_{n+1}$ en fonction de $u_n$. 
 	 \item Quelle est la nature de la suite \st{u} ? 
 	 \item Quel sera le chiffre d'affaire au bout de 10 ans ? 
 \end{enumerate} \end{frame}


\begin{frame} 
	\frametitle{Question 4}
 Un vendeur reçoit chaque année une prime de  2900 \euro{} qu'il place systématiquement, toujours à un taux annuel de 6 \% . 
 \begin{enumerate} 
 	 \item À combien s'élèvera le capital au bout de 1 an ? 2ans ?  
 	 \item On considère la suite \st{u} qui représente le capital au bout de $n$ années. Exprimer $u_{n+1}$ en fonction de $u_n$. 
 	 \item Quelle est la nature de la suite \st{u} ? 
 	 \item  À combien s'élèvera le capital au bout de 10 ans ? 
 \end{enumerate} \end{frame}


\begin{frame} 
	\frametitle{Question 5}
 La population d'un village de montagne diminue tous les ans de 18 \% . Sachant qu'en 2022 elle était de 2300 habitants. 
 \begin{enumerate} 
 	 \item Quelle sera la population du village en 2023 ? En 2024 ? 
 	 \item On considère la suite \st{u} qui représente la population du village au bout de $n$ années. Exprimer $u_{n+1}$ en fonction de $u_n$. 
 	 \item Quelle est la nature de la suite \st{u} ? 
 	 \item Combien de personne habiteront le village en 2032 ?. 
 \end{enumerate} \end{frame}


\begin{frame}
\vspace{-10mm}
	\frametitle{Correction 1}
\begin{enumerate} 
 	 \item Augmenter de 5 \% revient à multiplier par  1.05. En 2023, le capital sera de $2000 \times  1.05\simeq 2100.0$ et en 2024 le capital sera donc de  $2100.0 \times  1.05\simeq 2205.0$.  
 	 \item On a pour tout $n \in \N$, 
 
 \hfil$\left\{\begin{array}{rcl} 
 u_{n+1} & = & u_n \times  1.05\\ u_0 & = &  2000\end{array} \right.$ 
 	 \item La suite \st{u} est une suite géométrique car on multiplie à chaque fois par  1.05. 
 	 \item D'après la calculatrice, on a $u_{10}=3257.79$. 
 
 En 2032, le capital sera de 3257.79 \euro. 
 \end{enumerate} 
 
 \end{frame}


\begin{frame}
\vspace{-10mm}
	\frametitle{Correction 2}
\begin{enumerate} 
 	 \item Augmenter de 5 \% revient à multiplier par  1.05. En 2023, le capital sera de $1700 \times  1.05\simeq 1785.0$ et en 2024 le capital sera donc de  $1785.0 \times  1.05\simeq 1874.25$.  
 	 \item On a pour tout $n \in \N$, 
 
 \hfil$\left\{\begin{array}{rcl} 
 u_{n+1} & = & u_n \times  1.05\\ u_0 & = &  1700\end{array} \right.$ 
 	 \item La suite \st{u} est une suite géométrique car on multiplie à chaque fois par  1.05. 
 	 \item D'après la calculatrice, on a $u_{10}=2769.12$. 
 
 En 2032, le capital sera de 2769.12 \euro. 
 \end{enumerate} 
 
 \end{frame}


\begin{frame}
\vspace{-10mm}
	\frametitle{Correction 3}
\begin{enumerate} 
 	 \item Augmenter de 16 \% revient à multiplier par  1.16. En 2023, le chiffre d'affaires sera de $250000 \times  1.16\simeq 290000.0$ et en 2024 le chiffre d'affaires sera donc de  $290000.0 \times  1.16\simeq 336400.0$.  
 	 \item On a pour tout $n \in \N$, 
 
 \hfil$\left\{\begin{array}{rcl} 
 u_{n+1} & = & u_n \times  1.16\\ u_0 & = &  250000\end{array} \right.$ 
 	 \item La suite \st{u} est une suite géométrique car on multiplie à chaque fois par  1.16. 
 	 \item D'après la calculatrice, on a $u_{10}=1102858.77$. 
 
 En 2032, le chiffre d'affaires sera de 1102858.77 \euro. 
 \end{enumerate} 
 
 \end{frame}


\begin{frame}
\vspace{-10mm}
	\frametitle{Correction 4}
\begin{enumerate} 
 	 \item Augmenter de 6 \% revient à multiplier par  1.06. En 2023, le capital sera de $2900 \times  1.06\simeq 3074.0$ et en 2024 le capital sera donc de  $3074.0 \times  1.06\simeq 3258.44$.  
 	 \item On a pour tout $n \in \N$, 
 
 \hfil$\left\{\begin{array}{rcl} 
 u_{n+1} & = & u_n \times  1.06\\ u_0 & = &  2900\end{array} \right.$ 
 	 \item La suite \st{u} est une suite géométrique car on multiplie à chaque fois par  1.06. 
 	 \item D'après la calculatrice, on a $u_{10}=5193.46$. 
 
 En 2032, le capital sera de 5193.46 \euro. 
 \end{enumerate} 
 
 \end{frame}


\begin{frame}
\vspace{-10mm}
	\frametitle{Correction 5}
\begin{enumerate} 
 	 \item Diminuer de 18 \% revient à multiplier par  0.8200000000000001. En 2023, le nombre d'habitants sera de $2300 \times  0.8200000000000001\simeq 1886.0$ et en 2024 le nombre d'habitants sera donc de  $1886.0 \times  0.8200000000000001\simeq 1546.52$.  
 	 \item On a pour tout $n \in \N$, 
 
 \hfil$\left\{\begin{array}{rcl} 
 u_{n+1} & = & u_n \times  0.8200000000000001\\ u_0 & = &  2300\end{array} \right.$ 
 	 \item La suite \st{u} est une suite géométrique car on multiplie à chaque fois par  0.8200000000000001. 
 	 \item D'après la calculatrice, on a $u_{10}=316.13$. 
 
 En 2032, il y aura 316.13 habitants. 
 \end{enumerate} 
 
 \end{frame}




\end{document}