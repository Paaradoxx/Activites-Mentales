\documentclass[15pt, mathserif]{beamer}

\usepackage[french]{babel}
\usepackage[T1]{fontenc}
\usepackage[utf8]{inputenc}
%\usepackage{esvect}
\usepackage{bm}
\usepackage{eurosym}
\usepackage{tikz}
\usepackage{pgf,tikz,pgfplots}
\pgfplotsset{compat=1.15}
\usepackage{mathrsfs}
\usetikzlibrary{arrows}
\usetikzlibrary{arrows.meta}

\usetikzlibrary{mindmap}
\usepackage{multicol}
\usepackage[tikz]{bclogo}
\usepackage{tkz-tab}
\usepackage{amsmath, tabu}
\usepackage{esvect} %\vv{AB} pour le vecteur AB

\DeclareMathOperator{\e}{e}

%% Tableau

\usepackage{makecell}
\setcellgapes{1pt}
\makegapedcells
\newcolumntype{R}[1]{>{\raggedleft\arraybackslash }b{#1}}
\newcolumntype{L}[1]{>{\raggedright\arraybackslash }b{#1}}
\newcolumntype{C}[1]{>{\centering\arraybackslash }b{#1}}


%pour avoir des parenthèses rondes dans le package fourier
\DeclareSymbolFont{cmoperators}   {OT1}{cmr} {m}{n}
\DeclareSymbolFont{cmlargesymbols}{OMX}{cmex}{m}{n}

\usefonttheme{professionalfonts} %permet d'enlever un bug avec fourier
\usepackage{fourier}
\DeclareMathDelimiter{(}{\mathopen} {cmoperators}{"28}{cmlargesymbols}{"00}
\DeclareMathDelimiter{)}{\mathclose}{cmoperators}{"29}{cmlargesymbols}{"01}

%Graphiques 

\usepackage{pgf,tikz,pgfplots}
\pgfplotsset{compat=1.15}
\usepackage{mathrsfs}
\usetikzlibrary{arrows}
\usetikzlibrary{mindmap}

%ensembles de nbres

\newcommand{\R}{\mathbb{R}}			%permet d'écrire le R "ensemble des réels"'
\newcommand{\N}{\mathbb{N}}			%permet d'écrire le N "ensemble des entiers naturels"
\newcommand{\Z}{\mathbb{Z}}			%permet d'écrire le Z "ensemble des entiers relatifs"
\newcommand{\Prem}{\mathbb{P}}	%permet d'écrire le P "ensemble des nombres premiers" (qui n'a pas marché avec le \P car il existe déjà)
\newcommand{\D}{\mathbb{D}}
\newcommand{\Df}{\mathcal{D}_f}
\newcommand{\Cf}{\mathcal{C}_f}

\newcommand{\Q}{\mathbb{Q}}


\newcommand{\st}[1]{$(#1_n)_{n \in \N}$}

\usetheme{Madrid}
\useoutertheme{miniframes} % Alternatively: miniframes, infolines, split
\useinnertheme{circles}
\definecolor{UBCblue}{rgb}{0.1, 0.25, 0.4} % UBC Blue (primary)
\definecolor{bordeaux}{RGB}{128,0,0}
\usecolortheme[named=UBCblue]{structure}

\usepackage{color} % J'aime bien définir mes couleurs
\definecolor{propcolor}{rgb}{0, 0.5, 1}
\definecolor{thcolor}{rgb}{0.6, 0.07, 0.07}
\colorlet{louis}{blue!70!green!60!white}
\colorlet{sakura}{pink!40!red}

\title{Activités Mentales}
\date{24 Août 2023}

\newcommand{\vco}[2]{\begin{pmatrix} #1 \\ #2 \end{pmatrix}} %Coordonnées de vecteur
\newenvironment{eq}{\begin{cases}\begin{tabu}{ccccc}}{\end{tabu}\end{cases}}
\newenvironment{eql}{\begin{cases}\begin{tabu}{cccccl}}{\end{tabu}\end{cases}}
\newenvironment{eqrl}{\begin{cases}\begin{tabu}{rl}}{\end{tabu}\end{cases}}

\newenvironment{Eq}{\begin{center}\begin{tabular}{rrcl}}{\end{tabular}\end{center}}
\newcommand{\ligneq}[2]{$\Longleftrightarrow$ & $#1$ & $=$ & $#2$ \\}
\newcommand{\Ligneq}[2]{ & $#1$ & $=$ & $#2$ \\}

\newenvironment{RPN}{\begin{center}\begin{tabular}{rrclcrcl}}{\end{tabular}\end{center}}
\newcommand{\Lignerpn}[4]{ & $#1$ & $=$ & $#2$ & ou & $#3$ & $=$ & $#4$ \\}
\newcommand{\lignerpn}[4]{$\Longleftrightarrow$ & $#1$ & $=$ & $#2$ & ou & $#3$ & $=$ & $#4$ \\}

\newenvironment{TRPN}{\begin{center}\begin{tabular}{rrclcrclcrcl}}{\end{tabular}\end{center}}
\newcommand{\Lignetrpn}[6]{ & $#1$ & $=$ & $#2$ & ou & $#3$ & $=$ & $#4$ & ou & $#5$ & $=$ & $#6$ \\}
\newcommand{\lignetrpn}[6]{$\Longleftrightarrow$ & $#1$ & $=$ & $#2$ & ou & $#3$ & $=$ & $#4$ & ou & $#5$ & $=$ & $#6$ \\}
\begin{document}

\begin{frame}
    \titlepage
\end{frame}

\begin{frame} 
	\frametitle{Question 1}
On donne le point $S(1;-6)$ et le vecteur $\vec{u}\begin{pmatrix} -10\\ 6 \end{pmatrix}$ 
 
 Déterminer les coordonnées du points $E$ tels que $\vec{SE}=\vec{u}$.\end{frame}


\begin{frame} 
	\frametitle{Question 2}
On donne le point $O(-10;-8)$ et le vecteur $\vec{u}\begin{pmatrix} 4\\ 10 \end{pmatrix}$ 
 
 Déterminer les coordonnées du points $I$ tels que $\vec{OI}=\vec{u}$.\end{frame}


\begin{frame} 
	\frametitle{Question 3}
On donne le point $N(2;-10)$ et le vecteur $\vec{u}\begin{pmatrix} 7\\ 11 \end{pmatrix}$ 
 
 Déterminer les coordonnées du points $E$ tels que $\vec{NE}=\vec{u}$.\end{frame}


\begin{frame} 
	\frametitle{Question 4}
On donne le point $U(-3;1)$ et le vecteur $\vec{u}\begin{pmatrix} 1\\ -5 \end{pmatrix}$ 
 
 Déterminer les coordonnées du points $H$ tels que $\vec{UH}=\vec{u}$.\end{frame}


\begin{frame} 
	\frametitle{Question 5}
On donne le point $K(7;-8)$ et le vecteur $\vec{u}\begin{pmatrix} -9\\ 9 \end{pmatrix}$ 
 
 Déterminer les coordonnées du points $P$ tels que $\vec{KP}=\vec{u}$.\end{frame}


\begin{frame}
\vspace{-10mm}
	\frametitle{Correction 1}
\vspace*{0.5cm} 
 On donne le point $S(1;-6)$ et le vecteur $\vec{u}\begin{pmatrix} -10\\ 6 \end{pmatrix}$ 
 
 On cherche les coordonnées de $E$ et on sait que $\vec{SE}=\vec{u}$ donc $$\begin{pmatrix} x_{E}-x_S \\ y_{E}-y_S \end{pmatrix}=\begin{pmatrix} x_{E}-1 \\ y_{E}-\left(-6\right)\end{pmatrix}=\begin{pmatrix} -10 \\ 6 \end{pmatrix}$$ 
 
 On a donc 2 équations à résoudre :  
 $$ \left\{ \begin{array}{cc} 
  x_{E}-1=-10 \\ 
  y_{E}+6=6 
 \end{array} \right. \Leftrightarrow \left\{ \begin{array}{cc} 
  x_{E}=-9 \\ 
 y_{E}=0 
 \end{array} \right. $$ 
 
 Le point $E$ a pour coordonnées $(-9;0)$.\end{frame}


\begin{frame}
\vspace{-10mm}
	\frametitle{Correction 2}
\vspace*{0.5cm} 
 On donne le point $O(-10;-8)$ et le vecteur $\vec{u}\begin{pmatrix} 4\\ 10 \end{pmatrix}$ 
 
 On cherche les coordonnées de $I$ et on sait que $\vec{OI}=\vec{u}$ donc $$\begin{pmatrix} x_{I}-x_O \\ y_{I}-y_O \end{pmatrix}=\begin{pmatrix} x_{I}-\left(-10\right) \\ y_{I}-\left(-8\right)\end{pmatrix}=\begin{pmatrix} 4 \\ 10 \end{pmatrix}$$ 
 
 On a donc 2 équations à résoudre :  
 $$ \left\{ \begin{array}{cc} 
  x_{I}+10=4 \\ 
  y_{I}+8=10 
 \end{array} \right. \Leftrightarrow \left\{ \begin{array}{cc} 
  x_{I}=-6 \\ 
 y_{I}=2 
 \end{array} \right. $$ 
 
 Le point $I$ a pour coordonnées $(-6;2)$.\end{frame}


\begin{frame}
\vspace{-10mm}
	\frametitle{Correction 3}
\vspace*{0.5cm} 
 On donne le point $N(2;-10)$ et le vecteur $\vec{u}\begin{pmatrix} 7\\ 11 \end{pmatrix}$ 
 
 On cherche les coordonnées de $E$ et on sait que $\vec{NE}=\vec{u}$ donc $$\begin{pmatrix} x_{E}-x_N \\ y_{E}-y_N \end{pmatrix}=\begin{pmatrix} x_{E}-2 \\ y_{E}-\left(-10\right)\end{pmatrix}=\begin{pmatrix} 7 \\ 11 \end{pmatrix}$$ 
 
 On a donc 2 équations à résoudre :  
 $$ \left\{ \begin{array}{cc} 
  x_{E}-2=7 \\ 
  y_{E}+10=11 
 \end{array} \right. \Leftrightarrow \left\{ \begin{array}{cc} 
  x_{E}=9 \\ 
 y_{E}=1 
 \end{array} \right. $$ 
 
 Le point $E$ a pour coordonnées $(9;1)$.\end{frame}


\begin{frame}
\vspace{-10mm}
	\frametitle{Correction 4}
\vspace*{0.5cm} 
 On donne le point $U(-3;1)$ et le vecteur $\vec{u}\begin{pmatrix} 1\\ -5 \end{pmatrix}$ 
 
 On cherche les coordonnées de $H$ et on sait que $\vec{UH}=\vec{u}$ donc $$\begin{pmatrix} x_{H}-x_U \\ y_{H}-y_U \end{pmatrix}=\begin{pmatrix} x_{H}-\left(-3\right) \\ y_{H}-1\end{pmatrix}=\begin{pmatrix} 1 \\ -5 \end{pmatrix}$$ 
 
 On a donc 2 équations à résoudre :  
 $$ \left\{ \begin{array}{cc} 
  x_{H}+3=1 \\ 
  y_{H}-1=-5 
 \end{array} \right. \Leftrightarrow \left\{ \begin{array}{cc} 
  x_{H}=-2 \\ 
 y_{H}=-4 
 \end{array} \right. $$ 
 
 Le point $H$ a pour coordonnées $(-2;-4)$.\end{frame}


\begin{frame}
\vspace{-10mm}
	\frametitle{Correction 5}
\vspace*{0.5cm} 
 On donne le point $K(7;-8)$ et le vecteur $\vec{u}\begin{pmatrix} -9\\ 9 \end{pmatrix}$ 
 
 On cherche les coordonnées de $P$ et on sait que $\vec{KP}=\vec{u}$ donc $$\begin{pmatrix} x_{P}-x_K \\ y_{P}-y_K \end{pmatrix}=\begin{pmatrix} x_{P}-7 \\ y_{P}-\left(-8\right)\end{pmatrix}=\begin{pmatrix} -9 \\ 9 \end{pmatrix}$$ 
 
 On a donc 2 équations à résoudre :  
 $$ \left\{ \begin{array}{cc} 
  x_{P}-7=-9 \\ 
  y_{P}+8=9 
 \end{array} \right. \Leftrightarrow \left\{ \begin{array}{cc} 
  x_{P}=-2 \\ 
 y_{P}=1 
 \end{array} \right. $$ 
 
 Le point $P$ a pour coordonnées $(-2;1)$.\end{frame}




\end{document}