\documentclass[15pt, mathserif]{beamer}

\usepackage[french]{babel}
\usepackage[T1]{fontenc}
\usepackage[utf8]{inputenc}
%\usepackage{esvect}
\usepackage{bm}
\usepackage{eurosym}
\usepackage{tikz}
\usepackage{pgf,tikz,pgfplots}
\pgfplotsset{compat=1.15}
\usepackage{mathrsfs}
\usetikzlibrary{arrows}
\usetikzlibrary{arrows.meta}

\usetikzlibrary{mindmap}
\usepackage{multicol}
\usepackage[tikz]{bclogo}
\usepackage{tkz-tab}
\usepackage{amsmath, tabu}
\usepackage{esvect} %\vv{AB} pour le vecteur AB

\DeclareMathOperator{\e}{e}

%% Tableau

\usepackage{makecell}
\setcellgapes{1pt}
\makegapedcells
\newcolumntype{R}[1]{>{\raggedleft\arraybackslash }b{#1}}
\newcolumntype{L}[1]{>{\raggedright\arraybackslash }b{#1}}
\newcolumntype{C}[1]{>{\centering\arraybackslash }b{#1}}


%pour avoir des parenthèses rondes dans le package fourier
\DeclareSymbolFont{cmoperators}   {OT1}{cmr} {m}{n}
\DeclareSymbolFont{cmlargesymbols}{OMX}{cmex}{m}{n}

\usefonttheme{professionalfonts} %permet d'enlever un bug avec fourier
\usepackage{fourier}
\DeclareMathDelimiter{(}{\mathopen} {cmoperators}{"28}{cmlargesymbols}{"00}
\DeclareMathDelimiter{)}{\mathclose}{cmoperators}{"29}{cmlargesymbols}{"01}

%Graphiques 

\usepackage{pgf,tikz,pgfplots}
\pgfplotsset{compat=1.15}
\usepackage{mathrsfs}
\usetikzlibrary{arrows}
\usetikzlibrary{mindmap}

%ensembles de nbres

\newcommand{\R}{\mathbb{R}}			%permet d'écrire le R "ensemble des réels"'
\newcommand{\N}{\mathbb{N}}			%permet d'écrire le N "ensemble des entiers naturels"
\newcommand{\Z}{\mathbb{Z}}			%permet d'écrire le Z "ensemble des entiers relatifs"
\newcommand{\Prem}{\mathbb{P}}	%permet d'écrire le P "ensemble des nombres premiers" (qui n'a pas marché avec le \P car il existe déjà)
\newcommand{\D}{\mathbb{D}}
\newcommand{\Df}{\mathcal{D}_f}
\newcommand{\Cf}{\mathcal{C}_f}

\newcommand{\Q}{\mathbb{Q}}


\newcommand{\st}[1]{$(#1_n)_{n \in \N}$}

\usetheme{Madrid}
\useoutertheme{miniframes} % Alternatively: miniframes, infolines, split
\useinnertheme{circles}
\definecolor{UBCblue}{rgb}{0.1, 0.25, 0.4} % UBC Blue (primary)
\definecolor{bordeaux}{RGB}{128,0,0}
\usecolortheme[named=UBCblue]{structure}

\usepackage{color} % J'aime bien définir mes couleurs
\definecolor{propcolor}{rgb}{0, 0.5, 1}
\definecolor{thcolor}{rgb}{0.6, 0.07, 0.07}
\colorlet{louis}{blue!70!green!60!white}
\colorlet{sakura}{pink!40!red}

\title{Activités Mentales}
\date{24 Août 2023}

\newcommand{\vco}[2]{\begin{pmatrix} #1 \\ #2 \end{pmatrix}} %Coordonnées de vecteur
\newenvironment{eq}{\begin{cases}\begin{tabu}{ccccc}}{\end{tabu}\end{cases}}
\newenvironment{eql}{\begin{cases}\begin{tabu}{cccccl}}{\end{tabu}\end{cases}}
\newenvironment{eqrl}{\begin{cases}\begin{tabu}{rl}}{\end{tabu}\end{cases}}

\newenvironment{Eq}{\begin{center}\begin{tabular}{rrcl}}{\end{tabular}\end{center}}
\newcommand{\ligneq}[2]{$\Longleftrightarrow$ & $#1$ & $=$ & $#2$ \\}
\newcommand{\Ligneq}[2]{ & $#1$ & $=$ & $#2$ \\}

\newenvironment{RPN}{\begin{center}\begin{tabular}{rrclcrcl}}{\end{tabular}\end{center}}
\newcommand{\Lignerpn}[4]{ & $#1$ & $=$ & $#2$ & ou & $#3$ & $=$ & $#4$ \\}
\newcommand{\lignerpn}[4]{$\Longleftrightarrow$ & $#1$ & $=$ & $#2$ & ou & $#3$ & $=$ & $#4$ \\}

\newenvironment{TRPN}{\begin{center}\begin{tabular}{rrclcrclcrcl}}{\end{tabular}\end{center}}
\newcommand{\Lignetrpn}[6]{ & $#1$ & $=$ & $#2$ & ou & $#3$ & $=$ & $#4$ & ou & $#5$ & $=$ & $#6$ \\}
\newcommand{\lignetrpn}[6]{$\Longleftrightarrow$ & $#1$ & $=$ & $#2$ & ou & $#3$ & $=$ & $#4$ & ou & $#5$ & $=$ & $#6$ \\}
\begin{document}

\begin{frame}
    \titlepage
\end{frame}

\begin{frame} 
	\frametitle{Question 1}
Donner sous forme irréductible le résultat du calcul suivant \[\dfrac{150}{50} \times \dfrac{75}{100}.\]\end{frame}


\begin{frame} 
	\frametitle{Question 2}
Donner sous forme irréductible le résultat du calcul suivant \[\dfrac{80}{24} \times \dfrac{-6}{-32}.\]\end{frame}


\begin{frame} 
	\frametitle{Question 3}
Donner sous forme irréductible le résultat du calcul suivant \[\dfrac{18}{-45} \times \dfrac{-30}{-45}.\]\end{frame}


\begin{frame} 
	\frametitle{Question 4}
Donner sous forme irréductible le résultat du calcul suivant \[\dfrac{-40}{-8} \times \dfrac{0}{-20}.\]\end{frame}


\begin{frame} 
	\frametitle{Question 5}
Donner sous forme irréductible le résultat du calcul suivant \[\dfrac{90}{-100} \times \dfrac{50}{-120}.\]\end{frame}


\begin{frame}
\vspace{-10mm}
	\frametitle{Correction 1}
	\begin{align*}
	\dfrac{150}{50} \times \dfrac{75}{100}&= \dfrac{150 \times 75}{50 \times 100}\\
		&=\dfrac{10 \times 15 \times 5 \times 15}{5 \times 10 \times 10 \times 10}\\		&=\dfrac{10 \times 5 \times 3 \times 5 \times 5 \times 3}{5 \times 5 \times 2 \times 10 \times 5 \times 2}\\		&=\dfrac{3 \times 3}{2 \times 2}\\		&=\dfrac{9}{4}
	\end{align*}\end{frame}


\begin{frame}
\vspace{-10mm}
	\frametitle{Correction 2}
	\begin{align*}
	\dfrac{80}{24} \times \dfrac{-6}{-32}&= \dfrac{80 \times \left(-6\right)}{24 \times \left(-32\right)}\\
		&=\dfrac{4 \times 20 \times 2 \times \left(-3\right)}{2 \times 12 \times 4 \times \left(-8\right)}\\		&=\dfrac{4 \times 4 \times 5 \times 2 \times 3 \times \left(-1\right)}{2 \times 3 \times 4 \times 4 \times 4 \times \left(-2\right)}\\		&=\dfrac{5 \times \left(-1\right)}{4 \times \left(-2\right)}\\		&=\dfrac{5}{8}
	\end{align*}\end{frame}


\begin{frame}
\vspace{-10mm}
	\frametitle{Correction 3}
	\begin{align*}
	\dfrac{18}{-45} \times \dfrac{-30}{-45}&= \dfrac{18 \times \left(-30\right)}{\left(-45\right) \times \left(-45\right)}\\
		&=\dfrac{3 \times 6 \times 3 \times \left(-10\right)}{3 \times \left(-15\right) \times 3 \times \left(-15\right)}\\		&=\dfrac{3 \times 3 \times 2 \times 3 \times 5 \times \left(-2\right)}{3 \times 5 \times \left(-3\right) \times 3 \times 3 \times \left(-5\right)}\\		&=\dfrac{2 \times \left(-2\right)}{\left(-3\right) \times \left(-5\right)}\\		&=\dfrac{-4}{15}
	\end{align*}\end{frame}


\begin{frame}
\vspace{-10mm}
	\frametitle{Correction 4}
	\begin{align*}
	\dfrac{-40}{-8} \times \dfrac{0}{-20}&= \dfrac{\left(-40\right) \times 0}{\left(-8\right) \times \left(-20\right)}\\
		&=\dfrac{10 \times \left(-4\right) \times 2 \times 0}{2 \times \left(-4\right) \times 10 \times \left(-2\right)}\\		&=\dfrac{10 \times 2 \times \left(-2\right) \times 2 \times 4 \times 0}{2 \times 4 \times \left(-1\right) \times 10 \times 2 \times \left(-1\right)}\\		&=\dfrac{\left(-2\right) \times 0}{\left(-1\right) \times \left(-1\right)}\\		&=0
	\end{align*}\end{frame}


\begin{frame}
\vspace{-10mm}
	\frametitle{Correction 5}
	\begin{align*}
	\dfrac{90}{-100} \times \dfrac{50}{-120}&= \dfrac{90 \times 50}{\left(-100\right) \times \left(-120\right)}\\
		&=\dfrac{10 \times 9 \times 10 \times 5}{10 \times \left(-10\right) \times 10 \times \left(-12\right)}\\		&=\dfrac{10 \times 3 \times 3 \times 10 \times 5 \times 1}{10 \times 5 \times \left(-2\right) \times 10 \times 3 \times \left(-4\right)}\\		&=\dfrac{3 \times 1}{\left(-2\right) \times \left(-4\right)}\\		&=\dfrac{3}{8}
	\end{align*}\end{frame}




\end{document}