\documentclass[15pt, mathserif]{beamer}

\usepackage[french]{babel}
\usepackage[T1]{fontenc}
\usepackage[utf8]{inputenc}
%\usepackage{esvect}
\usepackage{bm}
\usepackage{eurosym}
\usepackage{tikz}
\usepackage{pgf,tikz,pgfplots}
\pgfplotsset{compat=1.15}
\usepackage{mathrsfs}
\usetikzlibrary{arrows}
\usetikzlibrary{arrows.meta}

\usetikzlibrary{mindmap}
\usepackage{multicol}
\usepackage[tikz]{bclogo}
\usepackage{tkz-tab}
\usepackage{amsmath, tabu}
\usepackage{esvect} %\vv{AB} pour le vecteur AB

\DeclareMathOperator{\e}{e}

%% Tableau

\usepackage{makecell}
\setcellgapes{1pt}
\makegapedcells
\newcolumntype{R}[1]{>{\raggedleft\arraybackslash }b{#1}}
\newcolumntype{L}[1]{>{\raggedright\arraybackslash }b{#1}}
\newcolumntype{C}[1]{>{\centering\arraybackslash }b{#1}}


%pour avoir des parenthèses rondes dans le package fourier
\DeclareSymbolFont{cmoperators}   {OT1}{cmr} {m}{n}
\DeclareSymbolFont{cmlargesymbols}{OMX}{cmex}{m}{n}

\usefonttheme{professionalfonts} %permet d'enlever un bug avec fourier
\usepackage{fourier}
\DeclareMathDelimiter{(}{\mathopen} {cmoperators}{"28}{cmlargesymbols}{"00}
\DeclareMathDelimiter{)}{\mathclose}{cmoperators}{"29}{cmlargesymbols}{"01}

%Graphiques 

\usepackage{pgf,tikz,pgfplots}
\pgfplotsset{compat=1.15}
\usepackage{mathrsfs}
\usetikzlibrary{arrows}
\usetikzlibrary{mindmap}

%ensembles de nbres

\newcommand{\R}{\mathbb{R}}			%permet d'écrire le R "ensemble des réels"'
\newcommand{\N}{\mathbb{N}}			%permet d'écrire le N "ensemble des entiers naturels"
\newcommand{\Z}{\mathbb{Z}}			%permet d'écrire le Z "ensemble des entiers relatifs"
\newcommand{\Prem}{\mathbb{P}}	%permet d'écrire le P "ensemble des nombres premiers" (qui n'a pas marché avec le \P car il existe déjà)
\newcommand{\D}{\mathbb{D}}
\newcommand{\Df}{\mathcal{D}_f}
\newcommand{\Cf}{\mathcal{C}_f}

\newcommand{\Q}{\mathbb{Q}}


\newcommand{\st}[1]{$(#1_n)_{n \in \N}$}

\usetheme{Madrid}
\useoutertheme{miniframes} % Alternatively: miniframes, infolines, split
\useinnertheme{circles}
\definecolor{UBCblue}{rgb}{0.1, 0.25, 0.4} % UBC Blue (primary)
\definecolor{bordeaux}{RGB}{128,0,0}
\usecolortheme[named=UBCblue]{structure}

\usepackage{color} % J'aime bien définir mes couleurs
\definecolor{propcolor}{rgb}{0, 0.5, 1}
\definecolor{thcolor}{rgb}{0.6, 0.07, 0.07}
\colorlet{louis}{blue!70!green!60!white}
\colorlet{sakura}{pink!40!red}

\title{Activités Mentales}
\date{24 Août 2023}

\newcommand{\vco}[2]{\begin{pmatrix} #1 \\ #2 \end{pmatrix}} %Coordonnées de vecteur
\newenvironment{eq}{\begin{cases}\begin{tabu}{ccccc}}{\end{tabu}\end{cases}}
\newenvironment{eql}{\begin{cases}\begin{tabu}{cccccl}}{\end{tabu}\end{cases}}
\newenvironment{eqrl}{\begin{cases}\begin{tabu}{rl}}{\end{tabu}\end{cases}}

\newenvironment{Eq}{\begin{center}\begin{tabular}{rrcl}}{\end{tabular}\end{center}}
\newcommand{\ligneq}[2]{$\Longleftrightarrow$ & $#1$ & $=$ & $#2$ \\}
\newcommand{\Ligneq}[2]{ & $#1$ & $=$ & $#2$ \\}

\newenvironment{RPN}{\begin{center}\begin{tabular}{rrclcrcl}}{\end{tabular}\end{center}}
\newcommand{\Lignerpn}[4]{ & $#1$ & $=$ & $#2$ & ou & $#3$ & $=$ & $#4$ \\}
\newcommand{\lignerpn}[4]{$\Longleftrightarrow$ & $#1$ & $=$ & $#2$ & ou & $#3$ & $=$ & $#4$ \\}

\newenvironment{TRPN}{\begin{center}\begin{tabular}{rrclcrclcrcl}}{\end{tabular}\end{center}}
\newcommand{\Lignetrpn}[6]{ & $#1$ & $=$ & $#2$ & ou & $#3$ & $=$ & $#4$ & ou & $#5$ & $=$ & $#6$ \\}
\newcommand{\lignetrpn}[6]{$\Longleftrightarrow$ & $#1$ & $=$ & $#2$ & ou & $#3$ & $=$ & $#4$ & ou & $#5$ & $=$ & $#6$ \\}
\begin{document}

\begin{frame}
    \titlepage
\end{frame}

\begin{frame} 
	\frametitle{Question 1}
Soit \st{u} une suite arithmétique de premier terme $u_0=30$ et de raison $r=12$. 
 
 \begin{enumerate} 
 	 \item Donner les trois premiers termes de la suite. 
 	 \item Exprimer $u_{n+1}$ en fonction de $u_n$. 
 	 \item Conjecturer le sens de variation de la suite \st{u}. 
 	 \item Démontrer le sens de variation. 
 	 \item On donne maintenant $u_n=30+12n$ pour tout $n \in \N$. Calculer $u_{10}$. 
 
 \end{enumerate} \end{frame}


\begin{frame} 
	\frametitle{Question 2}
Soit \st{u} une suite définie par $u_n=16-12n$ pour tout $n \in \N$. 
 
 \begin{enumerate} 
 	 \item Donner les trois premiers termes de la suite. 
 	 \item Exprimer $u_{n+1}$ en fonction de $n$. 
 	 \item Quelle est la nature de la suite ? On démontrera le résultat 
 	 \item Après avoir conjecturer le sens de variation de la suite, le démontrer. 
 
 \end{enumerate} \end{frame}


\begin{frame} 
	\frametitle{Question 3}
Soit \st{u} une suite définie par $u_n=10-5n$ pour tout $n \in \N$. 
 
 \begin{enumerate} 
 	 \item Donner les trois premiers termes de la suite. 
 	 \item Exprimer $u_{n+1}$ en fonction de $n$. 
 	 \item Quelle est la nature de la suite ? On démontrera le résultat 
 	 \item Après avoir conjecturer le sens de variation de la suite, le démontrer. 
 
 \end{enumerate} \end{frame}


\begin{frame} 
	\frametitle{Question 4}
Soit \st{u} une suite arithmétique de premier terme $u_0=21$ et de raison $r=-15$. 
 
 \begin{enumerate} 
 	 \item Donner les trois premiers termes de la suite. 
 	 \item Exprimer $u_{n+1}$ en fonction de $u_n$. 
 	 \item Conjecturer le sens de variation de la suite \st{u}. 
 	 \item Démontrer le sens de variation. 
 	 \item On donne maintenant $u_n=21-15n$ pour tout $n \in \N$. Calculer $u_{10}$. 
 
 \end{enumerate} \end{frame}


\begin{frame} 
	\frametitle{Question 5}
Soit \st{u} une suite arithmétique telle que $u_0=8$ et $u_1=-6$. 
 
 \begin{enumerate} 
 	 \item Quelle est la raison de la suite \st{u}? Donner la valeur de $u_2$. 
 	 \item Exprimer $u_{n+1}$ en fonction de $u_n$. 
 	 \item Conjecturer le sens de variation de la suite \st{u}. 
 	 \item Démontrer le sens de variation. 
 	 \item On donne maintenant $u_n=8-14n$ pour tout $n \in \N$. Calculer $u_{10}$. 
 
 \end{enumerate} \end{frame}


\begin{frame}
\vspace{-10mm}
	\frametitle{Correction 1}
Soit \st{u} une suite arithmétique de premier terme $u_0=30$ et de raison $r=12$. 
 
 \begin{enumerate} 
 	 \item \begin{multicols}{3} 
 
 $u_0=30$ 
 \columnbreak 
 \begin{align*} 
 u_1 &= u_0+r \\ 
 &= 30+12\\ 
 &= 42 
 \end{align*}  
 \columnbreak 
 \begin{align*} 
 u_2 &= u_1+r \\ 
 &= 42+12\\ 
 &= 54 
 \end{align*} 
 \end{multicols} 
 \vfil 
 	 \item On a de manière immédiate d'après l'énoncé : 
 \hfil$\begin{cases} 
 u_0=30 \\ 
 u_{n+1}=u_n+12 
 \end{cases}$ 
 \vfil 
 	 \item Comme $u_0<u_1<u_2$, on peut conjecturer que la suite est croissante.
 \end{enumerate} 
 
 \end{frame} 
 
 \begin{frame}  
 \begin{enumerate} \setcounter{enumi}{3} 
 	 \item \textcolor{green}{Soit $n \in \N$}, 
 
  \begin{align*} 
 u_{n+1}-u_n &= u_n +12-u_n \\ 
 &= 12>0 
 \end{align*}La suite est donc bien croissante.
 \vfil 
 	 \item  On donne maintenant $u_n=30+12n$ pour tout $n \in \N$. 
 
  \hfil$u_{10}=30+12\times 10=150$. 
 
 \end{enumerate} \end{frame}


\begin{frame}
\vspace{-10mm}
	\frametitle{Correction 2}
\vspace*{0.5cm} 
 
 Soit \st{u} une suite définie par $u_n=16-12n$ pour tout $n \in \N$. 
 
 \begin{enumerate} 
 	 \item \begin{multicols}{3} 
 
 \begin{align*} 
 u_0 &=16-12\times 0 \\ &= 16\end{align*}  
 \columnbreak 
 \begin{align*} 
 u_1 &= 16-12\times 1 \\ 
 &= 16-12\\ 
 &= 4 
 \end{align*}  
 \columnbreak 
 \begin{align*} 
 u_2 &= 16-12\times 2 \\ 
 &= 16-24\\ 
 &= -8 
 \end{align*} 
 \end{multicols} 
 \vfil 
 	 \item \begin{align*} u_{n+1} &= 16-12(n+1) \\ &=  16-12n -12 \\ &=  4-12n 
 \end{align*} \end{enumerate} 
 
 \end{frame} 
 
 \begin{frame}  
 \begin{enumerate} \setcounter{enumi}{2}  
 	 \item Il semblerait que la suite soit arithmétique. Démontrons le.  \textcolor{green}{Soit $n \in \N$}, \begin{align*} 
 u_{n+1}-u_n &=  4-12n -\left(16-12n \right) \\ 
 &= 4-12n -16+12n \\ 
 &= -12 
 \end{align*} 
 \vfil 
 	 \item  D'après la question précédente, comme $u_{n+1}-u_n= -12<0$, la suite est décroissante.
 
 \end{enumerate} \end{frame}


\begin{frame}
\vspace{-10mm}
	\frametitle{Correction 3}
\vspace*{0.5cm} 
 
 Soit \st{u} une suite définie par $u_n=10-5n$ pour tout $n \in \N$. 
 
 \begin{enumerate} 
 	 \item \begin{multicols}{3} 
 
 \begin{align*} 
 u_0 &=10-5\times 0 \\ &= 10\end{align*}  
 \columnbreak 
 \begin{align*} 
 u_1 &= 10-5\times 1 \\ 
 &= 10-5\\ 
 &= 5 
 \end{align*}  
 \columnbreak 
 \begin{align*} 
 u_2 &= 10-5\times 2 \\ 
 &= 10-10\\ 
 &= 0 
 \end{align*} 
 \end{multicols} 
 \vfil 
 	 \item \begin{align*} u_{n+1} &= 10-5(n+1) \\ &=  10-5n -5 \\ &=  5-5n 
 \end{align*} \end{enumerate} 
 
 \end{frame} 
 
 \begin{frame}  
 \begin{enumerate} \setcounter{enumi}{2}  
 	 \item Il semblerait que la suite soit arithmétique. Démontrons le.  \textcolor{green}{Soit $n \in \N$}, \begin{align*} 
 u_{n+1}-u_n &=  5-5n -\left(10-5n \right) \\ 
 &= 5-5n -10+5n \\ 
 &= -5 
 \end{align*} 
 \vfil 
 	 \item  D'après la question précédente, comme $u_{n+1}-u_n= -5<0$, la suite est décroissante.
 
 \end{enumerate} \end{frame}


\begin{frame}
\vspace{-10mm}
	\frametitle{Correction 4}
Soit \st{u} une suite arithmétique de premier terme $u_0=21$ et de raison $r=-15$. 
 
 \begin{enumerate} 
 	 \item \begin{multicols}{3} 
 
 $u_0=21$ 
 \columnbreak 
 \begin{align*} 
 u_1 &= u_0+r \\ 
 &= 21-15\\ 
 &= 6 
 \end{align*}  
 \columnbreak 
 \begin{align*} 
 u_2 &= u_1+r \\ 
 &= 6-15\\ 
 &= -9 
 \end{align*} 
 \end{multicols} 
 \vfil 
 	 \item On a de manière immédiate d'après l'énoncé : 
 \hfil$\begin{cases} 
 u_0=21 \\ 
 u_{n+1}=u_n-15 
 \end{cases}$ 
 \vfil 
 	 \item Comme $u_0>u_1>u_2$, on peut conjecturer que la suite est décroissante.
 \end{enumerate} 
 
 \end{frame} 
 
 \begin{frame}  
 \begin{enumerate} \setcounter{enumi}{3} 
 	 \item \textcolor{green}{Soit $n \in \N$}, 
 
  \begin{align*} 
 u_{n+1}-u_n &= u_n -15-u_n \\ 
 &= -15<0 
 \end{align*}La suite est donc bien décroissante
 \vfil 
 	 \item  On donne maintenant $u_n=21-15n$ pour tout $n \in \N$. 
 
  \hfil$u_{10}=21-15\times 10=-129$. 
 
 \end{enumerate} \end{frame}


\begin{frame}
\vspace{-10mm}
	\frametitle{Correction 5}
Soit \st{u} une suite arithmétique telle que $u_0=8$ et $u_1=-6$. 
 
 \begin{enumerate} 
 	 \item On sait que la suite est arithmétique donc la raison est donnée par $u_1-u_0= -6-8=-14$. 
 
 La raison de la suite \st{u} est -14
 
 On a alors $u_2=u_1+r=-6-14=-20$ 
 \vfil 
 	 \item On a de manière immédiate d'après la question précédente : 
 \hfil$\begin{cases} 
 u_0=8 \\ 
 u_{n+1}=u_n-14 
 \end{cases}$ 
 \vfil 
 	 \item Comme $u_0>u_1>u_2$, on peut conjecturer que la suite est décroissante.
 \end{enumerate} 
 
 \end{frame} 
 
 \begin{frame}  
 \begin{enumerate} \setcounter{enumi}{3} 
 	 \item \textcolor{green}{Soit $n \in \N$}, 
 
  \begin{align*} 
 u_{n+1}-u_n &= u_n -14-u_n \\ 
 &= -14<0 
 \end{align*}La suite est donc bien décroissante
 \vfil 
 	 \item On donne maintenant $u_n=8-14n$ pour tout $n \in \N$. 
 
  \hfil$u_{10}=8-14\times 10=-132$. 
 
 \end{enumerate} \end{frame}




\end{document}