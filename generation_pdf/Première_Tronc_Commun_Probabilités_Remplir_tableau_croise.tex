\documentclass[15pt, mathserif]{beamer}

\usepackage[french]{babel}
\usepackage[T1]{fontenc}
\usepackage[utf8]{inputenc}
%\usepackage{esvect}
\usepackage{bm}
\usepackage{eurosym}
\usepackage{tikz}
\usepackage{pgf,tikz,pgfplots}
\pgfplotsset{compat=1.15}
\usepackage{mathrsfs}
\usetikzlibrary{arrows}
\usetikzlibrary{arrows.meta}

\usetikzlibrary{mindmap}
\usepackage{multicol}
\usepackage[tikz]{bclogo}
\usepackage{tkz-tab}
\usepackage{amsmath, tabu}
\usepackage{esvect} %\vv{AB} pour le vecteur AB

\DeclareMathOperator{\e}{e}

%% Tableau

\usepackage{makecell}
\setcellgapes{1pt}
\makegapedcells
\newcolumntype{R}[1]{>{\raggedleft\arraybackslash }b{#1}}
\newcolumntype{L}[1]{>{\raggedright\arraybackslash }b{#1}}
\newcolumntype{C}[1]{>{\centering\arraybackslash }b{#1}}


%pour avoir des parenthèses rondes dans le package fourier
\DeclareSymbolFont{cmoperators}   {OT1}{cmr} {m}{n}
\DeclareSymbolFont{cmlargesymbols}{OMX}{cmex}{m}{n}

\usefonttheme{professionalfonts} %permet d'enlever un bug avec fourier
\usepackage{fourier}
\DeclareMathDelimiter{(}{\mathopen} {cmoperators}{"28}{cmlargesymbols}{"00}
\DeclareMathDelimiter{)}{\mathclose}{cmoperators}{"29}{cmlargesymbols}{"01}

%Graphiques 

\usepackage{pgf,tikz,pgfplots}
\pgfplotsset{compat=1.15}
\usepackage{mathrsfs}
\usetikzlibrary{arrows}
\usetikzlibrary{mindmap}

%ensembles de nbres

\newcommand{\R}{\mathbb{R}}			%permet d'écrire le R "ensemble des réels"'
\newcommand{\N}{\mathbb{N}}			%permet d'écrire le N "ensemble des entiers naturels"
\newcommand{\Z}{\mathbb{Z}}			%permet d'écrire le Z "ensemble des entiers relatifs"
\newcommand{\Prem}{\mathbb{P}}	%permet d'écrire le P "ensemble des nombres premiers" (qui n'a pas marché avec le \P car il existe déjà)
\newcommand{\D}{\mathbb{D}}
\newcommand{\Df}{\mathcal{D}_f}
\newcommand{\Cf}{\mathcal{C}_f}

\newcommand{\Q}{\mathbb{Q}}


\newcommand{\st}[1]{$(#1_n)_{n \in \N}$}

\usetheme{Madrid}
\useoutertheme{miniframes} % Alternatively: miniframes, infolines, split
\useinnertheme{circles}
\definecolor{UBCblue}{rgb}{0.1, 0.25, 0.4} % UBC Blue (primary)
\definecolor{bordeaux}{RGB}{128,0,0}
\usecolortheme[named=UBCblue]{structure}

\usepackage{color} % J'aime bien définir mes couleurs
\definecolor{propcolor}{rgb}{0, 0.5, 1}
\definecolor{thcolor}{rgb}{0.6, 0.07, 0.07}
\colorlet{louis}{blue!70!green!60!white}
\colorlet{sakura}{pink!40!red}

\title{Activités Mentales}
\date{24 Août 2023}

\newcommand{\vco}[2]{\begin{pmatrix} #1 \\ #2 \end{pmatrix}} %Coordonnées de vecteur
\newenvironment{eq}{\begin{cases}\begin{tabu}{ccccc}}{\end{tabu}\end{cases}}
\newenvironment{eql}{\begin{cases}\begin{tabu}{cccccl}}{\end{tabu}\end{cases}}
\newenvironment{eqrl}{\begin{cases}\begin{tabu}{rl}}{\end{tabu}\end{cases}}

\newenvironment{Eq}{\begin{center}\begin{tabular}{rrcl}}{\end{tabular}\end{center}}
\newcommand{\ligneq}[2]{$\Longleftrightarrow$ & $#1$ & $=$ & $#2$ \\}
\newcommand{\Ligneq}[2]{ & $#1$ & $=$ & $#2$ \\}

\newenvironment{RPN}{\begin{center}\begin{tabular}{rrclcrcl}}{\end{tabular}\end{center}}
\newcommand{\Lignerpn}[4]{ & $#1$ & $=$ & $#2$ & ou & $#3$ & $=$ & $#4$ \\}
\newcommand{\lignerpn}[4]{$\Longleftrightarrow$ & $#1$ & $=$ & $#2$ & ou & $#3$ & $=$ & $#4$ \\}

\newenvironment{TRPN}{\begin{center}\begin{tabular}{rrclcrclcrcl}}{\end{tabular}\end{center}}
\newcommand{\Lignetrpn}[6]{ & $#1$ & $=$ & $#2$ & ou & $#3$ & $=$ & $#4$ & ou & $#5$ & $=$ & $#6$ \\}
\newcommand{\lignetrpn}[6]{$\Longleftrightarrow$ & $#1$ & $=$ & $#2$ & ou & $#3$ & $=$ & $#4$ & ou & $#5$ & $=$ & $#6$ \\}
\begin{document}

\begin{frame}
    \titlepage
\end{frame}

\begin{frame} 
	\frametitle{Question 1}
On considère 600 personnes pouvant appartenir au groupe A ou au groupe B, les deux ou aucun des deux. On trouve que 
 \begin{itemize} 
 \item 60\%  des personnes n'appartiennent pas au groupe B; 
 \item 70\% des personnes n'appartiennent pas au groupe A; 
 \item 252 sont dans aucun des deux groupes.
 \end{itemize} 
 \begin{enumerate} 
 \item Compléter le tableau suivant : 
 \hfil \begin{tabular}{|c|c|c|c|} 
 \cline{2-4}   
 \multicolumn{1}{c|}{} & $A$ & $\overline{A}$ & Total\\ \hline 
 $B$  &    & &   \\\hline 
 $\overline{B}$   &  &    &    \\\hline	
 Total   & &  &  \\\hline  
 \end{tabular} 
 \item Calculer la proportion de personnes n'appartenant  à aucun des deux groupes. 
 \item Calculer la proportion de personnes n'appartenant pas à B mais appartenant à A parmi les personnes n'appartenant pas à B. 
  \end{enumerate} 
 \end{frame}


\begin{frame} 
	\frametitle{Question 2}
On considère 1400 personnes pouvant appartenir au groupe A ou au groupe B, les deux ou aucun des deux. On trouve que 
 \begin{itemize} 
 \item 40\%  des personnes appartiennent au groupe B; 
 \item 40\% des personnes appartiennent au groupe A; 
 \item 336 sont dans B mais pas dans A.
 \end{itemize} 
 \begin{enumerate} 
 \item Compléter le tableau suivant : 
 \hfil \begin{tabular}{|c|c|c|c|} 
 \cline{2-4}   
 \multicolumn{1}{c|}{} & $A$ & $\overline{A}$ & Total\\ \hline 
 $B$  &    & &   \\\hline 
 $\overline{B}$   &  &    &    \\\hline	
 Total   & &  &  \\\hline  
 \end{tabular} 
 \item Calculer la proportion de personnes n'appartenant pas à B mais appartenant à A. 
 \item Calculer la proportion de personnes appartenant à B et A parmi les personnes appartenant à A. 
  \end{enumerate} 
 \end{frame}


\begin{frame} 
	\frametitle{Question 3}
On considère 1200 personnes pouvant appartenir au groupe A ou au groupe B, les deux ou aucun des deux. On trouve que 
 \begin{itemize} 
 \item 60\%  des personnes appartiennent au groupe B; 
 \item 60\% des personnes n'appartiennent pas au groupe A; 
 \item 288 sont dans les deux groupes.
 \end{itemize} 
 \begin{enumerate} 
 \item Compléter le tableau suivant : 
 \hfil \begin{tabular}{|c|c|c|c|} 
 \cline{2-4}   
 \multicolumn{1}{c|}{} & $A$ & $\overline{A}$ & Total\\ \hline 
 $B$  &    & &   \\\hline 
 $\overline{B}$   &  &    &    \\\hline	
 Total   & &  &  \\\hline  
 \end{tabular} 
 \item Calculer la proportion de personnes n'appartenant à aucun des deux groupes. 
 \item Calculer la proportion de personnes n'appartenant pas à A mais appartenant à B parmi les personnes n'appartenant pas à A. 
  \end{enumerate} 
 \end{frame}


\begin{frame} 
	\frametitle{Question 4}
On considère 600 personnes pouvant appartenir au groupe A ou au groupe B, les deux ou aucun des deux. On trouve que 
 \begin{itemize} 
 \item 60\%  des personnes appartiennent au groupe B; 
 \item 70\% des personnes appartiennent au groupe A; 
 \item 108 sont dans B mais pas dans A.
 \end{itemize} 
 \begin{enumerate} 
 \item Compléter le tableau suivant : 
 \hfil \begin{tabular}{|c|c|c|c|} 
 \cline{2-4}   
 \multicolumn{1}{c|}{} & $A$ & $\overline{A}$ & Total\\ \hline 
 $B$  &    & &   \\\hline 
 $\overline{B}$   &  &    &    \\\hline	
 Total   & &  &  \\\hline  
 \end{tabular} 
 \item Calculer la proportion de personnes n'appartenant pas à B mais appartenant à A. 
 \item Calculer la proportion de personnes appartenant à B et A parmi les personnes appartenant à A. 
  \end{enumerate} 
 \end{frame}


\begin{frame} 
	\frametitle{Question 5}
On considère 800 personnes pouvant appartenir au groupe A ou au groupe B, les deux ou aucun des deux. On trouve que 
 \begin{itemize} 
 \item 60\%  des personnes appartiennent au groupe B; 
 \item 70\% des personnes n'appartiennent pas au groupe A; 
 \item 144 sont dans les deux groupes.
 \end{itemize} 
 \begin{enumerate} 
 \item Compléter le tableau suivant : 
 \hfil \begin{tabular}{|c|c|c|c|} 
 \cline{2-4}   
 \multicolumn{1}{c|}{} & $A$ & $\overline{A}$ & Total\\ \hline 
 $B$  &    & &   \\\hline 
 $\overline{B}$   &  &    &    \\\hline	
 Total   & &  &  \\\hline  
 \end{tabular} 
 \item Calculer la proportion de personnes n'appartenant à aucun des deux groupes. 
 \item Calculer la proportion de personnes n'appartenant pas à A mais appartenant à B parmi les personnes n'appartenant pas à A. 
  \end{enumerate} 
 \end{frame}


\begin{frame}
\vspace{-10mm}
	\frametitle{Correction 1}
\begin{center} 
 \begin{tabular}{|c|c|c|c|} 
 \cline{2-4} 
 \multicolumn{1}{c|}{} & A & $\overline{A}$ & Total \\\hline 
 B   &72  &168& 240 \\\hline 
 $\overline{B}$   &108 & 252 & 360 \\\hline 
 Total   &180&420 &600 \\\hline  
 \end{tabular} 
 \end{center}
 La proportion de personnes n'appartenant à aucun des deux groupes est de $\dfrac{72}{600 }= \dfrac{3}{25}$
 \\ La proportion de personnes n'appartenant pas à B mais appartenant à A parmi les personnes n'appartenant pas à B est de $\dfrac{108}{360 }= \dfrac{3}{10}$
\end{frame}


\begin{frame}
\vspace{-10mm}
	\frametitle{Correction 2}
\begin{center} 
 \begin{tabular}{|c|c|c|c|} 
 \cline{2-4} 
 \multicolumn{1}{c|}{} & A & $\overline{A}$ & Total \\\hline 
 B   &224  &336& 560 \\\hline 
 $\overline{B}$   &336 & 504 & 840 \\\hline 
 Total   &560&840 &1400 \\\hline  
 \end{tabular} 
 \end{center}
 La proportion de personnes n'appartenant pas à B mais appartenant à A est de $\dfrac{336}{1400 }= \dfrac{6}{25}$
 \\ La proportion de personnes appartenant à B et A parmi les personnes appartenant à A est de $\dfrac{224}{560 }= \dfrac{2}{5}$
\end{frame}


\begin{frame}
\vspace{-10mm}
	\frametitle{Correction 3}
\begin{center} 
 \begin{tabular}{|c|c|c|c|} 
 \cline{2-4} 
 \multicolumn{1}{c|}{} & A & $\overline{A}$ & Total \\\hline 
 B   &288  &432& 720 \\\hline 
 $\overline{B}$   &192 & 288 & 480 \\\hline 
 Total   &480&720 &1200 \\\hline  
 \end{tabular} 
 \end{center}
 La proportion de personnes n'appartenant à aucun des deux groupes est de $\dfrac{288}{1200 }= \dfrac{6}{25}$
 \\ La proportion de personnes n'appartenant pas à A mais appartenant à B parmi les personnes n'appartenant pas à A est de $\dfrac{432}{720 }= \dfrac{3}{5}$
\end{frame}


\begin{frame}
\vspace{-10mm}
	\frametitle{Correction 4}
\begin{center} 
 \begin{tabular}{|c|c|c|c|} 
 \cline{2-4} 
 \multicolumn{1}{c|}{} & A & $\overline{A}$ & Total \\\hline 
 B   &252  &108& 360 \\\hline 
 $\overline{B}$   &168 & 72 & 240 \\\hline 
 Total   &420&180 &600 \\\hline  
 \end{tabular} 
 \end{center}
 La proportion de personnes n'appartenant pas à B mais appartenant à A est de $\dfrac{168}{600 }= \dfrac{7}{25}$
 \\ La proportion de personnes appartenant à B et A parmi les personnes appartenant à A est de $\dfrac{252}{420 }= \dfrac{3}{5}$
\end{frame}


\begin{frame}
\vspace{-10mm}
	\frametitle{Correction 5}
\begin{center} 
 \begin{tabular}{|c|c|c|c|} 
 \cline{2-4} 
 \multicolumn{1}{c|}{} & A & $\overline{A}$ & Total \\\hline 
 B   &144  &336& 480 \\\hline 
 $\overline{B}$   &96 & 224 & 320 \\\hline 
 Total   &240&560 &800 \\\hline  
 \end{tabular} 
 \end{center}
 La proportion de personnes n'appartenant à aucun des deux groupes est de $\dfrac{224}{800 }= \dfrac{7}{25}$
 \\ La proportion de personnes n'appartenant pas à A mais appartenant à B parmi les personnes n'appartenant pas à A est de $\dfrac{336}{560 }= \dfrac{3}{5}$
\end{frame}




\end{document}