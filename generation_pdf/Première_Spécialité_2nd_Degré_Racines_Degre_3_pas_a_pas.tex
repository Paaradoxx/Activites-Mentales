\documentclass[15pt, mathserif]{beamer}

\usepackage[french]{babel}
\usepackage[T1]{fontenc}
\usepackage[utf8]{inputenc}
%\usepackage{esvect}
\usepackage{bm}
\usepackage{eurosym}
\usepackage{tikz}
\usepackage{pgf,tikz,pgfplots}
\pgfplotsset{compat=1.15}
\usepackage{mathrsfs}
\usetikzlibrary{arrows}
\usetikzlibrary{arrows.meta}

\usetikzlibrary{mindmap}
\usepackage{multicol}
\usepackage[tikz]{bclogo}
\usepackage{tkz-tab}
\usepackage{amsmath, tabu}
\usepackage{esvect} %\vv{AB} pour le vecteur AB

\DeclareMathOperator{\e}{e}

%% Tableau

\usepackage{makecell}
\setcellgapes{1pt}
\makegapedcells
\newcolumntype{R}[1]{>{\raggedleft\arraybackslash }b{#1}}
\newcolumntype{L}[1]{>{\raggedright\arraybackslash }b{#1}}
\newcolumntype{C}[1]{>{\centering\arraybackslash }b{#1}}


%pour avoir des parenthèses rondes dans le package fourier
\DeclareSymbolFont{cmoperators}   {OT1}{cmr} {m}{n}
\DeclareSymbolFont{cmlargesymbols}{OMX}{cmex}{m}{n}

\usefonttheme{professionalfonts} %permet d'enlever un bug avec fourier
\usepackage{fourier}
\DeclareMathDelimiter{(}{\mathopen} {cmoperators}{"28}{cmlargesymbols}{"00}
\DeclareMathDelimiter{)}{\mathclose}{cmoperators}{"29}{cmlargesymbols}{"01}

%Graphiques 

\usepackage{pgf,tikz,pgfplots}
\pgfplotsset{compat=1.15}
\usepackage{mathrsfs}
\usetikzlibrary{arrows}
\usetikzlibrary{mindmap}

%ensembles de nbres

\newcommand{\R}{\mathbb{R}}			%permet d'écrire le R "ensemble des réels"'
\newcommand{\N}{\mathbb{N}}			%permet d'écrire le N "ensemble des entiers naturels"
\newcommand{\Z}{\mathbb{Z}}			%permet d'écrire le Z "ensemble des entiers relatifs"
\newcommand{\Prem}{\mathbb{P}}	%permet d'écrire le P "ensemble des nombres premiers" (qui n'a pas marché avec le \P car il existe déjà)
\newcommand{\D}{\mathbb{D}}
\newcommand{\Df}{\mathcal{D}_f}
\newcommand{\Cf}{\mathcal{C}_f}

\newcommand{\Q}{\mathbb{Q}}


\newcommand{\st}[1]{$(#1_n)_{n \in \N}$}

\usetheme{Madrid}
\useoutertheme{miniframes} % Alternatively: miniframes, infolines, split
\useinnertheme{circles}
\definecolor{UBCblue}{rgb}{0.1, 0.25, 0.4} % UBC Blue (primary)
\definecolor{bordeaux}{RGB}{128,0,0}
\usecolortheme[named=UBCblue]{structure}

\usepackage{color} % J'aime bien définir mes couleurs
\definecolor{propcolor}{rgb}{0, 0.5, 1}
\definecolor{thcolor}{rgb}{0.6, 0.07, 0.07}
\colorlet{louis}{blue!70!green!60!white}
\colorlet{sakura}{pink!40!red}

\title{Activités Mentales}
\date{24 Août 2023}

\newcommand{\vco}[2]{\begin{pmatrix} #1 \\ #2 \end{pmatrix}} %Coordonnées de vecteur
\newenvironment{eq}{\begin{cases}\begin{tabu}{ccccc}}{\end{tabu}\end{cases}}
\newenvironment{eql}{\begin{cases}\begin{tabu}{cccccl}}{\end{tabu}\end{cases}}
\newenvironment{eqrl}{\begin{cases}\begin{tabu}{rl}}{\end{tabu}\end{cases}}

\newenvironment{Eq}{\begin{center}\begin{tabular}{rrcl}}{\end{tabular}\end{center}}
\newcommand{\ligneq}[2]{$\Longleftrightarrow$ & $#1$ & $=$ & $#2$ \\}
\newcommand{\Ligneq}[2]{ & $#1$ & $=$ & $#2$ \\}

\newenvironment{RPN}{\begin{center}\begin{tabular}{rrclcrcl}}{\end{tabular}\end{center}}
\newcommand{\Lignerpn}[4]{ & $#1$ & $=$ & $#2$ & ou & $#3$ & $=$ & $#4$ \\}
\newcommand{\lignerpn}[4]{$\Longleftrightarrow$ & $#1$ & $=$ & $#2$ & ou & $#3$ & $=$ & $#4$ \\}

\newenvironment{TRPN}{\begin{center}\begin{tabular}{rrclcrclcrcl}}{\end{tabular}\end{center}}
\newcommand{\Lignetrpn}[6]{ & $#1$ & $=$ & $#2$ & ou & $#3$ & $=$ & $#4$ & ou & $#5$ & $=$ & $#6$ \\}
\newcommand{\lignetrpn}[6]{$\Longleftrightarrow$ & $#1$ & $=$ & $#2$ & ou & $#3$ & $=$ & $#4$ & ou & $#5$ & $=$ & $#6$ \\}
\begin{document}

\begin{frame}
    \titlepage
\end{frame}

\begin{frame} 
	\frametitle{Question 1}
\begin{enumerate} 
 	 	 \item Montrer que pour tout réel $x$:~ $(x-9)(x-10)=x^2-19x+90$. 
 	  \item Résoudre dans $\R$ l'équation $(E):~x^2-19x+90= 0$.  
 	 \item En déduire les solutions réelles de l'équation $(E'):~x^3-19x^2+90x = 0$. 
 \end{enumerate}\end{frame}


\begin{frame} 
	\frametitle{Question 2}
\begin{enumerate} 
 	 	 \item Montrer que pour tout réel $x$:~ $(x-4)(x+8)=x^2+4x-32$. 
 	  \item Résoudre dans $\R$ l'équation $(E):~x^2+4x-32= 0$.  
 	 \item En déduire les solutions réelles de l'équation $(E'):~x^3+4x^2-32x = 0$. 
 \end{enumerate}\end{frame}


\begin{frame} 
	\frametitle{Question 3}
\begin{enumerate} 
 	 	 \item Montrer que pour tout réel $x$:~ $(x+6)(x+2)=x^2+8x+12$. 
 	  \item Résoudre dans $\R$ l'équation $(E):~x^2+8x+12= 0$.  
 	 \item En déduire les solutions réelles de l'équation $(E'):~x^3+8x^2+12x = 0$. 
 \end{enumerate}\end{frame}


\begin{frame} 
	\frametitle{Question 4}
\begin{enumerate} 
 	 	 \item Montrer que pour tout réel $x$:~ $(x-3)(x-4)=x^2-7x+12$. 
 	  \item Résoudre dans $\R$ l'équation $(E):~x^2-7x+12= 0$.  
 	 \item En déduire les solutions réelles de l'équation $(E'):~x^3-7x^2+12x = 0$. 
 \end{enumerate}\end{frame}


\begin{frame} 
	\frametitle{Question 5}
\begin{enumerate} 
 	 	 \item Montrer que pour tout réel $x$:~ $(x+5)(x+9)=x^2+14x+45$. 
 	  \item Résoudre dans $\R$ l'équation $(E):~x^2+14x+45= 0$.  
 	 \item En déduire les solutions réelles de l'équation $(E'):~x^3+14x^2+45x = 0$. 
 \end{enumerate}\end{frame}


\begin{frame}
\vspace{-10mm}
	\frametitle{Correction 1}
\begin{enumerate} 
 	 \item Montrer que pour tout réel $x$:~ $(x-9)(x-10)=x^2-19x+90$. 
 	 \item Résoudre dans $\R$ l'équation $(E):~x^2-19x+90= 0$. 
 	 \item En déduire les solutions réelles de l'équation $(E'):~x^3-19x^2+90x = 0$. 
 \end{enumerate} 
 
 \bigskip 
 \bigskip 
 \begin{enumerate} 
 	 \item On développe le membre de gauche pour retomber sur le membre de droite (plus facile que de factoriser). Soit $x$ un réel : 
 	 	 \begin{align*} 
 	 	 (x-9)(x-10)&=x^2-10x-9x+90 \\ 
 	 	 	 &=x^2-19x+90 
 	 \end{align*} 
 \end{enumerate} 
 \end{frame} 
 \begin{frame} 
 \begin{enumerate} \setcounter{enumi}{1}  
 	 \item C'est une \textbf{équation du second degré} : pour la résoudre on doit donc se ramener à une \textbf{équation produit nul}. On doit donc \textbf{factoriser} $x^2-19x+90$. On utilise la première question : 
 	 	  \begin{RPN} 
 	 	  	 \Ligneq{x^2-19x+90}{0} 
 	 	 	 \ligneq{(x-9)(x-10)}{0} 
 	 	 	 \lignerpn{x-9}{0}{x-10}{0} 
 	 	 	 \lignerpn{x}{9}{x}{10} 
 	 	  \end{RPN} 
 	 \item C'est une \textbf{équation autres de degré 3} : pour la résoudre on doit donc se ramener à une \textbf{équation produit nul}, pour cela on doit donc \textbf{factoriser} $x^3-19x^2+90x $ soit par un facteur commun soit via une identité remarquable. Ici le facteur commun est $x$. On a donc : 
 	 	 \begin{footnotesize} 
 \begin{TRPN} 
 	 	 	 \Ligneq{x^3-19x^2+90x}{0} 
 	 	 	 \ligneq{x(x^2-19x+90)}{0} 
 	 	 	 \lignerpn{x}{0}{x^2-19x+90}{0} 
 	 	 	 \lignetrpn{x}{0}{x}{9}{x}{10} 
 	 	 \end{TRPN} 
 \end{footnotesize} L'ensemble des solutions est $S=\{0;9;10\}$. 
 \end{enumerate} 
 
 \end{frame}


\begin{frame}
\vspace{-10mm}
	\frametitle{Correction 2}
\begin{enumerate} 
 	 \item Montrer que pour tout réel $x$:~ $(x-4)(x+8)=x^2+4x-32$. 
 	 \item Résoudre dans $\R$ l'équation $(E):~x^2+4x-32= 0$. 
 	 \item En déduire les solutions réelles de l'équation $(E'):~x^3+4x^2-32x = 0$. 
 \end{enumerate} 
 
 \bigskip 
 \bigskip 
 \begin{enumerate} 
 	 \item On développe le membre de gauche pour retomber sur le membre de droite (plus facile que de factoriser). Soit $x$ un réel : 
 	 	 \begin{align*} 
 	 	 (x-4)(x+8)&=x^2+8x-4x-32 \\ 
 	 	 	 &=x^2+4x-32 
 	 \end{align*} 
 \end{enumerate} 
 \end{frame} 
 \begin{frame} 
 \begin{enumerate} \setcounter{enumi}{1}  
 	 \item C'est une \textbf{équation du second degré} : pour la résoudre on doit donc se ramener à une \textbf{équation produit nul}. On doit donc \textbf{factoriser} $x^2+4x-32$. On utilise la première question : 
 	 	  \begin{RPN} 
 	 	  	 \Ligneq{x^2+4x-32}{0} 
 	 	 	 \ligneq{(x-4)(x+8)}{0} 
 	 	 	 \lignerpn{x-4}{0}{x+8}{0} 
 	 	 	 \lignerpn{x}{4}{x}{-8} 
 	 	  \end{RPN} 
 	 \item C'est une \textbf{équation autres de degré 3} : pour la résoudre on doit donc se ramener à une \textbf{équation produit nul}, pour cela on doit donc \textbf{factoriser} $x^3+4x^2-32x $ soit par un facteur commun soit via une identité remarquable. Ici le facteur commun est $x$. On a donc : 
 	 	 \begin{footnotesize} 
 \begin{TRPN} 
 	 	 	 \Ligneq{x^3+4x^2-32x}{0} 
 	 	 	 \ligneq{x(x^2+4x-32)}{0} 
 	 	 	 \lignerpn{x}{0}{x^2+4x-32}{0} 
 	 	 	 \lignetrpn{x}{0}{x}{4}{x}{-8} 
 	 	 \end{TRPN} 
 \end{footnotesize} L'ensemble des solutions est $S=\{-8;0;4\}$. 
 \end{enumerate} 
 
 \end{frame}


\begin{frame}
\vspace{-10mm}
	\frametitle{Correction 3}
\begin{enumerate} 
 	 \item Montrer que pour tout réel $x$:~ $(x+6)(x+2)=x^2+8x+12$. 
 	 \item Résoudre dans $\R$ l'équation $(E):~x^2+8x+12= 0$. 
 	 \item En déduire les solutions réelles de l'équation $(E'):~x^3+8x^2+12x = 0$. 
 \end{enumerate} 
 
 \bigskip 
 \bigskip 
 \begin{enumerate} 
 	 \item On développe le membre de gauche pour retomber sur le membre de droite (plus facile que de factoriser). Soit $x$ un réel : 
 	 	 \begin{align*} 
 	 	 (x+6)(x+2)&=x^2+2x+6x+12 \\ 
 	 	 	 &=x^2+8x+12 
 	 \end{align*} 
 \end{enumerate} 
 \end{frame} 
 \begin{frame} 
 \begin{enumerate} \setcounter{enumi}{1}  
 	 \item C'est une \textbf{équation du second degré} : pour la résoudre on doit donc se ramener à une \textbf{équation produit nul}. On doit donc \textbf{factoriser} $x^2+8x+12$. On utilise la première question : 
 	 	  \begin{RPN} 
 	 	  	 \Ligneq{x^2+8x+12}{0} 
 	 	 	 \ligneq{(x+6)(x+2)}{0} 
 	 	 	 \lignerpn{x+6}{0}{x+2}{0} 
 	 	 	 \lignerpn{x}{-6}{x}{-2} 
 	 	  \end{RPN} 
 	 \item C'est une \textbf{équation autres de degré 3} : pour la résoudre on doit donc se ramener à une \textbf{équation produit nul}, pour cela on doit donc \textbf{factoriser} $x^3+8x^2+12x $ soit par un facteur commun soit via une identité remarquable. Ici le facteur commun est $x$. On a donc : 
 	 	 \begin{footnotesize} 
 \begin{TRPN} 
 	 	 	 \Ligneq{x^3+8x^2+12x}{0} 
 	 	 	 \ligneq{x(x^2+8x+12)}{0} 
 	 	 	 \lignerpn{x}{0}{x^2+8x+12}{0} 
 	 	 	 \lignetrpn{x}{0}{x}{-6}{x}{-2} 
 	 	 \end{TRPN} 
 \end{footnotesize} L'ensemble des solutions est $S=\{-6;-2;0\}$. 
 \end{enumerate} 
 
 \end{frame}


\begin{frame}
\vspace{-10mm}
	\frametitle{Correction 4}
\begin{enumerate} 
 	 \item Montrer que pour tout réel $x$:~ $(x-3)(x-4)=x^2-7x+12$. 
 	 \item Résoudre dans $\R$ l'équation $(E):~x^2-7x+12= 0$. 
 	 \item En déduire les solutions réelles de l'équation $(E'):~x^3-7x^2+12x = 0$. 
 \end{enumerate} 
 
 \bigskip 
 \bigskip 
 \begin{enumerate} 
 	 \item On développe le membre de gauche pour retomber sur le membre de droite (plus facile que de factoriser). Soit $x$ un réel : 
 	 	 \begin{align*} 
 	 	 (x-3)(x-4)&=x^2-4x-3x+12 \\ 
 	 	 	 &=x^2-7x+12 
 	 \end{align*} 
 \end{enumerate} 
 \end{frame} 
 \begin{frame} 
 \begin{enumerate} \setcounter{enumi}{1}  
 	 \item C'est une \textbf{équation du second degré} : pour la résoudre on doit donc se ramener à une \textbf{équation produit nul}. On doit donc \textbf{factoriser} $x^2-7x+12$. On utilise la première question : 
 	 	  \begin{RPN} 
 	 	  	 \Ligneq{x^2-7x+12}{0} 
 	 	 	 \ligneq{(x-3)(x-4)}{0} 
 	 	 	 \lignerpn{x-3}{0}{x-4}{0} 
 	 	 	 \lignerpn{x}{3}{x}{4} 
 	 	  \end{RPN} 
 	 \item C'est une \textbf{équation autres de degré 3} : pour la résoudre on doit donc se ramener à une \textbf{équation produit nul}, pour cela on doit donc \textbf{factoriser} $x^3-7x^2+12x $ soit par un facteur commun soit via une identité remarquable. Ici le facteur commun est $x$. On a donc : 
 	 	 \begin{footnotesize} 
 \begin{TRPN} 
 	 	 	 \Ligneq{x^3-7x^2+12x}{0} 
 	 	 	 \ligneq{x(x^2-7x+12)}{0} 
 	 	 	 \lignerpn{x}{0}{x^2-7x+12}{0} 
 	 	 	 \lignetrpn{x}{0}{x}{3}{x}{4} 
 	 	 \end{TRPN} 
 \end{footnotesize} L'ensemble des solutions est $S=\{0;3;4\}$. 
 \end{enumerate} 
 
 \end{frame}


\begin{frame}
\vspace{-10mm}
	\frametitle{Correction 5}
\begin{enumerate} 
 	 \item Montrer que pour tout réel $x$:~ $(x+5)(x+9)=x^2+14x+45$. 
 	 \item Résoudre dans $\R$ l'équation $(E):~x^2+14x+45= 0$. 
 	 \item En déduire les solutions réelles de l'équation $(E'):~x^3+14x^2+45x = 0$. 
 \end{enumerate} 
 
 \bigskip 
 \bigskip 
 \begin{enumerate} 
 	 \item On développe le membre de gauche pour retomber sur le membre de droite (plus facile que de factoriser). Soit $x$ un réel : 
 	 	 \begin{align*} 
 	 	 (x+5)(x+9)&=x^2+9x+5x+45 \\ 
 	 	 	 &=x^2+14x+45 
 	 \end{align*} 
 \end{enumerate} 
 \end{frame} 
 \begin{frame} 
 \begin{enumerate} \setcounter{enumi}{1}  
 	 \item C'est une \textbf{équation du second degré} : pour la résoudre on doit donc se ramener à une \textbf{équation produit nul}. On doit donc \textbf{factoriser} $x^2+14x+45$. On utilise la première question : 
 	 	  \begin{RPN} 
 	 	  	 \Ligneq{x^2+14x+45}{0} 
 	 	 	 \ligneq{(x+5)(x+9)}{0} 
 	 	 	 \lignerpn{x+5}{0}{x+9}{0} 
 	 	 	 \lignerpn{x}{-5}{x}{-9} 
 	 	  \end{RPN} 
 	 \item C'est une \textbf{équation autres de degré 3} : pour la résoudre on doit donc se ramener à une \textbf{équation produit nul}, pour cela on doit donc \textbf{factoriser} $x^3+14x^2+45x $ soit par un facteur commun soit via une identité remarquable. Ici le facteur commun est $x$. On a donc : 
 	 	 \begin{footnotesize} 
 \begin{TRPN} 
 	 	 	 \Ligneq{x^3+14x^2+45x}{0} 
 	 	 	 \ligneq{x(x^2+14x+45)}{0} 
 	 	 	 \lignerpn{x}{0}{x^2+14x+45}{0} 
 	 	 	 \lignetrpn{x}{0}{x}{-5}{x}{-9} 
 	 	 \end{TRPN} 
 \end{footnotesize} L'ensemble des solutions est $S=\{-9;-5;0\}$. 
 \end{enumerate} 
 
 \end{frame}




\end{document}