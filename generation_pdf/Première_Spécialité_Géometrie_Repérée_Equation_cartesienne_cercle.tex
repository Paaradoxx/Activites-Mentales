\documentclass[15pt, mathserif]{beamer}

\usepackage[french]{babel}
\usepackage[T1]{fontenc}
\usepackage[utf8]{inputenc}
%\usepackage{esvect}
\usepackage{bm}
\usepackage{eurosym}
\usepackage{tikz}
\usepackage{pgf,tikz,pgfplots}
\pgfplotsset{compat=1.15}
\usepackage{mathrsfs}
\usetikzlibrary{arrows}
\usetikzlibrary{arrows.meta}

\usetikzlibrary{mindmap}
\usepackage{multicol}
\usepackage[tikz]{bclogo}
\usepackage{tkz-tab}
\usepackage{amsmath, tabu}
\usepackage{esvect} %\vv{AB} pour le vecteur AB

\DeclareMathOperator{\e}{e}

%% Tableau

\usepackage{makecell}
\setcellgapes{1pt}
\makegapedcells
\newcolumntype{R}[1]{>{\raggedleft\arraybackslash }b{#1}}
\newcolumntype{L}[1]{>{\raggedright\arraybackslash }b{#1}}
\newcolumntype{C}[1]{>{\centering\arraybackslash }b{#1}}


%pour avoir des parenthèses rondes dans le package fourier
\DeclareSymbolFont{cmoperators}   {OT1}{cmr} {m}{n}
\DeclareSymbolFont{cmlargesymbols}{OMX}{cmex}{m}{n}

\usefonttheme{professionalfonts} %permet d'enlever un bug avec fourier
\usepackage{fourier}
\DeclareMathDelimiter{(}{\mathopen} {cmoperators}{"28}{cmlargesymbols}{"00}
\DeclareMathDelimiter{)}{\mathclose}{cmoperators}{"29}{cmlargesymbols}{"01}

%Graphiques 

\usepackage{pgf,tikz,pgfplots}
\pgfplotsset{compat=1.15}
\usepackage{mathrsfs}
\usetikzlibrary{arrows}
\usetikzlibrary{mindmap}

%ensembles de nbres

\newcommand{\R}{\mathbb{R}}			%permet d'écrire le R "ensemble des réels"'
\newcommand{\N}{\mathbb{N}}			%permet d'écrire le N "ensemble des entiers naturels"
\newcommand{\Z}{\mathbb{Z}}			%permet d'écrire le Z "ensemble des entiers relatifs"
\newcommand{\Prem}{\mathbb{P}}	%permet d'écrire le P "ensemble des nombres premiers" (qui n'a pas marché avec le \P car il existe déjà)
\newcommand{\D}{\mathbb{D}}
\newcommand{\Df}{\mathcal{D}_f}
\newcommand{\Cf}{\mathcal{C}_f}

\newcommand{\Q}{\mathbb{Q}}


\newcommand{\st}[1]{$(#1_n)_{n \in \N}$}

\usetheme{Madrid}
\useoutertheme{miniframes} % Alternatively: miniframes, infolines, split
\useinnertheme{circles}
\definecolor{UBCblue}{rgb}{0.1, 0.25, 0.4} % UBC Blue (primary)
\definecolor{bordeaux}{RGB}{128,0,0}
\usecolortheme[named=UBCblue]{structure}

\usepackage{color} % J'aime bien définir mes couleurs
\definecolor{propcolor}{rgb}{0, 0.5, 1}
\definecolor{thcolor}{rgb}{0.6, 0.07, 0.07}
\colorlet{louis}{blue!70!green!60!white}
\colorlet{sakura}{pink!40!red}

\title{Activités Mentales}
\date{24 Août 2023}

\newcommand{\vco}[2]{\begin{pmatrix} #1 \\ #2 \end{pmatrix}} %Coordonnées de vecteur
\newenvironment{eq}{\begin{cases}\begin{tabu}{ccccc}}{\end{tabu}\end{cases}}
\newenvironment{eql}{\begin{cases}\begin{tabu}{cccccl}}{\end{tabu}\end{cases}}
\newenvironment{eqrl}{\begin{cases}\begin{tabu}{rl}}{\end{tabu}\end{cases}}

\newenvironment{Eq}{\begin{center}\begin{tabular}{rrcl}}{\end{tabular}\end{center}}
\newcommand{\ligneq}[2]{$\Longleftrightarrow$ & $#1$ & $=$ & $#2$ \\}
\newcommand{\Ligneq}[2]{ & $#1$ & $=$ & $#2$ \\}

\newenvironment{RPN}{\begin{center}\begin{tabular}{rrclcrcl}}{\end{tabular}\end{center}}
\newcommand{\Lignerpn}[4]{ & $#1$ & $=$ & $#2$ & ou & $#3$ & $=$ & $#4$ \\}
\newcommand{\lignerpn}[4]{$\Longleftrightarrow$ & $#1$ & $=$ & $#2$ & ou & $#3$ & $=$ & $#4$ \\}

\newenvironment{TRPN}{\begin{center}\begin{tabular}{rrclcrclcrcl}}{\end{tabular}\end{center}}
\newcommand{\Lignetrpn}[6]{ & $#1$ & $=$ & $#2$ & ou & $#3$ & $=$ & $#4$ & ou & $#5$ & $=$ & $#6$ \\}
\newcommand{\lignetrpn}[6]{$\Longleftrightarrow$ & $#1$ & $=$ & $#2$ & ou & $#3$ & $=$ & $#4$ & ou & $#5$ & $=$ & $#6$ \\}
\begin{document}

\begin{frame}
    \titlepage
\end{frame}

\begin{frame} 
	\frametitle{Question 1}
Déterminer l'ensemble des points $M(x;y)$ tels que $$x^2+y^2-14x+20y+148=0$$ et préciser les éléments caractéristiques de cet ensemble.\end{frame}


\begin{frame} 
	\frametitle{Question 2}
Déterminer l'ensemble des points $M(x;y)$ tels que $$x^2+y^2+10x+2y+22=0$$ et préciser les éléments caractéristiques de cet ensemble.\end{frame}


\begin{frame} 
	\frametitle{Question 3}
Déterminer l'ensemble des points $M(x;y)$ tels que $$x^2+y^2-4x-4y+4=0$$ et préciser les éléments caractéristiques de cet ensemble.\end{frame}


\begin{frame} 
	\frametitle{Question 4}
Déterminer l'ensemble des points $M(x;y)$ tels que $$x^2+y^2-10x-16y+8=0$$ et préciser les éléments caractéristiques de cet ensemble.\end{frame}


\begin{frame} 
	\frametitle{Question 5}
Déterminer l'ensemble des points $M(x;y)$ tels que $$x^2+y^2-18x+4y+69=0$$ et préciser les éléments caractéristiques de cet ensemble.\end{frame}


\begin{frame}
\vspace{-10mm}
	\frametitle{Correction 1}
Montrons que l'ensemble des points $M(x; y)$ tels que $x^2+y^2-14x+20y+148=0$ est un cercle dont on précisera les caractéristiques (centre et rayon). 
 $$\begin{array}{crcl} 
 & x^2+y^2-14x+20y+148& = & 0 \\ 
 \Leftrightarrow & x^2+y^2+2 \times\left(-7\right)x+2 \times 10y+148 & = & 0 \\ 
 \Leftrightarrow & x^2+2 \times\left(-7\right)x+ \left(-7\right)^2-\left(-7\right)^2+y^2+2 \times 10y + 10^2-10^2+148 & = & 0 \\ 
 \Leftrightarrow & (x-7)^2+(y+10)^2-\left(-7\right)^2-10^2+148 & = & 0 \\ 
 \Leftrightarrow & (x-7)^2+(y+10)^2 & = & 1 \\ 
 \Leftrightarrow & (x-7)^2+(y+10)^2 & = & 1^2 \\ 
 \end{array}$$ 
 L'ensemble des points $M(x; y)$ tels que $x^2+y^2-14x+20y+148=0$ est un cercle de centre $\Omega(7;-10)$ et de rayon $r =1$\end{frame}


\begin{frame}
\vspace{-10mm}
	\frametitle{Correction 2}
Montrons que l'ensemble des points $M(x; y)$ tels que $x^2+y^2+10x+2y+22=0$ est un cercle dont on précisera les caractéristiques (centre et rayon). 
 $$\begin{array}{crcl} 
 & x^2+y^2+10x+2y+22& = & 0 \\ 
 \Leftrightarrow & x^2+y^2+2 \times5x+2 \times 1y+22 & = & 0 \\ 
 \Leftrightarrow & x^2+2 \times5x+ 5^2-5^2+y^2+2 \times 1y + 1^2-1^2+22 & = & 0 \\ 
 \Leftrightarrow & (x+5)^2+(y+1)^2-5^2-1^2+22 & = & 0 \\ 
 \Leftrightarrow & (x+5)^2+(y+1)^2 & = & 4 \\ 
 \Leftrightarrow & (x+5)^2+(y+1)^2 & = & 2^2 \\ 
 \end{array}$$ 
 L'ensemble des points $M(x; y)$ tels que $x^2+y^2+10x+2y+22=0$ est un cercle de centre $\Omega(-5;-1)$ et de rayon $r =2$\end{frame}


\begin{frame}
\vspace{-10mm}
	\frametitle{Correction 3}
Montrons que l'ensemble des points $M(x; y)$ tels que $x^2+y^2-4x-4y+4=0$ est un cercle dont on précisera les caractéristiques (centre et rayon). 
 $$\begin{array}{crcl} 
 & x^2+y^2-4x-4y+4& = & 0 \\ 
 \Leftrightarrow & x^2+y^2+2 \times\left(-2\right)x+2 \times \left(-2\right)y+4 & = & 0 \\ 
 \Leftrightarrow & x^2+2 \times\left(-2\right)x+ \left(-2\right)^2-\left(-2\right)^2+y^2+2 \times \left(-2\right)y + \left(-2\right)^2-\left(-2\right)^2+4 & = & 0 \\ 
 \Leftrightarrow & (x-2)^2+(y-2)^2-\left(-2\right)^2-\left(-2\right)^2+4 & = & 0 \\ 
 \Leftrightarrow & (x-2)^2+(y-2)^2 & = & 4 \\ 
 \Leftrightarrow & (x-2)^2+(y-2)^2 & = & 2^2 \\ 
 \end{array}$$ 
 L'ensemble des points $M(x; y)$ tels que $x^2+y^2-4x-4y+4=0$ est un cercle de centre $\Omega(2;2)$ et de rayon $r =2$\end{frame}


\begin{frame}
\vspace{-10mm}
	\frametitle{Correction 4}
Montrons que l'ensemble des points $M(x; y)$ tels que $x^2+y^2-10x-16y+8=0$ est un cercle dont on précisera les caractéristiques (centre et rayon). 
 $$\begin{array}{crcl} 
 & x^2+y^2-10x-16y+8& = & 0 \\ 
 \Leftrightarrow & x^2+y^2+2 \times\left(-5\right)x+2 \times \left(-8\right)y+8 & = & 0 \\ 
 \Leftrightarrow & x^2+2 \times\left(-5\right)x+ \left(-5\right)^2-\left(-5\right)^2+y^2+2 \times \left(-8\right)y + \left(-8\right)^2-\left(-8\right)^2+8 & = & 0 \\ 
 \Leftrightarrow & (x-5)^2+(y-8)^2-\left(-5\right)^2-\left(-8\right)^2+8 & = & 0 \\ 
 \Leftrightarrow & (x-5)^2+(y-8)^2 & = & 81 \\ 
 \Leftrightarrow & (x-5)^2+(y-8)^2 & = & 9^2 \\ 
 \end{array}$$ 
 L'ensemble des points $M(x; y)$ tels que $x^2+y^2-10x-16y+8=0$ est un cercle de centre $\Omega(5;8)$ et de rayon $r =9$\end{frame}


\begin{frame}
\vspace{-10mm}
	\frametitle{Correction 5}
Montrons que l'ensemble des points $M(x; y)$ tels que $x^2+y^2-18x+4y+69=0$ est un cercle dont on précisera les caractéristiques (centre et rayon). 
 $$\begin{array}{crcl} 
 & x^2+y^2-18x+4y+69& = & 0 \\ 
 \Leftrightarrow & x^2+y^2+2 \times\left(-9\right)x+2 \times 2y+69 & = & 0 \\ 
 \Leftrightarrow & x^2+2 \times\left(-9\right)x+ \left(-9\right)^2-\left(-9\right)^2+y^2+2 \times 2y + 2^2-2^2+69 & = & 0 \\ 
 \Leftrightarrow & (x-9)^2+(y+2)^2-\left(-9\right)^2-2^2+69 & = & 0 \\ 
 \Leftrightarrow & (x-9)^2+(y+2)^2 & = & 16 \\ 
 \Leftrightarrow & (x-9)^2+(y+2)^2 & = & 4^2 \\ 
 \end{array}$$ 
 L'ensemble des points $M(x; y)$ tels que $x^2+y^2-18x+4y+69=0$ est un cercle de centre $\Omega(9;-2)$ et de rayon $r =4$\end{frame}




\end{document}