\documentclass[15pt, mathserif]{beamer}

\usepackage[french]{babel}
\usepackage[T1]{fontenc}
\usepackage[utf8]{inputenc}
%\usepackage{esvect}
\usepackage{bm}
\usepackage{eurosym}
\usepackage{tikz}
\usepackage{pgf,tikz,pgfplots}
\pgfplotsset{compat=1.15}
\usepackage{mathrsfs}
\usetikzlibrary{arrows}
\usetikzlibrary{arrows.meta}

\usetikzlibrary{mindmap}
\usepackage{multicol}
\usepackage[tikz]{bclogo}
\usepackage{tkz-tab}
\usepackage{amsmath, tabu}
\usepackage{esvect} %\vv{AB} pour le vecteur AB

\DeclareMathOperator{\e}{e}

%% Tableau

\usepackage{makecell}
\setcellgapes{1pt}
\makegapedcells
\newcolumntype{R}[1]{>{\raggedleft\arraybackslash }b{#1}}
\newcolumntype{L}[1]{>{\raggedright\arraybackslash }b{#1}}
\newcolumntype{C}[1]{>{\centering\arraybackslash }b{#1}}


%pour avoir des parenthèses rondes dans le package fourier
\DeclareSymbolFont{cmoperators}   {OT1}{cmr} {m}{n}
\DeclareSymbolFont{cmlargesymbols}{OMX}{cmex}{m}{n}

\usefonttheme{professionalfonts} %permet d'enlever un bug avec fourier
\usepackage{fourier}
\DeclareMathDelimiter{(}{\mathopen} {cmoperators}{"28}{cmlargesymbols}{"00}
\DeclareMathDelimiter{)}{\mathclose}{cmoperators}{"29}{cmlargesymbols}{"01}

%Graphiques 

\usepackage{pgf,tikz,pgfplots}
\pgfplotsset{compat=1.15}
\usepackage{mathrsfs}
\usetikzlibrary{arrows}
\usetikzlibrary{mindmap}

%ensembles de nbres

\newcommand{\R}{\mathbb{R}}			%permet d'écrire le R "ensemble des réels"'
\newcommand{\N}{\mathbb{N}}			%permet d'écrire le N "ensemble des entiers naturels"
\newcommand{\Z}{\mathbb{Z}}			%permet d'écrire le Z "ensemble des entiers relatifs"
\newcommand{\Prem}{\mathbb{P}}	%permet d'écrire le P "ensemble des nombres premiers" (qui n'a pas marché avec le \P car il existe déjà)
\newcommand{\D}{\mathbb{D}}
\newcommand{\Df}{\mathcal{D}_f}
\newcommand{\Cf}{\mathcal{C}_f}

\newcommand{\Q}{\mathbb{Q}}


\newcommand{\st}[1]{$(#1_n)_{n \in \N}$}

\usetheme{Madrid}
\useoutertheme{miniframes} % Alternatively: miniframes, infolines, split
\useinnertheme{circles}
\definecolor{UBCblue}{rgb}{0.1, 0.25, 0.4} % UBC Blue (primary)
\definecolor{bordeaux}{RGB}{128,0,0}
\usecolortheme[named=UBCblue]{structure}

\usepackage{color} % J'aime bien définir mes couleurs
\definecolor{propcolor}{rgb}{0, 0.5, 1}
\definecolor{thcolor}{rgb}{0.6, 0.07, 0.07}
\colorlet{louis}{blue!70!green!60!white}
\colorlet{sakura}{pink!40!red}

\title{Activités Mentales}
\date{24 Août 2023}

\newcommand{\vco}[2]{\begin{pmatrix} #1 \\ #2 \end{pmatrix}} %Coordonnées de vecteur
\newenvironment{eq}{\begin{cases}\begin{tabu}{ccccc}}{\end{tabu}\end{cases}}
\newenvironment{eql}{\begin{cases}\begin{tabu}{cccccl}}{\end{tabu}\end{cases}}
\newenvironment{eqrl}{\begin{cases}\begin{tabu}{rl}}{\end{tabu}\end{cases}}

\newenvironment{Eq}{\begin{center}\begin{tabular}{rrcl}}{\end{tabular}\end{center}}
\newcommand{\ligneq}[2]{$\Longleftrightarrow$ & $#1$ & $=$ & $#2$ \\}
\newcommand{\Ligneq}[2]{ & $#1$ & $=$ & $#2$ \\}

\newenvironment{RPN}{\begin{center}\begin{tabular}{rrclcrcl}}{\end{tabular}\end{center}}
\newcommand{\Lignerpn}[4]{ & $#1$ & $=$ & $#2$ & ou & $#3$ & $=$ & $#4$ \\}
\newcommand{\lignerpn}[4]{$\Longleftrightarrow$ & $#1$ & $=$ & $#2$ & ou & $#3$ & $=$ & $#4$ \\}

\newenvironment{TRPN}{\begin{center}\begin{tabular}{rrclcrclcrcl}}{\end{tabular}\end{center}}
\newcommand{\Lignetrpn}[6]{ & $#1$ & $=$ & $#2$ & ou & $#3$ & $=$ & $#4$ & ou & $#5$ & $=$ & $#6$ \\}
\newcommand{\lignetrpn}[6]{$\Longleftrightarrow$ & $#1$ & $=$ & $#2$ & ou & $#3$ & $=$ & $#4$ & ou & $#5$ & $=$ & $#6$ \\}
\begin{document}

\begin{frame}
    \titlepage
\end{frame}

\begin{frame} 
	\frametitle{Question 1}
On considère la suite \st{u} définie pour $n \in \N$ par 
 
 \hfil$u_n=\e^{-6-\frac{n}{10}}.$
 \begin{enumerate} 
 	 \item Calculer les trois premiers termes de la suite puis conjecturer son sens de variation. 
 	 \item Montrer que la suite \st{u} est une suite géométrique dont on déterminera la raison. 
 	 \item La conjecture précédente est-elle validée ? Justifier. 
 \end{enumerate}\end{frame}


\begin{frame} 
	\frametitle{Question 2}
On considère la suite \st{u} définie pour $n \in \N$ par 
 
 \hfil$u_n=\e^{-4+\frac{n}{5}}.$
 \begin{enumerate} 
 	 \item Calculer les trois premiers termes de la suite puis conjecturer son sens de variation. 
 	 \item Montrer que la suite \st{u} est une suite géométrique dont on déterminera la raison. 
 	 \item La conjecture précédente est-elle validée ? Justifier. 
 \end{enumerate}\end{frame}


\begin{frame} 
	\frametitle{Question 3}
On considère la suite \st{u} définie pour $n \in \N$ par 
 
 \hfil$u_n=\e^{-2-\frac{3n}{10}}.$
 \begin{enumerate} 
 	 \item Calculer les trois premiers termes de la suite puis conjecturer son sens de variation. 
 	 \item Montrer que la suite \st{u} est une suite géométrique dont on déterminera la raison. 
 	 \item La conjecture précédente est-elle validée ? Justifier. 
 \end{enumerate}\end{frame}


\begin{frame} 
	\frametitle{Question 4}
On considère la suite \st{u} définie pour $n \in \N$ par 
 
 \hfil$u_n=\e^{1+\frac{3n}{5}}.$
 \begin{enumerate} 
 	 \item Calculer les trois premiers termes de la suite puis conjecturer son sens de variation. 
 	 \item Montrer que la suite \st{u} est une suite géométrique dont on déterminera la raison. 
 	 \item La conjecture précédente est-elle validée ? Justifier. 
 \end{enumerate}\end{frame}


\begin{frame} 
	\frametitle{Question 5}
On considère la suite \st{u} définie pour $n \in \N$ par 
 
 \hfil$u_n=\e^{5+\frac{n}{2}}.$
 \begin{enumerate} 
 	 \item Calculer les trois premiers termes de la suite puis conjecturer son sens de variation. 
 	 \item Montrer que la suite \st{u} est une suite géométrique dont on déterminera la raison. 
 	 \item La conjecture précédente est-elle validée ? Justifier. 
 \end{enumerate}\end{frame}


\begin{frame}
\vspace{-10mm}
	\frametitle{Correction 1}

 \begin{enumerate} 
 	 \item  \hfil$u_0=e^{-6-\frac{1\times 0}{10}}=\e^{-6.0}$ \hfil$u_1=e^{-6-\frac{1\times 1}{10}}=\e^{-6.1} $ \hfil $u_2=e^{-6-\frac{1\times 2}{10}}=\e^{-6.2}$
 
 Comme $u_0>u_1>u_2$, il semblerait que la suite soit décroissante.
 	 \item $u_n=\e^{-6-\frac{n}{10}}=\e^{-6} \times \e^{\dfrac{-1}{10}\times n} =\e^{-6} \times \left( \e^{\dfrac{-1}{10}} \right)^n=u_0 \times q^n$ 
 
 \st{u} est donc géométrique de raison $q= \e^{\dfrac{-1}{10}}$ et de premier terme $u_0=\e^{-6}$ 
 	 \item Puisque $0<q= \e^{\dfrac{-1}{10}}<1$ et puisque $u_0=\e^{-6}>0$ la suite est bien décroissante.
 \end{enumerate}\end{frame}


\begin{frame}
\vspace{-10mm}
	\frametitle{Correction 2}

 \begin{enumerate} 
 	 \item  \hfil$u_0=e^{-4+\frac{1\times 0}{5}}=\e^{-4.0}$ \hfil$u_1=e^{-4+\frac{1\times 1}{5}}=\e^{-3.8} $ \hfil $u_2=e^{-4+\frac{1\times 2}{5}}=\e^{-3.6}$
 
 Comme $u_0<u_1<u_2$, il semblerait que la suite soit croissante.
 	 \item $u_n=\e^{-4+\frac{n}{5}}=\e^{-4} \times \e^{\dfrac{1}{5}\times n} =\e^{-4} \times \left( \e^{\dfrac{1}{5}} \right)^n=u_0 \times q^n$ 
 
 \st{u} est donc géométrique de raison $q= \e^{\dfrac{1}{5}}$ et de premier terme $u_0=\e^{-4}$ 
 	 \item Puisque $q= \e^{\dfrac{1}{5}}>1$ et puisque $u_0=\e^{-4}>0$ la suite est bien croissante.
 \end{enumerate}\end{frame}


\begin{frame}
\vspace{-10mm}
	\frametitle{Correction 3}

 \begin{enumerate} 
 	 \item  \hfil$u_0=e^{-2-\frac{3\times 0}{10}}=\e^{-2.0}$ \hfil$u_1=e^{-2-\frac{3\times 1}{10}}=\e^{-2.3} $ \hfil $u_2=e^{-2-\frac{3\times 2}{10}}=\e^{-2.6}$
 
 Comme $u_0>u_1>u_2$, il semblerait que la suite soit décroissante.
 	 \item $u_n=\e^{-2-\frac{3n}{10}}=\e^{-2} \times \e^{\dfrac{-3}{10}\times n} =\e^{-2} \times \left( \e^{\dfrac{-3}{10}} \right)^n=u_0 \times q^n$ 
 
 \st{u} est donc géométrique de raison $q= \e^{\dfrac{-3}{10}}$ et de premier terme $u_0=\e^{-2}$ 
 	 \item Puisque $0<q= \e^{\dfrac{-3}{10}}<1$ et puisque $u_0=\e^{-2}>0$ la suite est bien décroissante.
 \end{enumerate}\end{frame}


\begin{frame}
\vspace{-10mm}
	\frametitle{Correction 4}

 \begin{enumerate} 
 	 \item  \hfil$u_0=e^{1+\frac{3\times 0}{5}}=\e^{1.0}$ \hfil$u_1=e^{1+\frac{3\times 1}{5}}=\e^{1.6} $ \hfil $u_2=e^{1+\frac{3\times 2}{5}}=\e^{2.2}$
 
 Comme $u_0<u_1<u_2$, il semblerait que la suite soit croissante.
 	 \item $u_n=\e^{1+\frac{3n}{5}}=\e^{1} \times \e^{\dfrac{3}{5}\times n} =\e^{1} \times \left( \e^{\dfrac{3}{5}} \right)^n=u_0 \times q^n$ 
 
 \st{u} est donc géométrique de raison $q= \e^{\dfrac{3}{5}}$ et de premier terme $u_0=\e^{1}=\e$ 
 	 \item Puisque $q= \e^{\dfrac{3}{5}}>1$ et puisque $u_0=\e^{1}=\e>0$ la suite est bien croissante.
 \end{enumerate}\end{frame}


\begin{frame}
\vspace{-10mm}
	\frametitle{Correction 5}

 \begin{enumerate} 
 	 \item  \hfil$u_0=e^{5+\frac{1\times 0}{2}}=\e^{5.0}$ \hfil$u_1=e^{5+\frac{1\times 1}{2}}=\e^{5.5} $ \hfil $u_2=e^{5+\frac{1\times 2}{2}}=\e^{6.0}$
 
 Comme $u_0<u_1<u_2$, il semblerait que la suite soit croissante.
 	 \item $u_n=\e^{5+\frac{n}{2}}=\e^{5} \times \e^{\dfrac{1}{2}\times n} =\e^{5} \times \left( \e^{\dfrac{1}{2}} \right)^n=u_0 \times q^n$ 
 
 \st{u} est donc géométrique de raison $q= \e^{\dfrac{1}{2}}$ et de premier terme $u_0=\e^{5}$ 
 	 \item Puisque $q= \e^{\dfrac{1}{2}}>1$ et puisque $u_0=\e^{5}>0$ la suite est bien croissante.
 \end{enumerate}\end{frame}




\end{document}