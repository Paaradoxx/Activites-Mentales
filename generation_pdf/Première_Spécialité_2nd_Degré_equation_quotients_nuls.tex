\documentclass[15pt, mathserif]{beamer}

\usepackage[french]{babel}
\usepackage[T1]{fontenc}
\usepackage[utf8]{inputenc}
%\usepackage{esvect}
\usepackage{bm}
\usepackage{eurosym}
\usepackage{tikz}
\usepackage{pgf,tikz,pgfplots}
\pgfplotsset{compat=1.15}
\usepackage{mathrsfs}
\usetikzlibrary{arrows}
\usetikzlibrary{arrows.meta}

\usetikzlibrary{mindmap}
\usepackage{multicol}
\usepackage[tikz]{bclogo}
\usepackage{tkz-tab}
\usepackage{amsmath, tabu}
\usepackage{esvect} %\vv{AB} pour le vecteur AB

\DeclareMathOperator{\e}{e}

%% Tableau

\usepackage{makecell}
\setcellgapes{1pt}
\makegapedcells
\newcolumntype{R}[1]{>{\raggedleft\arraybackslash }b{#1}}
\newcolumntype{L}[1]{>{\raggedright\arraybackslash }b{#1}}
\newcolumntype{C}[1]{>{\centering\arraybackslash }b{#1}}


%pour avoir des parenthèses rondes dans le package fourier
\DeclareSymbolFont{cmoperators}   {OT1}{cmr} {m}{n}
\DeclareSymbolFont{cmlargesymbols}{OMX}{cmex}{m}{n}

\usefonttheme{professionalfonts} %permet d'enlever un bug avec fourier
\usepackage{fourier}
\DeclareMathDelimiter{(}{\mathopen} {cmoperators}{"28}{cmlargesymbols}{"00}
\DeclareMathDelimiter{)}{\mathclose}{cmoperators}{"29}{cmlargesymbols}{"01}

%Graphiques 

\usepackage{pgf,tikz,pgfplots}
\pgfplotsset{compat=1.15}
\usepackage{mathrsfs}
\usetikzlibrary{arrows}
\usetikzlibrary{mindmap}

%ensembles de nbres

\newcommand{\R}{\mathbb{R}}			%permet d'écrire le R "ensemble des réels"'
\newcommand{\N}{\mathbb{N}}			%permet d'écrire le N "ensemble des entiers naturels"
\newcommand{\Z}{\mathbb{Z}}			%permet d'écrire le Z "ensemble des entiers relatifs"
\newcommand{\Prem}{\mathbb{P}}	%permet d'écrire le P "ensemble des nombres premiers" (qui n'a pas marché avec le \P car il existe déjà)
\newcommand{\D}{\mathbb{D}}
\newcommand{\Df}{\mathcal{D}_f}
\newcommand{\Cf}{\mathcal{C}_f}

\newcommand{\Q}{\mathbb{Q}}


\newcommand{\st}[1]{$(#1_n)_{n \in \N}$}

\usetheme{Madrid}
\useoutertheme{miniframes} % Alternatively: miniframes, infolines, split
\useinnertheme{circles}
\definecolor{UBCblue}{rgb}{0.1, 0.25, 0.4} % UBC Blue (primary)
\definecolor{bordeaux}{RGB}{128,0,0}
\usecolortheme[named=UBCblue]{structure}

\usepackage{color} % J'aime bien définir mes couleurs
\definecolor{propcolor}{rgb}{0, 0.5, 1}
\definecolor{thcolor}{rgb}{0.6, 0.07, 0.07}
\colorlet{louis}{blue!70!green!60!white}
\colorlet{sakura}{pink!40!red}

\title{Activités Mentales}
\date{24 Août 2023}

\newcommand{\vco}[2]{\begin{pmatrix} #1 \\ #2 \end{pmatrix}} %Coordonnées de vecteur
\newenvironment{eq}{\begin{cases}\begin{tabu}{ccccc}}{\end{tabu}\end{cases}}
\newenvironment{eql}{\begin{cases}\begin{tabu}{cccccl}}{\end{tabu}\end{cases}}
\newenvironment{eqrl}{\begin{cases}\begin{tabu}{rl}}{\end{tabu}\end{cases}}

\newenvironment{Eq}{\begin{center}\begin{tabular}{rrcl}}{\end{tabular}\end{center}}
\newcommand{\ligneq}[2]{$\Longleftrightarrow$ & $#1$ & $=$ & $#2$ \\}
\newcommand{\Ligneq}[2]{ & $#1$ & $=$ & $#2$ \\}

\newenvironment{RPN}{\begin{center}\begin{tabular}{rrclcrcl}}{\end{tabular}\end{center}}
\newcommand{\Lignerpn}[4]{ & $#1$ & $=$ & $#2$ & ou & $#3$ & $=$ & $#4$ \\}
\newcommand{\lignerpn}[4]{$\Longleftrightarrow$ & $#1$ & $=$ & $#2$ & ou & $#3$ & $=$ & $#4$ \\}

\newenvironment{TRPN}{\begin{center}\begin{tabular}{rrclcrclcrcl}}{\end{tabular}\end{center}}
\newcommand{\Lignetrpn}[6]{ & $#1$ & $=$ & $#2$ & ou & $#3$ & $=$ & $#4$ & ou & $#5$ & $=$ & $#6$ \\}
\newcommand{\lignetrpn}[6]{$\Longleftrightarrow$ & $#1$ & $=$ & $#2$ & ou & $#3$ & $=$ & $#4$ & ou & $#5$ & $=$ & $#6$ \\}
\begin{document}

\begin{frame}
    \titlepage
\end{frame}

\begin{frame} 
	\frametitle{Question 1}
On définit l'équation $(E):\dfrac{-2x-4}{-4x+1}=\dfrac{-3x-1}{-6x-4}$. 
 \begin{enumerate} 
 	 \item Pour quelle(s) valeur(s) de $x$ l'équation $(E)$ existe-t-elle ? 
 	 \item Montrer que $(E) \Leftrightarrow \dfrac{31x+17}{(-4x+1)(-6x-4)}=0$. 
 	 \item En déduire les solutions de $(E)$. 
 \end{enumerate}\end{frame}


\begin{frame} 
	\frametitle{Question 2}
On définit l'équation $(E):\dfrac{-6x-2}{2x-6}=\dfrac{-12x+4}{4x-4}$. 
 \begin{enumerate} 
 	 \item Pour quelle(s) valeur(s) de $x$ l'équation $(E)$ existe-t-elle ? 
 	 \item Montrer que $(E) \Leftrightarrow \dfrac{-64x+32}{(2x-6)(4x-4)}=0$. 
 	 \item En déduire les solutions de $(E)$. 
 \end{enumerate}\end{frame}


\begin{frame} 
	\frametitle{Question 3}
On définit l'équation $(E):\dfrac{6x-1}{-x-4}=\dfrac{24x+6}{-4x-2}$. 
 \begin{enumerate} 
 	 \item Pour quelle(s) valeur(s) de $x$ l'équation $(E)$ existe-t-elle ? 
 	 \item Montrer que $(E) \Leftrightarrow \dfrac{94x+26}{(-x-4)(-4x-2)}=0$. 
 	 \item En déduire les solutions de $(E)$. 
 \end{enumerate}\end{frame}


\begin{frame} 
	\frametitle{Question 4}
On définit l'équation $(E):\dfrac{6x+5}{-4x+6}=\dfrac{9x+3}{-6x-4}$. 
 \begin{enumerate} 
 	 \item Pour quelle(s) valeur(s) de $x$ l'équation $(E)$ existe-t-elle ? 
 	 \item Montrer que $(E) \Leftrightarrow \dfrac{-96x-38}{(-4x+6)(-6x-4)}=0$. 
 	 \item En déduire les solutions de $(E)$. 
 \end{enumerate}\end{frame}


\begin{frame} 
	\frametitle{Question 5}
On définit l'équation $(E):\dfrac{4x+2}{-2x-3}=\dfrac{-6x-6}{3x-4}$. 
 \begin{enumerate} 
 	 \item Pour quelle(s) valeur(s) de $x$ l'équation $(E)$ existe-t-elle ? 
 	 \item Montrer que $(E) \Leftrightarrow \dfrac{-40x-26}{(-2x-3)(3x-4)}=0$. 
 	 \item En déduire les solutions de $(E)$. 
 \end{enumerate}\end{frame}


\begin{frame}
\vspace{-10mm}
	\frametitle{Correction 1}
 \vspace*{1cm} 
 
 On définit l'équation $(E):\dfrac{-2x-4}{-4x+1}=\dfrac{-3x-1}{-6x-4}$. 
 \begin{enumerate} 
 	 \item Pour quelle(s) valeur(s) de $x$ l'équation $(E)$ existe-t-elle ? 
 	 \item Montrer que $(E) \Leftrightarrow \dfrac{31x+17}{(-4x+1)(-6x-4)}=0$. 
 	 \item En déduire les solutions de $(E)$. 
 \end{enumerate} 
 
 \begin{enumerate} 
 	 \item On cherche les valeurs interdites. Pour cela les dénominateurs ne doivent pas valoir 0 car on ne peut pas diviser par 0. 
 
 	 On résout : 
 
  \begin{minipage}{0.45\linewidth} 
 	 $\begin{array}{crcl} 
 	 	 & -4x+1& = & 0 \\ 
 	 	 	 \Leftrightarrow &-4x & = &-1\\ 
 	 	 	 \Leftrightarrow & x & = & \dfrac{-1}{-4}=\dfrac{1}{4}
 	 	 \end{array} $ 
 	 \end{minipage} \hfil \begin{minipage}{0.45\linewidth} 
 	 $\begin{array}{crcl} 
 	 	 & -6x-4& = & 0 \\ 
 	 	 	 \Leftrightarrow &-6x & = &4 \\ 
 	 	 	 \Leftrightarrow & x & = & \dfrac{4}{-6}=\dfrac{-2}{3}
 	 	 \end{array} $ 
 
 	 \end{minipage} 
 
 Donc l'équation $(E)$ existe pour $x \in \R \setminus \left\{\dfrac{1}{4};\dfrac{-2}{3}\right\}$. 
 \end{enumerate} 
 \end{frame} 
 \begin{frame} 
 \begin{enumerate} \setcounter{enumi}{1}  
 	 \item On veut obtenir $0$ dans le membre de droite, et une seule fraction à gauche : 
 
 	 $\begin{array}{crcl} 
 	 	 & \dfrac{-2x-4}{-4x+1} & = &\dfrac{-3x-1}{-6x-4} \\ 
 	 \Leftrightarrow & \dfrac{-2x-4}{-4x+1}-\dfrac{-3x-1}{-6x-4} & = & 0 \\ 
 
 \Leftrightarrow & \dfrac{(-2x-4)(-6x-4)-(-3x-1)(-4x+1)}{(-4x+1)(-6x-4)} & = & 0 \\ \Leftrightarrow & \dfrac{12x^2+8x+24x+16-(+12x^2+4x-3x-1)}{(-4x+1)(-6x-4)}& = & 0 \\ \Leftrightarrow & \dfrac{31x+17}{(-4x+1)(-6x-4)} & = & 0 
 	 \end{array}$ 
 \end{enumerate} 
 \end{frame} 
 \begin{frame} 
 \begin{enumerate} \setcounter{enumi}{1}  
 	 \item C'est une équation quotient nul, on utilise donc la Règle du Quotient Nul :
 
  \tiny{$\begin{array}{crclcrclcrcl} 
 
 	  & \dfrac{-2x-4}{-4x+1} & = &\dfrac{-3x-1}{-6x-4} & & & & & & & & \\ 
 	 \Leftrightarrow & \dfrac{31x+17}{(-4x+1)(-6x-4)} & = & 0 & & & & & & & & \\ 
 	 \Leftrightarrow & 31x+17 & =& 0 & \text{et} & (-4x+1)(-6x-4) & \neq & 0 \\ 
 	 \Leftrightarrow & 31x&=&-17& \text{et} & -4x+1& \neq & 0 & \text{et} & -6x-4& \neq & 0 \\ 
 	 \Leftrightarrow & x&=&\dfrac{-17}{31} & \text{et} & x &\neq&\dfrac{1}{4} & \text{et} & x & \neq&\dfrac{-2}{3}
 
 \end{array}$} 
 
 \bigskip 
 
 \normalsize{ $ \dfrac{-17}{31} $ n'est pas une valeur interdite donc $S=\left\{ \dfrac{-17}{31}\right\}$.} 
 
 \end{enumerate} \end{frame}


\begin{frame}
\vspace{-10mm}
	\frametitle{Correction 2}
 \vspace*{1cm} 
 
 On définit l'équation $(E):\dfrac{-6x-2}{2x-6}=\dfrac{-12x+4}{4x-4}$. 
 \begin{enumerate} 
 	 \item Pour quelle(s) valeur(s) de $x$ l'équation $(E)$ existe-t-elle ? 
 	 \item Montrer que $(E) \Leftrightarrow \dfrac{-64x+32}{(2x-6)(4x-4)}=0$. 
 	 \item En déduire les solutions de $(E)$. 
 \end{enumerate} 
 
 \begin{enumerate} 
 	 \item On cherche les valeurs interdites. Pour cela les dénominateurs ne doivent pas valoir 0 car on ne peut pas diviser par 0. 
 
 	 On résout : 
 
  \begin{minipage}{0.45\linewidth} 
 	 $\begin{array}{crcl} 
 	 	 & 2x-6& = & 0 \\ 
 	 	 	 \Leftrightarrow &2x & = &6\\ 
 	 	 	 \Leftrightarrow & x & = & \dfrac{6}{2}=3
 	 	 \end{array} $ 
 	 \end{minipage} \hfil \begin{minipage}{0.45\linewidth} 
 	 $\begin{array}{crcl} 
 	 	 & 4x-4& = & 0 \\ 
 	 	 	 \Leftrightarrow &4x & = &4 \\ 
 	 	 	 \Leftrightarrow & x & = & \dfrac{4}{4}=1
 	 	 \end{array} $ 
 
 	 \end{minipage} 
 
 Donc l'équation $(E)$ existe pour $x \in \R \setminus \left\{3;1\right\}$. 
 \end{enumerate} 
 \end{frame} 
 \begin{frame} 
 \begin{enumerate} \setcounter{enumi}{1}  
 	 \item On veut obtenir $0$ dans le membre de droite, et une seule fraction à gauche : 
 
 	 $\begin{array}{crcl} 
 	 	 & \dfrac{-6x-2}{2x-6} & = &\dfrac{-12x+4}{4x-4} \\ 
 	 \Leftrightarrow & \dfrac{-6x-2}{2x-6}-\dfrac{-12x+4}{4x-4} & = & 0 \\ 
 
 \Leftrightarrow & \dfrac{(-6x-2)(4x-4)-(-12x+4)(2x-6)}{(2x-6)(4x-4)} & = & 0 \\ \Leftrightarrow & \dfrac{-24x^2+24x-8x+8-(-24x^2+8x+72x-24)}{(2x-6)(4x-4)}& = & 0 \\ \Leftrightarrow & \dfrac{-64x+32}{(2x-6)(4x-4)} & = & 0 
 	 \end{array}$ 
 \end{enumerate} 
 \end{frame} 
 \begin{frame} 
 \begin{enumerate} \setcounter{enumi}{1}  
 	 \item C'est une équation quotient nul, on utilise donc la Règle du Quotient Nul :
 
  \tiny{$\begin{array}{crclcrclcrcl} 
 
 	  & \dfrac{-6x-2}{2x-6} & = &\dfrac{-12x+4}{4x-4} & & & & & & & & \\ 
 	 \Leftrightarrow & \dfrac{-64x+32}{(2x-6)(4x-4)} & = & 0 & & & & & & & & \\ 
 	 \Leftrightarrow & -64x+32 & =& 0 & \text{et} & (2x-6)(4x-4) & \neq & 0 \\ 
 	 \Leftrightarrow & -64x&=&-32& \text{et} & 2x-6& \neq & 0 & \text{et} & 4x-4& \neq & 0 \\ 
 	 \Leftrightarrow & x&=&\dfrac{1}{2} & \text{et} & x &\neq&3 & \text{et} & x & \neq&1
 
 \end{array}$} 
 
 \bigskip 
 
 \normalsize{ $ \dfrac{1}{2} $ n'est pas une valeur interdite donc $S=\left\{ \dfrac{1}{2}\right\}$.} 
 
 \end{enumerate} \end{frame}


\begin{frame}
\vspace{-10mm}
	\frametitle{Correction 3}
 \vspace*{1cm} 
 
 On définit l'équation $(E):\dfrac{6x-1}{-x-4}=\dfrac{24x+6}{-4x-2}$. 
 \begin{enumerate} 
 	 \item Pour quelle(s) valeur(s) de $x$ l'équation $(E)$ existe-t-elle ? 
 	 \item Montrer que $(E) \Leftrightarrow \dfrac{94x+26}{(-x-4)(-4x-2)}=0$. 
 	 \item En déduire les solutions de $(E)$. 
 \end{enumerate} 
 
 \begin{enumerate} 
 	 \item On cherche les valeurs interdites. Pour cela les dénominateurs ne doivent pas valoir 0 car on ne peut pas diviser par 0. 
 
 	 On résout : 
 
  \begin{minipage}{0.45\linewidth} 
 	 $\begin{array}{crcl} 
 	 	 & -x-4& = & 0 \\ 
 	 	 	 \Leftrightarrow &-x & = &4\\ 
 	 	 	 \Leftrightarrow & x & = & \dfrac{4}{-1}=-4
 	 	 \end{array} $ 
 	 \end{minipage} \hfil \begin{minipage}{0.45\linewidth} 
 	 $\begin{array}{crcl} 
 	 	 & -4x-2& = & 0 \\ 
 	 	 	 \Leftrightarrow &-4x & = &2 \\ 
 	 	 	 \Leftrightarrow & x & = & \dfrac{2}{-4}=\dfrac{-1}{2}
 	 	 \end{array} $ 
 
 	 \end{minipage} 
 
 Donc l'équation $(E)$ existe pour $x \in \R \setminus \left\{-4;-1\right\}$. 
 \end{enumerate} 
 \end{frame} 
 \begin{frame} 
 \begin{enumerate} \setcounter{enumi}{1}  
 	 \item On veut obtenir $0$ dans le membre de droite, et une seule fraction à gauche : 
 
 	 $\begin{array}{crcl} 
 	 	 & \dfrac{6x-1}{-x-4} & = &\dfrac{24x+6}{-4x-2} \\ 
 	 \Leftrightarrow & \dfrac{6x-1}{-x-4}-\dfrac{24x+6}{-4x-2} & = & 0 \\ 
 
 \Leftrightarrow & \dfrac{(6x-1)(-4x-2)-(24x+6)(-x-4)}{(-x-4)(-4x-2)} & = & 0 \\ \Leftrightarrow & \dfrac{-24x^2-12x+4x+2-(-24x^2-6x-96x-24)}{(-x-4)(-4x-2)}& = & 0 \\ \Leftrightarrow & \dfrac{94x+26}{(-x-4)(-4x-2)} & = & 0 
 	 \end{array}$ 
 \end{enumerate} 
 \end{frame} 
 \begin{frame} 
 \begin{enumerate} \setcounter{enumi}{1}  
 	 \item C'est une équation quotient nul, on utilise donc la Règle du Quotient Nul :
 
  \tiny{$\begin{array}{crclcrclcrcl} 
 
 	  & \dfrac{6x-1}{-x-4} & = &\dfrac{24x+6}{-4x-2} & & & & & & & & \\ 
 	 \Leftrightarrow & \dfrac{94x+26}{(-x-4)(-4x-2)} & = & 0 & & & & & & & & \\ 
 	 \Leftrightarrow & 94x+26 & =& 0 & \text{et} & (-x-4)(-4x-2) & \neq & 0 \\ 
 	 \Leftrightarrow & 94x&=&-26& \text{et} & -x-4& \neq & 0 & \text{et} & -4x-2& \neq & 0 \\ 
 	 \Leftrightarrow & x&=&\dfrac{-13}{47} & \text{et} & x &\neq&-4 & \text{et} & x & \neq&\dfrac{-1}{2}
 
 \end{array}$} 
 
 \bigskip 
 
 \normalsize{ $ \dfrac{-13}{47} $ n'est pas une valeur interdite donc $S=\left\{ \dfrac{-13}{47}\right\}$.} 
 
 \end{enumerate} \end{frame}


\begin{frame}
\vspace{-10mm}
	\frametitle{Correction 4}
 \vspace*{1cm} 
 
 On définit l'équation $(E):\dfrac{6x+5}{-4x+6}=\dfrac{9x+3}{-6x-4}$. 
 \begin{enumerate} 
 	 \item Pour quelle(s) valeur(s) de $x$ l'équation $(E)$ existe-t-elle ? 
 	 \item Montrer que $(E) \Leftrightarrow \dfrac{-96x-38}{(-4x+6)(-6x-4)}=0$. 
 	 \item En déduire les solutions de $(E)$. 
 \end{enumerate} 
 
 \begin{enumerate} 
 	 \item On cherche les valeurs interdites. Pour cela les dénominateurs ne doivent pas valoir 0 car on ne peut pas diviser par 0. 
 
 	 On résout : 
 
  \begin{minipage}{0.45\linewidth} 
 	 $\begin{array}{crcl} 
 	 	 & -4x+6& = & 0 \\ 
 	 	 	 \Leftrightarrow &-4x & = &-6\\ 
 	 	 	 \Leftrightarrow & x & = & \dfrac{-6}{-4}=\dfrac{3}{2}
 	 	 \end{array} $ 
 	 \end{minipage} \hfil \begin{minipage}{0.45\linewidth} 
 	 $\begin{array}{crcl} 
 	 	 & -6x-4& = & 0 \\ 
 	 	 	 \Leftrightarrow &-6x & = &4 \\ 
 	 	 	 \Leftrightarrow & x & = & \dfrac{4}{-6}=\dfrac{-2}{3}
 	 	 \end{array} $ 
 
 	 \end{minipage} 
 
 Donc l'équation $(E)$ existe pour $x \in \R \setminus \left\{\dfrac{3}{2};\dfrac{-2}{3}\right\}$. 
 \end{enumerate} 
 \end{frame} 
 \begin{frame} 
 \begin{enumerate} \setcounter{enumi}{1}  
 	 \item On veut obtenir $0$ dans le membre de droite, et une seule fraction à gauche : 
 
 	 $\begin{array}{crcl} 
 	 	 & \dfrac{6x+5}{-4x+6} & = &\dfrac{9x+3}{-6x-4} \\ 
 	 \Leftrightarrow & \dfrac{6x+5}{-4x+6}-\dfrac{9x+3}{-6x-4} & = & 0 \\ 
 
 \Leftrightarrow & \dfrac{(6x+5)(-6x-4)-(9x+3)(-4x+6)}{(-4x+6)(-6x-4)} & = & 0 \\ \Leftrightarrow & \dfrac{-36x^2-24x-30x-20-(-36x^2-12x+54x+18)}{(-4x+6)(-6x-4)}& = & 0 \\ \Leftrightarrow & \dfrac{-96x-38}{(-4x+6)(-6x-4)} & = & 0 
 	 \end{array}$ 
 \end{enumerate} 
 \end{frame} 
 \begin{frame} 
 \begin{enumerate} \setcounter{enumi}{1}  
 	 \item C'est une équation quotient nul, on utilise donc la Règle du Quotient Nul :
 
  \tiny{$\begin{array}{crclcrclcrcl} 
 
 	  & \dfrac{6x+5}{-4x+6} & = &\dfrac{9x+3}{-6x-4} & & & & & & & & \\ 
 	 \Leftrightarrow & \dfrac{-96x-38}{(-4x+6)(-6x-4)} & = & 0 & & & & & & & & \\ 
 	 \Leftrightarrow & -96x-38 & =& 0 & \text{et} & (-4x+6)(-6x-4) & \neq & 0 \\ 
 	 \Leftrightarrow & -96x&=&+38& \text{et} & -4x+6& \neq & 0 & \text{et} & -6x-4& \neq & 0 \\ 
 	 \Leftrightarrow & x&=&\dfrac{-19}{48} & \text{et} & x &\neq&\dfrac{3}{2} & \text{et} & x & \neq&\dfrac{-2}{3}
 
 \end{array}$} 
 
 \bigskip 
 
 \normalsize{ $ \dfrac{-19}{48} $ n'est pas une valeur interdite donc $S=\left\{ \dfrac{-19}{48}\right\}$.} 
 
 \end{enumerate} \end{frame}


\begin{frame}
\vspace{-10mm}
	\frametitle{Correction 5}
 \vspace*{1cm} 
 
 On définit l'équation $(E):\dfrac{4x+2}{-2x-3}=\dfrac{-6x-6}{3x-4}$. 
 \begin{enumerate} 
 	 \item Pour quelle(s) valeur(s) de $x$ l'équation $(E)$ existe-t-elle ? 
 	 \item Montrer que $(E) \Leftrightarrow \dfrac{-40x-26}{(-2x-3)(3x-4)}=0$. 
 	 \item En déduire les solutions de $(E)$. 
 \end{enumerate} 
 
 \begin{enumerate} 
 	 \item On cherche les valeurs interdites. Pour cela les dénominateurs ne doivent pas valoir 0 car on ne peut pas diviser par 0. 
 
 	 On résout : 
 
  \begin{minipage}{0.45\linewidth} 
 	 $\begin{array}{crcl} 
 	 	 & -2x-3& = & 0 \\ 
 	 	 	 \Leftrightarrow &-2x & = &3\\ 
 	 	 	 \Leftrightarrow & x & = & \dfrac{3}{-2}=\dfrac{-3}{2}
 	 	 \end{array} $ 
 	 \end{minipage} \hfil \begin{minipage}{0.45\linewidth} 
 	 $\begin{array}{crcl} 
 	 	 & 3x-4& = & 0 \\ 
 	 	 	 \Leftrightarrow &3x & = &4 \\ 
 	 	 	 \Leftrightarrow & x & = & \dfrac{4}{3}
 	 	 \end{array} $ 
 
 	 \end{minipage} 
 
 Donc l'équation $(E)$ existe pour $x \in \R \setminus \left\{\dfrac{-3}{2};\dfrac{4}{3}\right\}$. 
 \end{enumerate} 
 \end{frame} 
 \begin{frame} 
 \begin{enumerate} \setcounter{enumi}{1}  
 	 \item On veut obtenir $0$ dans le membre de droite, et une seule fraction à gauche : 
 
 	 $\begin{array}{crcl} 
 	 	 & \dfrac{4x+2}{-2x-3} & = &\dfrac{-6x-6}{3x-4} \\ 
 	 \Leftrightarrow & \dfrac{4x+2}{-2x-3}-\dfrac{-6x-6}{3x-4} & = & 0 \\ 
 
 \Leftrightarrow & \dfrac{(4x+2)(3x-4)-(-6x-6)(-2x-3)}{(-2x-3)(3x-4)} & = & 0 \\ \Leftrightarrow & \dfrac{12x^2-16x+6x-8-(+12x^2+12x+18x+18)}{(-2x-3)(3x-4)}& = & 0 \\ \Leftrightarrow & \dfrac{-40x-26}{(-2x-3)(3x-4)} & = & 0 
 	 \end{array}$ 
 \end{enumerate} 
 \end{frame} 
 \begin{frame} 
 \begin{enumerate} \setcounter{enumi}{1}  
 	 \item C'est une équation quotient nul, on utilise donc la Règle du Quotient Nul :
 
  \tiny{$\begin{array}{crclcrclcrcl} 
 
 	  & \dfrac{4x+2}{-2x-3} & = &\dfrac{-6x-6}{3x-4} & & & & & & & & \\ 
 	 \Leftrightarrow & \dfrac{-40x-26}{(-2x-3)(3x-4)} & = & 0 & & & & & & & & \\ 
 	 \Leftrightarrow & -40x-26 & =& 0 & \text{et} & (-2x-3)(3x-4) & \neq & 0 \\ 
 	 \Leftrightarrow & -40x&=&+26& \text{et} & -2x-3& \neq & 0 & \text{et} & 3x-4& \neq & 0 \\ 
 	 \Leftrightarrow & x&=&\dfrac{-13}{20} & \text{et} & x &\neq&\dfrac{-3}{2} & \text{et} & x & \neq&\dfrac{4}{3}
 
 \end{array}$} 
 
 \bigskip 
 
 \normalsize{ $ \dfrac{-13}{20} $ n'est pas une valeur interdite donc $S=\left\{ \dfrac{-13}{20}\right\}$.} 
 
 \end{enumerate} \end{frame}




\end{document}