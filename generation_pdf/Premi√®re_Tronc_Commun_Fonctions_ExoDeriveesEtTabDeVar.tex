\documentclass[15pt, mathserif]{beamer}

\usepackage[french]{babel}
\usepackage[T1]{fontenc}
\usepackage[utf8]{inputenc}
%\usepackage{esvect}
\usepackage{bm}
\usepackage{eurosym}
\usepackage{tikz}
\usepackage{pgf,tikz,pgfplots}
\pgfplotsset{compat=1.15}
\usepackage{mathrsfs}
\usetikzlibrary{arrows}
\usetikzlibrary{arrows.meta}

\usetikzlibrary{mindmap}
\usepackage{multicol}
\usepackage[tikz]{bclogo}
\usepackage{tkz-tab}
\usepackage{amsmath, tabu}
\usepackage{esvect} %\vv{AB} pour le vecteur AB

\DeclareMathOperator{\e}{e}

%% Tableau

\usepackage{makecell}
\setcellgapes{1pt}
\makegapedcells
\newcolumntype{R}[1]{>{\raggedleft\arraybackslash }b{#1}}
\newcolumntype{L}[1]{>{\raggedright\arraybackslash }b{#1}}
\newcolumntype{C}[1]{>{\centering\arraybackslash }b{#1}}


%pour avoir des parenthèses rondes dans le package fourier
\DeclareSymbolFont{cmoperators}   {OT1}{cmr} {m}{n}
\DeclareSymbolFont{cmlargesymbols}{OMX}{cmex}{m}{n}

\usefonttheme{professionalfonts} %permet d'enlever un bug avec fourier
\usepackage{fourier}
\DeclareMathDelimiter{(}{\mathopen} {cmoperators}{"28}{cmlargesymbols}{"00}
\DeclareMathDelimiter{)}{\mathclose}{cmoperators}{"29}{cmlargesymbols}{"01}

%Graphiques 

\usepackage{pgf,tikz,pgfplots}
\pgfplotsset{compat=1.15}
\usepackage{mathrsfs}
\usetikzlibrary{arrows}
\usetikzlibrary{mindmap}

%ensembles de nbres

\newcommand{\R}{\mathbb{R}}			%permet d'écrire le R "ensemble des réels"'
\newcommand{\N}{\mathbb{N}}			%permet d'écrire le N "ensemble des entiers naturels"
\newcommand{\Z}{\mathbb{Z}}			%permet d'écrire le Z "ensemble des entiers relatifs"
\newcommand{\Prem}{\mathbb{P}}	%permet d'écrire le P "ensemble des nombres premiers" (qui n'a pas marché avec le \P car il existe déjà)
\newcommand{\D}{\mathbb{D}}
\newcommand{\Df}{\mathcal{D}_f}
\newcommand{\Cf}{\mathcal{C}_f}

\newcommand{\Q}{\mathbb{Q}}


\newcommand{\st}[1]{$(#1_n)_{n \in \N}$}

\usetheme{Madrid}
\useoutertheme{miniframes} % Alternatively: miniframes, infolines, split
\useinnertheme{circles}
\definecolor{UBCblue}{rgb}{0.1, 0.25, 0.4} % UBC Blue (primary)
\definecolor{bordeaux}{RGB}{128,0,0}
\usecolortheme[named=UBCblue]{structure}

\usepackage{color} % J'aime bien définir mes couleurs
\definecolor{propcolor}{rgb}{0, 0.5, 1}
\definecolor{thcolor}{rgb}{0.6, 0.07, 0.07}
\colorlet{louis}{blue!70!green!60!white}
\colorlet{sakura}{pink!40!red}

\title{Activités Mentales}
\date{24 Août 2023}

\newcommand{\vco}[2]{\begin{pmatrix} #1 \\ #2 \end{pmatrix}} %Coordonnées de vecteur
\newenvironment{eq}{\begin{cases}\begin{tabu}{ccccc}}{\end{tabu}\end{cases}}
\newenvironment{eql}{\begin{cases}\begin{tabu}{cccccl}}{\end{tabu}\end{cases}}
\newenvironment{eqrl}{\begin{cases}\begin{tabu}{rl}}{\end{tabu}\end{cases}}

\newenvironment{Eq}{\begin{center}\begin{tabular}{rrcl}}{\end{tabular}\end{center}}
\newcommand{\ligneq}[2]{$\Longleftrightarrow$ & $#1$ & $=$ & $#2$ \\}
\newcommand{\Ligneq}[2]{ & $#1$ & $=$ & $#2$ \\}

\newenvironment{RPN}{\begin{center}\begin{tabular}{rrclcrcl}}{\end{tabular}\end{center}}
\newcommand{\Lignerpn}[4]{ & $#1$ & $=$ & $#2$ & ou & $#3$ & $=$ & $#4$ \\}
\newcommand{\lignerpn}[4]{$\Longleftrightarrow$ & $#1$ & $=$ & $#2$ & ou & $#3$ & $=$ & $#4$ \\}

\newenvironment{TRPN}{\begin{center}\begin{tabular}{rrclcrclcrcl}}{\end{tabular}\end{center}}
\newcommand{\Lignetrpn}[6]{ & $#1$ & $=$ & $#2$ & ou & $#3$ & $=$ & $#4$ & ou & $#5$ & $=$ & $#6$ \\}
\newcommand{\lignetrpn}[6]{$\Longleftrightarrow$ & $#1$ & $=$ & $#2$ & ou & $#3$ & $=$ & $#4$ & ou & $#5$ & $=$ & $#6$ \\}
\begin{document}

\begin{frame}
    \titlepage
\end{frame}

\begin{frame} 
	\frametitle{Question 1}
On considère la fonction $f$ définie sur $\R$ par $f(x)=48x^3 x^2-16x-19$. \begin{enumerate} 
 	 \item Donner l'expression de la dérivée de la fonction $f$ que l'on notera $f'$. 
 	 \item Montrer que pour tout $x \in \R$, on a $-4(6x+2)(-6x+2)=144x^2x-16$. 
 	 \item Construire le tableau de signe de la fonction définie sur $\R$ par 
 \hfil$-4(6x+2)(-6x+2)$ 
 	 \item En déduire les variations de la fonction $f$. 
 
 \end{enumerate} 
 
 \end{frame}


\begin{frame} 
	\frametitle{Question 2}
On considère la fonction $f$ définie sur $\R$ par $f(x)=-16x^3 -56x^2+20x-10$. \begin{enumerate} 
 	 \item Donner l'expression de la dérivée de la fonction $f$ que l'on notera $f'$. 
 	 \item Montrer que pour tout $x \in \R$, on a $4(2x+5)(-6x+1)=-48x^2-112x+20$. 
 	 \item Construire le tableau de signe de la fonction définie sur $\R$ par 
 \hfil$4(2x+5)(-6x+1)$ 
 	 \item En déduire les variations de la fonction $f$. 
 
 \end{enumerate} 
 
 \end{frame}


\begin{frame} 
	\frametitle{Question 3}
On considère la fonction $f$ définie sur $\R$ par $f(x)=16x^3 -36x^2-48x+30$. \begin{enumerate} 
 	 \item Donner l'expression de la dérivée de la fonction $f$ que l'on notera $f'$. 
 	 \item Montrer que pour tout $x \in \R$, on a $6(-4x-2)(-2x+4)=48x^2-72x-48$. 
 	 \item Construire le tableau de signe de la fonction définie sur $\R$ par 
 \hfil$6(-4x-2)(-2x+4)$ 
 	 \item En déduire les variations de la fonction $f$. 
 
 \end{enumerate} 
 
 \end{frame}


\begin{frame} 
	\frametitle{Question 4}
On considère la fonction $f$ définie sur $\R$ par $f(x)=15x^3 +15x^2-120x+20$. \begin{enumerate} 
 	 \item Donner l'expression de la dérivée de la fonction $f$ que l'on notera $f'$. 
 	 \item Montrer que pour tout $x \in \R$, on a $5(-3x-6)(-3x+4)=45x^2+30x-120$. 
 	 \item Construire le tableau de signe de la fonction définie sur $\R$ par 
 \hfil$5(-3x-6)(-3x+4)$ 
 	 \item En déduire les variations de la fonction $f$. 
 
 \end{enumerate} 
 
 \end{frame}


\begin{frame} 
	\frametitle{Question 5}
On considère la fonction $f$ définie sur $\R$ par $f(x)=40x^3 -36x^2-48x-6$. \begin{enumerate} 
 	 \item Donner l'expression de la dérivée de la fonction $f$ que l'on notera $f'$. 
 	 \item Montrer que pour tout $x \in \R$, on a $-6(5x+2)(-4x+4)=120x^2-72x-48$. 
 	 \item Construire le tableau de signe de la fonction définie sur $\R$ par 
 \hfil$-6(5x+2)(-4x+4)$ 
 	 \item En déduire les variations de la fonction $f$. 
 
 \end{enumerate} 
 
 \end{frame}


\begin{frame}
\vspace{-10mm}
	\frametitle{Correction 1}
\begin{enumerate} 
 	 \item Soit $x \in \R$, on a $$f(x)=48\textcolor{blue}{x^3}-16\textcolor{blue}{x}-19$$
 
 On a alors pour tout $x \in  \R$, $$f'(x)= 48\times \textcolor{blue}{3x^2}-16\times \textcolor{blue}{1}+\textcolor{blue}{0}=144x^2-16$$
 	 \item Soit $x \in \R$, \begin{align*} 
 -4(6x+2)(-6x+2) & = -4\left( -36x^2 +12x -12x +4\right) \\ 
 &=  -4\left( -36x^2 x +4\right) \\ 
 &= 144x^2 x -16
 \end{align*} \end{enumerate} 
 
 \end{frame} 
 
 \begin{frame} 
 
 \begin{enumerate} 
 \setcounter{enumi}{2} 
 
 	 \item On pose $A(x)= 6x+2$ et $B(x) = -6x+2$.
 \bigskip 
 \begin{itemize}
	\item $A$ est une fonction affine avec $m =6>0$. $f$ est donc croissante sur $\mathbb{R}$. Elle est donc d'abord négative puis positive. .

	 De plus $A(x) = 0 \Leftrightarrow x = \dfrac{-1}{3}$. 
 \bigskip 
	\item $B$ est une fonction affine avec $m =-6<0$. $B$ est donc décroissante sur $\mathbb{R}$. Elle est donc d'abord positive puis négative. sur $\mathbb{R}$.

	 De plus $B(x) = 0 \Leftrightarrow x = \dfrac{1}{3}$.
\end{itemize}
 On compare les deux racines obtenues : $ \dfrac{-1}{3} < \dfrac{1}{3}$ 
 \end{enumerate} 
 
 \end{frame}


\begin{frame}On rappelle que $A(x) = 6x+2$ et $B(x) = -6x+2$ et $f'(x) = -4(6x+2)(-6x+2)$. Son tableau de signe est alors 

\medskip \hfil
\begin{tikzpicture}[scale = 0.75]
	\tkzTabInit[lgt = 1.5]{$x$/1.25, $-4$/ 1, $A(x)$ / 1, $B(x)$ / 1, $f'(x)$/1}{$-\infty$, $\dfrac{-1}{3}$, $\dfrac{1}{3}$, $+\infty$}
	\tkzTabLine{ , -, t, -, t, -, }
	\tkzTabLine{ , -, z, +, t, +, }
	\tkzTabLine{ , +, t, +, z, -, }
	\tkzTabLine{ , +, z, -, z, +, }
	\end{tikzpicture}

 \begin{enumerate} 
 \setcounter{enumi}{3} 
 	 \item On en déduit les variations de la fonction $f$ : 

  \medskip \hfil
\begin{tikzpicture}[scale = 0.75]
	\tkzTabInit[lgt = 1.5]{$x$/1.25, $f$/1}{$-\infty$, $\dfrac{-1}{3}$, $\dfrac{1}{3}$, $+\infty$}
	\tkzTabVar{-/ , +/ ,-/,+/}
	
 \end{tikzpicture}

 \end{enumerate} 
 
\end{frame}


\begin{frame}
\vspace{-10mm}
	\frametitle{Correction 2}
\begin{enumerate} 
 	 \item Soit $x \in \R$, on a $$f(x)=-16\textcolor{blue}{x^3}-56\textcolor{blue}{x^2}+20\textcolor{blue}{x-10}$$
 
 On a alors pour tout $x \in  \R$, $$f'(x)= -16\times \textcolor{blue}{3x^2} -56\times \textcolor{blue}{2x}+20\times \textcolor{blue}{1}+\textcolor{blue}{0}=-48x^2-112x+20$$
 	 \item Soit $x \in \R$, \begin{align*} 
 4(2x+5)(-6x+1) & = 4\left( -12x^2 +2x -30x +5\right) \\ 
 &=  4\left( -12x^2 -28x +5\right) \\ 
 &= -48x^2 -112x +20
 \end{align*} \end{enumerate} 
 
 \end{frame} 
 
 \begin{frame} 
 
 \begin{enumerate} 
 \setcounter{enumi}{2} 
 
 	 \item On pose $A(x)= 2x+5$ et $B(x) = -6x+1$.
 \bigskip 
 \begin{itemize}
	\item $A$ est une fonction affine avec $m =2>0$. $f$ est donc croissante sur $\mathbb{R}$. Elle est donc d'abord négative puis positive. .

	 De plus $A(x) = 0 \Leftrightarrow x = \dfrac{-5}{2}$. 
 \bigskip 
	\item $B$ est une fonction affine avec $m =-6<0$. $B$ est donc décroissante sur $\mathbb{R}$. Elle est donc d'abord positive puis négative. sur $\mathbb{R}$.

	 De plus $B(x) = 0 \Leftrightarrow x = \dfrac{1}{6}$.
\end{itemize}
 On compare les deux racines obtenues : $ \dfrac{-5}{2} < \dfrac{1}{6}$ 
 \end{enumerate} 
 
 \end{frame}


\begin{frame}On rappelle que $A(x) = 2x+5$ et $B(x) = -6x+1$ et $f'(x) = 4(2x+5)(-6x+1)$. Son tableau de signe est alors 

\medskip \hfil
\begin{tikzpicture}[scale = 0.75]
	\tkzTabInit[lgt = 1.5]{$x$/1.25, $4$/ 1, $A(x)$ / 1, $B(x)$ / 1, $f'(x)$/1}{$-\infty$, $\dfrac{-5}{2}$, $\dfrac{1}{6}$, $+\infty$}
	\tkzTabLine{ , +, t, +, t, +, }
	\tkzTabLine{ , -, z, +, t, +, }
	\tkzTabLine{ , +, t, +, z, -, }
	\tkzTabLine{ , -, z, +, z, -, }
	\end{tikzpicture}

 \begin{enumerate} 
 \setcounter{enumi}{3} 
 	 \item On en déduit les variations de la fonction $f$ : 

  \medskip \hfil
\begin{tikzpicture}[scale = 0.75]
	\tkzTabInit[lgt = 1.5]{$x$/1.25, $f$/1}{$-\infty$, $\dfrac{-5}{2}$, $\dfrac{1}{6}$, $+\infty$}
	\tkzTabVar{+/ , -/ ,+/,-/}
	
 \end{tikzpicture}

 \end{enumerate} 
 
\end{frame}


\begin{frame}
\vspace{-10mm}
	\frametitle{Correction 3}
\begin{enumerate} 
 	 \item Soit $x \in \R$, on a $$f(x)=16\textcolor{blue}{x^3}-36\textcolor{blue}{x^2}-48\textcolor{blue}{x+30}$$
 
 On a alors pour tout $x \in  \R$, $$f'(x)= 16\times \textcolor{blue}{3x^2} -36\times \textcolor{blue}{2x}-48\times \textcolor{blue}{1}+\textcolor{blue}{0}=48x^2-72x-48$$
 	 \item Soit $x \in \R$, \begin{align*} 
 6(-4x-2)(-2x+4) & = 6\left( 8x^2 -16x +4x -8\right) \\ 
 &=  6\left( 8x^2 -12x -8\right) \\ 
 &= 48x^2 -72x -48
 \end{align*} \end{enumerate} 
 
 \end{frame} 
 
 \begin{frame} 
 
 \begin{enumerate} 
 \setcounter{enumi}{2} 
 
 	 \item On pose $A(x)= -4x-2$ et $B(x) = -2x+4$.
 \bigskip 
 \begin{itemize}
	\item $A$ est une fonction affine avec $m =-4<0$. $f$ est donc décroissante sur $\mathbb{R}$. Elle est donc d'abord positive puis négative. .

	 De plus $A(x) = 0 \Leftrightarrow x = \dfrac{-1}{2}$. 
 \bigskip 
	\item $B$ est une fonction affine avec $m =-2<0$. $B$ est donc décroissante sur $\mathbb{R}$. Elle est donc d'abord positive puis négative. sur $\mathbb{R}$.

	 De plus $B(x) = 0 \Leftrightarrow x = 2$.
\end{itemize}
 On compare les deux racines obtenues : $ \dfrac{-1}{2} < 2$ 
 \end{enumerate} 
 
 \end{frame}


\begin{frame}On rappelle que $A(x) = -4x-2$ et $B(x) = -2x+4$ et $f'(x) = 6(-4x-2)(-2x+4)$. Son tableau de signe est alors 

\medskip \hfil
\begin{tikzpicture}[scale = 0.75]
	\tkzTabInit[lgt = 1.5]{$x$/1.25, $6$/ 1, $A(x)$ / 1, $B(x)$ / 1, $f'(x)$/1}{$-\infty$, $\dfrac{-1}{2}$, $2$, $+\infty$}
	\tkzTabLine{ , +, t, +, t, +, }
	\tkzTabLine{ , +, z, -, t, -, }
	\tkzTabLine{ , +, t, +, z, -, }
	\tkzTabLine{ , +, z, -, z, +, }
	\end{tikzpicture}

 \begin{enumerate} 
 \setcounter{enumi}{3} 
 	 \item On en déduit les variations de la fonction $f$ : 

  \medskip \hfil
\begin{tikzpicture}[scale = 0.75]
	\tkzTabInit[lgt = 1.5]{$x$/1.25, $f$/1}{$-\infty$, $\dfrac{-1}{2}$, $2$, $+\infty$}
	\tkzTabVar{-/ , +/ ,-/,+/}
	
 \end{tikzpicture}

 \end{enumerate} 
 
\end{frame}


\begin{frame}
\vspace{-10mm}
	\frametitle{Correction 4}
\begin{enumerate} 
 	 \item Soit $x \in \R$, on a $$f(x)=15\textcolor{blue}{x^3}+15\textcolor{blue}{x^2}-120\textcolor{blue}{x+20}$$
 
 On a alors pour tout $x \in  \R$, $$f'(x)= 15\times \textcolor{blue}{3x^2} +15\times \textcolor{blue}{2x}-120\times \textcolor{blue}{1}+\textcolor{blue}{0}=45x^2+30x-120$$
 	 \item Soit $x \in \R$, \begin{align*} 
 5(-3x-6)(-3x+4) & = 5\left( 9x^2 -12x +18x -24\right) \\ 
 &=  5\left( 9x^2 +6x -24\right) \\ 
 &= 45x^2 +30x -120
 \end{align*} \end{enumerate} 
 
 \end{frame} 
 
 \begin{frame} 
 
 \begin{enumerate} 
 \setcounter{enumi}{2} 
 
 	 \item On pose $A(x)= -3x-6$ et $B(x) = -3x+4$.
 \bigskip 
 \begin{itemize}
	\item $A$ est une fonction affine avec $m =-3<0$. $f$ est donc décroissante sur $\mathbb{R}$. Elle est donc d'abord positive puis négative. .

	 De plus $A(x) = 0 \Leftrightarrow x = -2$. 
 \bigskip 
	\item $B$ est une fonction affine avec $m =-3<0$. $B$ est donc décroissante sur $\mathbb{R}$. Elle est donc d'abord positive puis négative. sur $\mathbb{R}$.

	 De plus $B(x) = 0 \Leftrightarrow x = \dfrac{4}{3}$.
\end{itemize}
 On compare les deux racines obtenues : $ -2 < \dfrac{4}{3}$ 
 \end{enumerate} 
 
 \end{frame}


\begin{frame}On rappelle que $A(x) = -3x-6$ et $B(x) = -3x+4$ et $f'(x) = 5(-3x-6)(-3x+4)$. Son tableau de signe est alors 

\medskip \hfil
\begin{tikzpicture}[scale = 0.75]
	\tkzTabInit[lgt = 1.5]{$x$/1.25, $5$/ 1, $A(x)$ / 1, $B(x)$ / 1, $f'(x)$/1}{$-\infty$, $-2$, $\dfrac{4}{3}$, $+\infty$}
	\tkzTabLine{ , +, t, +, t, +, }
	\tkzTabLine{ , +, z, -, t, -, }
	\tkzTabLine{ , +, t, +, z, -, }
	\tkzTabLine{ , +, z, -, z, +, }
	\end{tikzpicture}

 \begin{enumerate} 
 \setcounter{enumi}{3} 
 	 \item On en déduit les variations de la fonction $f$ : 

  \medskip \hfil
\begin{tikzpicture}[scale = 0.75]
	\tkzTabInit[lgt = 1.5]{$x$/1.25, $f$/1}{$-\infty$, $-2$, $\dfrac{4}{3}$, $+\infty$}
	\tkzTabVar{-/ , +/ ,-/,+/}
	
 \end{tikzpicture}

 \end{enumerate} 
 
\end{frame}


\begin{frame}
\vspace{-10mm}
	\frametitle{Correction 5}
\begin{enumerate} 
 	 \item Soit $x \in \R$, on a $$f(x)=40\textcolor{blue}{x^3}-36\textcolor{blue}{x^2}-48\textcolor{blue}{x-6}$$
 
 On a alors pour tout $x \in  \R$, $$f'(x)= 40\times \textcolor{blue}{3x^2} -36\times \textcolor{blue}{2x}-48\times \textcolor{blue}{1}+\textcolor{blue}{0}=120x^2-72x-48$$
 	 \item Soit $x \in \R$, \begin{align*} 
 -6(5x+2)(-4x+4) & = -6\left( -20x^2 +20x -8x +8\right) \\ 
 &=  -6\left( -20x^2 +12x +8\right) \\ 
 &= 120x^2 -72x -48
 \end{align*} \end{enumerate} 
 
 \end{frame} 
 
 \begin{frame} 
 
 \begin{enumerate} 
 \setcounter{enumi}{2} 
 
 	 \item On pose $A(x)= 5x+2$ et $B(x) = -4x+4$.
 \bigskip 
 \begin{itemize}
	\item $A$ est une fonction affine avec $m =5>0$. $f$ est donc croissante sur $\mathbb{R}$. Elle est donc d'abord négative puis positive. .

	 De plus $A(x) = 0 \Leftrightarrow x = \dfrac{-2}{5}$. 
 \bigskip 
	\item $B$ est une fonction affine avec $m =-4<0$. $B$ est donc décroissante sur $\mathbb{R}$. Elle est donc d'abord positive puis négative. sur $\mathbb{R}$.

	 De plus $B(x) = 0 \Leftrightarrow x = 1$.
\end{itemize}
 On compare les deux racines obtenues : $ \dfrac{-2}{5} < 1$ 
 \end{enumerate} 
 
 \end{frame}


\begin{frame}On rappelle que $A(x) = 5x+2$ et $B(x) = -4x+4$ et $f'(x) = -6(5x+2)(-4x+4)$. Son tableau de signe est alors 

\medskip \hfil
\begin{tikzpicture}[scale = 0.75]
	\tkzTabInit[lgt = 1.5]{$x$/1.25, $-6$/ 1, $A(x)$ / 1, $B(x)$ / 1, $f'(x)$/1}{$-\infty$, $\dfrac{-2}{5}$, $1$, $+\infty$}
	\tkzTabLine{ , -, t, -, t, -, }
	\tkzTabLine{ , -, z, +, t, +, }
	\tkzTabLine{ , +, t, +, z, -, }
	\tkzTabLine{ , +, z, -, z, +, }
	\end{tikzpicture}

 \begin{enumerate} 
 \setcounter{enumi}{3} 
 	 \item On en déduit les variations de la fonction $f$ : 

  \medskip \hfil
\begin{tikzpicture}[scale = 0.75]
	\tkzTabInit[lgt = 1.5]{$x$/1.25, $f$/1}{$-\infty$, $\dfrac{-2}{5}$, $1$, $+\infty$}
	\tkzTabVar{-/ , +/ ,-/,+/}
	
 \end{tikzpicture}

 \end{enumerate} 
 
\end{frame}




\end{document}