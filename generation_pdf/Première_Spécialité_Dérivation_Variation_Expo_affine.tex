\documentclass[15pt, mathserif]{beamer}

\usepackage[french]{babel}
\usepackage[T1]{fontenc}
\usepackage[utf8]{inputenc}
%\usepackage{esvect}
\usepackage{bm}
\usepackage{eurosym}
\usepackage{tikz}
\usepackage{pgf,tikz,pgfplots}
\pgfplotsset{compat=1.15}
\usepackage{mathrsfs}
\usetikzlibrary{arrows}
\usetikzlibrary{arrows.meta}

\usetikzlibrary{mindmap}
\usepackage{multicol}
\usepackage[tikz]{bclogo}
\usepackage{tkz-tab}
\usepackage{amsmath, tabu}
\usepackage{esvect} %\vv{AB} pour le vecteur AB

\DeclareMathOperator{\e}{e}

%% Tableau

\usepackage{makecell}
\setcellgapes{1pt}
\makegapedcells
\newcolumntype{R}[1]{>{\raggedleft\arraybackslash }b{#1}}
\newcolumntype{L}[1]{>{\raggedright\arraybackslash }b{#1}}
\newcolumntype{C}[1]{>{\centering\arraybackslash }b{#1}}


%pour avoir des parenthèses rondes dans le package fourier
\DeclareSymbolFont{cmoperators}   {OT1}{cmr} {m}{n}
\DeclareSymbolFont{cmlargesymbols}{OMX}{cmex}{m}{n}

\usefonttheme{professionalfonts} %permet d'enlever un bug avec fourier
\usepackage{fourier}
\DeclareMathDelimiter{(}{\mathopen} {cmoperators}{"28}{cmlargesymbols}{"00}
\DeclareMathDelimiter{)}{\mathclose}{cmoperators}{"29}{cmlargesymbols}{"01}

%Graphiques 

\usepackage{pgf,tikz,pgfplots}
\pgfplotsset{compat=1.15}
\usepackage{mathrsfs}
\usetikzlibrary{arrows}
\usetikzlibrary{mindmap}

%ensembles de nbres

\newcommand{\R}{\mathbb{R}}			%permet d'écrire le R "ensemble des réels"'
\newcommand{\N}{\mathbb{N}}			%permet d'écrire le N "ensemble des entiers naturels"
\newcommand{\Z}{\mathbb{Z}}			%permet d'écrire le Z "ensemble des entiers relatifs"
\newcommand{\Prem}{\mathbb{P}}	%permet d'écrire le P "ensemble des nombres premiers" (qui n'a pas marché avec le \P car il existe déjà)
\newcommand{\D}{\mathbb{D}}
\newcommand{\Df}{\mathcal{D}_f}
\newcommand{\Cf}{\mathcal{C}_f}

\newcommand{\Q}{\mathbb{Q}}


\newcommand{\st}[1]{$(#1_n)_{n \in \N}$}

\usetheme{Madrid}
\useoutertheme{miniframes} % Alternatively: miniframes, infolines, split
\useinnertheme{circles}
\definecolor{UBCblue}{rgb}{0.1, 0.25, 0.4} % UBC Blue (primary)
\definecolor{bordeaux}{RGB}{128,0,0}
\usecolortheme[named=UBCblue]{structure}

\usepackage{color} % J'aime bien définir mes couleurs
\definecolor{propcolor}{rgb}{0, 0.5, 1}
\definecolor{thcolor}{rgb}{0.6, 0.07, 0.07}
\colorlet{louis}{blue!70!green!60!white}
\colorlet{sakura}{pink!40!red}

\title{Activités Mentales}
\date{24 Août 2023}

\newcommand{\vco}[2]{\begin{pmatrix} #1 \\ #2 \end{pmatrix}} %Coordonnées de vecteur
\newenvironment{eq}{\begin{cases}\begin{tabu}{ccccc}}{\end{tabu}\end{cases}}
\newenvironment{eql}{\begin{cases}\begin{tabu}{cccccl}}{\end{tabu}\end{cases}}
\newenvironment{eqrl}{\begin{cases}\begin{tabu}{rl}}{\end{tabu}\end{cases}}

\newenvironment{Eq}{\begin{center}\begin{tabular}{rrcl}}{\end{tabular}\end{center}}
\newcommand{\ligneq}[2]{$\Longleftrightarrow$ & $#1$ & $=$ & $#2$ \\}
\newcommand{\Ligneq}[2]{ & $#1$ & $=$ & $#2$ \\}

\newenvironment{RPN}{\begin{center}\begin{tabular}{rrclcrcl}}{\end{tabular}\end{center}}
\newcommand{\Lignerpn}[4]{ & $#1$ & $=$ & $#2$ & ou & $#3$ & $=$ & $#4$ \\}
\newcommand{\lignerpn}[4]{$\Longleftrightarrow$ & $#1$ & $=$ & $#2$ & ou & $#3$ & $=$ & $#4$ \\}

\newenvironment{TRPN}{\begin{center}\begin{tabular}{rrclcrclcrcl}}{\end{tabular}\end{center}}
\newcommand{\Lignetrpn}[6]{ & $#1$ & $=$ & $#2$ & ou & $#3$ & $=$ & $#4$ & ou & $#5$ & $=$ & $#6$ \\}
\newcommand{\lignetrpn}[6]{$\Longleftrightarrow$ & $#1$ & $=$ & $#2$ & ou & $#3$ & $=$ & $#4$ & ou & $#5$ & $=$ & $#6$ \\}
\begin{document}

\begin{frame}
    \titlepage
\end{frame}

\begin{frame} 
	\frametitle{Question 1}
On considère la fonction $f$, définie et dérivable sur $\mathbb{R}$ d'expression \[f(x) = (2x-8)e^{-8x-8}.\]

Étudier les variations de $f$ sur $\mathbb{R}$\end{frame}


\begin{frame} 
	\frametitle{Question 2}
On considère la fonction $f$, définie et dérivable sur $\mathbb{R}$ d'expression \[f(x) = (x+4)e^{-3x-7}.\]

Étudier les variations de $f$ sur $\mathbb{R}$\end{frame}


\begin{frame} 
	\frametitle{Question 3}
On considère la fonction $f$, définie et dérivable sur $\mathbb{R}$ d'expression \[f(x) = (10x-8)e^{5x}.\]

Étudier les variations de $f$ sur $\mathbb{R}$\end{frame}


\begin{frame} 
	\frametitle{Question 4}
On considère la fonction $f$, définie et dérivable sur $\mathbb{R}$ d'expression \[f(x) = (-9x-9)e^{-6x+10}.\]

Étudier les variations de $f$ sur $\mathbb{R}$\end{frame}


\begin{frame} 
	\frametitle{Question 5}
On considère la fonction $f$, définie et dérivable sur $\mathbb{R}$ d'expression \[f(x) = (-9x+9)e^{3x-6}.\]

Étudier les variations de $f$ sur $\mathbb{R}$\end{frame}


\begin{frame}
\vspace{-10mm}
	\frametitle{Correction 1}
Pour tout réel $x$, on a $f(x) =(2x-8)e^{-8x-8}$. On pose pour tout $x \in \mathbb{R}$,\[\begin{matrix}u(x) = 2x-8& v(x) = e^{-8x-8}\\ u'(x) = 2& v'(x) = -8e^{-8x-8}\end{matrix}\]et pour tout réel $x$ on a \begin{align*}f'(x)&= u'(x) \times v(x) + u(x) \times v'(x) \\ &=2e^{-8x-8}+(2x-8)\times \left(-8\right) e^{-8x-8} \\
	 &= (2-16x+64)e^{-8x-8} \\
	 &= (-16x+66)e^{-8x-8}
\end{align*}

\end{frame}

\begin{frame}On a pour tout réel $x$, $f'(x) = (-16x+66)e^{-8x-8}$.

Le signe de la dérivée est donnée par la fonction affine $x \mapsto -16x+66$ car pour tout réel $x$, $e^{-8x-8}>0$.

Or comme $-16<0$, cette fonction est décroissante sur $\mathbb{R}$ et s'annule en $x = \dfrac{33}{8}$.

Finalement, le tableau de variation de $f$ est

\hfil\begin{tikzpicture}
\tkzTabInit{$x$ / 1, $f'(x)$ / 1, $f$ / 1.5}{$-\infty$,$\dfrac{33}{8}$, $+\infty$}
\tkzTabLine{ , +, z, -, }
\tkzTabVar{-/, +/, -/}

\end{tikzpicture}\end{frame}


\begin{frame}
\vspace{-10mm}
	\frametitle{Correction 2}
Pour tout réel $x$, on a $f(x) =(x+4)e^{-3x-7}$. On pose pour tout $x \in \mathbb{R}$,\[\begin{matrix}u(x) = x+4& v(x) = e^{-3x-7}\\ u'(x) = 1& v'(x) = -3e^{-3x-7}\end{matrix}\]et pour tout réel $x$ on a \begin{align*}f'(x)&= u'(x) \times v(x) + u(x) \times v'(x) \\ &=e^{-3x-7}+(x+4)\times \left(-3\right) e^{-3x-7} \\
	 &= (1-3x-12)e^{-3x-7} \\
	 &= (-3x-11)e^{-3x-7}
\end{align*}

\end{frame}

\begin{frame}On a pour tout réel $x$, $f'(x) = (-3x-11)e^{-3x-7}$.

Le signe de la dérivée est donnée par la fonction affine $x \mapsto -3x-11$ car pour tout réel $x$, $e^{-3x-7}>0$.

Or comme $-3<0$, cette fonction est décroissante sur $\mathbb{R}$ et s'annule en $x = \dfrac{-11}{3}$.

Finalement, le tableau de variation de $f$ est

\hfil\begin{tikzpicture}
\tkzTabInit{$x$ / 1, $f'(x)$ / 1, $f$ / 1.5}{$-\infty$,$\dfrac{-11}{3}$, $+\infty$}
\tkzTabLine{ , +, z, -, }
\tkzTabVar{-/, +/, -/}

\end{tikzpicture}\end{frame}


\begin{frame}
\vspace{-10mm}
	\frametitle{Correction 3}
Pour tout réel $x$, on a $f(x) =(10x-8)e^{5x}$. On pose pour tout $x \in \mathbb{R}$,\[\begin{matrix}u(x) = 10x-8& v(x) = e^{5x}\\ u'(x) = 10& v'(x) = 5e^{5x}\end{matrix}\]et pour tout réel $x$ on a \begin{align*}f'(x)&= u'(x) \times v(x) + u(x) \times v'(x) \\ &=10e^{5x}+(10x-8)\times 5 e^{5x} \\
	 &= (10+50x-40)e^{5x} \\
	 &= (50x-30)e^{5x}
\end{align*}

\end{frame}

\begin{frame}On a pour tout réel $x$, $f'(x) = (50x-30)e^{5x}$.

Le signe de la dérivée est donnée par la fonction affine $x \mapsto 50x-30$ car pour tout réel $x$, $e^{5x}>0$.

Or comme $50>0$, cette fonction est croissante sur $\mathbb{R}$ et s'annule en $x = \dfrac{3}{5}$.

Finalement, le tableau de variation de $f$ est

\hfil\begin{tikzpicture}
\tkzTabInit{$x$ / 1, $f'(x)$ / 1, $f$ / 1.5}{$-\infty$,$\dfrac{3}{5}$, $+\infty$}
\tkzTabLine{ , -, z, +, }
\tkzTabVar{+/, -/, +/}

\end{tikzpicture}\end{frame}


\begin{frame}
\vspace{-10mm}
	\frametitle{Correction 4}
Pour tout réel $x$, on a $f(x) =(-9x-9)e^{-6x+10}$. On pose pour tout $x \in \mathbb{R}$,\[\begin{matrix}u(x) = -9x-9& v(x) = e^{-6x+10}\\ u'(x) = -9& v'(x) = -6e^{-6x+10}\end{matrix}\]et pour tout réel $x$ on a \begin{align*}f'(x)&= u'(x) \times v(x) + u(x) \times v'(x) \\ &=-9e^{-6x+10}+(-9x-9)\times \left(-6\right) e^{-6x+10} \\
	 &= (-9+54x+54)e^{-6x+10} \\
	 &= (54x+45)e^{-6x+10}
\end{align*}

\end{frame}

\begin{frame}On a pour tout réel $x$, $f'(x) = (54x+45)e^{-6x+10}$.

Le signe de la dérivée est donnée par la fonction affine $x \mapsto 54x+45$ car pour tout réel $x$, $e^{-6x+10}>0$.

Or comme $54>0$, cette fonction est croissante sur $\mathbb{R}$ et s'annule en $x = \dfrac{-5}{6}$.

Finalement, le tableau de variation de $f$ est

\hfil\begin{tikzpicture}
\tkzTabInit{$x$ / 1, $f'(x)$ / 1, $f$ / 1.5}{$-\infty$,$\dfrac{-5}{6}$, $+\infty$}
\tkzTabLine{ , -, z, +, }
\tkzTabVar{+/, -/, +/}

\end{tikzpicture}\end{frame}


\begin{frame}
\vspace{-10mm}
	\frametitle{Correction 5}
Pour tout réel $x$, on a $f(x) =(-9x+9)e^{3x-6}$. On pose pour tout $x \in \mathbb{R}$,\[\begin{matrix}u(x) = -9x+9& v(x) = e^{3x-6}\\ u'(x) = -9& v'(x) = 3e^{3x-6}\end{matrix}\]et pour tout réel $x$ on a \begin{align*}f'(x)&= u'(x) \times v(x) + u(x) \times v'(x) \\ &=-9e^{3x-6}+(-9x+9)\times 3 e^{3x-6} \\
	 &= (-9-27x+27)e^{3x-6} \\
	 &= (-27x+18)e^{3x-6}
\end{align*}

\end{frame}

\begin{frame}On a pour tout réel $x$, $f'(x) = (-27x+18)e^{3x-6}$.

Le signe de la dérivée est donnée par la fonction affine $x \mapsto -27x+18$ car pour tout réel $x$, $e^{3x-6}>0$.

Or comme $-27<0$, cette fonction est décroissante sur $\mathbb{R}$ et s'annule en $x = \dfrac{2}{3}$.

Finalement, le tableau de variation de $f$ est

\hfil\begin{tikzpicture}
\tkzTabInit{$x$ / 1, $f'(x)$ / 1, $f$ / 1.5}{$-\infty$,$\dfrac{2}{3}$, $+\infty$}
\tkzTabLine{ , +, z, -, }
\tkzTabVar{-/, +/, -/}

\end{tikzpicture}\end{frame}




\end{document}