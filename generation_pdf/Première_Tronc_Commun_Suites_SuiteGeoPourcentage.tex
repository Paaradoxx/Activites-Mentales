\documentclass[15pt, mathserif]{beamer}

\usepackage[french]{babel}
\usepackage[T1]{fontenc}
\usepackage[utf8]{inputenc}
%\usepackage{esvect}
\usepackage{bm}
\usepackage{eurosym}
\usepackage{tikz}
\usepackage{pgf,tikz,pgfplots}
\pgfplotsset{compat=1.15}
\usepackage{mathrsfs}
\usetikzlibrary{arrows}
\usetikzlibrary{arrows.meta}

\usetikzlibrary{mindmap}
\usepackage{multicol}
\usepackage[tikz]{bclogo}
\usepackage{tkz-tab}
\usepackage{amsmath, tabu}
\usepackage{esvect} %\vv{AB} pour le vecteur AB

\DeclareMathOperator{\e}{e}

%% Tableau

\usepackage{makecell}
\setcellgapes{1pt}
\makegapedcells
\newcolumntype{R}[1]{>{\raggedleft\arraybackslash }b{#1}}
\newcolumntype{L}[1]{>{\raggedright\arraybackslash }b{#1}}
\newcolumntype{C}[1]{>{\centering\arraybackslash }b{#1}}


%pour avoir des parenthèses rondes dans le package fourier
\DeclareSymbolFont{cmoperators}   {OT1}{cmr} {m}{n}
\DeclareSymbolFont{cmlargesymbols}{OMX}{cmex}{m}{n}

\usefonttheme{professionalfonts} %permet d'enlever un bug avec fourier
\usepackage{fourier}
\DeclareMathDelimiter{(}{\mathopen} {cmoperators}{"28}{cmlargesymbols}{"00}
\DeclareMathDelimiter{)}{\mathclose}{cmoperators}{"29}{cmlargesymbols}{"01}

%Graphiques 

\usepackage{pgf,tikz,pgfplots}
\pgfplotsset{compat=1.15}
\usepackage{mathrsfs}
\usetikzlibrary{arrows}
\usetikzlibrary{mindmap}

%ensembles de nbres

\newcommand{\R}{\mathbb{R}}			%permet d'écrire le R "ensemble des réels"'
\newcommand{\N}{\mathbb{N}}			%permet d'écrire le N "ensemble des entiers naturels"
\newcommand{\Z}{\mathbb{Z}}			%permet d'écrire le Z "ensemble des entiers relatifs"
\newcommand{\Prem}{\mathbb{P}}	%permet d'écrire le P "ensemble des nombres premiers" (qui n'a pas marché avec le \P car il existe déjà)
\newcommand{\D}{\mathbb{D}}
\newcommand{\Df}{\mathcal{D}_f}
\newcommand{\Cf}{\mathcal{C}_f}

\newcommand{\Q}{\mathbb{Q}}


\newcommand{\st}[1]{$(#1_n)_{n \in \N}$}

\usetheme{Madrid}
\useoutertheme{miniframes} % Alternatively: miniframes, infolines, split
\useinnertheme{circles}
\definecolor{UBCblue}{rgb}{0.1, 0.25, 0.4} % UBC Blue (primary)
\definecolor{bordeaux}{RGB}{128,0,0}
\usecolortheme[named=UBCblue]{structure}

\usepackage{color} % J'aime bien définir mes couleurs
\definecolor{propcolor}{rgb}{0, 0.5, 1}
\definecolor{thcolor}{rgb}{0.6, 0.07, 0.07}
\colorlet{louis}{blue!70!green!60!white}
\colorlet{sakura}{pink!40!red}

\title{Activités Mentales}
\date{24 Août 2023}

\newcommand{\vco}[2]{\begin{pmatrix} #1 \\ #2 \end{pmatrix}} %Coordonnées de vecteur
\newenvironment{eq}{\begin{cases}\begin{tabu}{ccccc}}{\end{tabu}\end{cases}}
\newenvironment{eql}{\begin{cases}\begin{tabu}{cccccl}}{\end{tabu}\end{cases}}
\newenvironment{eqrl}{\begin{cases}\begin{tabu}{rl}}{\end{tabu}\end{cases}}

\newenvironment{Eq}{\begin{center}\begin{tabular}{rrcl}}{\end{tabular}\end{center}}
\newcommand{\ligneq}[2]{$\Longleftrightarrow$ & $#1$ & $=$ & $#2$ \\}
\newcommand{\Ligneq}[2]{ & $#1$ & $=$ & $#2$ \\}

\newenvironment{RPN}{\begin{center}\begin{tabular}{rrclcrcl}}{\end{tabular}\end{center}}
\newcommand{\Lignerpn}[4]{ & $#1$ & $=$ & $#2$ & ou & $#3$ & $=$ & $#4$ \\}
\newcommand{\lignerpn}[4]{$\Longleftrightarrow$ & $#1$ & $=$ & $#2$ & ou & $#3$ & $=$ & $#4$ \\}

\newenvironment{TRPN}{\begin{center}\begin{tabular}{rrclcrclcrcl}}{\end{tabular}\end{center}}
\newcommand{\Lignetrpn}[6]{ & $#1$ & $=$ & $#2$ & ou & $#3$ & $=$ & $#4$ & ou & $#5$ & $=$ & $#6$ \\}
\newcommand{\lignetrpn}[6]{$\Longleftrightarrow$ & $#1$ & $=$ & $#2$ & ou & $#3$ & $=$ & $#4$ & ou & $#5$ & $=$ & $#6$ \\}
\begin{document}

\begin{frame}
    \titlepage
\end{frame}

\begin{frame} 
	\frametitle{Question 1}
 Un vendeur reçoit chaque année une prime de  2000 \euro{} qu'il place systématiquement, toujours à un taux annuel de 8 \% . 
 \begin{enumerate} 
 	 \item À combien s'élèvera le capital au bout de 1 an ? 2ans ?  
 	 \item On considère la suite \st{u} qui représente le capital au bout de $n$ années. Exprimer $u_{n+1}$ en fonction de $u_n$. 
 	 \item Quelle est la nature de la suite \st{u} ? 
 	 \item  À combien s'élèvera le capital au bout de 10 ans ? 
 \end{enumerate} \end{frame}


\begin{frame} 
	\frametitle{Question 2}
 Une voiture, achetée neuve coûtait 16000 \euro ~(en 2022), perd chaque année  17 \% de sa valeur. 
 \begin{enumerate} 
 	 \item Quelle serait la valeur de la voiture en 2023 ? En 2024 ? 
 	 \item On considère la suite \st{u} qui représente la valeur de la voiture au bout de $n$ années. Exprimer $u_{n+1}$ en fonction de $u_n$. 
 	 \item Quelle est la nature de la suite \st{u} ?
 	 \item Quelle serait la valeur de la voiture en 2032. 
 \end{enumerate} \end{frame}


\begin{frame} 
	\frametitle{Question 3}
 Un vendeur reçoit chaque année une prime de  1700 \euro{} qu'il place systématiquement, toujours à un taux annuel de 7 \% . 
 \begin{enumerate} 
 	 \item À combien s'élèvera le capital au bout de 1 an ? 2ans ?  
 	 \item On considère la suite \st{u} qui représente le capital au bout de $n$ années. Exprimer $u_{n+1}$ en fonction de $u_n$. 
 	 \item Quelle est la nature de la suite \st{u} ? 
 	 \item  À combien s'élèvera le capital au bout de 10 ans ? 
 \end{enumerate} \end{frame}


\begin{frame} 
	\frametitle{Question 4}
 Une société du secteur des nouvelles technologies prévoit une augmentation de son chiffre d'affaire de 24 \% . La première année, leur chiffre d'affaire était de  190000 habitants. 
 \begin{enumerate} 
 	 \item Quelle sera le chiffre d'affaire la première année ? La deuxième année ? 
 	 \item On considère la suite \st{u} qui représente le chiffre d'affaire de l'entreprise au bout de $n$ années. Exprimer $u_{n+1}$ en fonction de $u_n$. 
 	 \item Quelle est la nature de la suite \st{u} ? 
 	 \item Quel sera le chiffre d'affaire au bout de 10 ans ? 
 \end{enumerate} \end{frame}


\begin{frame} 
	\frametitle{Question 5}
 Une société du secteur des nouvelles technologies prévoit une augmentation de son chiffre d'affaire de 22 \% . La première année, leur chiffre d'affaire était de  250000 habitants. 
 \begin{enumerate} 
 	 \item Quelle sera le chiffre d'affaire la première année ? La deuxième année ? 
 	 \item On considère la suite \st{u} qui représente le chiffre d'affaire de l'entreprise au bout de $n$ années. Exprimer $u_{n+1}$ en fonction de $u_n$. 
 	 \item Quelle est la nature de la suite \st{u} ? 
 	 \item Quel sera le chiffre d'affaire au bout de 10 ans ? 
 \end{enumerate} \end{frame}


\begin{frame}
\vspace{-10mm}
	\frametitle{Correction 1}
\begin{enumerate} 
 	 \item Augmenter de 8 \% revient à multiplier par  1.08. En 2023, le capital sera de $2000 \times  1.08\simeq 2160.0$ et en 2024 le capital sera donc de  $2160.0 \times  1.08\simeq 2332.8$.  
 	 \item On a pour tout $n \in \N$, 
 
 \hfil$\left\{\begin{array}{rcl} 
 u_{n+1} & = & u_n \times  1.08\\ u_0 & = &  2000\end{array} \right.$ 
 	 \item La suite \st{u} est une suite géométrique car on multiplie à chaque fois par  1.08. 
 	 \item D'après la calculatrice, on a $u_{10}=4317.85$. 
 
 En 2032, le capital sera de 4317.85 \euro. 
 \end{enumerate} 
 
 \end{frame}


\begin{frame}
\vspace{-10mm}
	\frametitle{Correction 2}
\begin{enumerate} 
 	 \item Diminuer de 17 \% revient à multiplier par  0.83. En 2023, la valeur de la voiture sera donc de  $16000 \times  0.83\simeq 13280.0$ et en 2024 la valeur de la voiture sera donc de  $13280.0 \times  0.83\simeq 11022.4$.  
 	 \item On a pour tout $n \in \N$, 
 
 \hfil$\left\{\begin{array}{rcl} 
 u_{n+1} & = & u_n \times  0.83\\ u_0 & = &  16000\end{array} \right.$ 
 	 \item La suite \st{u} est une suite géométrique car on multiplie à chaque fois par  0.83. 
 	 \item D'après la calculatrice, on a $u_{10}=2482.57$. La valeur de la voiture dans 10 ans sera d'environ 2482.57. 
 \end{enumerate} 
 
 \end{frame}


\begin{frame}
\vspace{-10mm}
	\frametitle{Correction 3}
\begin{enumerate} 
 	 \item Augmenter de 7 \% revient à multiplier par  1.07. En 2023, le capital sera de $1700 \times  1.07\simeq 1819.0$ et en 2024 le capital sera donc de  $1819.0 \times  1.07\simeq 1946.33$.  
 	 \item On a pour tout $n \in \N$, 
 
 \hfil$\left\{\begin{array}{rcl} 
 u_{n+1} & = & u_n \times  1.07\\ u_0 & = &  1700\end{array} \right.$ 
 	 \item La suite \st{u} est une suite géométrique car on multiplie à chaque fois par  1.07. 
 	 \item D'après la calculatrice, on a $u_{10}=3344.16$. 
 
 En 2032, le capital sera de 3344.16 \euro. 
 \end{enumerate} 
 
 \end{frame}


\begin{frame}
\vspace{-10mm}
	\frametitle{Correction 4}
\begin{enumerate} 
 	 \item Augmenter de 24 \% revient à multiplier par  1.24. En 2023, le chiffre d'affaires sera de $190000 \times  1.24\simeq 235600.0$ et en 2024 le chiffre d'affaires sera donc de  $235600.0 \times  1.24\simeq 292144.0$.  
 	 \item On a pour tout $n \in \N$, 
 
 \hfil$\left\{\begin{array}{rcl} 
 u_{n+1} & = & u_n \times  1.24\\ u_0 & = &  190000\end{array} \right.$ 
 	 \item La suite \st{u} est une suite géométrique car on multiplie à chaque fois par  1.24. 
 	 \item D'après la calculatrice, on a $u_{10}=1632940.85$. 
 
 En 2032, le chiffre d'affaires sera de 1632940.85 \euro. 
 \end{enumerate} 
 
 \end{frame}


\begin{frame}
\vspace{-10mm}
	\frametitle{Correction 5}
\begin{enumerate} 
 	 \item Augmenter de 22 \% revient à multiplier par  1.22. En 2023, le chiffre d'affaires sera de $250000 \times  1.22\simeq 305000.0$ et en 2024 le chiffre d'affaires sera donc de  $305000.0 \times  1.22\simeq 372100.0$.  
 	 \item On a pour tout $n \in \N$, 
 
 \hfil$\left\{\begin{array}{rcl} 
 u_{n+1} & = & u_n \times  1.22\\ u_0 & = &  250000\end{array} \right.$ 
 	 \item La suite \st{u} est une suite géométrique car on multiplie à chaque fois par  1.22. 
 	 \item D'après la calculatrice, on a $u_{10}=1826157.85$. 
 
 En 2032, le chiffre d'affaires sera de 1826157.85 \euro. 
 \end{enumerate} 
 
 \end{frame}




\end{document}